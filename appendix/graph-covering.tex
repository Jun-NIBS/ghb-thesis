% Copyright (c) 2018, Gabriel Hjort Blindell <ghb@kth.se>
%
% This work is licensed under a Creative Commons 4.0 International License (see
% LICENSE file or visit <http://creativecommons.org/licenses/by/4.0/> for a copy
% of the license).

\chapter{Graph Covering}
\labelAppendix{graph-covering}

\todo{rewrite introduction}

\todo{add figures}

Although \gls{DAG covering} is more general than \gls{tree covering} -- and thus
offers a more powerful approach to \gls{instruction selection} -- it is still
not enough for handling all aspects featured in the \glspl{program} and
\glspl{instruction}.
%
For example, control flow incurred by loop statements cannot be modeled in a
\gls{block DAG} as it requires the use of \glspl{cycle}, which violates the
definition of \glspl{DAG}.
%
By resorting to \glspl{function graph} we attain \gls{graph covering}, which is
the most general form of covering.
%
Unfortunately it also constitutes the shortest appendix among the
\glspl{principle} discussed in this dissertation.

This appendix is based on material presented in
\cite[Chap.\thinspace5]{HjortBlindell:2016:Survey} that has been adapted for
this dissertation.
%
To not disturb the flow of reading, material already presented in
\refChapter{existing-isel-techniques-and-reps} is duplicated in this appendix.


\section{The Principle}

\todo{fix inline figure}

%\begin{inParFigure}{3.5cm}[r]%
%  \centering%
%
%  % LAYOUT FIX:
%  % Make the figure more aligned with the text
%  \vspace*{0.25\baselineskip}
%
%  \input{\figurePath/graph-covering/graph-covering-example}%
%
%  % LAYOUT FIX:
%  % Make some space below
%  \vspace*{\baselineskip}
%\end{inParFigure}%
%\noindent
In \gls{DAG covering}-based \gls{instruction selection}, \glspl{program} can
only be modeled one \gls{block} at a time as \glspl{cycle} are forbidden to
appear in the \glspl{block DAG}.
%
Lifting this restriction makes it possible to incorporate both data and
control-flow information into the \glspl{function graph}, which in turn enables
entire \glspl{function} to be modeled as a single \gls{graph}.
%
Selecting \glspl{instruction} for such \glspl{graph} is known as
\gls!{global.is}[ \gls{instruction selection}] and has several advantages over
\gls{local.is} \gls{instruction selector}.
%
First, with an entire \gls{function} as input, a \gls{global.is}
\gls{instruction selector} can account for the effects of \gls{local.is}
\gls{pattern selection} across the \gls{block} boundaries and is thereby better
informed when making its decisions.
%
In addition, it can move operations from one \gls{block} to another if that
enables better use of the \gls{instruction set} (this is known as \gls{global
  code motion}).
%
Second, to support \gls{inter-block.ic} \glspl{instruction} -- which require
modeling of both data and control-flow information -- it is imperative that the
\glspl{pattern} be expressible using \glspl{graph} that may contain
\glspl{cycle}.
%
This makes \gls{graph covering} one of the key approaches for making use of
fewer but more efficient \glspl{instruction}, which is becoming more and more
crucial for modern \glspl{target machine} -- especially embedded systems --
where both power consumption and heat emission are becoming increasingly
important factors.

%\begin{figure}[t]
%  \centering
%  \begin{framedBoxWI}{7.3cm}
%    \centering%
%    \mathDisplayFontSize\figureFontStyle%
%    \begin{tabular}{c@{~~}|>{\centering\arraybackslash}p{2.5cm}%
%                    >{\centering\arraybackslash}p{3cm}}
%             & Pattern matching & Optimal pattern selection\\
%      \hline
%      % LAYOUT FIX: prevent first row from touching the line above
%      \rule{0pt}{2ex}%
%      Trees  & Linear                     & Linear\\
%      DAGs   & \glsentrytext{NP complete} & \glsentrytext{NP complete}\\
%      Graphs & \glsentrytext{NP complete} & \glsentrytext{NP complete}\\
%    \end{tabular}
%  \end{framedBoxWI}
%
%  \figCaption[Time complexities for solving \glsentrytext{pattern matching} and
%    optimal \glsentrytext{pattern selection}]%
%    {Time complexities for solving the \glsentrytext{pattern matching} and
%      optimal \glsentrytext{pattern selection} problems using various
%      program representations}
%  \labelFigure{time-complexities}
%\end{figure}

However, when transitioning from \glspl{pattern DAG} to \glspl{pattern graph},
we can no longer apply \gls{pattern matching} techniques designed for
\glspl{tree} and \glspl{DAG} but must resort to methods from the field of
\gls{subgraph isomorphism} in solving this problem (see
\refFigure{time-complexities} for a time complexity comparison).
%
The \gls{pattern selection} problem, on the other hand, can still be solved
using many of the techniques which were discussed in \refAppendix{dag-covering}.
%
Therefore, in this appendix we will only examine techniques that were originally
designated for \gls{graph covering}.


\section{Pattern Matching Is a Subgraph Isomorphism Problem}

The \gls{subgraph isomorphism}[ problem] is to detect whether an arbitrary
\gls{graph}~$G_a$ can be turned, twisted, or mirrored such that it forms a
\gls{subgraph} of another \gls{graph}~$G_b$.
%
In such cases one says that $G_a$ is an \glsshort!{isomorphism}[ \gls{subgraph}]
of~$G_b$, and deciding this is known to be NP-complete~\cite{Cook:1971}.
%
It should be clear that this is a generalization of the \gls{pattern matching}
problem, and with appropriate constraints a solution to the corresponding
\gls{subgraph isomorphism} problem can be directly translated into a solution
for the original \gls{pattern matching} problem.\!%
%
\footnote{%
  Most \glspl{pattern} derived from \glspl{instruction} are restrictive in how
  they can be twisted and turned without changing the semantics.
  %
  For example, the \gls{ingoing.e} \glspl{edge} to a addition \gls{node} can be
  swapped due to the commutative nature of addition, but the same does not apply
  to subtraction or division.%
}

As \gls{subgraph isomorphism} is found in many other fields, a vast amount of
research has been devoted to this problem (see for example~\cite{Ullmann:1976,
  CordellaEtAl:2001, GuoEtAl:2003, KrissinelHenrick:2004, SorlinSolnon:2004,
  Gallagher:2006, FanEtAl:2010, FanEtAl:2011, HinoEtAl:2012}).
%
In this section we will mainly look at \citeauthor{Ullmann:1976}'s algorithm and
another commonly used algorithm by \citeauthor{CordellaEtAl:2001}.
%
As a side note, we will also discuss an algorithm that solves the \gls{graph
  isomorphism} problem in polynomial time for a certain class of \glspl{graph}.

It should be noted that although we are now using the most generic \gls{graph}
form for representing the \glspl{program} and \glspl{pattern}, these methods of
\gls{pattern matching} are still only capable of finding the \glspl{match set}
where the \glspl{pattern} have matching \emph{structure}, and not matching
\emph{semantics}.
%
For example, the expressions \mbox{$a * (b + c)$} and \mbox{$a * b + a * c$} are
semantically equivalent but will yield differently structured \glspl{graph}.
%
Hence the \glspl{pattern} selected to cover one \gls{function graph} may differ
from those covering the other -- and consequently may yield different code
quality -- even though both versions of the \gls{assembly code} will produce the
same value in the end.
%
In an attempt to mitigate this issue \textcite{AroraEtAl:2010} introduced a
method where the \glspl{function graph} are first normalized before \gls{pattern
  matching}, but the design is limited to arithmetic \glspl{block DAG} and still
does not guarantee that all \glspl{match} will be found.


\subsection{Ullmann's Algorithm}

One of the first and most well-known methods for deciding \gls{subgraph
  isomorphism} is an algorithm developed by \textcite{Ullmann:1976}.
%
In a seminal paper from~1976, \citeauthor{Ullmann:1976} expresses the problem of
determining whether a \gls{graph} \mbox{$G_a = \mTuple{N_a, E_a}$} is
\glsshort{subgraph isomorphism} to another \gls{graph} \mbox{$G_b = \mTuple{N_b,
    E_b}$} as a problem of finding a Boolean \mbox{$|N_a| \times |N_b|$}
matrix~$\mMatrix{M}$ such that the following conditions holds:
%
\begin{equation}
  \forall 1 \leq i \leq |N_a|, 1 \leq j \leq |N_b| :
  \mMatrix{C} = \mMatrix{M} \cdot (\mMatrix{M} \cdot \mMatrix{B})^T,
  a_{ij} = 1 \implies c_{ij} = 1.
\end{equation}
%
$\mMatrix{A}$ and $\mMatrix{B}$ are the respective \glspl!{adjacency matrix} of
$G_a$ and~$G_b$, where $a_{ij}$ is an element of~$\mMatrix{A}$.
%
Similarly, $c_{ij}$ is an element of~$\mMatrix{C}$.
%
When these conditions hold, every row in $\mMatrix{M}$ will contain exactly
one~1, and every column in $\mMatrix{M}$ will contain at most one~1.

A simple method for finding $\mMatrix{M}$ is to first set every element~$m_{ij}$
to~1, and then iteratively reset them to~0 until a solution is found.
%
As expected, this brute-force approach suffers from ensuing combinatorial
explosion and will thus not be effective.
%
\citeauthor{Ullmann:1976} therefore tried to reduce the search space by
developing a procedure that eliminates some of the 1s that will never appear in
any solution.
%
But according to \textcite{CordellaEtAl:2001}, even with this improvement the
worst-case time complexity of the algorithm is still \mbox{$\mBigO(n!n^2)$}.


\subsection{The VF2 Algorithm}

In 2001, \textcite{CordellaEtAl:2001} introduced another \gls{subgraph isomorphism}
algorithm called \gls{VF2} -- the authors did not say what it stands for --
which has been used in several \glsshort{DAG covering} and \gls{graph
  covering}-based instruction selectors.

In broad outline, the \gls{VF2} algorithm constructs a mapping set consisting of
\mbox{$\mTuple{n, m}$}~pairs, where \mbox{$n \in G_a$} and \mbox{$m \in G_b$}.
%
This set is grown by recursively adding new pairs, one at a time, until either a
solution is found or a dead end is reached.
%
To detect the latter as soon as possible, each new pair is checked against a set
of rules before it is added to the mapping set.
%
The rules are composed by a set of syntactic feasibility checks given
by~$F_{\text{syn}}$, followed by a semantic feasibility check given
by~$F_{\text{sem}}$.
%
Without going into too much detail, we define $F_{\text{syn}}$ as
%
\begin{displaymath}
  F_{\text{syn}}(s, n, m) = R_{\text{pred}} \mOr R_{\text{succ}} \mOr
  R_{\text{in}} \mOr R_{\text{out}} \mOr R_{\text{new}},
\end{displaymath}
%
where $n$ and $m$ constitute the candidate pair under consideration and $s$
represents the current (partial) mapping set.
%
The first two rules, $R_{\text{pred}}$ and $R_{\text{succ}}$, ensure that the
new mapping set is consistent with the structures of $G_a$ and~$G_b$, and the
remaining three rules are used to prune the search space.
%
$R_{\text{in}}$ and $R_{\text{out}}$ perform one-step look-aheads in the search
process and ensure that there will still exist enough unmapped nodes in $G_b$ to
allow the remaining \glspl{node} in~$G_a$ to be mapped.
%
Similarly, $R_{\text{new}}$ performs a two-step look-ahead (but I am not certain
about the intuition behind this rule).
%
If necessary, the rules can be modified with minimal effort to check \gls{graph
  isomorphism} instead of \gls{subgraph isomorphism}.
%
In the former the structures of $G_a$ and $G_b$ must be rigid -- and thus cannot
be twisted and turned in order to match -- which is more fitting for our
purposes.
%
Additional checks, such as ensuring that the \gls{node} types are compatible,
can be added by customizing the definition of~$F_{\text{sem}}$.

Although this algorithm exhibits a worst-case time complexity of
\mbox{$\mBigO(n!n)$}, its best-case time complexity -- which is polynomial --
still makes it an efficient method for performing \gls{pattern matching} over
very large \glspl{graph}.
%
For example, \citeauthor{CordellaEtAl:2001} report in \cite{CordellaEtAl:2001} that the
\gls{VF2} algorithm has been successfully used on \glspl{graph} containing
several thousand \glspl{node}.


\subsection{Graph Isomorphism in Quadratic Time}

\citeauthor{JiangBunke:1996}~\cite{JiangBunke:1996, JiangBunke:1998:Article,
  JiangBunke:1998:Chapter} discovered that if the \glspl{graph} are
\gls!{ordered.g}, meaning all \glspl{edge} belonging to the same node have a
predetermined order among each other, then the \gls{graph isomorphism} problem
can be solved in polynomial time for \gls{undirected.g} \glspl{graph}.
%
This is because \gls{ordered.g} \glspl{graph} contain additional structural
information that can be exploited during \gls{pattern matching}.
%
Although it is unclear whether this discovery can be applied to \gls{instruction
  selection} -- the reasons will become apparent shortly -- I decided to include
its discussion out of personal interest.

\citeauthor{JiangBunke:1996}'s algorithm essentially works as follows.
%
Starting from some \gls{node}, traverse the first \gls{graph}~$G_a$ using
breadth-first search in the order dictated by the \gls{edge} ordering for the
current \gls{node}.
%
When a \gls{node}~$n$ is visited for the first time, assign $n$ a number such
that the number of every \gls{node} is unique for all \glspl{node} in~$G_a$.
%
In addition, every time a \gls{node}~$n$ is encountered, no matter if it has
already been visited or not, record the number of~$n$ onto a sequence.
%
This sequence will always be of length~$2m$, where $m$ is the number of
\glspl{edge} in~$G_a$, as it can be proven that every \gls{edge} will be
traversed exactly twice (once in each direction).
%
Let us denote the sequence produced for $G_a$ when starting from \gls{node}~$n$
as~\mbox{$S(G_a, n)$}.
%
We then do the same for the second \gls{graph}~$G_b$, and if there exists some
\gls{node}~$m$ in $G_b$ such that \mbox{$S(G_b, m) = S(G_a, n)$}, then $G_b$
must be \glsshort{isomorphism} to~$G_a$.
%
In the worst case this can be checked in $\mBigO(e^2)$, where $e$ is the total
number of \glspl{edge} in the two \glspl{graph}.

This algorithm is clearly a vast improvement over those by
\citeauthor{Ullmann:1976} and \citeauthor{CordellaEtAl:2001}, but it also has several
significant limitations that make it difficult to use in practice.
%
First, it requires that all \gls{program} and \glspl{pattern graph} be
\gls{ordered.g}, which is not always the case for \gls{instruction selection}
(for example, \glspl{graph} containing commutative \glspl{node} violate this
restriction).
%
Second, in its current form it detects \gls{graph isomorphism} instead of
\gls{subgraph isomorphism}, meaning the algorithm can only be applied on
\glspl{pattern} that match the entire \gls{function graph}.
%
Although the first problem can be mitigated by duplicating \glspl{pattern} with
commutative \glspl{node}, the second problem is much harder to overcome.


\section{Graph-Based Intermediate Representations}

With \glsshort{tree covering} and \gls{DAG covering}, it is sufficient to
represent the \gls{program} on \gls{block} level.
%
Consequently, \glspl{program} are typically modeled as a \gls{forest} of
\glspl{expression tree} or a set of \glspl{block DAG}.
%
But, as previously stated, this becomes an impediment when applied in \gls{graph
  covering}-based techniques, forcing us to instead use a \gls{graph}-based
\glsdesc{IR}.
%
We will therefore continue by looking briefly at how \glspl{program} can be
expressed using such representations (an excellent survey of this field was
recently made by \textcite{StanierWatson:2013}).


\subsection{Static-Single-Assignment Graphs}

Most modern \glspl{compiler} -- including \gls{GCC} and~\gls{LLVM} -- use
\glspl{IR} based on \gls{SSA}, which is a form where each variable or
\gls{temporary} within a \gls{function} is restricted to being defined only
once.
%
This is typically achieved by rewriting the \gls{program} such that each
variable assignment receives its own uniquely named definition, and an example
of this is given in \refFigure{ssa-example}.
%
One of the main benefits of this is that the \gls{live range} of each variable
is contiguous.
%
The \gls{live range} of a variable can be loosely described as the length within
the \gls{program} where the value of that variable must not be destroyed.
%
This in turn means that each variable corresponds to a single value, which
simplifies many \gls{program} optimization routines.

%\begin{figure}[t]
%  \centering%
%
%  \begin{subfigure}[C function][ssa-example-not-ssa]
%    \centering%
%    \begin{minipage}{5cm}
%      \begin{cCodeNumbered}
%int factorial(int n) {
%  init:
%    int f = 1;
%  loop:
%    if (n <= 1) goto end;
%    f = f * n;
%    n = n - 1;
%    goto loop;
%  end:
%    return f;
%}
%      \end{cCodeNumbered}
%    \end{minipage}
%  \end{subfigure}%
%  \hspace{1cm}%
%  \begin{subfigure}[Same C function in \glsentrytext{SSA} form][ssa-example-ssa]
%    \centering%
%    \begin{minipage}{5cm}
%      \lstset{mathescape,escapechar=|}
%      \begin{cCodeNumbered}
%int factorial(int |\dT{n}[1]|) {
%  init:
%    int |\dT{f}[1]| = 1;
%  loop:
%    int |\dT{f}[2]| = $\phi$(|\dT{f}[1]|, |\dT{f}[3]|);
%    int |\dT{n}[2]| = $\phi$(|\dT{n}[1]|, |\dT{n}[3]|);
%    if (|\dT{n}[2]| <= 1) goto end;
%    int |\dT{f}[3]| = |\dT{f}[2]| * |\dT{n}[2]|;
%    int |\dT{n}[3]| = |\dT{n}[2]| - 1;
%    goto loop;
%  end:
%    return |\dT{f}[2]|;
%}
%      \end{cCodeNumbered}
%    \end{minipage}
%  \end{subfigure}%
%
%  \figCaption{Example of converting a regular \glsentrytext{program} into
%    \glsentrytext{SSA} form}
%  \labelFigure{ssa-example}
%\end{figure}

This \emph{define-only-once} restriction, however, causes problems for variables
whose value can come from more than one source.
%
For example, in \refFigure{ssa-example-not-ssa} we see the factorial
\gls{function} (introduced in \refAppendix{introduction}), which is not in
\gls{SSA} form as \cVar*{n} and \cVar*{f} are defined multiple times (first at
lines~1 and~3, and then at lines~7 and~6, respectively).
%
We could try to rename the variables to avoid redefinition conflicts --
\cVar*{n} is renamed to \cVar*{n}[1] at line~1 and to \cVar*{n}[2] at line~7,
and \cVar*{f} is renamed to \cVar*{f}[1] at line~3 and to \cVar*{f}[2] at line~6
-- but which variable do we use for the \cCode*{return} at line~10? These
situations are addressed through the use of \glspl{phi-function}, which allow
variables to be defined using one of several values that originate from a
separate \gls{block}.
%
Hence, by declaring additional variables and inserting \glspl{phi-function} at
the beginning of the \cCode*{loop}~\gls{block}, the aforementioned problem of
\cVar*{f} is resolved as shown in \refFigure{ssa-example-ssa}.\!%
%
\footnote{%
  Although this in turn introduces two new problems -- where do we insert these
  \glspl{phi-function}, and how do we know which value within the
  \gls{phi-function} to choose during execution? -- we will ignore these issues
  as they are out of scope for our purposes.
  %
  There is also \gls!{gated single assignment}~\cite{BallanceEtAl:1990}, where
  the \glspl{phi-function} take a predicate as additional input, which enables
  executable interpretation of the \gls{SSA}-based \gls{IR} code.%
}

%\begin{figure}[t]
%  \centering%
%  \begin{minipage}{5cm}
%    \centering%
%    \input{\figurePath/graph-covering/ssa-example-graph}
%  \end{minipage}%
%  \hfill%
%  \begin{minipage}{6cm}
%    \figCaption[\glsentrytext{SSA graph} example]%
%      {Corresponding \glsentrytext{SSA graph} of the function from
%        \refFigure{ssa-example-ssa}.
%%
% Note that, unlike in expression trees and
%        DAGs, the data flows \emph{downwards} along the \glsentrytext{SSA
%          graph}}
%    \labelFigure{ssa-example-graph}
%  \end{minipage}
%\end{figure}

From the \gls{SSA} representation we can extract an \gls{SSA graph}, which is
basically a \gls{function graph} where \glspl{loop} are permitted (see for
example \refFigure{ssa-example-ssa}).
%
Unlike \glspl{block DAG}, the \gls{SSA graph} captures the data flow across
\glspl{block} as well as within them, modeling an entire \gls{function} which
gives a more complete view of the \gls{program}.
%
But since the \gls{SSA graph} is devoid of any control-flow information, it is
often used as a supplement alongside one or more other~\glspl{IR}.
%
Obviously this also prevents selection of \glspl{instruction} for implementing
branches and procedure calls.


\subsection{Program-Dependence Graphs}

Another popular \glsdesc{IR} is the \gls!{PDG}.
%
Introduced by \textcite{FerranteEtAl:1987} in~1987, the \gls{PDG} only models
the \emph{essential} control dependencies of the \gls{program} -- we will soon
elaborate on what this means -- and several subsequent \gls{IR}~designs, such as
the \gls!{program-dependence web} by \textcite{BallanceEtAl:1990} and the
\gls!{value-dependence graph} by \textcite{WeiseEtAl:1994}, are either direct
extensions of the \gls{PDG} or have been heavily influenced by it.

Instead of directly modeling the control flow within a \gls{function}, the
\gls{PDG} conveys information about which computational operations are dependent
on which predicates.
%
This information is given as a \gls!{CDG}, which is a \gls{subgraph} of
the~\gls{PDG}.
%
Although it is called a \emph{\gls{graph}}, the \gls{CDG} is actually shaped
like a \gls{tree}, with the operations appearing as \glspl{leaf} and the
predicates as intermediate \glspl{node}, and a special \gls{node} as the
\gls{root} denoting the \gls{function} entry point.
%
There are also so-called \glspl{region node}, which are used for eliminating
common subexpressions that appear as part of the control flow.
%
Using the~\gls{CDG}, the predicates that an operation depends on can be found by
following the \gls{path} from the \gls{leaf} \gls{node} of that operation to the
\gls{root}.
%
The \gls{PDG} is then constructed from the \gls{CDG} simply by adding the
data-flow dependencies.

%\begin{figure}[t]
%  \centering%
%
%  \begin{subfigure}[Control-flow graph][pdg-example-cfg]
%    \centering%
%    \begin{minipage}{3cm}
%      \centering%
%      \input{\figurePath/graph-covering/pdg-example-cfg}
%    \end{minipage}
%  \end{subfigure}%
%  \hfill%
%  \begin{subfigure}[Control-dependence graph][pdg-example-cdg]
%    \centering%
%    \begin{minipage}{4cm}
%      \centering%
%      \input{\figurePath/graph-covering/pdg-example-cdg}
%    \end{minipage}
%  \end{subfigure}
%  \hfill%
%  \begin{subfigure}[Program-dependence graph][pdg-example-pdg]
%    \centering%
%    \begin{minipage}{4cm}
%      \centering%
%      \input{\figurePath/graph-covering/pdg-example-pdg}
%    \end{minipage}
%  \end{subfigure}
%
%  \figCaption[Converting a function into a \glsentrytext{PDG}]
%    {Converting the function from \refFigure{ssa-example-not-ssa} into a
%      program-dependence graph.
%%
%Node numbers correspond to line numbers in the
%      function.
%%
%Control flow is shown using solid lines, and data flow as dashed
%      lines.
%%
%Shaded nodes represent region nodes}
%  \labelFigure{pdg-example}
%\end{figure}

An advantage of using a \gls{PDG} is that \gls{program} analysis tends to become
simpler.
%
In the factorial \gls{function} from \refFigure{ssa-example}, for example, the
\cCode*{end}~\gls{block} will always be executed when the
\cCode*{init}~\gls{block} is executed.
%
Hence, if there are no data dependencies between the operations within these
\glspl{block}, then they could be executed in parallel.
%
In a \gls{control-flow graph} (see \refFigure{pdg-example-cfg}) this information
is not immediately visible, but in the \gls{CDG} (see
\refFigure{pdg-example-cdg}) this fact becomes clear, as these operations only
depend on the \gls{function} entry.
%
On the other hand, generating correct \gls{assembly code} directly from the
\gls{PDG} becomes more complex, as the boundaries of \glspl{block} are obscured.


\subsubsection{Applications}

Despite its widespread use in many \gls{program} optimization routines, I have
not seen any techniques that use \glspl{PDG} for selecting \glspl{instruction},
although \textcite{PalecznyEtAl:2001} come close with their implementation of the
\gls!{JHSC}.
%
Internally, \gls{JHSC} uses a \gls{graph}-based \gls{IR} format that was
originally designed by \textcite{ClickPaleczny:1995} and is similar to
the~\gls{PDG}.
%
The \glspl{function graph} are covered by a \gls{BURS}-based \gls{instruction
  selector} (see \refSection{precomputing-cost-computations}) which selects the
least-cost \glspl{pattern} for each \gls{subtree} that is rooted either at a
value with multiple uses, or at an operation which may not be duplicated due to
side effects.
%
In other words, the \gls{function graph} is essentially converted into multiple
\glspl{block DAG} that are then covered individually.
%
This, however, may incur overlapping and thus lead to redundant code
duplication.
%
After \gls{pattern selection} the \glspl{instruction} are placed into
\glspl{block} using a heuristic \gls{global code motion} method
(see~\cite{ClickCooper:1995}).


\section{Modeling Pattern Selection as a PBQP}

In 2003, \textcite{EcksteinEtAl:2003} recognized that limiting \gls{instruction
  selection} to \gls{local.is} scope can decrease code quality of \gls{assembly
  code} generated for fixed-point arithmetic \glsdesc{DSP}s.
%
A common idiosyncrasy of such \glspl{DSP} is that their fixed-point
multiplication units will often leave the result shifted one bit to the left.
%
Hence, if a value is computed by accumulating values from fixed-point
multiplications -- as in the factorial \gls{function} given in
\refFigure{ssa-example} -- it should remain in shifted mode until all
fixed-point multiplications have been performed.
%
Otherwise the accumulated value will needlessly be shifted back and forth.
%
But this is difficult to achieve using \gls{local.is} \gls{instruction
  selection}.

To overcome this problem, \citeauthor{EcksteinEtAl:2003} developed a design that
takes \glspl{SSA graph} as input -- making this technique the first to do so --
and transforms the \gls{pattern selection} problem to a \gls!{PBQP}.
%
The \gls{PBQP} is an extension of the \gls!{QAP} -- a fundamental combinatorial
optimization problem (see~\cite{LoiolaEtAl:2007} for a recent survey) -- and was
first introduced \textcite{ScholzEckstein:2002} as a means of tackling
\gls{register allocation}.
%
\gls{QAP} and \gls{PBQP} are both NP-complete, and
\citeauthor{EcksteinEtAl:2003} therefore developed their own heuristic solver as
described in~\cite{EcksteinEtAl:2003}.
%
We will explain the \glsshort{PBQP} approach by building the model bottom-up,
starting with the definitions.

To begin with, the design assumes that the \glspl{instruction} are given as a
\glsshort{normal form.g} \gls{grammar} (see
\refAppendix{tree-covering}, \refPage{linear-form-grammar}), where each
\gls{rule} is either a \gls!{base.r}[ \gls{rule}] or a \gls!{chain.r}[
  \gls{rule}].
%
For each \gls{node}~$n$ in the \gls{SSA graph} we introduce a Boolean
vector~$\mVector{r}_n$, whose length is equal to the number of \gls{base.r}
\glspl{rule} that match~$n$, and \mbox{$\mVector{r}_n[i] = 1$} indicates that
the \gls{rule}~$i$ has been selected to cover $n$.
%
The costs for \gls{rule} selection are given as another vector~$\mVector{c}_n$
of the same length, where each element is the \gls{rule} cost times the
estimated relative execution frequency of the operation represented by $n$.
%
This is needed to give higher priority to low-cost \glspl{instruction} for
operations that reside in tight loops since those \glspl{instruction} will have
a greater impact on performance.
%
With these vectors we can define, for a given \gls{SSA graph}~$\mTuple{N, E}$, a
cost function~$f$ as
%
\begin{displaymath}
  f = \sum_{1 \leq n \leq |N|} \mVector{r}_n^T \cdot \mVector{c}_n.
\end{displaymath}
%
We call this the accumulated \gls!{base cost} as it gives the total cost of
applying \gls{base.r} \glspl{rule} to cover the \gls{SSA graph}.
%
The goal is then to cover each \gls{node} in the \gls{SSA graph} exactly once
(meaning \mbox{$\forall n \in N : \mVector{r}_n^T \cdot \mVector{1} = 1$}) such
that $f$ is minimized.

Unfortunately this does not necessarily produce a valid covering as there is no
connection \emph{between} the \gls{base.r} \glspl{rule}.
%
Consequently, the \gls{nonterminal} to which one selected \gls{base.r}
\gls{rule} is reduced may differ from what is expected by another selected
\gls{base.r} \gls{rule}.
%
This problem is addressed by introducing a cost matrix~$\mMatrix{C}_{nm}$ for
every pair of \glspl{node}~$n$ and $m$ in the \gls{SSA graph} where there exists
a directed \gls{edge} from $m$ to~$n$.
%
An element~$c_{ij}$ in $\mMatrix{C}_{nm}$ then reflects the aggregated cost of
selecting rule~$i$ for \gls{node}~$n$ and rule~$j$ for \gls{node}~$m$, which is
computed as follows:
%
\begin{enumerate}
  \item If rule~$j$ reduces $m$ to the \gls{nonterminal} expected at a certain
    position on the right-hand side in the \gls{production} of rule~$i$, then
    \mbox{$c_{ij} = 0$}.
  \item If the previous condition does not hold, but the \gls{nonterminal}
    produced by rule~$j$ can be reduced to the expected \gls{nonterminal} using
    a series of \gls{chain.r} \glspl{rule}, then \mbox{$c_{ij} = \sum c_k$},
    where $c_k$ denotes the cost of an applied \gls{chain.r} \gls{rule}~$k$.
  \item Otherwise \mbox{$c_{ij} = \infty$}, preventing this rule combination
    from being selected.
\end{enumerate}
%
The chain costs for the second condition are calculated by first computing the
\gls{transitive closure} for all \gls{chain.r} \glspl{rule} (see
\refAppendix{tree-covering}, \refPage{transitive-closure}).
%
For this, \citeauthor{EcksteinEtAl:2003} seem to have used the Floyd-Warshall
algorithm~\cite{Floyd:1962}, and \textcite{SchaeferScholz:2007} later discovered
a method that computes the lowest cost for each~$c_{ij}$ by finding the optimal
sequence of \gls{chain.r} \glspl{rule}.
%
Lastly, the costs are weighted according to the estimated execution frequency
for the \glspl{node}.

We now extend $f$ by adding the accumulated chain costs, resulting in the
following cost function:
%
\begin{displaymath}
  f =
  \sum_{1 \leq n < m \leq |N|}
    \mVector{r}_n^T \cdot \mMatrix{C}_{nm} \cdot \mVector{r}_m
  +
  \sum_{1 \leq n \leq |N|} \mVector{r}_n^T \cdot \mVector{c}_n.
\end{displaymath}
%
The model is then solved using a heuristic \glsshort{PBQP} solver, which was
also developed by \citeauthor{EcksteinEtAl:2003} (we will omit details on how it
works).

Using a prototype implementation, \citeauthor{EcksteinEtAl:2003} ran experiments on a
selected set of fixed-point \glspl{program} exhibiting the behavior discussed
earlier.
%
The results indicate that their scheme improved performance by \mbox{40--60\%}
on average -- and at most 82\% for one \gls{program} -- compared to a
traditional \gls{tree}-based \gls{instruction selector}.
%
According to \citeauthor{EcksteinEtAl:2003}, this considerable gain in performance
comes from more efficient use of value modes to which \gls{tree covering}-based
techniques must make premature assignments, and thus could have a detrimental
effect on code quality.
%
For example, if chosen poorly, the \gls{instruction selector} may need to emit
additional \glspl{instruction} in order to undo decisions regarding value modes,
which obviously reduces performance and needlessly increases the code size.
%
Although the technique by \citeauthor{EcksteinEtAl:2003} clearly mitigates these
concerns, their design also has limitations of its own.
%
Most importantly, their \glsshort{PBQP} model can only support \glspl{pattern
  tree} and consequently hinders exploitation of many common \gls{target
  machine} features, such as \gls{multi-output.ic} \glspl{instruction}.


\subsection{Extending the PBQP Approach to Pattern DAGs}

In 2008, \textcite{EbnerEtAl:2008} addressed the aforementioned concern by extending
the original \glsshort{PBQP} model by \citeauthor{EcksteinEtAl:2003} to also support
\glspl{pattern DAG}.
%
When replacing the default \gls{instruction selector} in \gls{LLVM}~2.1 -- which
is a greedy \gls{DAG}~rewriter (see \refSection{llvm}) -- the performance of the
\gls{assembly code} targeting an \gls{ARM}~processor improved by an average of
13\% for a selected set of \glspl{program}.
%
In addition, the impact on compilation time was shown to be negligible.

\citeauthor{EbnerEtAl:2008} first extended the \gls{grammar} to allow
\glspl{rule} to contain multiple \glspl{production} in a similar fashion to that
of \textcite{ScharwaechterEtAl:2007} (see
\refAppendix{dag-covering}, \refPage{rules-multiple-productions}).
%
We will refer to such \glspl{rule} as \gls!{complex.r}[ \glspl{rule}], and the
\glspl{production} within a \gls{complex.r} \gls{rule} will be called
\gls!{proxy.r}[ \glspl{rule}].
%
The \glsshort{PBQP} model is then augmented to accommodate the selection of
\gls{complex.r} \glspl{rule}, which essentially entails introducing new vectors
and matrices that decide whether a \gls{complex.r} \gls{rule} is selected,
together with constraints to enforce that all corresponding \gls{proxy.r}
\glspl{rule} are also selected.

With more than one cost matrix, it becomes necessary to be able to distinguish
one from another.
%
\newcommand{\mCategory}[1]{#1}
%
We say that all \gls{base.r} and \gls{proxy.r} \glspl{rule} belong to
category~$\mCategory{B}$, and all \gls{complex.r} \glspl{rule} belong to
category~$\mCategory{C}$.
%
A cost matrix written as $\mMatrix{C}^{\mCategory{X} \rightarrow \mCategory{Y}}$
therefore indicates that it concerns the costs of transitioning from
category~$\mCategory{X}$ to category~$\mCategory{Y}$.
%
As an example, the cost matrix used to compute the accumulated chain cost is
henceforth written as $\mMatrix{C}_{nm}^{\mCategory{B} \rightarrow
  \mCategory{B}}$ since it only concerns the \gls{base.r} \glspl{rule}.
%
We can now proceed with extending the \glsshort{PBQP} model.

First, each $\mVector{r}_n$~vector is extended with the \gls{proxy.r}
\glspl{rule} that also match at \gls{node}~$n$ in the \gls{SSA graph}.
%
If a set of identical \gls{proxy.r} \glspl{rule} is derived from multiple
\gls{complex.r} \glspl{rule}, the length of the vector only increases by one
element for each such set.
%
Second, we create an instance of a \gls{complex.r} \gls{rule} for every
permutation of distinct \glspl{node} where the matched \gls{proxy.r}
\glspl{rule} can be combined into a \gls{complex.r} \gls{rule}.
%
Each such instance~$i$ in turn gives rise to a \mbox{two-element} decision
vector~$\mVector{d}_i$ which indicates whether $i$ is selected or not (hence a~1
in the first element indicates \emph{not selected}, and a~1 in the second
element indicates \emph{selected}).\!%
%
\footnote{%
  A \mbox{two-element} vector is chosen instead of a single Boolean variable as
  the \glsshort{PBQP} model must consist of only matrices and vectors, and for
  all vectors the sum of its elements must always be exactly~1.%
}
%
Then, as with the \gls{base.r} \glspl{rule}, we accumulate the costs of the
selected \gls{complex.r} \glspl{rule} by extending $f$ with
%
\begin{displaymath}
  \sum_{1 \leq i \leq |I|} \mVector{d}_i^T \cdot
  \mVector{c}_i^{\mCategory{C}},
\end{displaymath}
%
where $I$ is the set of \gls{complex.r} \gls{rule} instances, and
$\mVector{c}_i^{\mCategory{C}}$ is a \mbox{two-element} cost vector whose
elements consist of the value~0 and the cost of the \gls{complex.r} \gls{rule}.

Selecting a \gls{complex.r} \gls{rule} means that all of its \gls{proxy.r}
\glspl{rule} must also be selected.
%
We enforce this through a cost matrix~$\mMatrix{C}_{ni}^{\mCategory{B}
  \rightarrow \mCategory{C}}$, where $n$ is a particular \gls{node} in the
\gls{SSA graph} and $i$ is a particular instance of a \gls{complex.r}
\gls{rule}.
%
An element~$c_{mj}$ in $\mMatrix{C}_{ni}^{\mCategory{B} \rightarrow
  \mCategory{C}}$ is then set as follows:
%
\begin{itemize}
  \item If $j$ represents that $i$ is not selected, then \mbox{$c_{mj} = 0$}.
  \item If $m$ is a \gls{base.r} \gls{rule} or \gls{proxy.r} \gls{rule} not
    associated with the \gls{complex.r} \gls{rule} of~$i$, then \mbox{$c_{mj} =
      0$}.
  \item Otherwise \mbox{$c_{mj} = \infty$}.
\end{itemize}
%
We then force the selection of necessary \gls{proxy.r} \glspl{rule} by appending
the following to~$f$:
%
\begin{displaymath}
  \sum_{\substack{1 \leq n \leq |N|\\1 \leq i \leq |I|}}
    \mVector{r}_n^T \cdot
    \mMatrix{C}_{ni}^{\mCategory{B} \rightarrow \mCategory{C}} \cdot \mVector{d}_i.
\end{displaymath}

An issue with this model is that if the cost of all \gls{proxy.r} \glspl{rule}
is~0, then solutions are allowed where all \gls{proxy.r} \glspl{rule} of a
\gls{complex.r} \gls{rule} are selected but the \gls{complex.r} \gls{rule}
itself is not selected.
%
\citeauthor{EbnerEtAl:2008} solved this problem by first setting a high cost~$M$ to
all \gls{proxy.r} \glspl{rule} and then setting the cost of all \gls{complex.r}
\glspl{rule} to \mbox{$\mCost{i} - |l_i|M$}, where $l_i$ is the set of
\gls{proxy.r} \glspl{rule} of a \gls{complex.r} \gls{rule}~$i$.
%
Hence, the overhead of selecting the \gls{proxy.r} \glspl{rule} is only offset
if the \gls{complex.r} \gls{rule} is also selected.

In some cases, selecting certain \gls{complex.r} \glspl{rule} can incur cyclic
data dependencies.
%
To prevent this, we introduce a cost matrix~$\mMatrix{C}_{ij}^{\mCategory{C}
  \rightarrow \mCategory{C}}$ that prevents two instances~$i$ and $j$ from being
selected simultaneously if such a combination incurs a cyclic data dependency.
%
In their model, \citeauthor{EbnerEtAl:2008} also forbade selection of \gls{complex.r}
\gls{rule} instances that overlap.
%
As before, these restrictions are enforced by setting the elements in
$\mMatrix{C}_{ij}^{\mCategory{C} \rightarrow \mCategory{C}}$ corresponding to
such situations to~$\infty$, and to 0 for all other elements.

Hence the complete definition of $f$ in the \glsshort{PBQP} model by
\citeauthor{EbnerEtAl:2008} becomes
%
\todo{fix}
%\begin{gather*}
%  f =
%  \sum_{1 \leq i < j \leq |I|} \mVector{d}_i^T \cdot
%    \mMatrix{C}_{ij}^{\mCategory{C} \rightarrow \mCategory{C}} \cdot \mVector{d}_j
%  +
%  \sum_{\substack{1 \leq n \leq |N|\\1 \leq i \leq |I|}}
%    \mVector{r}_n^T \cdot
%    \mMatrix{C}_{ni}^{\mCategory{B} \rightarrow \mCategory{C}} \cdot \mVector{d}_i
%  +
%  \sum_{1 \leq i \leq \mNumberOf{I}} \mVector{d}_i^T \cdot
%    \mVector{c}_i^{\mCategory{C}}\\
%  +
%  \sum_{1 \leq n < m \leq \mNumberOf{N}} \mVector{r}_n^T \cdot
%     \mMatrix{C}_{nm}^{\mCategory{B} \rightarrow \mCategory{B}} \cdot \mVector{r}_m
%  +
%  \sum_{1 \leq n \leq \mNumberOf{N}} \mVector{r}_n^T \cdot
%     \mVector{c}_n^{\mCategory{B}}.
%\end{gather*}


\subsection{Using Rewrite Rules Instead of Production Rules}

In 2010, \textcite{BuchwaldZwinkau:2010} introduced another technique based on
\glspl{PBQP}.
%
But unlike \citeauthor{EcksteinEtAl:2003} and \citeauthor{EbnerEtAl:2008},
\citeauthor{BuchwaldZwinkau:2010} approached the task of \gls{instruction selection} as
a formal \gls{graph} transformation problem, for which much previous work
already exist.
%
Hence, in \citeauthor{BuchwaldZwinkau:2010}'s design the \glspl{instruction} are
expressed as \gls!{rewrite.r}[ \glspl{rule}] instead of \glspl{production rule}.
%
As these \gls{rewrite.r} \glspl{rule} are based on a formal foundation, the
resulting \gls{instruction selector} can be automatically verified to handle all
possible \glspl{program}.
%
If this check fails, the verification tool can also provide the necessary
\gls{rewrite.r} \glspl{rule} that are currently missing from the
\gls{instruction set}.

The technique works as follows.
%
First the \gls{SSA graph} is converted into a \gls{DAG}-like form by duplicating
each \mbox{$\phi$-\gls{node}} into two \glspl{node}, which effectively breaks
any \glspl{cycle} appearing in the \gls{SSA graph}.
%
After finding all applicable \gls{rewrite.r} \glspl{rule} for this \gls{DAG}
(this is done using traditional \gls{pattern matching}), a corresponding
instance of the \gls{PBQP} is formulated and solved as before.

\citeauthor{BuchwaldZwinkau:2010} also discovered and addressed flaws in the
\glsshort{PBQP} solver by \citeauthor{EcksteinEtAl:2003}, which may fail to find a
solution in certain situations due to inadequate propagation of information.
%
However, \citeauthor{BuchwaldZwinkau:2010} also cautioned that their own implementation
does not scale well when the number of overlapping \glspl{pattern} grows.
%
In addition, since the \gls{SSA graph} is devoid of control-flow information,
none of the \glsshort{PBQP}-based techniques can support \gls{inter-block.ic}
\glspl{instruction}.


\section{Other Graph-Based Approaches}

\subsection{More Hardware Modeling Techniques}

In \refAppendix{dag-covering} we saw a technique for performing \gls{microcode
  generation} where the entire processor of the \gls{target machine} is modeled
as a \gls{graph} instead of by just deriving the \glspl{pattern} for the
available \glspl{instruction}.
%
Here we will look at a few techniques that rely on the same modeling scheme, but
address the more traditional problem of \gls{instruction selection}.


\subsubsection{\glsentrytext{Chess}}

\textcite{LanneerEtAl:1990} developed in 1990 a design that was later adopted by
\citeauthor{VanPraetEtAl:1994}~\cite{VanPraetEtAl:1994, VanPraetEtAl:2001} in their
implementation of \gls{Chess}, a well-known compiler targeting \glspl{DSP} and
\glspl{ASIP}.

Comparing \gls{Chess} to \gls{MSSQ} (see
\refSection{modeling-entire-target-machines}), we find two striking differences.
%
First, in \gls{MSSQ} the data paths of the processor are given by a manually
written \gls{machine description}, whereas \gls{Chess} derives these
automatically from an \mbox{\gls{nML}-based} specification~\cite{FauthEtAl:1993,
  FauthEtAl:1995}.

Second, the method of \gls{bundling} -- which is the task of scheduling
operations for parallel execution -- is different.
%
The \gls{instruction selector} in \gls{MSSQ} uses techniques from \gls{DAG
  covering} to find \glspl{pattern} in the hardware \gls{graph}, which can
subsequently be used to cover the \glspl{expression tree}.
%
After \gls{pattern selection}, another routine attempts to schedule the selected
\glspl{instruction} for parallel execution.
%
In contrast, \gls{Chess} takes a more incremental approach.
%
From the \gls{program} \gls{Chess} first constructs a \gls{chaining graph},
where each \gls{node} represents an operation in the \gls{program} that has been
annotated with a set of functional units capable of executing that operation.
%
Since the functional units on a \gls{DSP} are commonly grouped into
\glspl!{FBB}, the \gls{chaining graph} also contains an \gls{edge} between every
pair of \glspl{node} that could potentially be executed on the same \gls{FBB}.
%
A heuristic algorithm then attempts to collapse the \glspl{node} in the
\gls{chaining graph} by selecting an \gls{edge}, replacing the two \glspl{node}
with a new \gls{node}, and then removing the \glspl{edge} between \glspl{node}
of operations that can no longer be executed on the same \gls{FBB} as the
operations of the new \gls{node}.
%
This process iterates until no more \glspl{node} can be collapsed, and every
remaining \gls{node} in the \gls{chaining graph} thus constitutes a
\gls!{bundle}.
%
The same authors later extended this design in~\cite{VanPraetEtAl:2001} to consider
selection between \emph{all} possible \glspl{bundle} using \glshyphened{branch
  and bound} \gls{search}, and to enable some measure of code duplication by
allowing the same operations in the \gls{function graph} to appear in multiple
\glspl{bundle}.

Using this incremental scheme to form the \glspl{bundle}, the design by
\citeauthor{VanPraetEtAl:1994} is capable of \gls{bundling} operations that
potentially reside in different \glspl{block}.
%
Their somewhat-\gls{integrated.cg} \gls{code generation} approach also allows
efficient \gls{assembly code} to be generated for complex architectures, making
it suitable for \glspl{DSP} and \glspl{ASIP} where the data paths are very
irregular.
%
It may be possible to also extend the technique to support \gls{inter-block.ic}
\glspl{instruction} as well, but \gls{interdependent.ic} \glspl{instruction} are
most likely out of reach due to its heuristic nature.


\subsubsection{Generating Assembly Code Using Simulated Annealing}

Another, albeit unusual, \gls{code generation} technique was proposed by
\textcite{Visser:1999} in~1999.
%
Like \gls{MSSQ} and \gls{Chess}, \citeauthor{Visser:1999}'s approach is an
\gls{integrated.cg} \gls{code generation} design but solves the problem using
\gls!{simulated annealing}, which is a meta-heuristic to avoid getting stuck in
a local maximum when searching for solutions (see for example
\cite{KirkpatrickEtAl:1983} for an overview).
%
In brief terms, an initial solution is found by randomly mapping each \gls{node}
in the \gls{function graph} to a \gls{node} in the hardware \gls{graph} -- which
models the entire processor -- and then a schedule is found using traditional
\gls{list scheduling}.
%
A \gls{fitness function} is then applied to judge the effectiveness of the
solution, but the exact details are omitted from the paper.
%
A proof-of-concept prototype was developed and tested on a simple \gls{program},
but it appears no further research has been conducted on this idea.


\subsection{Improving Code Quality with Mutation Scheduling}

The last item we will discuss is a technique called \gls!{mutation
  scheduling},\!%
%
\footnote{%
  Despite its name, the idea of \gls{mutation scheduling} is completely
  orthogonal to the theory of \glsdesc{GA}s.%
}
%
which was introduced in 1994 by
\citeauthor{NovackEtAl:2002}~\cite{NovackNicolau:1994, NovackEtAl:2002}.
%
\Gls{mutation scheduling} is technically a form of \gls{instruction scheduling}
that primarily targets \gls{VLIW} architectures, but it also integrates a
sufficient amount of \gls{instruction selection} to warrant being included in
this dissertation.
%
On the other hand, the amount of \gls{instruction selection} that \emph{is}
incorporated is in turn not really based on \gls{graph covering}, but, as with
\glspl{trellis diagram} (see \refAppendix{tree-covering} on
\refPageOfSection{tc-trellis-diagrams-trees} and \refAppendix{dag-covering} on
\refPageOfSection{dc-trellis-diagrams-dags}), I decided against discussing it in
its own appendix.

Broadly speaking, \gls{mutation scheduling} essentially tries to reduce the
makespan of \glspl{program} for which \gls{assembly code} have already been
generated (hence \gls{instruction selection}, \gls{instruction scheduling}, and
\gls{register allocation} has already been performed).\!%
%
\footnote{%
  Although it is depicted here primarily as a post-step to \gls{code
    generation}, one could just as well design a \glsshort{Davidson-Fraser
    approach}-style \gls{compiler} (see \refAppendix{macro-expansion}) where
  simple methods are applied to generate correct but naive \gls{assembly code},
  and then rely on \gls{mutation scheduling} to improve the code quality.%
}
%
This is done by progressively moving the computations, one at a time, such that
they can be executed in parallel with other \glspl{instruction} and thus finish
sooner.
%
If such a move cannot be made, for example due to violation of some resource
constraint or data dependency, then \gls{mutation scheduling} tries to alter the
value -- this is called \gls!{value mutation} -- which means that the current
operation is replaced by other, equivalent operations that conform to the
restrictions.
%
These operations are selected from a \gls!{mutation set}, which is conceptually
a recursive data structure, as an expression in the \gls{mutation set} may use
intermediate values that in turn necessitate \glspl{mutation set} of their own.
%
\citeauthor{NovackEtAl:2002} compute these \glspl{mutation set} by first taking the
original operation and then applying a series of semantic-preserving functions
that have been derived from various logical axioms, algebraic theorems, and the
characteristics of the \gls{target machine}.
%
For example, if the value~$X$ is computed as \mbox{$Y + 5$}, then $Y$ can later
be obtained by computing \mbox{$X - 5$}.
%
Another example is multiplication by powers of~\num{2}, which can be replaced
with \cCode*{shift} \glspl{instruction}, provided such \glspl{instruction} are
available.
%
If this is beneficial, a value can also be recomputed instead of copied from its
current location.
%
This idea is known as \gls!{recomputation} or \gls!{rematerialization}, which is
a method for reducing \gls{register pressure}, as it allows \glspl{register} to
be released at an earlier point in the \gls{assembly code}.

In \gls{mutation scheduling}, ``shorter'' mutations are preferred over longer
ones.
%
This is because a \gls{value mutation} of $v$ can lead to a cascade of new
computations, which all will need to be scheduled before $v$ can be is computed.
%
Note that these computations can be scheduled such that they appear in
\glspl{block} preceding the \gls{block} in which the computations of $v$ appear.
%
Hence the length of a mutation is loosely defined as the number of
\gls{instruction} \glspl{bundle} that may need to be modified in order to
realize the mutation.
%
Moreover, since the new computations of a successful mutation consume resources
and incur dependencies of their own, the existing candidates appearing in
\glspl{mutation set} may need to be removed or modified.
%
The ``best'' combination of mutations is then decided heuristically, but the
paper is vague on how this is done exactly.

\citeauthor{NovackEtAl:2002} implemented a prototype by extending an existing
scheduler based on \gls!{GRiP}, which is another global \gls{instruction
  scheduling} technique developed by the same authors
(see~\cite{NicolauNovack:1992}).
%
Subsequent experiments using a selected set of benchmark \glspl{program}
demonstrated that the \gls{mutation scheduling}-based design yielded a two- to
threefold performance improvement over the \gls{GRiP}-only-based counterpart,
partly due to its ability to apply \gls{rematerialization} in regions where
\gls{register pressure} is high.
%
Unfortunately the authors neglected to say anything about the time complexity of
\gls{mutation scheduling}, and whether it scales to larger \glspl{program}.


\section{Summary}

In this appendix we have considered a number of techniques that are founded, in
one form or another, on the \gls{principle} of \gls{graph covering}.
%
Such techniques are among the most powerful methods of \gls{instruction
  selection} since they perform \gls{global.is} \gls{instruction selection} as
well as have more extensive \gls{instruction} support compared to most
\glsshort{tree covering} and \gls{DAG covering}-based designs.

Unfortunately this has not been fully exploited in existing techniques, partly
due to limitations in the \gls{program} representations or to restrictions
enforced by the underlying solving techniques.
%
Moreover, performing \gls{global.is} \gls{instruction selection} is
computationally much harder compared to \gls{local.is} \gls{instruction
  selection}, and therefore most likely we will only see these techniques
applied in \glspl{compiler} whose users can afford very long compilation times
(for example when targeting embedded systems with extremely high demands on
performance, code size, power consumption, or a combination thereof).
