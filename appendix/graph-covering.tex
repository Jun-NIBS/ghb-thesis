% Copyright (c) 2018, Gabriel Hjort Blindell <ghb@kth.se>
%
% This work is licensed under a Creative Commons 4.0 International License (see
% LICENSE file or visit <http://creativecommons.org/licenses/by/4.0/> for a copy
% of the license).

\chapter{Graph Covering}
\labelAppendix{graph-covering}

This appendix considers techniques based on \gls{graph covering}.
%
First, we introduce the principle in \refSection{gc-principle} and describe
representations that enable this principle in \refSection{gc-representations}.
%
In \refSection{gc-extending-tree-covering-techniques-to-dags}, we describe
techniques that extend methods from \gls{tree covering} to \glspl{graph}.
%
\glsunset{PBQP}
%
In \refSection{gc-pbqp-based-approaches}, we describe techniques that model
\gls{instruction selection} as a \gls{PBQP}.
%
\glsreset{PBQP}
%
Other \gls{graph}-based approaches, that do not fit into any of the sections
above, are discussed in \refSection{gc-other-approaches}.
%
Lastly, we summarize in \refSection{gc-summary}.

The appendix is based on material presented in
\cite[Chap.\thinspace5]{HjortBlindell:2016:Survey} that has been adapted for
this dissertation.
%
To not disturb the flow of reading, material already presented in
\refChapter{existing-isel-techniques-and-reps} is duplicated in this appendix.


\section{The Principle}
\labelSection{gc-principle}

In \gls{DAG covering}-based \gls{instruction selection}, \glspl{function} can
only be modeled one \gls{block} at a time as \glspl{cycle} are forbidden to
appear in the \glspl{block DAG}.
%
Lifting this restriction results in a \glspl!{function graph}, thus capturing
the data flow for an entire \gls{function} as a single \gls{graph}.
%
Depending on the representation, some \glspl{function graph} also capture the
control flow for the \gls{function}.

Selecting \glspl{instruction} for such \glspl{graph} is known as
\gls!{global.is}[ \gls{instruction selection}] and has several advantages over
\gls{local.is} \gls{instruction selector}.
%
First, with an entire \gls{function} as input, a \gls{global.is}
\gls{instruction selector} can account for the effects of \gls{local.is}
\gls{pattern selection} across the \gls{block} boundaries and is thereby better
informed when making its decisions.
%
In addition, it can move operations from one \gls{block} to another if that
enables better use of the \gls{instruction set} (this is known as \gls{global
  code motion}).
%
Second, to support \gls{inter-block.ic} \glspl{instruction} -- which require
modeling of both data and control-flow information -- it is imperative that the
\glspl{pattern} be expressible using \glspl{graph} that may contain
\glspl{cycle}.
%
This makes \gls{graph covering} one of the key approaches for making use of
fewer but more efficient \glspl{instruction}, which is becoming more and more
crucial for modern \glspl{target machine} -- especially embedded systems --
where both power consumption and heat emission are becoming increasingly
important factors.

However, when transitioning from \glspl{pattern DAG} to \glspl{pattern graph},
we can no longer apply \gls{pattern matching} techniques designed for
\glspl{tree} and \glspl{DAG} but must resort to methods from the field of
\gls{subgraph isomorphism} in solving this problem (see
\refTable{time-complexities} for time complexity comparison).
%
\begin{table}
  \centering%
  \figureFont\figureFontSize%
  \begin{tabular}{ccc}%
    \toprule
        \tabhead representation
      & \tabhead pattern matching
      & \tabhead optimal pattern selection\\
    \midrule
        trees  & linear      & linear     \\
        DAGs   & NP-complete & NP-complete\\
        graphs & NP-complete & NP-complete\\
    \bottomrule
  \end{tabular}

  \caption[%
            Time complexities for solving pattern matching and optimal pattern
            selection%
          ]%
          {%
            Time complexities for solving the pattern matching and optimal
            pattern selection problems using various representations%
          }
  \labelTable{time-complexities}
\end{table}
%
The \gls{pattern selection} problem, on the other hand, can still be solved
using many of the techniques which were discussed in \refAppendix{dag-covering}.
%
Therefore, in this appendix we will only examine techniques that were originally
designated for \gls{graph covering}.


\section{Sea-of-nodes IRs}
\labelSection{gc-representations}

With \glsshort{tree covering} and \gls{DAG covering}, it is sufficient to
represent the \gls{function} on \gls{block} level.
%
Consequently, \glspl{function} are typically modeled as a \gls{forest} of
\glspl{expression tree} or a set of \glspl{block DAG}.
%
But, as previously stated, this becomes an impediment when applied in \gls{graph
  covering}-based techniques, forcing us to instead use a \gls{graph}-based
\glsdesc{IR}.
%
We will therefore continue by looking briefly at how \glspl{function} can be
expressed using such representations, which are colloquially referred to as
\glspl!{sea-of-nodes IR}.

In the context of \gls{instruction selection}, there are two \glspl{sea-of-nodes
  IR} that are of interest.
%
The first captures the data flow for entire \glspl{function}, and the second is
an extension of the first in order to also capture control flow.
%
For a comprehensive survey on \gls{function} representations,
see~\cite{StanierWatson:2013}.


\paragraph{Capturing Data Flow of Entire Functions}

In order to simply many \gls{compiler} tasks, \textcite{CytronEtAl:1991}
introduced a \gls{function} representation called \gls!{SSA}[ form].

A \gls{program} is said to be in \gls{SSA}~form if every \gls{variable} is
defined exactly once.
%
One of the main benefits of this is that the \gls{live range} of each variable
is contiguous.
%
The \gls{live range} of a variable can be loosely described as the length within
the \gls{program} where the value of that variable must not be destroyed.
%
This in turn means that each variable corresponds to a single value, which
simplifies many \gls{program} optimization routines.

For example, the \gls{function} shown in \refFigure{sea-of-nodes-example-2-c} is
not in \gls{SSA}~form as \glspl{variable}~\irCode*{f} and~\irCode*{n} are
redefined within the loop.
%
\begin{filecontents*}{fact.c}
int factorial(int n) {
  entry:
    int f = 1;
  head:
    if (n <= 1) goto end;
  body:
    f = f * n;
    n = n - 1;
    goto head;
  end:
    return f;
}
\end{filecontents*}
%
\begin{filecontents*}{fact-ssa.c}
int factorial(int $\irVar{n}[1]$) {
  entry:
    int $\irVar{f}[1]$ = 1;
  head:
    int $\irVar{f}[2]$ = $\mPhi$($\irVar{f}[1]$:entry, $\irVar{f}[3]$:body);
    int $\irVar{n}[2]$ = $\mPhi$($\irVar{n}[1]$:entry, $\irVar{n}[3]$:body);
    if ($\irVar{n}[2]$ <= 1) goto end;
  body:
    int $\irVar{f}[3]$ = $\irVar{f}[2]$ * $\irVar{n}[2]$;
    int $\irVar{n}[3]$ = $\irVar{n}[2]$ - 1;
    goto head;
  end:
    return $\irVar{f}[2]$;
}
\end{filecontents*}
%
\begin{figure}
  \centering%
  \mbox{}%
  \hfill%
  \subcaptionbox{%
                  C implementation of factorial%
                  \labelFigure{sea-of-nodes-example-2-c}%
                }%
                [50mm]%
                {%
                  \begin{lstpage}{45mm}%
                    \lstinputlisting[language=c]{fact.c}%
                  \end{lstpage}%
                }%
  \hfill\hfill%
  \subcaptionbox{Code in SSA form\labelFigure{sea-of-nodes-example-2-ssa-c}}%
                {%
                  \begin{lstpage}{59mm}%
                    \lstinputlisting[language=c,mathescape]{fact-ssa.c}%
                  \end{lstpage}%
                }%
  \hfill%
  \mbox{}

  \vspace*{\betweensubfigures}

  \subcaptionbox{SSA graph\labelFigure{sea-of-nodes-example-2-ssa-graph}}%
                {%
                  ../existing-isel-techniques-and-reps/sea-of-nodes-example-ssa-graph.tex%
                }

  \caption{Example of an SSA graph}
  \labelFigure{ssa-example-2}
\end{figure}
%
By introducing new \glspl{variable} and connecting these using
\glspl!{phi-function} where the value depends on control flow, the
\gls{function} can be rewritten into \gls{SSA}~form, as shown in
\refFigure{sea-of-nodes-example-2-ssa-c}.

From an \gls{SSA}-based \gls{function}, we can construct a \gls{data-flow graph}
called the \gls!{SSA graph}~\cite{GerlekEtAl:1995}.
%
Like in \glspl{data-flow graph}, each \gls{operation} in the \gls{function}
(including the \glspl{phi-function}) is represented as a \gls{node}.
%
These nodes are connected using \glspl{data-flow edge}, ignoring the fact that
the \glspl{operation} may belong to different \glspl{block}.
%
For the example above, this results in the \gls{SSA graph} shown in
\refFigure{sea-of-nodes-example-2-ssa-graph}, thus giving a complete view of the
data flow in the \gls{function}.

But since the \gls{SSA graph} is devoid of any control-flow information, it is
often used as a supplement alongside one or more other~\glspl{IR}.
%
Obviously this also prevents selection of \glspl{instruction} for implementing
branches and procedure calls.


\paragraph{Capturing Both Data And Control Flow}

\textcite{ClickPaleczny:1995} introduced a \gls{sea-of-nodes IR} that captures
both data and control flow.
%
The data flow is modeled exactly as in the \gls{SSA graph}, and the control flow
is captured using \glspl{node} to represent the \glspl{block} in the
\gls{function} and \glspl{edge} to represent jumps between \glspl{block}.
%
To capture dependencies between the data and control flow -- for example, when
the target of a jump depends on a Boolean value -- such jumps flow through
special \emph{if}~\glspl{node}.
%
An example of the \gls{Click-Paleczny graph} is shown in
\refFigure{click-paleczny-graph-example-2}.

\begin{figure}
  \centering%
  ../existing-isel-techniques-and-reps/sea-of-nodes-example-click-paleczny-graph.tex

  \caption[Example of a Click-Paleczny graph]%
          {%
            Example of a Click-Paleczny graph, corresponding to the program
            shown in \refFigure{ssa-example-2}.
            %
            Thin-lined nodes and edges denote data operations and data flow.
            %
            Thick-lined nodes and edges denote control operations and control
            flow.
            %
            Dashed edges indicate to which block an operation belongs%
          }
  \labelFigure{click-paleczny-graph-example-2}
\end{figure}


\section{Extending Tree Covering Techniques to Graphs}
\labelSection{gc-extending-tree-covering-techniques-to-dags}

\textcite{PalecznyEtAl:2001} introduced an approach for performing
\gls{instruction selection} based on the \gls{Click-Paleczny graph}.

The approach first divides the \gls{function graph} into a set of possibly
overlapping \glspl{expression tree}.
%
This is done by labeling certain \glspl{node} in the \gls{function graph} as
tree roots.
%
Root candidates are \glspl{node} representing \glspl{operation} whose result are
shared or \glspl{operation} with side effects and may therefore not be
\glsshort{node duplication}[ed].
%
The selection of roots is geared towards duplicating address computations and
other expressions that can be subsumed into a single \gls{instruction}.
%
Once labeled, each \gls{expression tree} is covered using a variant of
\refAlgorithm{burs-labeling-algorithm} (see \refAppendix{tree-covering} on
\refPageOfAlgorithm{burs-labeling-algorithm}).
%
The \glspl{instruction} are then emitted and placed in \glspl{block} using a
\gls{global code motion} method described in~\cite{Click:1995}.

Although the \gls{function} is represented as a \gls{function graph}, the
\glspl{instruction} must still be modeled as \glspl{pattern tree}.
%
Consequently, only \gls{single-output.ic} \glspl{instruction} can be selected
using this approach.


\section{Modeling Instruction Selection as a PBQP}
\labelSection{gc-pbqp-based-approaches}

In 2003, \textcite{EcksteinEtAl:2003} recognized that limiting \gls{instruction
  selection} to \gls{local.is} scope can decrease code quality of \gls{assembly
  code} generated for fixed-point arithmetic \glspl{DSP}.
%
For more details, see \refAppendix{dag-covering} on
\refPageOfSection{dc-limitations}.

To overcome this problem, \citeauthor{EcksteinEtAl:2003} developed a design that
takes \glspl{SSA graph} as input -- making this technique the first to do so --
and transforms the \gls{pattern selection} problem into a \gls!{PBQP}.
%
First introduced by \textcite{ScholzEckstein:2002} to model and solve
\gls{register allocation}, \gls{PBQP} is a variant of the \gls!{QAP}, which is a
fundamental combinatorial optimization problem in the field of operations
research (see \cite{LoiolaEtAl:2007} for a survey).
%
Although both problems are NP-complete in general, a subclass of \gls{PBQP} can
be solved in linear time which inspired \citeauthor{ScholzEckstein:2002} in
developing a greedy, linear-time solver.

A formal definition of \gls{PBQP} is given in
\refChapter{existing-isel-techniques-and-reps} on \refPageOfDefinition{pbqp},
which can intuitively be explained as follows.
%
Assume a problem consists of $n$~decisions, each with $k$~choices.
%
Then $\mVector{\mVar{x}}_i$ is a \gls{decision variable} with $k$~elements,
where \mbox{$\mVector{\mVar{x}}_i[j] = 1$} means that choice~$j$ has been
selected for decision~$i$.
%
For each \gls{variable}, the condition
\raisebox{0pt}[\height-2pt]{$\mVector{1}^{\transp} \mVector{\mVar{x}}_i = 1$}
ensures that exactly one choice is selected.
%
The cost of selecting a particular choice for decision~$i$ is represented
through a cost vector~$\mVector{c}_i$, and the cost of combining two
decisions~$i$ and~$j$ are represented through a cost matrix~$\mMatrix{C}_{ij}$.

In this context, $\mVector{\mVar{x}}_i$ decides whether to select a particular
\gls{match} to cover \gls{node}~$i$, $\mVector{c}_i$ contains the cost for each
such \gls{match}, and $\mMatrix{C}_{ij}$ contains the cost of additional
\glspl{instruction} that may need to be selected due to certain combinations of
\glspl{match}.
%
For example, assume two nodes~$i$ and~$j$ where $j$ depends on $i$.
%
Assume further that the \glspl{instruction} are represented as a
\glshyphened{normal form.g} \gls{machine grammar}, and that~$i$ and~$j$ can be
covered using two \glspl{rule}, $r_i$ and $r_j$, with \glspl{production}
\mbox{$\mNT{A} \rightarrow \irCode*{op}_{\mathit{i}} \; \mNT{A} \; \mNT{A}$} and
\mbox{$\mNT{B} \rightarrow \irCode*{op}_{\mathit{j}} \; \mNT{B} \; \mNT{B}$},
respectively.
%
Since the result of $r_i$ does not match the operands of $r_j$, this \gls{rule}
combination requires a \gls{chain.r} \gls{rule} -- or a chain of these, if
necessary -- that derives $\mNT{B}$ from $\mNT{A}$.
%
The $\mMatrix{C}_{ij}$ costs are calculated by computing the \gls{transitive
  closure} for all \gls{chain.r} \glspl{rule}.
%
For this \citeauthor{EcksteinEtAl:2003} seem to have used the Floyd-Warshall
algorithm~\cite{Floyd:1962}, and \textcite{SchaeferScholz:2007} later discovered
a method that finds the optimal sequence of \gls{chain.r} \glspl{rule}.
%
Illegal combinations are prevented by assigning infinite cost.
%
An example of a \gls{PBQP} instance is shown in \refFigure{pbqp-example-2}.

\begin{figure}
  \centering%
  \mbox{}%
  \hfill%
  \subcaptionbox%
    {%
      Rules.
      %
      For brevity, the actions are not included%
    }%
    [68mm]%
    {%
      \figureFontSize%
      \begin{tabular}{r@{ $\rightarrow$ }lc}
        \toprule
        \multicolumn{2}{c}{\tabhead rules} & \tabhead cost\\
        \midrule
        $\mNT{Reg}$ & \irCode{var} & 0\\
        $\mNT{Reg}$ & $\irCode{\irAddText} \; \mNT{Reg} \; \mNT{Reg}$ & 1\\
        $\mNT{Reg}$ & $\irCode{\irLoadText} \; \mNT{Addr}$ & 3\\
        $\mNT{Reg}$ & $\irCode{\irLoadText} \; \irAddText \; \mNT{Reg} \;
                      \mNT{Reg}$
                    & 5\\
        $\mNT{Addr}$ & $\mNT{Reg}$ & 2\\
        \bottomrule
      \end{tabular}%
    }%
  \hfill%
  \subcaptionbox{%
                  SSA graph%
                  \labelFigure{pbqp-example-2-ssa-graph}%
                }{%
                  \input{%
                    figures/existing-isel-techniques-and-reps/%
                    pbqp-example-ssa-graph%
                  }%
                }%
  \hfill%
  \mbox{}

  \vspace{\betweensubfigures}

  \subcaptionbox{%
                  PBQP instance.
                  %
                  The rows and columns in the cost vectors and matrices are
                  labeled with the matches they represent.
                  %
                  Cost matrices for uninteresting combinations are assumed to
                  consist of~0s%
                  \labelFigure{pbqp-example-2-instance}%
                }{%
                  \figureFontSize%
                  \apptocmd{\irFont}{\scriptsize}{}{}%
                  \newcommand{\lskip}{0.1ex}%
                  \newcommand{\newlineskip}{-1em}%
                  \newcolumntype{R}{r@{\;=\;}}%
                  \BAnewcolumntype{C}{>{\;}c<{\:}}%
                  \BAnewcolumntype{X}{@{\;\;\,}c}%
                  \newcommand{\mlabel}[1]{\raisebox{.2ex}{$\scriptstyle #1$}}%
                  \begin{minipage}{22mm}%
                    \begin{displaymath}
                      \begin{array}{r@{\;\in\;}l}
                          \mVector{\mVar{x}}_{\irVar{a}} & \mSet{0, 1} \\[.7ex]
                          \mVector{\mVar{x}}_{\irVar{b}} & \mSet{0, 1} \\[.7ex]
                          \mVector{\mVar{x}}_{\irCode{\irAddText}}
                        & \mSet{0, 1}^{2} \\[.7ex]
                          \mVector{\mVar{x}}_{\irCode{\irLoadText}}
                        & \mSet{0, 1}^{2}
                      \end{array}
                    \end{displaymath}%
                  \end{minipage}%
                  \hspace{5mm}%
                  \trimbox{0 8pt 0 0}{%
                    \begin{minipage}{27mm}%
                      \begin{displaymath}
                        \begin{array}{Rl}
                          \mVector{c}_{\irVar{a}}
                            & \begin{adjblockarray}{cc}{1.3ex}
                                \begin{block}{[C]X}
                                  0 & \mlabel{m_1} \\
                                \end{block}
                              \end{adjblockarray} \\[1.5ex]
                          \mVector{c}_{\irVar{b}}
                            & \begin{adjblockarray}{cc}{1.3ex}
                                \begin{block}{[C]X}
                                  0 & \mlabel{m_2} \\
                                \end{block}
                              \end{adjblockarray} \\[1.5ex]
                          \mVector{c}_{\irCode{\irAddText}}
                            & \begin{adjblockarray}{cc}{1.3ex}
                                \begin{block}{[C]X}
                                  1 & \mlabel{m_3} \\
                                  5 & \mlabel{m_5} \\
                                \end{block}
                              \end{adjblockarray} \\[3ex]
                          \mVector{c}_{\irCode{\irLoadText}}
                            & \begin{adjblockarray}{cc}{1.3ex}
                                \begin{block}{[C]X}
                                  3 & \mlabel{m_4} \\
                                  5 & \mlabel{m_5} \\
                                \end{block}
                              \end{adjblockarray}
                        \end{array}
                      \end{displaymath}%
                    \end{minipage}%
                  }%
                  \hspace{5mm}%
                  \begin{minipage}{33mm}%
                    \begin{displaymath}
                      \begin{array}{Rc}
                        \mMatrix{C}_{\irVar{a} \irCode{\irAddText}}
                          & \begin{adjblockarray}{ccc}{-0.1ex}
                                \scriptstyle m_3
                              & \scriptstyle m_5
                              & \\[\lskip]
                              \begin{block}{[cc]X}
                                0 & 0 & \mlabel{m_1} \\
                              \end{block}
                            \end{adjblockarray} \\[-2.2ex]
                        \mMatrix{C}_{\irVar{b} \irCode{\irAddText}}
                          & \begin{adjblockarray}{ccc}{-0.1ex}
                                \scriptstyle m_3
                              & \scriptstyle m_5
                              & \\[\lskip]
                              \begin{block}{[cc]X}
                                0 & 0 & \mlabel{m_1} \\
                              \end{block}
                            \end{adjblockarray} \\[-2.2ex]
                        \mMatrix{C}_{\irCode{\irAddText} \irCode{\irLoadText}}
                          & \begin{adjblockarray}{ccc}{0ex}
                                \scriptstyle m_4
                              & \scriptstyle m_5
                              & \\[\lskip]
                              \begin{block}{[cc]X}
                                2      & \infty & \mlabel{m_3} \\
                                \infty & 0      & \mlabel{m_5} \\
                              \end{block}
                            \end{adjblockarray}
                      \end{array}
                    \end{displaymath}%
                  \end{minipage}%
                }

  \caption{Example of modeling instruction selection as a PBQP}
  \labelFigure{pbqp-example-2}
\end{figure}

Using a prototype implementation, \citeauthor{EcksteinEtAl:2003} ran experiments
on a selected set of fixed-point \glspl{program} exhibiting the behavior
discussed earlier.
%
The results indicate that their scheme improved performance by
\mbox{\num{40}--\SI{60}{\percent}} on average -- and at most \SI{82}{\percent}
for one \gls{function} -- compared to traditional \gls{tree covering}-based
\glspl{instruction selector}.
%
According to \citeauthor{EcksteinEtAl:2003}, this considerable gain in
performance comes from more efficient use of value modes to which \gls{tree
  covering}-based techniques must make premature assignments, and thus could
have a detrimental effect on code quality.
%
For example, if chosen poorly, the \gls{instruction selector} may need to emit
additional \glspl{instruction} in order to undo decisions regarding value modes,
which obviously reduces performance and needlessly increases the code
size.
%
Although the technique by \citeauthor{EcksteinEtAl:2003} clearly mitigates these
concerns, their design also has limitations of its own.
%
Most importantly, their \gls{PBQP} model can only support \glspl{pattern tree}
and consequently hinders exploitation of many common \gls{target machine}
features, such as \gls{multi-output.ic} \glspl{instruction}.


\subsection{Handling DAG-shaped Patterns}

To handle \gls{DAG}-shaped \glspl{pattern}, the \gls{PBQP} model must be
extended.
%
First assume an extended \gls{grammar} where \gls{multi-output.ic}
\glspl{instruction} are described using \gls{complex.r} \glspl{rule} (described
in \refAppendix{dag-covering} on
\refPageOfSection{dc-rules-multiple-productions}, see also
\refFigure{extended-machine-grammar-rule-anatomy}).
%
For each combination of \glspl{match} derived from \gls{proxy.r} \glspl{rule}
that can be combined into an instance of a \gls{complex.r} \gls{rule}, a
\gls!{complex.m}[ \gls{match}] is created.
%
Each \gls{complex.m} \gls{match}~$i$ in turn introduces a
\gls{variable}~\raisebox{0pt}[\height-2pt]{$\mVector{\mVar{x}}_i \in \mSet{0,
    1}^2$} to decide whether $i$ is selected.
%
Because of the \raisebox{0pt}[\height-2pt]{$\mVector{1}^{\transp}
  \mVector{\mVar{x}}_i = 1$} condition, every such \gls{variable} has exactly
two elements (one representing \emph{on} and the other \emph{off}).
%
Like with the \gls{simple.r} \glspl{rule}, the costs of selecting a
\gls{complex.m} \gls{rule} and interactions between these -- for example, two
\gls{complex.m} \glspl{match} are not allowed to overlap or cause cyclic data
dependencies -- are represented through the cost vectors and matrices.

In order to select a \gls{complex.r} \gls{rule}, all of its \gls{proxy.r}
\glspl{rule} must also be selected.
%
This is achieved by first extending, for each node~$i$, the domain of its
\gls{variable}~$\mVector{\mVar{x}}_i$ with \glspl{match} derived from
\gls{proxy.r} \glspl{rule}.
%
Then a new set of cost matrices~$D_{ij}$ is created such that, for a node~$i$
and \gls{complex.m} \gls{match}~$j$, the costs are 0 if
\mbox{$\mVector{\mVar{x}}_j = \text{\emph{off}}$} or $\mVector{\mVar{x}}_i$ is
set to a \gls{proxy.r} \gls{rule} associated with $j$.
%
Otherwise the costs are $\infty$.
%
Consequently, if a \gls{complex.m} \gls{match} covering some node~$n$ is
selected, then the only choice for $\mVector{\mVar{x}}_n$ with non-infinite cost
is an associated \gls{proxy.r} \gls{rule}.
%
The \gls{PBQP} model is thus augmented with another sum
%
\begin{equation}
  \sum_{\mathclap{i \,\in\, N, \: j \,\in\, M}}
  \mVector{\mVar{x}}_i^{\transp} D_{ij} \, \mVector{\mVar{x}}_j
\end{equation}
%
where $N$ denotes the set of \glspl{node} in the \gls{SSA graph} and $M$ denotes
the set of \gls{complex.m} \glspl{match}.

This alone, however, allows \glspl{solution} where all \gls{proxy.r}
\glspl{rule} but none of the \gls{complex.r} \glspl{rule} are selected.
%
This is resolved by assigning an artificially large cost~$K$ to the selection of
\gls{proxy.r} \glspl{rule}, which is offset when selecting the corresponding
\gls{complex.r} \gls{rule}.
%
For example, if a \gls{complex.r} \gls{rule}~$r$ with cost~2 consists of three
\gls{proxy.r} \glspl{rule}, then the new cost of selecting $r$ is \mbox{$2 -
  3K$}.


\subsection{Using Rewrite Rules Instead of Production Rules}

In 2010, \textcite{BuchwaldZwinkau:2010} introduced another technique based on
\glspl{PBQP}.
%
But unlike \citeauthor{EcksteinEtAl:2003} and \citeauthor{EbnerEtAl:2008},
\citeauthor{BuchwaldZwinkau:2010} approached the task of \gls{instruction
  selection} as a formal \gls{graph} transformation problem, for which much
previous work already exist.
%
Hence, in \citeauthor{BuchwaldZwinkau:2010}'s design the \glspl{instruction} are
expressed as \gls!{rewrite.r}[ \glspl{rule}] instead of \glspl{production rule}.
%
As these \gls{rewrite.r} \glspl{rule} are based on a formal foundation, the
resulting \gls{instruction selector} can be automatically verified to handle all
possible \glspl{program}.
%
If this check fails, the verification tool can also provide the necessary
\gls{rewrite.r} \glspl{rule} that are currently missing from the
\gls{instruction set}.

The technique works as follows.
%
First the \gls{SSA graph} is converted into a \gls{DAG}-like form by duplicating
each \mbox{$\phi$-\gls{node}} into two \glspl{node}, which effectively breaks
any \glspl{cycle} appearing in the \gls{SSA graph}.
%
After finding all applicable \gls{rewrite.r} \glspl{rule} for this \gls{DAG}
(this is done using traditional \gls{pattern matching}), a corresponding
instance of the \gls{PBQP} is formulated and solved as before.

\citeauthor{BuchwaldZwinkau:2010} also discovered and addressed flaws in the
\glsshort{PBQP} solver by \citeauthor{EcksteinEtAl:2003}, which may fail to find
a solution in certain situations due to inadequate propagation of information.
%
However, \citeauthor{BuchwaldZwinkau:2010} also cautioned that their own
implementation does not scale well when the number of overlapping
\glspl{pattern} grows.
%
In addition, since the \gls{SSA graph} is devoid of control-flow information,
none of the \glsshort{PBQP}-based techniques can support \gls{inter-block.ic}
\glspl{instruction}.


\section{Other Graph-Based Approaches}
\labelSection{gc-other-approaches}

\subsection{More Hardware Modeling Techniques}

In \refAppendix{dag-covering} we saw a technique for performing \gls{microcode
  generation} where the entire processor of the \gls{target machine} is modeled
as a \gls{graph} instead of by just deriving the \glspl{pattern} for the
available \glspl{instruction}.
%
Here we will look at a few techniques that rely on the same modeling scheme, but
address the more traditional problem of \gls{instruction selection}.


\subsubsection{\glsentrytext{Chess}}

\textcite{LanneerEtAl:1990} developed in 1990 a design that was later adopted by
\citeauthor{VanPraetEtAl:1994}~\cite{VanPraetEtAl:1994, VanPraetEtAl:2001} in
their implementation of \gls{Chess}, a well-known compiler targeting \glspl{DSP}
and \glspl{ASIP}.

Comparing \gls{Chess} to \gls{MSSQ} (see \refAppendix{dag-covering} on
\refPageOfSection{dc-modeling-entire-target-machines}), we find two striking
differences.
%
First, in \gls{MSSQ} the data paths of the processor are given by a manually
written \gls{machine description}, whereas \gls{Chess} derives these
automatically from an \mbox{\gls{nML}-based} specification~\cite{FauthEtAl:1993,
  FauthEtAl:1995}.

Second, the method of \gls{bundling} -- which is the task of scheduling
operations for parallel execution -- is different.
%
The \gls{instruction selector} in \gls{MSSQ} uses techniques from \gls{DAG
  covering} to find \glspl{pattern} in the hardware \gls{graph}, which can
subsequently be used to cover the \glspl{expression tree}.
%
After \gls{pattern selection}, another routine attempts to schedule the selected
\glspl{instruction} for parallel execution.
%
In contrast, \gls{Chess} takes a more incremental approach.
%
From the \gls{program} \gls{Chess} first constructs a \gls{chaining graph},
where each \gls{node} represents an operation in the \gls{program} that has been
annotated with a set of functional units capable of executing that operation.
%
Since the functional units on a \gls{DSP} are commonly grouped into
\glspl!{FBB}, the \gls{chaining graph} also contains an \gls{edge} between every
pair of \glspl{node} that could potentially be executed on the same \gls{FBB}.
%
A heuristic algorithm then attempts to collapse the \glspl{node} in the
\gls{chaining graph} by selecting an \gls{edge}, replacing the two \glspl{node}
with a new \gls{node}, and then removing the \glspl{edge} between \glspl{node}
of operations that can no longer be executed on the same \gls{FBB} as the
operations of the new \gls{node}.
%
This process iterates until no more \glspl{node} can be collapsed, and every
remaining \gls{node} in the \gls{chaining graph} thus constitutes a
\gls!{bundle}.
%
The same authors later extended this design in~\cite{VanPraetEtAl:2001} to
consider selection between \emph{all} possible \glspl{bundle} using
\glshyphened{branch and bound} \gls{search}, and to enable some measure of code
duplication by allowing the same operations in the \gls{function graph} to
appear in multiple \glspl{bundle}.

Using this incremental scheme to form the \glspl{bundle}, the design by
\citeauthor{VanPraetEtAl:1994} is capable of \gls{bundling} operations that
potentially reside in different \glspl{block}.
%
Their somewhat-\gls{integrated.cg} \gls{code generation} approach also allows
efficient \gls{assembly code} to be generated for complex architectures, making
it suitable for \glspl{DSP} and \glspl{ASIP} where the data paths are very
irregular.
%
It may be possible to also extend the technique to support \gls{inter-block.ic}
\glspl{instruction} as well, but \gls{interdependent.ic} \glspl{instruction} are
most likely out of reach due to its heuristic nature.


\subsubsection{Generating Assembly Code Using Simulated Annealing}

Another, albeit unusual, \gls{code generation} technique was proposed by
\textcite{Visser:1999} in~1999.
%
Like \gls{MSSQ} and \gls{Chess}, \citeauthor{Visser:1999}'s approach is an
\gls{integrated.cg} \gls{code generation} design but solves the problem using
\gls!{simulated annealing}, which is a meta-heuristic to avoid getting stuck in
a local maximum when searching for solutions (see for example
\cite{KirkpatrickEtAl:1983} for an overview).
%
In brief terms, an initial solution is found by randomly mapping each \gls{node}
in the \gls{function graph} to a \gls{node} in the hardware \gls{graph} -- which
models the entire processor -- and then a schedule is found using traditional
\gls{list scheduling}.
%
A \gls{fitness function} is then applied to judge the effectiveness of the
solution, but the exact details are omitted from the paper.
%
A proof-of-concept prototype was developed and tested on a simple \gls{program},
but it appears no further research has been conducted on this idea.


\subsection{Improving Code Quality with Mutation Scheduling}

The last item we will discuss is a technique called \gls!{mutation
  scheduling},\!%
%
\footnote{%
  Despite its name, the idea of \gls{mutation scheduling} is completely
  orthogonal to the theory of \glsdesc{GA}s.%
}
%
which was introduced in 1994 by
\citeauthor{NovackEtAl:2002}~\cite{NovackNicolau:1994, NovackEtAl:2002}.
%
\Gls{mutation scheduling} is technically a form of \gls{instruction scheduling}
that primarily targets \gls{VLIW} architectures, but it also integrates a
sufficient amount of \gls{instruction selection} to warrant being included in
this dissertation.
%
On the other hand, the amount of \gls{instruction selection} that \emph{is}
incorporated is in turn not really based on \gls{graph covering}, but, as with
\glspl{trellis diagram} (see \refAppendixList{tree-covering, dag-covering} on
\refPagesOfSectionList{tc-trellis-diagrams-trees, dc-trellis-diagrams-dags},
respectively), I decided against discussing it in its own appendix.

Broadly speaking, \gls{mutation scheduling} essentially tries to reduce the
makespan of \glspl{program} for which \gls{assembly code} have already been
generated (hence \gls{instruction selection}, \gls{instruction scheduling}, and
\gls{register allocation} has already been performed).\!%
%
\footnote{%
  Although it is depicted here primarily as a post-step to \gls{code
    generation}, one could just as well design a \glsshort{Davidson-Fraser
    approach}-style \gls{compiler} (see \refAppendix{macro-expansion} on
  \refPageOfSection{macro-expansion-with-peephole-optimization}) where simple
  methods are applied to generate correct but naive \gls{assembly code}, and
  then rely on \gls{mutation scheduling} to improve the code quality.%
}
%
This is done by progressively moving the computations, one at a time, such that
they can be executed in parallel with other \glspl{instruction} and thus finish
sooner.
%
If such a move cannot be made, for example due to violation of some resource
constraint or data dependency, then \gls{mutation scheduling} tries to alter the
value -- this is called \gls!{value mutation} -- which means that the current
operation is replaced by other, equivalent operations that conform to the
restrictions.
%
These operations are selected from a \gls!{mutation set}, which is conceptually
a recursive data structure, as an expression in the \gls{mutation set} may use
intermediate values that in turn necessitate \glspl{mutation set} of their own.
%
\citeauthor{NovackEtAl:2002} compute these \glspl{mutation set} by first taking
the original operation and then applying a series of semantic-preserving
functions that have been derived from various logical axioms, algebraic
theorems, and the characteristics of the \gls{target machine}.
%
For example, if the value~$X$ is computed as \mbox{$Y + 5$}, then $Y$ can later
be obtained by computing \mbox{$X - 5$}.
%
Another example is multiplication by powers of~\num{2}, which can be replaced
with \cCode*{shift} \glspl{instruction}, provided such \glspl{instruction} are
available.
%
If this is beneficial, a value can also be recomputed instead of copied from its
current location.
%
This idea is known as \gls!{recomputation} or \gls!{rematerialization}, which is
a method for reducing \gls{register pressure}, as it allows \glspl{register} to
be released at an earlier point in the \gls{assembly code}.

In \gls{mutation scheduling}, ``shorter'' mutations are preferred over longer
ones.
%
This is because a \gls{value mutation} of $v$ can lead to a cascade of new
computations, which all will need to be scheduled before $v$ can be is computed.
%
Note that these computations can be scheduled such that they appear in
\glspl{block} preceding the \gls{block} in which the computations of $v$ appear.
%
Hence the length of a mutation is loosely defined as the number of
\gls{instruction} \glspl{bundle} that may need to be modified in order to
realize the mutation.
%
Moreover, since the new computations of a successful mutation consume resources
and incur dependencies of their own, the existing candidates appearing in
\glspl{mutation set} may need to be removed or modified.
%
The ``best'' combination of mutations is then decided heuristically, but the
paper is vague on how this is done exactly.

\citeauthor{NovackEtAl:2002} implemented a prototype by extending an existing
scheduler based on \gls!{GRiP}, which is another global \gls{instruction
  scheduling} technique developed by the same authors
(see~\cite{NicolauNovack:1992}).
%
Subsequent experiments using a selected set of benchmark \glspl{program}
demonstrated that the \gls{mutation scheduling}-based design yielded a two- to
threefold performance improvement over the \gls{GRiP}-only-based counterpart,
partly due to its ability to apply \gls{rematerialization} in regions where
\gls{register pressure} is high.
%
Unfortunately the authors neglected to say anything about the time complexity of
\gls{mutation scheduling}, and whether it scales to larger \glspl{program}.


\section{Summary}
\labelSection{gc-summary}

In this appendix we have considered a number of techniques that are founded, in
one form or another, on the \gls{principle} of \gls{graph covering}.
%
Such techniques are among the most powerful methods of \gls{instruction
  selection} since they perform \gls{global.is} \gls{instruction selection} as
well as have more extensive \gls{instruction} support compared to most
\glsshort{tree covering} and \gls{DAG covering}-based designs.

Unfortunately this has not been fully exploited in existing techniques, partly
due to limitations in the \gls{program} representations or to restrictions
enforced by the underlying solving techniques.
%
Moreover, performing \gls{global.is} \gls{instruction selection} is
computationally much harder compared to \gls{local.is} \gls{instruction
  selection}, and therefore most likely we will only see these techniques
applied in \glspl{compiler} whose users can afford very long compilation times
(for example when targeting embedded systems with extremely high demands on
performance, code size, power consumption, or a combination thereof).
