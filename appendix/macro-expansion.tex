% Copyright (c) 2017, Gabriel Hjort Blindell <ghb@kth.se>
%
% This work is licensed under a Creative Commons 4.0 International License (see
% LICENSE file or visit <http://creativecommons.org/licenses/by/4.0/> for a copy
% of the license).

\chapter{Macro Expansion}
\labelAppendix{macro-expansion}

\todo{write introduction}

The selection of techniques in this appendix includes all those discussed in
earlier surveys by \textcite{Cattell:1977} and
\textcite{GanapathiEtAl:1982:Survey}.
%
In the latter, this \gls{principle} is called \gls!{interpretative.cg}[
  \gls{code generation}].
%
Several of these techniques are also discussed in depth by
\textcite{Lunell:1983}.

This appendix is based on material presented in
\cite[Chap.\thinspace2]{HjortBlindell:2016:Survey} that has been adapted for
this dissertation.


\section{The Principle}

\begin{inParFigure}{54mm}[r]
  \centering%
  % Copyright (c) 2017, Gabriel Hjort Blindell <ghb@kth.se>
%
% This work is licensed under a Creative Commons 4.0 International License (see
% LICENSE file or visit <http://creativecommons.org/licenses/by/4.0/> for a copy
% of the license).
%
\begingroup%
\figureFont\figureFontSize\relsize{-.5}%
\def\braceSep{2mm}%
\begin{tikzpicture}[%
    code/.style={%
      nothing,
      node distance=0,
      font=\strut\ttfamily,
    },
    asm/.style={%
      code,
    },
    brace/.style={%
      decorate,
      decoration={brace},
      line width=\normalLineWidth,
    },
    line/.style={%
      -,
      line width=\normalLineWidth,
      shorten <=2pt,
      shorten >=2pt,
    },
    label/.style={
      nothing,
      inner ysep=1mm,
      node distance=0,
      font=\relsize{-.5}\scshape\bfseries,
    },
  ]

  % Function
  \node [code]                               (c1) {int a = 1;};
  \node [code, below right=1pt and 0 of c1.south west] (c2) {int b = a + 4;};
  \node [code, below right=1pt and 0 of c2.south west] (c3) {p[4] = b;};

  % Assembly code
  \node [asm, right=12mm of c1.east -| c2.east] (a1) {mv \irTemp{1}, 1};
  \node [asm, right=of c2.east -| a1.west] (a2) {add \irTemp{2}, \irTemp{1}, 4};
  \node [asm, right=of c3.east -| a1.west] (a3) {mv \irTemp{3}, @p};
  \node [asm, below right=of a3.south west] (a4)
        {add \irTemp{4}, \irTemp{3}, 16};
  \node [asm, below right=of a4.south west] (a5) {st 0(\irTemp{4}), \irTemp{2}};

  % Expansion braces and lines
  \coordinate (f-right) at ([xshift=.5*\braceSep] c2.east);
  \foreach \i in {1, 2, 3} {
    \draw [brace]
          (f-right |- c\i.north east)
          -- coordinate (cc\i)
          (f-right |- c\i.south east);
  }
  \coordinate (a-left) at ([xshift=-\braceSep] a1.west);
  \draw [brace]
        (a-left |- a1.south west)
        -- coordinate (ac1)
        (a-left |- a1.north west);
  \draw [brace]
        (a-left |- a2.south west)
        -- coordinate (ac2)
        (a-left |- a2.north west);
  \draw [brace, decoration={aspect=.835}]
        (a-left |- a5.south west)
        -- coordinate (ac3)
        (a-left |- a3.north west);
  \draw [line] (cc1) -- (ac1);
  \draw [line] (cc2) -- (ac2);
  \draw [line] (cc3) -- (ac3 |- cc3);

  % Labels
  \node [label, above right=of c1.north west] {function};
  \node [label, above right=of c1.north -| a1.west] {assembly code};
\end{tikzpicture}%
\endgroup%
%

  % LAYOUT FIX:
  % Allow for some more space underneath the figure
  \vspace*{.5\baselineskip}%
\end{inParFigure}%
\noindent%
The first \gls{principle} to emerge was \gls!{macro expansion}, with
applications starting to appear in the 1960s.
%
In \gls{macro expansion}, the \glspl{instruction} are expressed as
\glspl!{macro} which consist of two parts: a \gls!{template} to be matched over
the \gls{function}, and an \gls!{expand procedure} to be executed upon the part
of the \gls{function} that was matched (see \refFigure{macro-example} on
\refPageOfFigure{macro-example} for an example).
%
A \gls!{macro expander} traverses the \gls{function} under compilation and
attempts one \gls{macro} after another, typically in the order they are declared
in the \gls{machine description}.
%
Upon a \gls{match} it executes the corresponding \gls{expand procedure} and then
resumes the traversal with the next, unmatched part until the entire
\gls{function} has been expanded.
%
Consequently, \gls{matching} and \gls{selection} is combined into a single task
as the first \gls{macro} matched is the \gls{macro} to be selected.

The main benefit of \gls{macro expansion} is that it is intuitive and
straightforward to apply.
%
Because the \gls{macro expander} is implemented separately from the
\glspl{macro}, the former can be kept generic and simple while the latter can be
made as customized as needed for the \gls{target machine}.
%
This also allows the \gls{macro expander} to be void of any \glsshort{target
  machine}-specific details, thus necessiting only the \glspl{macro} to be
rewritten when retargeting the \gls{compiler} to another machine.
%
To this end, the \glspl{macro} are typically written in some dedicated language
in order to simplify this task by raising the level of abstraction.


\section{Naive Macro Expansion}

\subsection{Early Applications}

We will refer to \glspl{instruction selector} that directly apply the
\gls{principle} just described as \gls!{naive.me}[ \glspl{macro expander}], for
reasons that will soon become apparent.
%
In the first such implementations, the \glspl{macro} were either written by hand
-- like in the \gls{Pascal} \gls{compiler} developed by
\citeauthor{AmmannEtAl:1974}~\cite{AmmannEtAl:1974, AmmannEtAl:1977} -- or
generated automatically from a \gls{machine description}, which was typically
written in some dedicated language.
%
Consequently, many such languages and related tools have appeared -- and then
disappeared -- over the years (see for example \cite{Brown:1969} for an early
survey).

One such example is \gls{Simcmp}, a \gls{macro expander} developed in 1969 by
\textcite{OrgassWaite:1969}.
%
Designed to facilitate \gls{bootstrapping},\!%
%
\footnote{%
  \Gls!{bootstrapping} is the process of writing a \gls{compiler} in the
  programming language it is intended to compile.%
}
%
\gls{Simcmp} read its input line by line, compared the line against the
\glspl{template} of the available \glspl{macro} (see \refFigure{simcmp-example}
for an example), and then executed the first macro that matched.

\begin{filecontents*}{simcmp-example-macro.c}
* = CAR.*.
    I = CDR('21)
    CDR('11) = CAR(I).
.X
\end{filecontents*}
%
\begin{filecontents*}{simcmp-example-input.c}
A = CAR B.
\end{filecontents*}
%
\begin{filecontents*}{simcmp-example-result.c}
I = CDR(38)
CDR(36) = CAR(I)
\end{filecontents*}
%
\begin{figure}[b]
  \figureFont\figureFontSize%
  \centering%

  \subcaptionbox{%
                  A macro definition%
                  \labelFigure{sicmp-example-macro}%
                }%
                [40mm]%
                {%
                  \lstinputlisting{simcmp-example-macro.c}
                }%
  \hfill%
  \subcaptionbox{%
                  String that matches the template%
                  \labelFigure{sicmp-example-input}%
                }%
                [32mm]%
                {%
                  \lstinputlisting{simcmp-example-input.c}
                }%
  \hfill%
  \subcaptionbox{%
                  After macro expansion%
                  \labelFigure{sicmp-example-result}%
                }%
                [36mm]%
                {%
                  \lstinputlisting{simcmp-example-result.c}
                }

  \caption[Example of macro expansion using \glsentrytext{Simcmp}]%
          {%
            Example of macro expansion using \glsentrytext{Simcmp}
            \cite{OrgassWaite:1969}%
          }
  \labelFigure{simcmp-example}
\end{figure}

Another example is the \gls{GCL}, developed by \textcite{ElsonRake:1970}, which
was used in a \gls{PL/1} \gls{compiler} for generating \gls{assembly code} from
\glspl!{AST}, which are \gls{graph}-based representations of the source code
that are always shaped like \glspl{tree}.
%
The most important feature of these \glspl{tree} is that only a syntactically
valid \gls{program} can be transformed into an~\gls{AST}, which simplifies the
task of the \gls{instruction selector}. However, the basic \gls{principle} of
\gls{macro expansion} remains the same.


\subsubsection{Using IR Instead of ASTs}

\glsreset{IR}

Performing \gls{instruction selection} directly on the source code, either in
its textual form or on the \gls{AST}, carries the disadvantage of tightly
coupling the \gls{backend} to a particular programming language.
%
Most \gls{compiler} infrastructures therefore rely on some lower-level,
machine-independent \gls!{IR} which isolates the subsequent
\glsshort{target machine}-independent optimizations and the \gls{backend} from
the details of the programming language.
%
The \gls{IR} code is often represented as an \gls!{expression tree}, which is a
\gls{tree}-shaped \gls{data-flow graph} (see
\refFigure{expression-tree-example}).
%
\begin{filecontents*}{ir-code-example.c}
$\irTemp{t}$ = a + b
c = $\irTemp{1}$ * 2
\end{filecontents*}
%
\begin{figure}[b]%
  \centering%

  \mbox{}%
  \hfill%
  \subcaptionbox{%
                  IR code%
                  \labelFigure{expression-tree-example-ir-code}%
                }{%
                  \lstinputlisting[mathescape]{ir-code-example.c}
                }%
  \hfill%
  \subcaptionbox{%
                  Expression tree%
                  \labelFigure{expression-tree-example-tree}%
                }{%
                  % Copyright (c) 2017-2018, Gabriel Hjort Blindell <ghb@kth.se>
%
% This work is licensed under a Creative Commons Attribution-NoDerivatives 4.0
% International License (see LICENSE file or visit
% <http://creativecommons.org/licenses/by-nc-nd/4.0/> for details).
%
\begingroup%
\figureFont\figureFontSize%
\begin{tikzpicture}[%
    value node/.style={%
      computation node,
    },
  ]

  % Graph
  \node [computation node] (mul) {\nMul};
  \node [computation node, position=-135 degrees from mul] (add) {\nAdd};
  \node [value node, position=-45 degrees from mul] (const) {\nVar{2}};
  \node [value node, position=-135 degrees from add] (a) {\nVar{a}};
  \node [value node, position=- 45 degrees from add] (b) {\nVar{b}};
  \begin{scope}[data flow]
    \draw (mul) -- (const);
    \draw (mul) -- (add);
    \draw (add) -- (a);
    \draw (add) -- (b);
  \end{scope}
\end{tikzpicture}%
\endgroup%
%
                }%
  \hfill%
  \mbox{}

  \caption{Example of an expression tree}
  \labelFigure{expression-tree-example}
\end{figure}
%
It is common to omit any intermediate variables from the \gls{expression tree}
and only keep those signifying the input and output values of the expression, as
shown in the example.
%
This also means that an \gls{expression tree} can only represent a set of
computations performed within the same \gls{block}, which thus may contain more
than one \gls{expression tree}.
%
Since these representations only capture data flow, the \gls{function}'s control
flow is represented separately as a \gls{control-flow graph}.

One of the first \gls{IR}-based schemes was developed by \textcite{Wilcox:1971}.
%
Implemented in a \gls{PL/C} \gls{compiler}, the \gls{AST} is first transformed
into machine-independent code consisting of \glspl!{SLM instruction}.
%
The \gls{instruction selector} then maps each \gls{SLM instruction} into one or
more target-specific \glspl{instruction} using \glspl{macro} defined in a
language called \gls!{ICL} (see \refFigure{icl-example} for an example).
%
\newcommand{\commentize}[1]{\textit{\figureFont#1}}%
\begin{filecontents*}{icl-example-macro.c}
ADDB  BR   A,ADDB1      $\commentize{If A is in a register, jump to ADDB1}$
      BR   B,ADDB2      $\commentize{If B is in a register, jump to ADDB2}$
      LGPR A            $\commentize{Generate code to load A into register}$

ADDB1 BR  B,ADDB3       $\commentize{If B is in a register, jump to ADDB3}$
      GRX A,A,B         $\commentize{Generate A+B}$
      B   ADDB4         $\commentize{Merge}$

ADDB3 GRR  AR,A,B       $\commentize{Generate A+B}$
ADDB4 FREE B            $\commentize{Release resources assigned to B}$
ADDB5 POP  1            $\commentize{Remove B descriptor from stack}$
      EXIT

ADDB2 GRI  A,B,A        $\commentize{Generate A+B}$
      FREE A            $\commentize{Release resources assigned to A}$
      SET  A,B          $\commentize{A now designates result location}$
      B    ADDB5        $\commentize{Merge}$
\end{filecontents*}
%
\begin{figure}%
  \centering%
  \begin{minipage}{8.5cm}
    \lstinputlisting[mathescape]{icl-example-macro.c}%
  \end{minipage}

  \caption[A binary addition macro in \glsentrytext{ICL}]%
          {A binary addition macro in \glsentrytext{ICL} \cite{Wilcox:1971}]}
  \labelFigure{icl-example}
\end{figure}
%
In practice, these \glspl{macro} turned out to be tedious and difficult to
write.
%
Many details, such as addressing modes and data locations, had to be dealt with
manually from within the \glspl{macro}.
%
In the case of \gls{ICL}, the macro writer also had to keep track of which
variables were part of the final \gls{assembly code}, and which variables were
auxiliary and only used to aid the \gls{code generation} process.
%
In an attempt to simplify this task, \textcite{Young:1974} proposed (but never
implemented) a higher-level language called \gls!{TEL} that would abstract away
some of the implementation-oriented details.
%
The idea was to first express the \glspl{macro} as \gls{TEL}~code and then to
automatically generate the lower-level \gls{ICL}~\glspl{macro} from the
\gls{machine description}.


\subsection{Generating the Macros from a Machine Description}
\labelSection{separating-macros-and-machine-description}

As with \citeauthor{Wilcox:1971}'s design, many of the early
\gls{macro}-expanding \glspl{instruction selector} depended on \glspl{macro}
that were intricate and difficult to write.
%
In addition, many \gls{compiler} developers often incorporated \gls{register
  allocation} into these \glspl{macro}, which further exacerbated the
problem.
%
For example, if the \gls{target machine} exhibits multiple sets of
\glspl{register}, called \glspl!{register class}, and has special
\glspl{instruction} to move data from one \glsshort{register class} to another,
a record must be kept of which value reside in which \gls{register}.
%
Then, depending on the \gls{register} assignment, the \gls{instruction selector}
needs to emit the appropriate data-transfer \glspl{instruction} in addition to
the rest of the \gls{assembly code}.
%
Due to the exponential number of possible situations, the complexity that the
macro designer has to manage can be immense.


\subsubsection{Automatically Inferring the Necessary Data Transfers}

The first attempt to address this problem was made by \textcite{Miller:1971}.
%
In his master's thesis from~1971, \citeauthor{Miller:1971} introduces a \gls{code
  generation} system called \gls!{Dmacs} that automatically infers the necessary
data transfers between memory and different \glspl{register class}.
%
By encapsulating this information in a separate \gls{machine description},
\gls{Dmacs} was also the first system to allow the details of the \gls{target
  machine} to be declared separately instead of being implicitly embedded into
the \glspl{macro}.

\gls{Dmacs} relies on two proprietary languages: \gls!{MIML}, which declares a
set of procedural two-argument commands that serves as the \gls{IR} format (see
\refFigure{miml-example} for an example);
%
\begin{filecontents*}{miml-example.c}
1:    SS      C,J
2:    IMUL    1,D
3:    IADD    2,B
4:    SS      A,I
5:    ASSG    4,3
\end{filecontents*}
%
\begin{figure}%
  \centering%
  \begin{minipage}{3cm}%
    \lstinputlisting{miml-example.c}%
  \end{minipage}

  \caption[MIML example]%
          {%
            An example on how an arithmetic expression
            \cCode*{A[I] = B + C[J] * D} is represented using MIML
            commands.
            %
            The \cCode*{SS} command is used for data referencing and the
            \cCode*{ASSG} command assigns a value to a variable.
            %
            The arguments to the MIML commands are referred to
            either by a variable symbol or by line number \cite{Miller:1971}%
          }
          \labelFigure{miml-example}
\end{figure}
%
and a declarative language called \gls!{OMML} for implementing the \glspl{macro}
that will transform each \gls{MIML} command into \gls{assembly code}.
%
So far this scheme is similar to the one applied by \citeauthor{Wilcox:1971}.

When adding support for a new \gls{target machine}, a macro designer first
specifies the set of available \glspl{register class} (including memory) as well
as the permissible transfer paths between these \glsplshort{register class}.
%
The macro designer then defines the \gls{OMML} \glspl{macro} by providing, for
each macro, a list of \glspl{instruction} that implements the corresponding
\gls{MIML} command on the \gls{target machine}.
%
If necessary, a sequence of \glspl{instruction} can be given to emulate the
effect of a single \gls{MIML} command.
%
Lastly, constraints are added that force the input and output data to reside in
the locations expected of the \gls{instruction}.
%
\RefFigure{omml-example} shows excerpts of an \gls{OMML} specification for an
\gls{IBM}~machine.

\begin{filecontents*}{omml-example.c}
rclass REG:r2,r3,r4,r5,r6
rclass FREG:fr0,fr2,fr4,fr6
...
rpath WORD->REG: L REG,WORD
rpath REG->WORD: ST REG,WORD
rpath FREG-WORD: LE FREG,WORD
rpath WORD->FREG: STE FREG,WORD
...
ISUB s1,s2
from REG(s1),REG(s2) emit SR s1,s2  result REG(s1)
from REG(s1),WORD(s2) emit S s1,s2  result REG(s2)

FMUL m1,m2 (commutative)
from FREG(m1),FREG(m2) emit MER m1,m2  result FREG(m1)
from FREG(m1),WORD(m2) emit ME m1,m2   result FREG(m1)
\end{filecontents*}
%
\begin{figure}%
  \centering%
  \begin{minipage}{9.2cm}%
    \lstinputlisting{omml-example.c}%
  \end{minipage}

  \caption[OMML example]%
          {%
            Partial machine description for IBM-360 in OMML.
            %
            The \cCode*{rclass} command declares a register class, and the
            \cCode*{rpath} command declares a permissible transfer between a
            register class and memory (or vice versa) along with the instruction
            that implements the transfer~\cite{Miller:1971}%
          }
  \labelFigure{omml-example}
\end{figure}

\gls{Dmacs} uses this information to generate a collection of \glspl{finite
  state automaton} (or \glspl{state machine}, as they are also called) to
determine how a given set of input values can be transferred into locations that
are permissible for a given \gls{OMML} \gls{macro}.
%
Each \gls{state machine} consists of a \gls{directed.g} \gls{graph} where a
\gls{node} represents a specific configuration of \glspl{register class} and
memory, some of which are marked as permissible.
%
The edges indicate how to transition from one state to another, and are labeled
with the machine instruction that will enable the transition when executed on a
particular input value.
%
During compilation the \gls{instruction selector} consults the appropriate
\gls{state machine} as it traverses from one \gls{MIML} command to the next,
using the input values of the former to initialize the \gls{state machine}.
%
As the \gls{state machine} transitions from one state to another, the machine
instructions appearing on the edges are emitted until the \gls{state machine}
reaches a permissible state.

The work by \citeauthor{Miller:1971} was pioneering but limited: \gls{Dmacs}
only handled arithmetic expressions consisting of integer and floating-point
values, its addressing mode support was limited, and it could not model other
\gls{target machine} classes such as stack-based architectures.
%
In his 1973 doctoral dissertation, \textcite{Donegan:1973} extended
\citeauthor{Miller:1971}'s ideas by proposing a new language called \gls!{CGPL}.
%
\citeauthor{Donegan:1973}'s proposal was put to the test in the 1978 master's
thesis by \textcite{Maltz:1978}, and was later extended by
\textcite{Donegan:1979}.
%
Similar techniques have also been developed by \textcite{Tirrell:1973} and
\textcite{Simoneaux:1975}, and in their survey
\textcite{GanapathiEtAl:1982:Survey} describe another \gls{state machine}-based
\gls{compiler} called \gls!{Ugen}, which was derived from a virtual machine
called \gls!{Ucode}~\cite{PerkinsSites:1979}.


\subsubsection{Further Improvements}

\glsunset{PCC}

In 1975, \textcite{Snyder:1975} implemented one of the first fully operational
and portable \gls{C}~\glspl{compiler}, where the \gls{target machine}-dependent
parts could be automatically generated from a \gls{machine description}.
%
The design is similar to \citeauthor{Miller:1971}'s in that the \gls{frontend}
first transforms the \gls{program} into an equivalent representation for an
abstract machine.
%
In \citeauthor{Snyder:1975}'s design this representation consists of
\glspl{AMOP}, which are then expanded into target-specific \glspl{instruction}
via \glspl{macro}.
%
The abstract machine and \glspl{macro} are specified in a \gls{machine
  description} language which is also similar to \citeauthor{Miller:1971}'s, but
handles more complex data types, addressing modes, alignment, as well as
branching and function calls.
%
If needed, more complicated \glspl{macro} can be defined as customized
\gls{C}~functions.
%
We mention \citeauthor{Snyder:1975}'s work primarily because it was later
adapted by \textcite{Johnson:1978} in his implementation of \gls{PCC}, which we
will discuss in \refAppendix{tree-covering}.

\glsreset{PCC}

\citeauthor{Fraser:1977a}~\cite{Fraser:1977a, Fraser:1977b} also recognized the
need for human knowledge to guide the \gls{code generation} process, and
implemented a system with the aim of facilitating the addition of handwritten
rules when these are required.
%
First the \gls{program} is transformed into a representation based on a
programming language called \gls!{XL}, which is akin to high-level \gls{assembly
  code}.
%
For example, \gls{XL} provides primitives for array accesses and for~loops.
%
As in the cases of \citeauthor{Miller:1971} and \citeauthor{Snyder:1975}, the
\glspl{instruction} are provided via a separate description that maps directly
to a distinct \gls{XL}~primitive.
%
If some portion of the \gls{program} cannot be implemented by any of the
available \glspl{instruction}, the \gls{instruction selector} will invoke a set
of rules to rewrite the \gls{XL}~code until a solution is found.
%
For example, array accesses are broken down into simpler primitives, and the
same rule base can also be used to improve the code quality of the generated
\gls{assembly code}.
%
Since these rules are provided as a separate \gls{machine description}, they can
be customized and augmented as needed to fit a particular \gls{target machine}.

As we will see, this idea of ``massaging'' the \gls{program} until a solution
can be found has been applied, in one form or another, by many
\glspl{instruction selector} that both predate and succeed
\citeauthor{Fraser:1977a}'s design.
%
Although they represent a popular approach, a significant drawback of such
schemes is that the \gls{instruction selector} may get stuck in an infinite loop
if the set of rules is incomplete for a particular \gls{target machine}, and
determining if this is the case is often far from trivial.
%
Moreover, such rules tend to be hard to reuse for other \glspl{target machine}.


\subsection{Reducing Compilation Time with Tables}

Despite their already simplistic nature, \gls{macro}-expanding
\glspl{instruction selector} can be made even more so by representing the
\mbox{1-to-1} or \mbox{1-to-$n$} mappings as sets of tables. This further
emphasizes the separation between the machine-independent core of the
\gls{instruction selector} from the machine-dependent mappings, as well as
allows for denser implementations that require less memory and potentially
reduce the compilation time, which is the time it takes to compile a given
\gls{program}.


\subsubsection{Representing Instructions as Coding Skeletons}

In 1969 \textcite{LowryMedlock:1969} introduced one of the first table-driven
methods for \gls{code generation}.
%
%\begin{inParFigure}{6cm}[p][2]
%  \centering%
%
%  % LAYOUT FIX:
%  % Create a bit of space above
%  \vspace{0.9\baselineskip}
%
%  % Table cannot be full textwidth or it will overflow by 5-6 pt
%  \begin{framedBoxWI}{0.98\textwidth}
%    \centering%
%    \figureFont% The usual patch doesn't work for inParFigure
%    \begin{tabular}{lll}
%      \dataTerm{L}  & \dataTerm{B2,D(0,BD)} & \dataTerm{XXXXXXXX00000000}\\
%      \dataTerm{LH} & \dataTerm{B2,D(0,B2)} & \dataTerm{0000111100000000}\\
%      \dataTerm{LR} & \dataTerm{R1,R2}      & \dataTerm{0000110100001101}\\
%    \end{tabular}
%  \end{framedBoxWI}
%
%  % LAYOUT FIX:
%  % Create a bit of space below
%  \vspace{\baselineskip}
%\end{inParFigure}%
%
In their implementation of \gls{FHC}, \citeauthor{LowryMedlock:1969} used a bit
string, called a \gls!{coding skeleton} (see \refFigure{coding-skeleton} for an
example), for each \gls{instruction}.
%
The bits represent the restrictions of the \glspl{instruction}, such as the
modes permitted for the operands and the result (for example, ``load from
memory,'' ``load from \gls{register},'' ``do not store,'' ``use this or that
base \gls{register}'').
%
These \glspl{coding skeleton} are then matched against the bit strings
corresponding to the \gls{program} under compilation.
%
An `\cVar*{X}' appearing in the \gls{coding skeleton} means that it will
always match any bit.

The main disadvantage of \citeauthor{LowryMedlock:1969}'s design was that the
tables could only be used for the most basic of \glspl{instruction}, and had to
be written by hand in the case of \gls{FHC}.
%
More extensive designs were later developed by \textcite{Tirrell:1973} and
\textcite{Donegan:1973}, but these also suffered from similar disadvantages of
making too many assumptions about the \gls{target machine}, which hindered
\gls{compiler} retargetability.


\subsubsection{Expanding Macros Top-Down}

Later \textcite{KrummeAckley:1982} introduced a table-driven design which,
unlike the earlier techniques, exhaustively enumerates all valid combinations of
selectable \glspl{instruction}, schedules, and \glspl{register allocation} for a
given \gls{expression tree}.
%
Implemented in a \gls{C}~\gls{compiler} targeting \gls{DEC-10} machines, the
technique also allows code size to be factored in as an optimization goal, which
was an uncommon feature at the time.
%
\citeauthor{KrummeAckley:1982}'s \gls{backend} applies a recursive algorithm
that begins by selecting \glspl{instruction} for the \gls{root} in the
\gls{expression tree}, and then working its way down.
%
In comparison, the bottom-up techniques we have examined so far all start at the
leaves and then traverse upwards.
%
We settle with this distinction for now as we will resume and deepen the
discussion of bottom-up \versus top-down \gls{instruction selection} in
\refAppendix{tree-covering}.

\labelPage{branch-and-bound}

Enumerating all valid combinations in code generation leads to a combinatorial
explosion, thus making it impossible to actually produce and check each and
every one of them.
%
To curb this immense complexity, \citeauthor{KrummeAckley:1982} applied a
variant of \gls!{branch and bound} as \gls{search} strategy.\!
%
\footnote{%
  In their paper, \citeauthor{KrummeAckley:1982} actually call this \gls{AB
    pruning}, which is an entirely different search strategy, but their
  description of it fits more the \gls{branch and bound} approach.
  %
  Both are well explained in~\cite{Norvig:2010}.%
}
%
The problem is how to prove that a given branch in the \gls{search space} will
definitely lead to solutions that are worse than what we already have (and can
thus be skipped).
%
\citeauthor{KrummeAckley:1982} only partially tackled this problem by pruning
away branches that for sure will eventually lead to failure and thus yield no
solution whatsoever.
%
Without going into too much detail, this is done by using not just a single
\gls{instruction} table but several -- one for each so-called \emph{mode} --
which are constructed in a hierarchical manner.
%
In this context, a mode is oriented around the result of an expression, for
example whether it is to be stored in a \gls{register} or in memory.
%
Using these tables, the \gls{instruction selector} can look ahead and detect
whether the current set of already-selected \glspl{instruction} will lead to a
dead end.
%
With this as the only method of branch pruning, however, the \gls{instruction
  selector} will make many needless revisits in the \gls{search space}, and
consequently does not scale to larger \glspl{expression tree}.


\subsection{Falling Out of Fashion}

Despite the improvements we have just discussed, they still do not resolve the
main disadvantage of \gls{macro}-expanding \glspl{instruction selector}, namely
that they can only handle \glspl{macro} that expand a single \gls{AST} or
\gls{IR}~\gls{node} at a time.
%
The limitation can be somewhat circumvented by allowing information about the
visited \glspl{node} to be forwarded from one \gls{macro} to the next, thereby
postponing \gls{assembly code} emission in the hopes that more efficient
\glspl{instruction} can be used.
%
However, if done manually -- which was often the case -- this quickly becomes an
unmanageable task for the \gls{macro} writer, in particular if backtracking
becomes necessary due to faulty decisions made in prior \gls{macro} invocations.

Thus \gls{naive.me} \glspl{macro expander} are effectively limited to supporting
only \gls{single-output.ic} \glspl{instruction}.\!%
%
\footnote{%
  This is a truth with modification: a \gls{macro expander} can emit
  \gls{multi-output.ic} \glspl{instruction}, but only one of its output values
  will be retained in the \gls{assembly code}.%
}
%
As this has a detrimental effect on code quality for \glspl{target machine}
exhibiting more complicated features, such as \gls{multi-output.ic}
\glspl{instruction}, \glspl{instruction selector} based solely on \gls{naive.me}
\gls{macro expansion} were quickly replaced by newer, more powerful techniques
when these started to appear in the late 1970s.
%
One of these we will discuss later in this appendix.



\subsubsection{%
  Rekindled Application in the First Dynamic Code Generation Systems%
}

Having fallen out of fashion, \gls{naive.me}[ly] \gls{macro}-expanding
\glspl{instruction selector} later made a brief reappearance in the first
dynamic \gls{code generation} systems that were developed in the 1980s and
1990s.
%
In such systems the \gls{program} is first compiled into \gls!{byte code}, which
is a kind of target-independent \gls{machine code} that can be interpreted by an
underlying runtime environment.
%
By providing an identical environment on every \gls{target machine}, the same
\gls{byte code} can be executed on multiple systems without having to be
recompiled.

The cost of this portability is that running a \gls{program} in interpretive
mode is typically much slower than executing native \gls{machine code}.
%
This performance loss can be mitigated by incorporating a \gls{compiler} into
the runtime environment.
%
First, the \gls{byte code} is profiled as it is executed.
%
Frequently executed segments, such as inner loops, are then compiled into native
\gls{machine code}.
%
Since the code segments are compiled at runtime, this scheme is called \gls!{JIT
  compilation}, which allows performance to be increased while retaining the
benefits of the \gls{byte code}.
%
If the performance gap between running \gls{byte code} instead of native
\gls{machine code} is large, then the \gls{compiler} can afford to produce
\gls{assembly code} of low quality in order to decrease the overhead in the
runtime environment.
%
This was of great importance for the earliest dynamic runtime systems where
hardware resources were typically scarce, which made \gls{macro}-expanding
\gls{instruction selection} a reasonable option.
%
A few examples include interpreters for
\gls{Smalltalk-80}~\cite{DeutschSchiffman:1984} and
\gls{Omniware}~\cite{Adl-TabatabaiEtAl:1996} (a predecessor to \gls{Java}), and
\gls{code generation} systems, such as \gls{Vcode}~\cite{Engler:1996},
\gls{GBurg}~\cite{FraserProebsting:1999} (which was used within a small virtual
machine), and \gls{Gnu Lightning}~\cite{GNUlightning} (which was directly
inspired by \gls{Vcode}).

As technology progressed, however, dynamic \gls{code generation} systems also
began to transition to more powerful techniques for \gls{instruction selection}
such as \gls{tree covering}, which will be described in
\refAppendix{tree-covering}.


\section{Improving Code Quality with Peephole Optimization}

An early but still applied method of improving the quality of generated
\gls{assembly code} is to perform a subsequent \gls{program} optimization step
that attempts to combine and replace several \glspl{instruction} with shorter,
more efficient alternatives.
%
These routines are known as \glspl!{peephole optimizer} for reasons which will
soon become apparent.


\subsection{What Is Peephole Optimization?}

In 1965, \textcite{McKeeman:1965} advocated the use of a simple but often
neglected \gls{program} optimization procedure which, as a post-step to
\gls{code generation}, inspects a small sequence of \glspl{instruction} in the
\gls{assembly code} and attempts to combine two or more adjacent
\glspl{instruction} with a single instruction.
%
Similar ideas were also suggested by \textcite{LowryMedlock:1969} around the
same time.
%
Doing this reduces code size and also improves performance as using complex
\glspl{instruction} is often more efficient than using several simpler
\glspl{instruction} to implement the same functionality\footnote{On a related
  note, this idea was applied by \textcite{Cho:2006} for reselecting
  instructions in order to improve iterative modulo schedules for \glspl{DSP}.}.
%
Because of its narrow window of observation, this technique became known as
\gls{peephole optimization}.


\subsubsection{Modeling Instructions with Register Transfer Lists}
\labelSection{register-transfer-lists}
\labelPage{register-transfer-lists}

Since this kind of optimization is tailored for a particular \gls{target
  machine}, the earliest implementations were (and still often are) done ad hoc
and by hand.
%
For example, in 2002, \textcite{Krishnaswamy:2002} wrote a \gls{peephole
  optimizer} by hand which reduces code size by replacing known \glspl{pattern}
of \gls{ARM}~code with smaller equivalents.
%
Recognizing the need for automation, \textcite{Fraser:1979} introduced in 1979
the first technique that allowed \glspl{peephole optimizer} to be generated from
a formal description.
%
The technique is also described in a longer article by \textcite{Davidson:1980}.

Like \citeauthor{Miller:1971}, \citeauthor{Fraser:1979} described the semantics
of the \glspl{instruction} separately in a symbolic \gls{machine description}.
%
The \gls{machine description} describes the observable effects that each
\gls{instruction} has on the \gls{target machine}'s \glspl{register}.
\citeauthor{Fraser:1979} called these effects \glspl!{RT}, and each
\gls{instruction} thus has a corresponding \gls!{RTL}.
%
For example, assume that we have a three-address \cCode*{add}~\gls{instruction}
which adds an immediate value~\cVar*{imm} to the value in
\gls{register}~\cVar*{r}[s], stores the result in \gls{register}~\cVar*{r}[d],
and sets a zero flag~\cVar*{Z}.
%
For this \gls{instruction}, the corresponding \gls{RTL} would be expressed as
\begin{displaymath}
  \mFunFont{\gls{RTL}}(\text{\cCode*{add}}) =
  \left\{
    \begin{array}{r@{\;\; \leftarrow \;\;}l}
        \text{\cVar*{r}[d]}
      & \text{\cVar*{r}[s]} + \text{\cCode{imm}} \\
        \text{\cVar*{Z}}
      & \text{\cVar*{r}[s]} + \text{\cCode{imm}} = 0
    \end{array}
  \right\}\!.
\end{displaymath}

The \glspl{RTL} are then fed to a \gls{program} called \gls!{PO}, which produces
a \gls{program} optimization routine that makes two passes over the generated
\gls{assembly code}.
%
The first pass runs backwards across the \gls{assembly code} to determine the
observable effects (that is, the \gls{RTL}) of each \gls{instruction} in the
\gls{assembly code}.
%
This allows effects that have no impact on the \gls{program}'s observable
behavior to be removed.
%
For example, if the value of a \gls{status flag} is not read by any subsequent
\gls{instruction}, it is considered to be \gls!{unobservable.RT} and can thus be
ignored.
%
The second pass then checks whether the combined \glspl{RTL} of two adjacent
\glspl{instruction} are equal to that of some other \gls{instruction} (in
\gls{PO} this check is done via a series of string comparisons).
%
If such an instruction is found, the pair is replaced and the routine backs up
one \gls{instruction} in order to check the combination of the new
\gls{instruction} with the following \gls{instruction} in the \gls{assembly
  code}.
%
This way replacements can be cascaded and many \glspl{instruction} reduced into
a single equivalent, provided there exists an appropriate \gls{instruction} for
each intermediate step.

Pioneering as it was, \gls{PO} also had several limitations.
%
The main drawbacks were that it only supported combinations of two
\glspl{instruction} at a time, and that these had to be lexicographically
adjacent in the \gls{assembly code}.
%
The \glspl{instruction} were also not allowed to cross \gls{block} boundaries,
meaning that they had to belong to the same \gls{block}.
%
\textcite{Davidson:1984} later removed the limitation of lexicographical
adjacency by making use of \glspl{data-flow graph} instead of operating directly
on the \gls{assembly code}, and they also extended the size of the
\gls{instruction} window from pairs to triples.


\subsubsection{Further Developments}

Much research has been dedicated to improving automated approaches to
\gls{peephole optimization}.
%
In 1983, \textcite{Giegerich:1983} proposed a formal design that eliminates the
need for a fixed-size \gls{instruction} window.
%
Shortly after, \textcite{Kessler:1984} introduced a method where \gls{RTL}
combinations and comparisons can be precomputed as the \gls{compiler} is built,
thus decreasing compilation time.
%
\textcite{Kessler:1986} later expanded his work to incorporate an
\mbox{$n$-size} \gls{instruction} window, similar to that of
\citeauthor{Giegerich:1983}, although at an exponential cost.

Another scheme was developed by \textcite{Massalin:1987} who implemented a
system called the \gls!{Superoptimizer}, and similar systems have subsequently
been referred to as \glspl!{superoptimizer}.
%
The \gls{Superoptimizer} accepts small \glspl{program} written in \gls{assembly
  code}, and then exhaustively combines sequences of \glspl{instruction} to find
shorter implementations that exhibit the same behavior as the original
\gls{program}.\!%
%
\footnote{%
  The same idea has also been applied by \textcite{El-Khalil:2004} and
  \textcite{Anckaert:2005}, where the \gls{assembly code} of compiled
  \glspl{program} is modified in order to support \gls!{steganography} (the
  covert insertion of secret messages).
  %
  For example, \citeauthor{Anckaert:2005} used this technique on nine
  \glspl{program} from the \gls{SPECint 2000} benchmark suite in order to embed
  and extract William Shakespeare's play \textit{King Lear}.%
}
%
\textcite{Granlund:1992} later adapted \citeauthor{Massalin:1987}'s ideas into a
method that minimizes the number of branches.
%
Both implementations, however, were implemented by hand and
customized for a particular \gls{target machine}.
%
Moreover, neither makes any
guarantees on correctness.
%
A technique for automatically generating
\gls{peephole optimization}-based \glspl{superoptimizer} was developed by
\textcite{Bansal:2006}, where the \gls{superoptimizer} learns to optimize short
sequences of \glspl{instruction} from a set of training \glspl{program}.
%
A couple of designs that guarantee correctness have been developed by
\citeauthor{Joshi:2002}~\cite{Joshi:2002, Joshi:2006} and \textcite{Crick:2009},
who applied automatic theorem proving and \gls{answer set programming},
respectively.
%
Recently, a similar technique based on \gls{quantifier-free
  bit-vector logic} formulas was introduced by \textcite{Srinivasan:2015}.


\subsection{Combining Naive Macro Expansion with Peephole Optimization}
\labelSection{macro-expansion-with-peephole-optimization}

Up to this point peephole optimizers had mainly been used to improve
already-generated \gls{assembly code} -- in other words, \emph{after}
instruction selection had been performed.
%
In 1984, however, \textcite{Davidson:1984} developed an \gls{instruction
  selection} technique that incorporates the power of \gls{peephole
  optimization} with the simplicity of \gls{macro expansion}.
%
Similar yet unsuccessful strategies had already been proposed earlier by
\textcite{Auslander:1982} and \textcite{Harrison:1979}, but
\citeauthor{Davidson:1984} struck the right balance between \gls{compiler}
retargetability and code quality, which made their design a viable option for
production-quality \glspl{compiler}.
%
This scheme has hence become known as the \gls!{Davidson-Fraser approach}, and
variants of it have been used in several \glspl{compiler}, such as the
\gls{YC}~\cite{Davidson:1982}, the \gls{ZephyrVPO}~system~\cite{Appel:1998}, the
\gls{ACK}~\cite{Tanenbaum:1983}, and, most famously the
\gls{GCC}~\cite{Stallman:1988, Khedker:2012}.


\subsubsection{The Davidson-Fraser Approach}

%\begin{figure}
%  \centering%
%  \input{\figurePath/macro-expansion/davidson-fraser}
%
%  \figCaption{Overview of the \glsentrytext{Davidson-Fraser approach}}%
%    [Davidson:1984]
%  \labelFigure{davidson-fraser}
%\end{figure}

In the \gls{Davidson-Fraser approach} the \gls{instruction selector} consists of
two parts: an \gls!{expander} and a \gls!{combiner} (see
\refFigure{davidson-fraser}).
%
The task of the \gls{expander} is to transform the \gls{program} into a series
of \glspl{RTL}.
%
The transformation is done by executing simple \glspl{macro} that expand every
\gls{node} in the \gls{expression tree} (assuming the \gls{program} is
represented as such) into a corresponding \gls{RTL} that describes the effects
of that \gls{node}.
%
Unlike the previous \glspl{macro expander} we have discussed, these
\glspl{macro} do not incorporate \gls{register allocation}.
%
Instead the \gls{expander} assigns each result to a virtual storage location
called a \gls{temporary}, of which it is assumed there exists an infinite
amount.
%
A subsequent \gls{register allocator} then assigns each temporary to a
\gls{register}, potentially inserting additional code that saves some values to
memory for later retrieval when the number of available \glspl{register} is not
enough (this is called \gls{spilling.r}).
%
After expansion, but before \gls{register allocation}, the \gls{combiner} is
run.
%
Using the same technique as that behind \gls{PO}, the \gls{combiner} tries to
improve code quality by combining several \glspl{RTL} in the \gls{program} into
a single, larger \gls{RTL} that corresponds to some \gls{instruction} on the
\gls{target machine}.
%
For this to work, both the \gls{expander} and the \gls{combiner} must at every
step adhere to a rule, called the \gls!{machine invariant}, which dictates that
every \gls{RTL} in the \gls{program} must be implementable by a single
\gls{instruction}.

By using a subsequent \gls{peephole optimizer} to combine the effects of
multiple \glspl{RTL}, the \gls{instruction selector} can effectively extend over
multiple \glspl{node} in the \gls{AST} or \gls{expression tree}, potentially
across \gls{block} boundaries.
%
The \gls{instruction} support in \citeauthor{Davidson:1984}'s design is
therefore in theory only restricted by the number of \glspl{instruction} that
the \gls{peephole optimizer} can compare at a time.
%
For example, opportunities to replace three \glspl{instruction} by a single
\gls{instruction} will be missed if the \gls{peephole optimizer} only checks
pair combinations.
%
But increasing the window size typically incurs an exponential cost in terms of
added complexity, thus making it difficult to handle complicated
\glspl{instruction} that require large \gls{instruction} windows.


\subsubsection{Further Improvements}

\textcite{Fraser:1988} later expanded the work by \citeauthor{Davidson:1984}.
%
In a paper from~1988, \citeauthor{Fraser:1988} describe a method where the
\gls{expander} and \gls{combiner} are effectively fused together into a single
component.
%
The idea is to generate the \gls{instruction selector} in two steps.
%
The first step produces a \gls{naive.me} \gls{macro expander} that is capable of
expanding a single \gls{IR} \gls{node} at a time.
%
Unlike \citeauthor{Davidson:1984}, who implemented the \gls{expander} by hand,
\citeauthor{Fraser:1988} applied an elaborate scheme consisting of a series of
switch and goto statements -- effectively implementing a \gls{state machine} --
which allowed their \gls{expander} to be generated automatically from a
\gls{machine description}.
%
Once produced, the \gls{macro expander} is executed on a carefully designed
training set.
%
Using function calls embedded into the \gls{instruction selector}, a
retargetable \gls{peephole optimizer} is executed in tandem which discovers and
gathers statistics on target-specific optimizations that can be done on the
generated \gls{assembly code}.
%
Based on these results, the beneficial optimization decisions are then selected
and incorporated directly into the \gls{macro expander}.
%
This effectively enables the \gls{macro expander} to expand multiple \gls{IR}
\glspl{node} at a time, thus removing the need for a separate \gls{peephole
  optimizer} in the final \gls{compiler}.
%
\citeauthor{Fraser:1988} argued that as the \gls{instruction selector} only
implements the optimization decisions that are deemed to be ``useful,'' the code
quality is improved with minimal overhead.
%
\textcite{Wendt:1990} later improved the technique by providing a more powerful
\gls{machine description} format, also based on \glspl{RTL}, which subsequently
evolved into a compact standalone language used for implementing \glspl{code
  generator} (see \textcite{Fraser:1989}).


\subsubsection{Enforcing the Machine Invariant with a Recognizer}

The \gls{Davidson-Fraser approach} was also recently extended by
\textcite{Dias:2006}.
%
Instead of requiring each separate \gls{RTL}-oriented optimization routine to
abide by the \gls{machine invariant}, \citeauthor{Dias:2006}'s design employs a
\gls!{recognizer} to determine whether an optimization decision violates the
aforementioned restriction (see \refFigure{dias-ramsey}).
%
The idea is that, by doing so, the optimization routines can be simplified and
generated automatically as they no longer need to internalize the \gls{machine
  invariant}.

%\begin{figure}
%  \centering%
%  \input{\figurePath/macro-expansion/dias-ramsey}
%
%  \figCaption[An extension of the \glsentrytext{Davidson-Fraser approach}]%
%    {Overview of \citeauthor{Dias:2006}'s design}[Dias:2006]
%  \labelFigure{dias-ramsey}
%\end{figure}

%\begin{figure}
%  \centering%
%  \begin{minipage}{9cm}%
%    \centering%
%    \lstset{escapechar=|}
%    \begin{plainCode}
%default attribute
%  add(rd, rs1, rs2) is |\$|r[rd] := |\$|rs[rs1] + |\$|r[rs2]
%    \end{plainCode}
%  \end{minipage}
%
%  \figCaption[An \glsentrytext{instruction} expressed in
%      \glsentrytext{lambdaRTL}]
%    {A \glsentrytext{PowerPC} \dataTerm{add} \glsentrytext{instruction}
%      specified using \glsentrytext{lambdaRTL}}[Dias:2010]
%  \labelFigure{lambda-rtl-example}
%\end{figure}

In a paper from~2006, \citeauthor{Dias:2006} demonstrate how the
\gls{recognizer} can be produced from a declarative \gls{machine description}
written in \gls{lambdaRTL}.
%
Originally developed by \textcite{Ramsey:1998}, \gls!{lambdaRTL} is a high-level
functional language based on \gls!{ML} and raises the level of abstraction for
writing \glspl{RTL} (see \refFigure{lambda-rtl-example} for an example).
%
In their paper, \citeauthor{Dias:2006} claim that \gls{lambdaRTL}-based
\glspl{machine description} are more concise and simpler to write compared to
those of many other designs, including \gls{GCC}.
%
In particular, \gls{lambdaRTL} is precise and unambiguous, which makes it
suitable for automated tool generation and verification.
%
The latter has been explored by \textcite{Fernandez:1997} and
\textcite{Bailey:2003}.

The \gls{recognizer} checks whether an \gls{RTL} in the \gls{program} fulfills
the \gls{machine invariant} by performing a syntactic comparison between that
\gls{RTL} and the \glspl{RTL} of the \glspl{instruction}.
%
However, if a given \gls{RTL} in the \gls{program} has $n$~operations, and a
given \gls{lambdaRTL} description contains $m$~\glspl{instruction} whose
\gls{RTL} contains $l$~operations, then a naive implementation would take
$\mBigO(nml)$ time to check a single \gls{RTL}.
%
Instead, using techniques to be discussed in
\refAppendix{tree-covering}, \citeauthor{Dias:2006} automatically generate the
\gls{recognizer} as a \gls{finite state automaton} that can compare a given
\gls{RTL} against all \glspl{RTL} in the \gls{lambdaRTL} description with a
single check.


\subsubsection{``One Program to Expand Them All''}

In 2010, \citeauthor{Dias:2010} introduced a scheme, described in
\cite{Dias:2010} and \cite{Ramsey:2011}, where the \gls{macro expander} only
needs to be implemented once per every distinct \emph{architecture family}
instead of once per every distinct \emph{instruction set}.
%
For example, \gls{register}-based and stack-based machines are two separate
architecture families, whereas \gls{X86}, \gls{PowerPC}, and \gls{Sparc} are
three different \glspl{instruction set}.
%
In other words, if two \glspl{target machine} belong to the same architecture
family, then the same \gls{expander} can be used despite the differing details
in their \glspl{instruction set}.
%
This is useful because the correctness of the \gls{expander} only needs to be
proven once, which is a difficult and time-consuming process if it is written by
hand.

The idea is to have a predefined set of \glspl{tile} that are specific for a
particular architecture family.
%
A \gls!{tile} represents a simple operation which is required for any
\gls{target machine} belonging to that architecture family.
%
For example, stack-based machines require \glspl{tile} for push and pop
operations, which are not necessary on \gls{register}-based machines.
%
Then, instead of expanding each \gls{IR} \gls{node} in the \gls{program} into a
sequence of~\glspl{RTL}, the \gls{expander} expands it into a sequence of
\glspl{tile}.
%
Since the set of \glspl{tile} is identical for all \glspl{target machine} within
the same architecture family, the \gls{expander} only needs to be implemented
once.
%
After \gls{macro expansion} the \glspl{tile} are replaced by the
\glspl{instruction} used to implement each \gls{tile}, and the resulting
\gls{assembly code} can then be improved by the \gls{combiner}.

A remaining problem is how to find \glspl{instruction} to implement a given
\gls{tile} for a particular \gls{target machine}.
%
In the same papers, \citeauthor{Dias:2010} describe a scheme for doing this
automatically.
%
By expressing both the \glspl{tile} and the \glspl{instruction} as
\gls{lambdaRTL}, \citeauthor{Dias:2010} developed a technique where the
\glspl{RTL} of the \glspl{instruction} are combined such that the effects equal
that of a \gls{tile}.
%
In broad outline, the algorithm maintains a pool of \glspl{RTL} which initially
contains those of the \glspl{instruction} found in the \gls{machine
  description}.
%
Using algebraic laws and combining existing \glspl{RTL} to produce new
\glspl{RTL}, the pool is grown iteratively until either all \glspl{tile} have
been implemented, or some termination criterion is reached.
%
The latter is necessary, as \citeauthor{Dias:2010} proved that the general
problem of finding implementations for arbitrary \glspl{tile} is undecidable.

Although the primary aim of \citeauthor{Dias:2010}'s design is to facilitate
\gls{compiler} retargetability, some experiments suggest that it potentially
also yields better code quality than the original \gls{Davidson-Fraser
  approach}.
%
When a prototype was compared against the default \gls{instruction selector} in
\gls{GCC}, the results favored the former.
%
However, this was seen only when all target-independent optimizations were
disabled; when they were reactivated, \gls{GCC} still produced better results.


\subsection{Running Peephole Optimization Before Instruction Selection}

In the techniques just discussed, the \gls{peephole optimizer} runs after
\gls{code generation}.
%
But in a scheme developed in 1989 by \textcite{Genin:1989}, a similar routine is
executed \emph{before} \gls{code generation}.
%
Targeting digital signal processors, their \gls{compiler} first transforms the
\gls{program} into an \gls!{ISFG}, and then executes a routine --
\citeauthor{Genin:1989} called it a \gls!{pattern matcher} -- which attempts to
find several low-level operations in the \gls{ISFG} that can be merged into
single \glspl{node}.\!%
%
\footnote{%
  The paper is not clear on how this is done exactly, but presumably
  \citeauthor{Genin:1989} implemented the routine as a handwritten \gls{peephole
    optimizer} since the intermediate format is fixed and does not change from
  one \gls{target machine} to another.%
}
%
\Gls{code generation} is then done following the conventional \gls{macro
  expansion} approach.
%
For each \gls{node} the \gls{instruction selector} invokes a rule along with the
information about the current context.
%
The invoked rule produces the \gls{assembly code} appropriate for the given
context, and can also insert new \glspl{node} to offload decisions that are
deemed better handled by the rules corresponding to the inserted \glspl{node}.

According to \citeauthor{Genin:1989}, experiments show that their \gls{compiler}
generated \gls{assembly code} that was five to \num{50}~times faster than that
produced by other, contemporary \gls{DSP} \glspl{compiler}, and comparable with
manually optimized \gls{assembly code}.
%
A disadvantage of this design is that it is limited to \glspl{program} where
prior knowledge about the application area, in this case digital signal
processing, can be encoded into specific optimization routines, which most
likely has to be done manually.



\subsection{Interactive Code Generation}

The aforementioned techniques yield \glspl{peephole optimizer} which are static
once they have been generated, meaning they will only recognize and optimize
\gls{assembly code} for a fixed set of \glspl{pattern}.
%
A method to overcome this issue has been designed by \textcite{Kulkarni:2006},
which is also the first and only one to my knowledge.

In a paper from~2006, \citeauthor{Kulkarni:2006} describe a \gls{compiler}
system called \gls{Vista}, which is an interactive compilation environment where
the user is given greater control over the \gls{compiler}.
%
Among other things, the user can alter \glspl{RTL} derived from the
\gls{program}'s source code and add new customized \gls{peephole optimization}
\glspl{pattern}.
%
Hence optimization privileges which normally are limited to low-level
\gls{assembly code} programmers are also granted to higher-level programming
language users.
%
In addition, \citeauthor{Kulkarni:2006} employed genetic algorithms -- these
will be explained in \refAppendix{tree-covering} -- in an attempt to
automatically derive a combination of user-provided optimization guidelines to
improve the code quality of a particular \gls{program}.
%
Experiments show that this scheme reduced code size on average by
\SI{4}{\percent} and up to \SI{12}{\percent} for a selected set of
\glspl{program}.


\section{Summary}

In this appendix we have discussed techniques and designs based on a
\gls{principle} known as \gls{macro expansion}, which was the first approach to
perform \gls{instruction selection}.
%
The idea behind the \gls{principle} is to expand the \glspl{node} in the
\gls{AST} or \gls{IR} code into one or more target-specific \glspl{instruction}.
%
The expansion is done via \gls{template} matching and \gls{macro} invocation,
which yields \glspl{instruction selector} that are resource-effective and
straightforward to implement.

But because \gls{macro}-expanding \glspl{instruction selector} only visit and
execute \glspl{macro} one \gls{IR} \gls{node} at a time, they require a
\mbox{1-to-1} or \mbox{1-to-$n$} mapping between the \gls{IR} \glspl{node} and
the \glspl{instruction} provided by the \gls{target machine} in order to
generate efficient \gls{assembly code}.
%
The limitation can be mitigated by incorporating additional logic and
bookkeeping into the \glspl{macro}, but this quickly becomes an unmanageable
task for the \gls{macro} writer if done manually.
%
Consequently, the code quality yielded by these techniques will typically be
low.
%
Moreover, as \glspl{instruction} are often emitted one at a time, it also
becomes difficult to make use of \glspl{instruction} that can have unintended
effects on other \glspl{instruction}.

A more robust remedy for improving code quality is to append a \gls{peephole
  optimizer} into the component chain of the \gls{backend}.
%
A \gls{peephole optimizer} combines the effects of multiple \glspl{instruction}
in the \gls{assembly code} with more efficient alternatives, thereby amending
some of the poor decisions made by the \gls{instruction selector}.
%
\Gls{peephole optimization} can also be incorporated directly into the
\gls{instruction selector} -- a scheme which has become known as the
\gls{Davidson-Fraser approach} -- and thereby extend its \gls{instruction}
support.
%
Because of this versatility, the \gls{Davidson-Fraser approach} remains one of
the most powerful \gls{instruction selection} techniques to date (a variant is
still applied in \gls{GCC} as of version 4.8.2).

In \refAppendix{tree-covering} we will explore another \gls{principle} of
\gls{instruction selection}, which solves the problem of implementing several
\gls{AST} or \gls{IR} \glspl{node} using a single \gls{instruction} in a more
direct fashion.
