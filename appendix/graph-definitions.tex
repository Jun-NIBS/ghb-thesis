% Copyright (c) 2017, Gabriel Hjort Blindell <ghb@kth.se>
%
% This work is licensed under a Creative Commons 4.0 International License (see
% LICENSE file or visit <http://creativecommons.org/licenses/by/4.0/> for a copy
% of the license).

\chapter{Graph Definitions}
\labelAppendix{graph-definitions}

A \gls!{graph} is defined as a tuple~\mbox{$\mPair{N}{E}$} where $N$ is a set of
\glspl!{node} (also known as \glspl!{vertex}) and $E$ is a set of \glspl!{edge},
each consisting of a pair of \glspl{node}~\mbox{$n, m \in N$}.
%
A \gls{graph} is \gls!{undirected.g} if its \glspl{edge} have no direction, and
\gls!{directed.g} if they do.
%
\Glspl{edge} with a direction are written either $\mPair{n}{m}$ or
$\mEdge{n}{m}$, whichever is most convenient, and \glspl{edge} without are
written $\mSet{n, m}$.
%
In a \gls{directed.g} \gls{graph}, we say that an \gls{edge}~\mbox{$e =
  \mPair{n}{m}$} is \gls!{outgoing.e} (or \gls!{outbound.e}) with respect
to~$n$, and \gls!{ingoing.e} (or \gls!{inbound.e}) with respect to~$m$.
%
We also call $n$ and $m$ the \gls!{source} and \gls!{target}, respectively, of
$e$, and introduce two functions, \mbox{$\mSource \! : E \rightarrow N$} and
\mbox{$\mTarget \! : E \rightarrow N$}, that respectively returns an
\gls{edge}'s \gls{source} and \gls{target}.
%
\Glspl{edge} for which \mbox{$\mSource(e) = \mTarget(e)$} are known as
\glspl!{loop edge} (or simply \glspl!{loop}).
%
If there exists more than one \gls{edge} between the same pair of \glspl{node}
then the \gls{graph} is a \gls!{multigraph}, otherwise it is a \gls!{simple.g}
\gls{graph}.

A sequence of \glspl{edge} that describe how to get from one \gls{node} to
another is called a \gls!{path}.
%
More formally, we define a \gls{path} between two \glspl{node}~$n$ and $m$ in a
\gls{directed.g} \gls{graph}~\mbox{$\mPair{N}{E}$} as an ordered
sequence~\mbox{$p = \mSequence{e_1, \ldots, e_l}$} for which \mbox{$e_i \in E,
  \forall e_i \in p$}, and \mbox{$\mTarget(e_i) = \mSource(e_{i+1}), \forall 1
  \leq i < l-1$}.
%
\Glspl{path} for \gls{undirected.g} \glspl{graph} are similarly defined and
will thus be skipped.
%
A \gls{path} for which \mbox{$\mSource(e_1) = \mTarget(e_l)$} is known as a
\gls!{cycle}.
%
Two \glspl{node}~$n$ and~$m$, where \mbox{$n \neq m$}, are said to be
\gls!{connected.n} if there exists a \gls{path} from $n$ to~$m$, and if the
\gls{path} is of length~1 then $n$ and $m$ are also \gls!{adjacent.n}.
%
A \gls{directed.g} \gls{graph} containing no \glspl{cycle} is known as a
\glsreset{DAG}\gls!{DAG}.
%
An \gls{undirected.g} \gls{graph} is \gls!{connected.g} if and only if there
exists a \gls{path} for every distinct pair of \glspl{node}.
%
For completeness, a \gls{directed.g} \gls{graph} is \gls!{strongly connected.g}
iff, for every pair of distinct \glspl{node}~$n$ and~$m$, there exists a
\gls{path} from $n$ to $m$ and a \gls{path} from $m$ to $n$.
%
Also, a \gls{directed.g} \gls{graph} is \gls!{weakly connected.g} if replacing
all \glspl{edge} with \gls{undirected.g} \glspl{edge} yields a \gls{connected.g}
\gls{undirected.g} \gls{graph}.
%
An example is shown in \refFigure{graph-example}.

\begin{figure}
  \centering%
  % Copyright (c) 2018, Gabriel Hjort Blindell <ghb@kth.se>
%
% This work is licensed under a Creative Commons 4.0 International License (see
% LICENSE file or visit <http://creativecommons.org/licenses/by/4.0/> for a copy
% of the license).
%
\begingroup%
\figureFont\figureFontSize%
\setlength{\nodeDist}{20pt}
\setlength{\nodeSize}{16pt}
\begin{tikzpicture}[
    node/.append style={
      minimum size=0.8\nodeSize,
      node distance=0.8\nodeDist,
    },
    edge/.style={
      data flow,
    },
    mapping/.style={
      data flow,
      dashed,
      shorten >=4pt,
      shorten <=6pt,
    },
    graph label/.style={
      nothing,
      node distance=1em,
    },
  ]

  % Graph m
  \node [node] (m1) {$m_1$};
  \node [node, below=of m1] (m2) {$m_2$};
  \node [node, position=-120 degrees from m2] (m3) {$m_3$};
  \begin{scope}[edge]
    \draw (m1) -- coordinate (m12) (m2);
    \draw (m2) -- coordinate (m23) (m3);
  \end{scope}
  \node [graph label, above=of m1] {$G_m$};

  % Graph n
  \node [node, right=3\nodeDist of m1] (n1) {$n_1$};
  \node [node, below=of n1] (n2) {$n_2$};
  \node [node, position=-120 degrees from n2] (n3) {$n_3$};
  \node [node, position=-60 degrees from n2] (n4) {$n_4$};
  \begin{scope}[edge]
    \draw (n1) -- coordinate (n12) (n2);
    \draw (n2) -- coordinate (n23) (n3);
    \draw (n3) -- (n4);
    \draw (n4) -- (n2);
  \end{scope}
  \node [graph label, above=of n1] {$G_n$};

  % Mappings
  \foreach \i in {1, ..., 3} {%
    \draw [mapping] (m\i) -- node [above] {$f$} (n\i);
  }
\end{tikzpicture}%
\endgroup%


  \caption[Example of two simple directed graphs]%
          {%
            An example of two simple, directed graphs~$G_m$ and~$G_n$.
            %
            Through the graph homomorphism~$f$ we see that $G_m$ is an
            isomorphic subgraph of~$G_n$.
            %
            Both graphs are weakly connected, and $G_n$ also has a strongly
            connected subgraph, consisting of $n_2$, $n_3$, and~$n_4$ which form
            a cycle%
          }%
          \labelFigure{graph-example}%
\end{figure}

A \gls{simple.g}, \gls{undirected.g} \gls{graph} that is \gls{connected.g},
contains no \glspl{cycle}, and has exactly one \gls{path} between any two
\glspl{node}, is called a \gls!{tree}.
%
A set of \glspl{tree} constitutes a \gls!{forest}.
%
\Glspl{node} in a \gls{tree} with exactly one neighbor are known as
\glspl!{leaf}.
%
A \gls!{directed.t}[ \gls{tree}] is a \gls{directed.g} \gls{graph} that would
become a \gls{tree} when ignoring the direction of its \glspl{edge}.
%
A \gls!{rooted directed.t}[ \gls{tree}] is a \gls{directed.t} \gls{tree} where
one \gls{node} has been assigned as \gls!{root} and all \glspl{edge} either
point away or towards the \gls{root}.
%
In a \gls{rooted directed.t} \gls{tree}, a \gls!{parent} of a \gls{node}~$n$ is
the \gls{node} \gls{adjacent.n} to $n$ that is closest to the \gls{root}.
%
Likewise, if a node~$n$ is the parent of another node~$m$, then $m$ is a
\gls!{child} of $n$.
%
In this dissertation, we assume all \glspl{tree} to be \gls{rooted directed.t}
\glspl{tree}, and a \gls{tree} will always be drawn with its \gls{root}
appearing at the top.

A \gls{graph}~\mbox{$G = \mPair{N}{E}$} is a \gls!{subgraph} of another
\gls{graph}~\mbox{$G' = \mPair{N'}{E'}$}, written \mbox{$G \subseteq G'$}, iff
\mbox{$N \subseteq N'$} and \mbox{$E \subseteq E'$}.
%
Likewise, a \gls{tree}~$T$ is a \gls!{subtree} of another \gls{tree}~$T'$ iff
\mbox{$T \subseteq T'$}.

A \gls!{graph homomorphism} is a mapping between two \glspl{graph} such that
their structure is preserved.
%
More formally, a \gls{graph homomorphism}~$f$ from a \gls{graph}~\mbox{$G =
  \mPair{N}{E}$} to another \gls{graph} ~\mbox{$G' = \mPair{N'}{E'}$} is a
mapping $f : N \rightarrow N'$ such that \mbox{$\mSet{u, v} \in E$} implies
\mbox{$\mSet{f(u), f(v)} \in E'$}.
%
If the \gls{graph homomorphism}~$f$ is an injective function, then $f$ is also
called a \gls!{subgraph isomorphism}.
%
If there exists such a mapping then we say that $G$ is an
\glsshort!{isomorphism}[ \gls{subgraph}] of~$G'$, and an example of this is
given in \refFigure{graph-example}.
%
If $f$ is a bijection, whose inverse function is also a \gls{graph
  homomorphism}, then $f$ is called a \gls!{graph isomorphism}.

Lastly we introduce the notion of \gls!{topological sort}, where the
\glspl{node} of a \gls{graph}~\mbox{$\mPair{N}{E}$} are arranged in an ordered
sequence~\mbox{$\mSequence{n_1, \ldots, n_n}$} such that \mbox{$n_i \in N,
  \forall 1 \leq i \leq n$}, and for no pair of \glspl{node}~$n_i$ and $n_j$,
where \mbox{$i < j$}, does there exist an \gls{edge} \mbox{$\mPair{n_j}{n_i} \in
  E$}.
%
In other words, if the \glspl{edge} are added to the list then all \glspl{edge}
will go point forward from left to right (hence \glspl{topological sort} are
only defined for \glspl{DAG}).
%
Several methods exists for achieving a \gls{topological sort}, see for example
in \textcite[Sect.\thinspace22.4]{CormenEtAl:2009}.
