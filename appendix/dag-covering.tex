% Copyright (c) 2017, Gabriel Hjort Blindell <ghb@kth.se>
%
% This work is licensed under a Creative Commons 4.0 International License (see
% LICENSE file or visit <http://creativecommons.org/licenses/by/4.0/> for a copy
% of the license).

\chapter{DAG Covering}
\labelAppendix{dag-covering}

\todo{write introduction}

As we saw in \refAppendix{tree-covering}, the \gls{principle} of \gls{tree
  covering} has two significant disadvantages.
%
The first is that common subexpressions cannot be properly expressed in
\glspl{expression tree}, and the second is that many \glspl{instruction
  characteristic} -- such as \gls{multi-output.ic} \glspl{instruction} -- cannot
be modeled as \glspl{pattern tree}.
%
As these shortcomings are primarily due to the restricted use of \glspl{tree},
we can achieve a more powerful approach to \gls{instruction selection} by
extending \gls{tree covering} to \gls{DAG covering}.

This appendix is based on material presented in
\cite[Chap.\thinspace4]{HjortBlindell:2016:Survey} that has been adapted for
this dissertation.


\section{The Principle}

\todo{fix inline figure}

%\def\inparfigwidth{2.5cm}
%\begin{inParFigure}{\inparfigwidth}[r]%
%  \centering%
%  \input{\figurePath/dag-covering/dag-covering-example}%
%
%  % LAYOUT FIX:
%  % Add some space below
%  \vspace*{\baselineskip}
%\end{inParFigure}%
%\noindent
%
By lifting the restriction that every \gls{node} in the \gls{expression tree}
have exactly one \gls{parent}, we attain a \gls!{block DAG}.
%
Because \glspl{DAG} permit \glspl{node} to have multiple \glspl{parent}, the
intermediate values in an expression can be shared and reused within the same
\gls{block DAG}.
%
This also enables \glspl{pattern DAG} to contain multiple \gls{root}
\glspl{node}, which signify the production of multiple output values and thus
extend the \gls{instruction} support to include \gls{multi-output.ic}
\glspl{instruction}.

%\begin{inParFigure}{\inparfigwidth}[r]%
%  \vspace*{2\baselineskip}
%\end{inParFigure}%
%
Since \glspl{DAG} are less restrictive compared to \glspl{tree}, transitioning
from \gls{tree covering} to \gls{DAG covering} requires new methods for solving
the problems of \gls{pattern matching} and \gls{pattern selection}.
%
\Gls{pattern matching} is typically addressed using one of the following
methods:
\begin{itemize}
  \item First split the \glspl{pattern DAG} into \glspl{tree}, then match these
    individually, and then recombine the matched \glspl{pattern tree} into their
    original \gls{DAG} form.
    %
    In general, matching \glspl{tree} on \glspl{DAG} is
    NP-complete~\cite{GareyJohnson:1979}, but designs applying this technique typically
    sacrifice completeness to retain linear time complexity.
  \item Match the \glspl{pattern DAG} directly using a generic \gls{subgraph
    isomorphism} algorithm.
    %
    Although such algorithms exhibit exponential worst-case time complexity, in
    the average case they often finish in polynomial time and are therefore used
    by several \gls{DAG covering}-based designs discussed in this appendix.
\end{itemize}

\Gls{optimal.ps} \gls{pattern selection} on \glspl{block DAG}, however, does not
offer the same range of choices in terms of complexity.


\section{Optimal Pattern Selection on DAGs Is NP-Complete}

The cost of the gain in generality and modeling capabilities that \glspl{DAG}
give us is a substantial increase in complexity.
%
As we saw in \refAppendix{tree-covering}, selecting an \gls{optimal.ps} set of
\glspl{pattern} to cover a \gls{expression tree} can be done in linear time, but
doing the same for \glspl{block DAG} is an NP-complete problem.
%
Proofs were given in 1976 by \textcite{BrunoSethi:1976} and
\textcite{AhoEtal:1976}, but these were most concerned with the optimality of
\gls{instruction scheduling} and \gls{register allocation}.
%
In 1995, \textcite{Proebsting:1995:Proof} gave a very concise proof for optimal
\gls{pattern selection}, and a longer, more detailed proof was given by
\textcite{KoesGoldstein:2008} in~2008.
%
In this dissertation, we will paraphrase the longer proof.


\subsection{The Proof}

The idea behind the proof is to transform the \gls{SAT} problem into an
\gls{optimal.ps} -- that is, \gls{least-cost.c} -- \gls{DAG covering} problem.
%
The \gls{SAT} problem is the task of deciding whether a Boolean formula, written
in \gls!{CNF}, can be satisfied.
%
A \gls{CNF} formula is an expression consisting of Boolean variables and the
Boolean operations $\mOr$ (or) and $\mAnd$ (and) with the following structure:
%
\begin{displaymath}
  (x_{1,1} \mOr x_{1,2} \mOr \ldots) \mAnd
  (x_{\,2,1} \mOr x_{\,2,2} \mOr \ldots) \mAnd
  \ldots
\end{displaymath}
%
A variable~$x$ can also be negated, written as \mbox{$\mNot x$}.

Since the \gls{SAT} problem is NP-complete, all polynomial-time transformations
from \gls{SAT} to any other problem $P$ must also render $P$ NP-complete.


\subsubsection{Modeling SAT as a Covering Problem}

First, we transform an instance~$S$ of the \gls{SAT} problem into a \gls{block
  DAG}.
%
The goal is then to find an exact cover for the \gls{DAG} in order to deduce the
truth assignment for the Boolean variables from the set of selected
\glspl{pattern}.
%
For this purpose we will use $\mOr$, $\mAnd$, $\mNot$, $v$, $\mBox$, and
$\mStop$ as \gls{node} types, and define $\mType(n)$ as the type of a
\gls{node}~$n$.
%
\Glspl{node} of type~$\mBox$ and $\mStop$ will be referred to as \glspl!{box
  node} and \glspl!{stop node}, respectively.
%
Now, for each Boolean variable~\mbox{$x \in S$} we create two \glspl{node}~$n_1$
and $n_2$ such that \mbox{$\mType(n_1) = v$} and \mbox{$\mType(n_2) = \mBox$},
and add these to the \gls{block DAG}.
%
At the same time we also add an \gls{edge}~$\mEdge{n_1}{n_2}$.
%
The same is done for each binary Boolean operator~\mbox{$\mathit{op} \in S$} by
creating two \glspl{node}~$n'_1$ and $n'_2$ such that \mbox{$\mType{n'_1} = op$}
and \mbox{$\mType{n'_2} = \mBox$}, along with an
\gls{edge}~$\mEdge{n'_1}{n'_2}$.
%
To model the connection between the $\mathit{op}$ operation and its two input
operands~$x$ and $y$, we add two \glspl{edge}~$\mEdge{n_x}{n'_1}$ and
$\mEdge{n_y}{n'_1}$ such that \mbox{$\mType{n_x} = \mType{n_y} = \mBox$}.
%
For the unary operation~$\neg$ we obviously only need one such \gls{edge}, and
since $\mOr$ and $\mAnd$ are commutative it does not matter in what order the
\glspl{edge} are arranged with respect to the operator \gls{node}.
%
Hence, in the resulting \gls{block DAG}, only \glspl{box node} will have more
than one \gls{outgoing.e} \gls{edge}.
%
An example of such a \gls{DAG} is shown in \refFigure{sat-example}, which can be
constructed in linear time simply by traversing the Boolean formula.


\subsubsection{Boolean Operations as Patterns}

%\begin{figure}[b]
%
%  \begin{subfigure}[The SAT patterns.
%%
%For brevity, the patterns for the
%                    $\mAnd$~operation are omitted (but these can be easily
%                    inferred from the $\mOr$~patterns) All patterns are
%                    assumed to have the same unit cost]%
%                   [sat-patterns]
%    \begin{minipage}{8cm}
%      % LAYOUT FIX:
%      % Make this subfigure align properly with the next subfigure
%      \vspace*{20pt}
%
%      \centering%
%      \begin{minipage}[t]{1cm}
%        \centering%
%        \input{\figurePath/dag-covering/sat-pattern-v-F}
%
%        $x: F$
%      \end{minipage}%
%      \hfill%
%      \begin{minipage}[t]{1cm}
%        \centering%
%        \input{\figurePath/dag-covering/sat-pattern-v-T}
%
%        $x: T$
%      \end{minipage}%
%      \hfill%
%      \begin{minipage}[t]{1.5cm}
%        \centering%
%        % Copyright (c) 2017-2018, Gabriel Hjort Blindell <ghb@kth.se>
%
% This work is licensed under a Creative Commons Attribution-NoDerivatives 4.0
% International License (see LICENSE file or visit
% <http://creativecommons.org/licenses/by-nc-nd/4.0/> for details).
%
\begingroup%
% Copyright (c) 2017-2018, Gabriel Hjort Blindell <ghb@kth.se>
%
% This work is licensed under a Creative Commons Attribution-NoDerivatives 4.0
% International License (see LICENSE file or visit
% <http://creativecommons.org/licenses/by-nc-nd/4.0/> for details).
%
\figureFont\figureFontSize%
\setlength{\nodeDist}{14pt}%
\setlength{\nodeSize}{12pt}%
\tikzset{%
  var node/.style={%
    node,
  },
  op node/.style={%
    node,
  },
  box node/.style={%
    value node,
  },
  stop node/.style={%
    node,
    regular polygon,
    regular polygon sides=6,
  },
  anchor node/.style={%
    node,
    minimum size=0.3\nodeSize,
    fill=black,
  },
}%
%
\begin{tikzpicture}
  \node [stop node] (stop) {};
  \node [box node, below=of stop] (box) {};
  \node [anchor node, below=of box] (anchor) {};
  \begin{scope}[data flow]
    \draw (anchor) -- (box);
    \draw (box) -- (stop);
  \end{scope}
\end{tikzpicture}%
\endgroup%

%
%        \textit{satisfied}
%      \end{minipage}%
%      \hfill%
%      \begin{minipage}[t]{1cm}
%        \centering%
%        % Copyright (c) 2017-2018, Gabriel Hjort Blindell <ghb@kth.se>
%
% This work is licensed under a Creative Commons Attribution-NoDerivatives 4.0
% International License (see LICENSE file or visit
% <http://creativecommons.org/licenses/by-nc-nd/4.0/> for details).
%
\begingroup%
% Copyright (c) 2017-2018, Gabriel Hjort Blindell <ghb@kth.se>
%
% This work is licensed under a Creative Commons Attribution-NoDerivatives 4.0
% International License (see LICENSE file or visit
% <http://creativecommons.org/licenses/by-nc-nd/4.0/> for details).
%
\figureFont\figureFontSize%
\setlength{\nodeDist}{14pt}%
\setlength{\nodeSize}{12pt}%
\tikzset{%
  var node/.style={%
    node,
  },
  op node/.style={%
    node,
  },
  box node/.style={%
    value node,
  },
  stop node/.style={%
    node,
    regular polygon,
    regular polygon sides=6,
  },
  anchor node/.style={%
    node,
    minimum size=0.3\nodeSize,
    fill=black,
  },
}%
%
\begin{tikzpicture}
  \node [box node] (not box) {};
  \node [op node, below=of not box] (not) {$\mNot$};
  \node [box node, below=of not] (box) {};
  \node [anchor node, below=of box] (anchor) {};
  \begin{scope}[data flow]
    \draw (anchor) -- (box);
    \draw (box) -- (not);
    \draw (not) -- (not box);
  \end{scope}
\end{tikzpicture}%
\endgroup%

%
%        $F: \mNot T$
%      \end{minipage}%
%      \hfill%
%      \begin{minipage}[t]{1cm}
%        \centering%
%        % Copyright (c) 2018, Gabriel Hjort Blindell <ghb@kth.se>
%
% This work is licensed under a Creative Commons 4.0 International License (see
% LICENSE file or visit <http://creativecommons.org/licenses/by/4.0/> for a copy
% of the license).
%
\begingroup%
% Copyright (c) 2017-2018, Gabriel Hjort Blindell <ghb@kth.se>
%
% This work is licensed under a Creative Commons Attribution-NoDerivatives 4.0
% International License (see LICENSE file or visit
% <http://creativecommons.org/licenses/by-nc-nd/4.0/> for details).
%
\figureFont\figureFontSize%
\setlength{\nodeDist}{14pt}%
\setlength{\nodeSize}{12pt}%
\tikzset{%
  var node/.style={%
    node,
  },
  op node/.style={%
    node,
  },
  box node/.style={%
    value node,
  },
  stop node/.style={%
    node,
    regular polygon,
    regular polygon sides=6,
  },
  anchor node/.style={%
    node,
    minimum size=0.3\nodeSize,
    fill=black,
  },
}%
%
\begin{tikzpicture}
  \node [op node] (not) {$\mNot$};
  \node [anchor node, below=of not] (anchor) {};
  \begin{scope}[data flow]
    \draw (anchor) -- (not);
  \end{scope}
\end{tikzpicture}%
\endgroup%

%
%        $T: \mNot F$
%      \end{minipage}
%
%      \vspace{14pt}
%
%      \begin{minipage}[t]{2cm}
%        \centering%
%        % Copyright (c) 2018, Gabriel Hjort Blindell <ghb@kth.se>
%
% This work is licensed under a Creative Commons 4.0 International License (see
% LICENSE file or visit <http://creativecommons.org/licenses/by/4.0/> for a copy
% of the license).
%
\begingroup%
% Copyright (c) 2017-2018, Gabriel Hjort Blindell <ghb@kth.se>
%
% This work is licensed under a Creative Commons Attribution-NoDerivatives 4.0
% International License (see LICENSE file or visit
% <http://creativecommons.org/licenses/by-nc-nd/4.0/> for details).
%
\figureFont\figureFontSize%
\setlength{\nodeDist}{14pt}%
\setlength{\nodeSize}{12pt}%
\tikzset{%
  var node/.style={%
    node,
  },
  op node/.style={%
    node,
  },
  box node/.style={%
    value node,
  },
  stop node/.style={%
    node,
    regular polygon,
    regular polygon sides=6,
  },
  anchor node/.style={%
    node,
    minimum size=0.3\nodeSize,
    fill=black,
  },
}%
%
\begin{tikzpicture}
  \node [op node] (or) {$\mOr$};
  \node [box node, position=-135 degrees from or] (box1) {};
  \node [box node, position=-45  degrees from or] (box2) {};
  \node [anchor node, below=of box1] (anchor1) {};
  \node [anchor node, below=of box2] (anchor2) {};
  \begin{scope}[data flow]
    \draw (anchor1) -- (box1);
    \draw (anchor2) -- (box2);
    \draw (box1) -- (or);
    \draw (box2) -- (or);
  \end{scope}
\end{tikzpicture}%
\endgroup%

%
%        $T: T \mOr T$
%      \end{minipage}%
%      \hfill%
%      \begin{minipage}[t]{2cm}
%        \centering%
%        % Copyright (c) 2018, Gabriel Hjort Blindell <ghb@kth.se>
%
% This work is licensed under a Creative Commons 4.0 International License (see
% LICENSE file or visit <http://creativecommons.org/licenses/by/4.0/> for a copy
% of the license).
%
\begingroup%
% Copyright (c) 2017-2018, Gabriel Hjort Blindell <ghb@kth.se>
%
% This work is licensed under a Creative Commons Attribution-NoDerivatives 4.0
% International License (see LICENSE file or visit
% <http://creativecommons.org/licenses/by-nc-nd/4.0/> for details).
%
\figureFont\figureFontSize%
\setlength{\nodeDist}{14pt}%
\setlength{\nodeSize}{12pt}%
\tikzset{%
  var node/.style={%
    node,
  },
  op node/.style={%
    node,
  },
  box node/.style={%
    value node,
  },
  stop node/.style={%
    node,
    regular polygon,
    regular polygon sides=6,
  },
  anchor node/.style={%
    node,
    minimum size=0.3\nodeSize,
    fill=black,
  },
}%
%
\begin{tikzpicture}
  \node [op node] (or) {$\mOr$};
  \node [box node, position=-133 degrees from or] (box) {};
  \node [anchor node, position=-45 degrees from or] (anchor1) {};
  \node [anchor node, below=of box] (anchor2) {};
  \begin{scope}[data flow]
    \draw (anchor1) -- (or);
    \draw (anchor2) -- (box);
    \draw (box) -- (or);
  \end{scope}
\end{tikzpicture}%
\endgroup%

%
%        $T: T \mOr F$
%      \end{minipage}%
%      \hfill%
%      \begin{minipage}[t]{2cm}
%        \centering%
%        % Copyright (c) 2018, Gabriel Hjort Blindell <ghb@kth.se>
%
% This work is licensed under a Creative Commons 4.0 International License (see
% LICENSE file or visit <http://creativecommons.org/licenses/by/4.0/> for a copy
% of the license).
%
\begingroup%
% Copyright (c) 2017-2018, Gabriel Hjort Blindell <ghb@kth.se>
%
% This work is licensed under a Creative Commons Attribution-NoDerivatives 4.0
% International License (see LICENSE file or visit
% <http://creativecommons.org/licenses/by-nc-nd/4.0/> for details).
%
\figureFont\figureFontSize%
\setlength{\nodeDist}{14pt}%
\setlength{\nodeSize}{12pt}%
\tikzset{%
  var node/.style={%
    node,
  },
  op node/.style={%
    node,
  },
  box node/.style={%
    value node,
  },
  stop node/.style={%
    node,
    regular polygon,
    regular polygon sides=6,
  },
  anchor node/.style={%
    node,
    minimum size=0.3\nodeSize,
    fill=black,
  },
}%
%
\begin{tikzpicture}
  \node [op node] (or) {$\mOr$};
  \node [anchor node, position=-130  degrees from or] (anchor1) {};
  \node [box node, position=-50 degrees from or] (box) {};
  \node [anchor node, below=of box] (anchor2) {};
  \begin{scope}[data flow]
    \draw (anchor1) -- (or);
    \draw (anchor2) -- (box);
    \draw (box) -- (or);
  \end{scope}
\end{tikzpicture}%
\endgroup%

%
%        $T: F \mOr T$
%      \end{minipage}%
%      \hfill%
%      \begin{minipage}[t]{2cm}
%        \centering%
%        % Copyright (c) 2017-2018, Gabriel Hjort Blindell <ghb@kth.se>
%
% This work is licensed under a Creative Commons Attribution-NoDerivatives 4.0
% International License (see LICENSE file or visit
% <http://creativecommons.org/licenses/by-nc-nd/4.0/> for details).
%
\begingroup%
% Copyright (c) 2017-2018, Gabriel Hjort Blindell <ghb@kth.se>
%
% This work is licensed under a Creative Commons Attribution-NoDerivatives 4.0
% International License (see LICENSE file or visit
% <http://creativecommons.org/licenses/by-nc-nd/4.0/> for details).
%
\figureFont\figureFontSize%
\setlength{\nodeDist}{14pt}%
\setlength{\nodeSize}{12pt}%
\tikzset{%
  var node/.style={%
    node,
  },
  op node/.style={%
    node,
  },
  box node/.style={%
    value node,
  },
  stop node/.style={%
    node,
    regular polygon,
    regular polygon sides=6,
  },
  anchor node/.style={%
    node,
    minimum size=0.3\nodeSize,
    fill=black,
  },
}%
%
\begin{tikzpicture}
  \node [box node] (or box) {};
  \node [op node, below=of or box] (or) {$\mOr$};
  \node [anchor node, position=-135  degrees from or] (anchor1) {};
  \node [anchor node, position=-45 degrees from or] (anchor2) {};
  \begin{scope}[data flow]
    \draw (anchor1) -- (or);
    \draw (anchor2) -- (or);
    \draw (or) -- (or box);
  \end{scope}
\end{tikzpicture}%
\endgroup%

%
%        $F: F \mOr F$
%      \end{minipage}%
%
%    \end{minipage}%
%  \end{subfigure}
%  \hfill%
%  \begin{subfigure}[Example of a SAT problem represented as a DAG
%                    covering problem][sat-example]
%    \begin{minipage}{3cm}
%      \centering%
%      \input{\figurePath/dag-covering/sat-example}
%
%      $(x_1 \mOr x_2) \mAnd (\mNot x_2)$
%    \end{minipage}
%  \end{subfigure}
%
%  \figCaption{Transforming SAT to DAG covering}[KoesGoldstein:2008]
%  \labelFigure{sat-to-instruction-selection-reduction}
%\end{figure}

\def\mPSat{P_\text{\smaller SAT}}%

To cover the \gls{block DAG}, we will use the \glspl{pattern tree} given in
\refFigure{sat-patterns}, and we will refer to this \gls{pattern set}
as~$\mPSat$.
%
Every \gls{pattern} in $\mPSat$ adheres to the following invariant:
%
\begin{enumerate}
  \item If a variable~$x$ is set to~$T$~(true), then the selected \gls{pattern}
    covering the $x$~\gls{node} will also cover the corresponding \gls{box node}
    of~$x$.
  \item If the result of an operation $\mathit{op}$ evaluates to~$F$~(false),
    then that \gls{pattern} will not cover the corresponding \gls{box node}
    of~$\mathit{op}$.
\end{enumerate}
%
Another way of looking at it is that an operator in a \gls{pattern}
\emph{consumes} a \gls{box node} if its corresponding value must be set to~$T$,
and \emph{produces} a \gls{box node} if the result must evaluate to~$F$.
%
Using this scheme, we can easily deduce the truth assignments to the variables
by inspecting whether the \glspl{pattern} selected to cover the \gls{DAG}
consume the \glspl{box node} of the variables.
%
Since the only \gls{pattern} to contain a \gls{stop node} also consumes a
\gls{box node}, the entire expression will be forced to evaluate to~$T$.

In addition to the \gls{node} types that can appear in the \gls{block DAG}, the
\glspl{pattern} can also contain \glspl{node} of an additional type, $\mAnchor$,
which we will refer to as \glspl!{anchor node}.
%
Let $\mNumChildren(n)$ denote the number of \glspl{child} of~$n$, and
$\mChild(i, n)$ the $i$th child of~$n$.
%
We now say that a \gls{pattern}~$p$, with \gls{root} \gls{node}~$p_r$,
\emph{matches} the part of a \gls{block DAG}~\mbox{$\mTuple{N, E}$} which is
rooted at a \gls{node}~\mbox{$n \in N$} if and only if:
%
\begin{enumerate}
  \item $\mType(n) = \mType{p_r}$,
  \item $\mNumChildren(n) = \mNumChildren(p_r)$, and
  \item $\forall 1 \leq i \leq \mNumChildren(n) : \mType{\mChild(i, n)} =
    \mAnchor \mOr \text{$\mChild(i, n)$ matches $\mChild(i, p_r)$}$.
\end{enumerate}
%
In other words, the structure of the \gls{pattern tree} -- which includes the
\gls{node} types and \glspl{edge} -- must correspond to the structure of the
matched \gls{subgraph}, with the exception of \glspl{anchor node}, which match
any \gls{node} in the \gls{block DAG}.

We introduce several new definitions.
%
Given a \gls{block DAG}~\mbox{$G = \mTuple{N, E}$}, let $\mMatchSet[n]$ be the
set of \glspl{pattern} in $\mPSat$ that match at \gls{node}~\mbox{$n \in N$}.
%
Also, given a \gls{pattern}~\mbox{$\mTuple{N_p, E_p}$}, let \mbox{$\mMatched(p,
  n_p)$} be the set of \glspl{node} in $N$ that are matched by a \gls{node}~$n_p
\in N_p$.
%
Lastly, we say that $G$ is \emph{covered} by a function~\mbox{$f : N \rightarrow
  2^{\mPSat}$}, which maps \glspl{node} in the \gls{block DAG} to a set of
\glspl{pattern}, if and only if, for each \mbox{$n \in N$},
%
\begin{enumerate}
  \item $\forall p \in f(n) : \text{$p$ matches $n$}$,
  \item $\mType(n) = \mStop \implies f(n) \neq \emptyset$, and
  \item $\forall p = \mTuple{N_p, E_p} \in f(v), n_p \in N_p :
    \mType{n_p} = \mAnchor \implies f(\mMatched(n, n_p)) \neq
    \emptyset$.
\end{enumerate}
%
The first \gls{constraint} enforces that only valid \glspl{match} are selected.
%
The second \gls{constraint} enforces that some \gls{match} has been selected to
cover the \gls{stop node}.
%
The third \gls{constraint} enforces that \glspl{match} have been selected to
cover the rest of the \gls{DAG}.
%
An \gls{optimal.ps} cover is thus a mapping~$f$ which covers the \gls{block
  DAG}~\mbox{$\mTuple{N, E}$} and also minimize
%
\begin{displaymath}
  \sum_{n \, \in \, N} \sum_{p \, \in \, f(n)} \mCost{p},
\end{displaymath}
%
where $\mCost{p}$ is the cost of \gls{pattern}~$p$.


\subsubsection{Optimal Solution to DAG Covering $\Rightarrow$ Solution to SAT}

We now postulate that if the \gls{optimal.ps} cover has a total cost equal to
the number of non-\glspl{box node} in the~\gls{block DAG}, then the
corresponding \gls{SAT} problem is satisfiable.
%
Since all \glspl{pattern} in $\mPSat$ cover exactly one non-\gls{box node} and
have equal unit cost, the condition above means that every \gls{node} in the
\gls{DAG} is covered by exactly one \gls{pattern}.
%
This in turn means that exactly one value will be assumed for every Boolean
variable and operator result, which is easy to deduce through inspection of the
selected \glspl{match}.

We have thereby shown that an instance of the \gls{SAT} problem can be solved by
transforming it, in polynomial time, to an instance of the \gls{optimal.ps}
\gls{DAG covering} problem.
%
Hence \gls{optimal.ps} \gls{DAG covering} -- and therefore also optimal
\gls{instruction selection} based on \gls{DAG covering} -- is NP-complete.
%
\hfill\qedsymbol


\section{Straightforward, Greedy Techniques}

Since \gls{instruction selection} on \glspl{DAG} with \gls{optimal.ps}
\gls{pattern selection} is computationally difficult, most \glspl{instruction
  selector} based on this \gls{principle} are suboptimal.
%
One of the first \glspl{code generator} to operate on \glspl{DAG} was developed
by \textcite{AhoEtAl:1976}.
%
In a paper from~1976, \citeauthor{AhoEtAl:1976} introduce some simple greedy
heuristics for producing \gls{assembly code} for a commutative
one-\gls{register} \gls{target machine}, but these methods assume a one-to-one
mapping between the \glspl{node} in a \gls{block DAG} and the
\glspl{instruction} and thus effectively ignore the \gls{instruction selection}
problem.


\subsection{LLVM}
\labelSection{llvm}

A more flexible, but still greedy, heuristic is applied in the well-known
\gls{LLVM} \gls{compiler} infrastructure~\cite{LattnerAdve:2004}.
%
According to a blog entry by \textcite{Bendersky:2013} -- which at the time of
writing provides the only documentation, except for the source code itself --
the \gls{instruction selector} is basically a greedy
\mbox{\gls{DAG}-to-\gls{DAG}} rewriter.\!%
%
\footnote{%
  \gls{LLVM} is also equipped with a ``fast'' \gls{instruction selector}, but it
  is implemented as a typical \gls{macro expander} and is only intended to be
  used when compiling without extensive \gls{program} optimization.%
}

The \glspl{pattern} -- which are limited to \glspl{tree} -- are expressed in a
\gls{machine description} that allows common features to be factored out into
abstract \glspl{instruction}.
%
A tool called \gls!{Tablegen} expands the abstract \glspl{instruction} into
\glspl{pattern tree}, which are then processed by a matcher generator.
%
To ensure a partial order among all \glspl{pattern}, the matcher generator first
performs a lexicographical sort on the \gls{pattern set}: first by decreasing
complexity, which is the sum of the \gls{pattern}'s size and a constant that can
be tweaked to give higher priority for particular \glspl{instruction}; then by
increasing cost; and lastly by increasing the size of the \gls{subgraph} that
replaces the covered part in the \gls{block DAG} if the corresponding
\gls{pattern} is selected.
%
Once sorted, the \glspl{pattern} are converted into small recursive
\glspl{program} which essentially check whether the corresponding \gls{pattern}
matches at a given \gls{node} in the \gls{block DAG}.
%
These \glspl{program} are then compiled into a form of byte code and assembled
into a matcher table, arranged such that the lexicographical sort is preserved.
%
The \gls{instruction selector} applies this table by simply executing the byte
code, starting with the first element.
%
When a \gls{match} is found, the \gls{pattern} is greedily selected and the
matched \gls{subgraph} is replaced with the output (usually a single \gls{node})
of the selected \gls{pattern}.
%
This process repeats until there are no \glspl{node} remaining in the original
\gls{block DAG}.

Although in extensive use (as of version 3.4), \gls{LLVM}'s \gls{instruction
  selector} has several drawbacks.
%
The main disadvantage is that any \gls{pattern} that is not supported by
\gls{Tablegen} has to be handled manually through custom \gls{C}~functions.
%
Unlike \gls{GCC} -- which applies \gls{macro expansion} combined with
\gls{peephole optimization} (see
\refSection{macro-expansion-with-peephole-optimization}) -- this includes all
\gls{multi-output.ic} \glspl{instruction}, since \gls{LLVM} is restricted to
\glspl{pattern tree} only.
%
In addition, the greedy scheme compromises code quality.


\section{Techniques Based on Exhaustive Search}

Although \gls{optimal.ps} \gls{pattern selection} can be achieved through
exhaustive search, in practice this is typically infeasible due to the
exponential number of possible combinations.
%
Nonetheless, there do exist a few techniques that do exactly this, but they
apply various techniques to prune the search space.


\subsection{Extending Means-End Analysis to DAGs}

Twenty years after \citeauthor{Newcomer:1975} and \citeauthor{CattellEtAl:1979}
(see \refSection{tree-covering-first-approaches}),
\citeauthor{YuHu:1994a}~\cite{YuHu:1994a, YuHu:1994b} rediscovered
\gls{means-end analysis} as a method for \gls{instruction selection} and also
made two major improvements.
%
First, \citeauthor{YuHu:1994a}'s design supports \glsshort{block DAG} and
\glspl{pattern DAG} whereas those by~\citeauthor{Newcomer:1975} and Cattell~\etal
are both limited to \glspl{tree}.
%
Second, it combines \gls{means-end analysis} with \gls{hierarchical
  planning}~\cite{Sacerdoti:1973}, which is a \gls{search} strategy that relies
on the fact that many problems can be arranged in a hierarchical manner for
handling larger and more complex problem instances.
%
Using hierarchical planning enables exhaustive exploration of the search space
while at the same time avoiding the situations of dead ends and infinite looping
that may occur in straightforward implementations of \gls{means-end analysis}
(\citeauthor{Newcomer:1975} and Cattell~\etal both circumvented this problem by
enforcing a cut-off when a certain depth in the search space had been reached).

Although this technique exhibits a worst time execution that is exponential in
the depth of the search, \citeauthor{YuHu:1994a} assert that a depth of~3 is
sufficient to yield results of equal quality to that of handwritten
\gls{assembly code}.
%
This claim notwithstanding, it is unclear whether it can be extended to support
complex \glspl{instruction} such as \gls{inter-block.ic} and
\gls{interdependent.ic} \glspl{instruction}.


\subsection{Relying on Semantic-Preserving Transformations}

In 1996, \textcite{HooverZadeck:1996} developed a system called \gls!{Toast} with the
goal of automating the generation of entire \gls{compiler} frameworks --
including \gls{instruction scheduling} and \gls{register allocation} -- from a
declarative \gls{machine description}.
%
In \gls{Toast} the \gls{instruction selection} is done by applying
semantic-preserving transformations during \gls{pattern selection} to make
better use of the \gls{instruction set}.
%
For example, although \mbox{$x * 2$} is semantically equivalent to \mbox{$x \ll
  1$} -- meaning that $x$ is arithmetically shifted one bit to the right, which
is faster than executing a multiplication -- most \glspl{instruction selector}
will fail to select \glspl{instruction} implementing the latter when the former
appears in the \gls{block DAG}, as the \glspl{pattern} are syntactically
different from one another.

Their design works as follows.
%
First, the \gls{frontend} emits \glspl{block DAG} consisting of \glspl{semantic
  primitive}, a kind of \gls{IR} code also used for describing the
\glspl{instruction}.
%
The \gls{block DAG} is then semantically matched using \gls{single-output.ic}
\glspl{pattern} derived from the \glspl{instruction}.
%
Semantic \glspl{match} -- which \citeauthor{HooverZadeck:1996} call \glspl!{toe print}
-- and are found by a \gls{semantic comparator}.
%
The \gls{semantic comparator} first performs syntactic matching -- that is,
checking that the \glspl{node} are of the same type, which is done using a
straightforward \mbox{$\mBigO(nm)$}~algorithm -- and then resorts to
semantic-preserving transformations for when syntactic matching fails.
%
To bound the exhaustive search for all possible \glspl{toe print}, a
transformation is only applied if it will lead to a syntactic \gls{match} later
on.
%
Once all \glspl{toe print} have been found, they are combined into \glspl!{foot
  print}, which correspond to the full effects of an \gls{instruction}.
%
A \gls{foot print} can consist of just a single \gls{toe print} (as with
\gls{single-output.ic} \glspl{instruction}) or several (as with
\gls{multi-output.ic} \glspl{instruction}), but the paper lacks details on how
this is done exactly.
%
Lastly, all combinations of \glspl{foot print} are considered in pursuit of the
one leading to the most effective implementation of the \gls{block DAG}.
%
To further prune the search space, this process only considers combinations
where each selected \gls{foot print} syntactically matches at least one
\gls{semantic primitive} in the \gls{block DAG}, and only ``trivial amounts'' of
the \gls{block DAG} (for example constants) may be included in more than one
\gls{foot print}.

Using a prototype implementation, \citeauthor{HooverZadeck:1996} reported that almost
$10^{70}$~``implied \gls{instruction} \glspl{match}'' were found for one of the
test cases, but it is unclear how many of them were actually useful.
%
Moreover, in its current form the design appears to be unpractical for
generating \gls{assembly code} for all but very small \glspl{program}.


\section{Extending Tree Covering Techniques to DAGs}

Another common approach to \gls{DAG covering} is to reuse already-known,
linear-time methods from \gls{tree covering}.
%
This can be achieved either by transforming the \glspl{block DAG} into
\glspl{tree}, or by generalizing the \gls{tree}-based algorithms for
\gls{pattern matching} and \gls{pattern selection}.
%
We begin by discussing designs that apply the first technique.


\subsection{Undagging Program DAGs}

The simplest approach for reusing \gls{tree covering} techniques is to transform
the \gls{block DAG} into several \glspl{expression tree}.
%
We will refer to this idea as \gls!{undagging}.

As illustrated in \refFigure{undagging-example}, a \gls{block DAG} can be
\glsshort{undagging}[ged] into \glspl{expression tree} in two ways.
%
The first approach is to split the edges involving shared \glspl{node} -- these
are \glspl{node} where reuse occurs due to the presence of common subexpressions
-- which results in a set of disconnected \glspl{expression tree} that can then
be covered individually.
%
Not surprisingly, this approach is called \gls!{edge splitting}.
%
An implicit connection between the \glspl{expression tree} is maintained by
forcing the values computed at the shared \glspl{node} to be stored and read
from a specific location, typically in memory.
%
An example of such an implementation is \gls{Dagon}, a technology binder
developed by \textcite{Keutzer:1987}, which maps technology-independent
descriptions onto circuits.
%
The second approach is to duplicate the \glspl{node} involved in computing the
shared value, which is known as \gls!{node duplication}.
%
This results in a single but larger \gls{expression tree} compared to those
produced with \gls{edge splitting}.

%\begin{figure}[t]
%  \centering%
%  \begin{subfigure}[DAG][undagging-example-dag]
%    \begin{minipage}{2.5cm}
%      \centering%
%      \input{\figurePath/dag-covering/undagging-1}
%    \end{minipage}%
%  \end{subfigure}
%  \hfill%
%  \begin{subfigure}[Result of edge splitting][undagging-example-splitting]
%    \begin{minipage}{4.3cm}
%      % LAYOUT FIX:
%      % Make this subfigure align properly with the previous subfigure
%      \vspace*{18pt}
%
%      \centering%
%      \input{\figurePath/dag-covering/undagging-2}
%    \end{minipage}%
%  \end{subfigure}
%  \hfill%
%  \begin{subfigure}[Result of node duplication][undagging-example-duplication]
%    \begin{minipage}{4.5cm}
%      % LAYOUT FIX:
%      % Make this subfigure align properly with the previous subfigure
%      \vspace*{4pt}
%
%      \centering%
%      \input{\figurePath/dag-covering/undagging-3}
%    \end{minipage}
%  \end{subfigure}
%
%  \figCaption{Undagging a \glsentrytext{block DAG} with a common
%    subexpression}
%  \labelFigure{undagging-example}
%\end{figure}

Common for both schemes is that they compromise code quality: too aggressive
\gls{edge splitting} produces many small \glspl{tree} that cannot be covered
using larger \glspl{pattern}, inhibiting use of more efficient
\glspl{instruction}; and too aggressive \gls{node duplication} incurs a larger
computational workload where many operations are needlessly re-executed in the
final \gls{assembly code}.
%
Moreover, the intermediate results of an edge-split \gls{block DAG} must be
forcibly stored in specific locations, which can be troublesome for
heterogeneous memory-\gls{register} architectures (this particular problem was
studied by \textcite{AraujoEtAl:1996}).


\subsubsection{Balancing Splitting and Duplication}

In 1994, \citeauthor{FauthEtAl:1994}~\cite{FauthEtAl:1994, Muller:1994} developed a
technique that tries to mitigate the deficiencies of \gls{undagging} by
balancing the use of \gls{node duplication} and \gls{edge splitting}.
%
Implemented in the \gls!{CBC}, the \gls{instruction selector} applies a
heuristic algorithm that first favors \gls{node duplication}, and resorts to
\gls{edge splitting} when the former is deemed too costly.
%
The decision about whether to duplicate or to split is taken by comparing the
cost of the two solutions and selecting the cheapest one.
%
The cost is calculated as a weighted sum \mbox{$w_1 n_{\text{dup}} + w_2
  n_{\text{split}}$}, where $n_{\text{dup}}$ is the number of \glspl{node} in
the \gls{block DAG} (a rough estimate of code size), and $n_{\text{split}}$ is
the expected number of \glspl{node} executed along each execution path (a rough
estimate of execution time).
%
Once this is done, each resulting \gls{expression tree} is covered by an
improved version of \gls{IBurg} (see \refSection{iburg}) with extended match
condition support.
%
However, the experimental data is too limited for us to judge how efficient this
technique is compared to a design where the \glspl{block DAG} have been
transformed into \glspl{expression tree} using just one method.


\subsubsection{Register-Sensitive Instruction Selection}

In 2001, \textcite{SarkarEtAl:2001} developed an \gls{instruction selection}
technique that attempts to reduce the \gls!{register pressure} -- that is, the
number of \glspl{register} needed by the \gls{program} -- in order to facilitate
\gls{register allocation}.\!%
%
\footnote{%
  Another register-aware \gls{instruction selection} technique was developed in
  2014 by \textcite{Xie:2014}, with the aim of reducing the number of writes to a
  nonvolatile \gls{register} file.
  %
  However, the \glspl{instruction} are selected using a proprietary and greedy
  heuristic hat does not warrant in-depth discussion.%
}

The design works as follows.
%
The \gls{block DAG} is first augmented with additional \glspl{edge} to signify
scheduling dependencies between memory operations, and then it is split into a
several \glspl{expression tree} using a heuristic to decide which \glspl{edge}
to break.
%
The \glspl{expression tree} are then covered individually using conventional
methods based on \gls{tree covering}, but instead of being the usual number of
execution cycles, the cost of each \gls{instruction} is set so as to reflect the
amount of \gls{register pressure} incurred by that instruction (unfortunately,
the paper lacks details on how these costs are computed exactly).
%
Once \glspl{pattern} have been selected, the \glspl{node} which are covered by
the same \gls{pattern} are merged into \glspl{super node}.
%
The resulting \gls{graph} is then checked for whether it contains any
\glspl{cycle}, which may appear due to the extra data dependencies that were
added at the earlier stage.
%
If it does, it means that there exist cyclic scheduling dependencies between two
or more memory operations, making it an illegal cover.
%
The splits are then reverted and the process repeats until a legal cover is
found.

\citeauthor{SarkarEtAl:2001} implemented their \gls{register}-sensitive design in
\gls{Jalapeno}, a \gls{register}-based \gls{Java} virtual machine developed by
\gls{IBM}.
%
For a small set of problems the performance increased by about 10\%, which
\citeauthor{SarkarEtAl:2001} claim to be due to fewer \glspl{instruction} needed for
\gls{register} \gls{spilling.r} compared to the default \gls{instruction
  selector}.
%
Although innovative, it is doubtful that the technique can be extended much
further.


\subsection{Extending the Dynamic Programming Approach to DAGs}

To avoid the application of ad hoc heuristics, several \gls{DAG}-based
\glspl{instruction selector} perform \gls{pattern selection} by applying an
extension of the \gls{tree}-based \gls{DP}~algorithm originally developed by
\textcite{AhoJohnson:1976}.
%
According to the literature, \citeauthor{LiemEtAl:1994}~\cite{LiemEtAl:1994,
  PaulinEtAl:1994, PaulinEtAl:1995} appear to have been the first to have done so.

In a seminal paper from~1994, \citeauthor{LiemEtAl:1994} introduce a design which is
part of \gls!{CodeSyn}, a well-known code synthesis system, which in turn is
part of a development environment for embedded systems called \gls!{FlexWare}.
%
For \gls{pattern matching}, \citeauthor{LiemEtAl:1994} applied the same technique as
\textcite{Weingart:1973} (see
\refSection{first-techniques-to-use-tree-pattern-matching}) by combining all
available \glspl{pattern tree} into a single \gls{tree} of \glspl{pattern}.
%
This \gls{pattern tree} is traversed in tandem with the \gls{block DAG}, and for
each \gls{node} an \mbox{$\mBigO(nm)$} \gls{pattern matcher} is used to find all
\glspl{match set}.
%
\Gls{pattern selection} is then performed using an extended version of the
\gls{DP}~algorithm, but the paper does not explain how this is done exactly.
%
Moreover, the algorithm is only applied on the data flow of the \gls{block DAG}
-- control flow is covered separately using a simple heuristic -- and no
guarantees are made that the \gls{pattern selection} is \gls{optimal.ps}, as
that is an NP-complete problem.


\subsubsection{Potentially Optimal Pattern Selection}

In a paper from~1999, \textcite{Ertl:1999} introduces a design which guarantees
\gls{optimal.ps} \gls{pattern selection} on \glspl{block DAG} for certain
\glspl{machine grammar}.
%
The idea is to first make a bottom-up pass over the \gls{block DAG} and compute
the costs using the conventional \gls{DP}~algorithm as discussed in
\refAppendix{tree-covering}.
%
Each \gls{node} is thus labeled with the same costs, as if the \gls{block DAG}
had first been transformed into a \gls{tree} through \gls{node} duplication; but
\citeauthor{Ertl:1999} recognized that if several \glspl{pattern} reduce the same
\gls{node} to the same \gls{nonterminal}, then the reduction to that
\gls{nonterminal} can be shared between several \glspl{rule} whose
\glspl{pattern} contain the \gls{nonterminal}.
%
Hence the \glspl{instruction} for implementing shared \glspl{nonterminal} only
need to be emitted once, decreasing code size and also improving performance,
since the amount of redundant computation is reduced.
%
With appropriate data structures, a linear-time implementation can be achieved.

%\begin{figure}[b]
%  \centering
%
%  \begin{minipage}{3.5cm}
%    % Copyright (c) 2017-2018, Gabriel Hjort Blindell <ghb@kth.se>
%
% This work is licensed under a Creative Commons Attribution-NoDerivatives 4.0
% International License (see LICENSE file or visit
% <http://creativecommons.org/licenses/by-nc-nd/4.0/> for details).
%
\begingroup%
\figureFont\figureFontSize%
\setlength{\nodeDist}{28pt}%
%\setlength{\nodeSize}{12pt}%
\pgfdeclarelayer{b2}%
\pgfdeclarelayer{b1}%
\pgfsetlayers{b2,b1,main}%
\colorlet{m1shade}{shade1}%
\colorlet{m2shade}{shade2!25!shade3}%
\begin{tikzpicture}[
    reduction label/.style={
      nothing,
      inner xsep=2pt,
      auto,
    },
    outer match node/.style={
      match node,
      inner sep=-1pt,
    },
  ]

  % DAG
  \node [node] (n1) {\nMul};
  \node [node, position=-130 degrees from n1] (n2) {\nLoad};
  \node [node, position=-50  degrees from n2] (n3) {\nAdd};
  \node [node, position=-130 degrees from n3] (n4) {\nVar{4}};
  \node [node, position=-50  degrees from n3] (n5) {\nVar{a}};
  \begin{scope}[data flow]
    \draw (n5)
          -- node [reduction label, swap] {$\mNT{Reg}, \mNT{Reg}$}
          (n3);
    \draw (n4)
          -- node [reduction label, auto=none, above, sloped, inner ysep=2pt]
                  {$\mNT{Reg}$}
          (n3);
    \draw (n3)
          -- node [reduction label] {$\mNT{Addr}$}
          (n2);
    \draw (n3)
          to [bend right=45]
          node [reduction label, swap, pos=.26] {$\mNT{Reg}$}
          (n1);
    \draw (n2)
          -- node [reduction label, inner xsep=1pt] {$\mNT{Reg}$}
          (n1);
  \end{scope}

  % Matches
  \begin{pgfonlayer}{b1}
    \node [match node, fit=(n1), shading=axis, shading angle=90,
           left color=m2shade, right color=m1shade]
          (m1) {};
    \node [match node, fit=(n2), fill=m2shade] (m2) {};
    \node [match node, fit=(n3), fill=m1shade] (m3) {};
    \node [match node, fit=(n4), fill=m1shade] (m4) {};
    \node [match node, fit=(n5), fill=m1shade] (m5) {};
  \end{pgfonlayer}
  \begin{pgfonlayer}{b2}
    \node [outer match node, fit=(m5), fill=m2shade] (m5) {};

    \node [outer match node, fit=(m4), draw=none] (m5a) {};
    \node [outer match node, fit=(m3), draw=none] (m5b) {};
    \draw [match line, bend left=45, fill=m2shade]
          (m5a.south east)
          to
          (m5a.south west)
          to
          (m5a.north west)
          --
          (m5b.north west)
          to
          (m5b.north east)
          to
          (m5b.south east)
          --
          cycle;
  \end{pgfonlayer}
\end{tikzpicture}%
\endgroup%

%  \end{minipage}
%  \hfill%
%  \begin{minipage}{7.5cm}
%    \figCaption[Sharing nonterminals of \glsentrytext{tree covering} on a
%        \glsentrytext{block DAG}]%
%      {A \glsentrytext{tree covering} of a \glsentrytext{block DAG} where the
%        patterns have been selected optimally.
%%
%The two shades indicate the
%        relation between rules, and the text along the edges indicates the
%        nonterminals to which each pattern is reduced.
%%
%Note that the
%        \cVar*{Reg}~\gls{node} is covered by two patterns (as indicated by the
%        double dash pattern) which both reduce to the same nonterminal and can
%        thus be shared}[Ertl:1999]
%    \labelFigure{ertl-example}
%  \end{minipage}
%\end{figure}

An example illustrating such a situation is given in \refFigure{ertl-example},
where we see an addition that will have to be implemented twice, as its
\gls{node} is covered by two separate \glspl{pattern} each of which reduces the
\gls{subtree} to a different \gls{nonterminal}.
%
The \cVar*{reg}~\gls{node}, on the other hand, is reduced twice to the same
\gls{nonterminal} ($\mNT{Reg}$), and can thus be shared between the \glspl{rule}
that use this \gls{nonterminal} in the \glspl{pattern}.

As said earlier, however, this technique yields \gls{optimal.ps} \gls{pattern
  selection} only for certain \glspl{machine grammar}.
%
\citeauthor{Ertl:1999} therefore devised a checker, called \gls{DBurg}, that
detects when the \gls{grammar} does not belong into this category and thus
cannot guarantee optimality.
%
The basic idea is to check whether every locally \gls{optimal.ps} decision is
also globally \gls{optimal.ps} by performing inductive proofs over the set of
all possible \glspl{block DAG}.
%
To do this efficiently, \citeauthor{Ertl:1999} implemented \gls{DBurg} using the
ideas behind \gls{Burg} (hence the name).


\subsubsection{Combining DP and Edge Splitting}

\def\overlapCost{overlap-cost\xspace}
\def\cseCost{cse-cost\xspace}

\textcite{KoesGoldstein:2008} extended \citeauthor{Ertl:1999}'s ideas by providing a
heuristic that splits the \gls{block DAG} at points where \gls{node duplication}
is estimated to have a detrimental effect on code quality.
%
Like \citeauthor{Ertl:1999}'s algorithm, \citeauthor{KoesGoldstein:2008}'s first selects
\glspl{pattern} optimally by performing a \gls{tree}-like, bottom-up
\gls{DP}~pass which ignores the fact that the input is a~\gls{DAG}.
%
Then, at points where multiple \glspl{pattern} overlap, two costs are
calculated: an \emph{\overlapCost} and a \emph{\cseCost}.
%
The \overlapCost is an estimate of the cost of letting the \glspl{pattern}
overlap and thus incur duplication of operations in the final \gls{assembly
  code}.
%
The \cseCost is an estimate of the cost of splitting the \glspl{edge} at such
points.
%
If \cseCost is lower than \overlapCost, then the \gls{node} where overlapping
occurs is marked as \gls!{fixed.n}.
%
Once all such \glspl{node} have been processed, a second bottom-up \gls{DP}~pass
is performed on the \gls{block DAG}, but this time no \glspl{pattern} are
allowed to span across \gls{fixed.n} \glspl{node}, which can only be matched at
the \gls{root} of a \gls{pattern}.
%
Lastly, a top-down pass emits the \gls{assembly code}.

For evaluation purposes \citeauthor{KoesGoldstein:2008} compared their own implementation,
called \gls!{Noltis}, against an implementation based on \glsdesc{IP} -- we will
discuss such techniques later in this appendix -- and found that \gls{Noltis}
achieved \gls{optimal.ps} \gls{pattern selection} in 99.7\% of the test cases.
%
More details are given in \citeauthor{Koes:2009}'s doctoral
dissertation~\cite{Koes:2009}.
%
But like \citeauthor{Ertl:1999}'s design, \citeauthor{KoesGoldstein:2008}'s is limited to
\glspl{pattern tree} and thus cannot support more complex \glspl{instruction}
such as \gls{multi-output.ic} \glspl{instruction}.


\subsubsection{Supporting Multi-output Instructions}

In most \gls{instruction selection} techniques based on \gls{DAG covering}, it
is assumed that the outputs of a \gls{pattern DAG} always occur at the
\gls{root}~\glspl{node}.
%
But in a design by \citeauthor{ArnoldCorporaal:1999}~\cite{ArnoldCorporaal:1999, ArnoldCorporaal:2001}
(originally introduced in a technical report by \textcite{Arnold:1999}), the
\glspl{node} representing output can be marked explicitly.
%
The advantage of this is that it allows the \glspl{pattern DAG} to be fully
decomposed into \glspl{tree} such that each output value receives its own
\gls{pattern tree}, which \citeauthor{ArnoldCorporaal:1999} call \gls!{partial.p}[
  \glspl{pattern}]
%
An example is given in \refFigure{arnold-example}.

%\begin{figure}[t]
%  \centering%
%  \begin{subfigure}[Original \glsentrytext{pattern DAG}][arnold-example-dag]
%    \begin{minipage}{2.5cm}%
%      \centering%
%      % Copyright (c) 2018, Gabriel Hjort Blindell <ghb@kth.se>
%
% This work is licensed under a Creative Commons 4.0 International License (see
% LICENSE file or visit <http://creativecommons.org/licenses/by/4.0/> for a copy
% of the license).
%
\begingroup%
% Copyright (c) 2018, Gabriel Hjort Blindell <ghb@kth.se>
%
% This work is licensed under a Creative Commons 4.0 International License (see
% LICENSE file or visit <http://creativecommons.org/licenses/by/4.0/> for a copy
% of the license).
%
\figureFont\figureFontSize%
\setlength{\nodeDist}{14pt}%
\setlength{\nodeSize}{12pt}%
\tikzset{%
  output node/.style={
    node,
    fill=black,
    text=white,
  },
}%
%
\begin{tikzpicture}
  \node [node] (x) {\nVar{x}};
  \node [node, right=of x] (y) {\strut\nVar{y}};
  \node [output node, above=of x] (add-output) {\nAdd};
  \node [node, above=of y] (add-not-output) {\nAdd};
  \node [node, right=of add-not-output] (zero) {\nVar{0}};
  \node [output node, above=of $(add-not-output) !.5! (zero)$] (eq) {\nEQ};
  \begin{scope}[data flow]
    \draw (add-not-output) -- (eq);
    \draw (zero) -- (eq);
    \draw (x) -- (add-not-output);
    \draw (y) -- (add-not-output);
    \draw (x) -- (add-output);
    \draw (y) -- (add-output);
  \end{scope}
\end{tikzpicture}%
\endgroup%
%
%    \end{minipage}%
%  \end{subfigure}
%  \hspace{40pt}%
%  \begin{subfigure}[Corresponding partial \glsentrytext{pattern tree}s]%
%    [arnold-example-trees]
%    \begin{minipage}{4.5cm}%
%      % LAYOUT FIX:
%      % Make this subfigure align properly with the previous subfigure
%      \vspace*{6pt}
%
%      \centering%
%      \begin{minipage}{1cm}%
%        \centering%
%        \input{\figurePath/dag-covering/arnold-example-tree1}%
%      \end{minipage}%
%      \hfill%
%      \begin{minipage}{2cm}%
%        \centering%
%        \input{\figurePath/dag-covering/arnold-example-tree2}%
%      \end{minipage}%
%    \end{minipage}
%  \end{subfigure}
%
%  \figCaption[Converting a \glsentrytext{pattern DAG} into partial (tree)
%    patterns]%
%    {Converting a \glsentrytext{pattern DAG}, which represents an
%      \dataTerm{add} instruction
%      that also sets a status flag if the result is equal to~0, into partial
%      pattern trees.
%%
%The darkly shaded nodes indicate the output nodes}
%  \labelFigure{arnold-example}
%\end{figure}

The \gls{partial.p} \glspl{pattern} are then matched over the \gls{block DAG}
using an \mbox{$\mBigO(nm)$} algorithm.
%
After matching, another algorithm attempts to merge appropriate combinations of
partial \glspl{match} into \glspl{match} of the original \gls{pattern DAG}.
%
This is done in a straightforward manner by maintaining, for each \gls{match},
an array that maps the \glspl{node} in the \gls{pattern DAG} to the covered
\glspl{node} in the \gls{block DAG}, and then checking whether two
\gls{partial.p} \glspl{pattern} belong to the same original \gls{pattern DAG}
and have compatible mappings.
%
This means that no pair of \gls{pattern} \glspl{node} belonging to different
\gls{partial.p} \glspl{pattern} but corresponding to the same \gls{node} in the
original \gls{pattern DAG} may cover different \glspl{node} in the \gls{block
  DAG}.
%
For \gls{pattern selection} \citeauthor{ArnoldCorporaal:1999} applied a variant of the
\gls{DP}~scheme but combined it with a greedy heuristic in order to enforce that
each \gls{node} is covered exactly once.
%
Hence code quality is compromised.


\section{Transforming Pattern Selection to an M(W)IS Problem}
\labelSection{maximum-independent-set}

In the techniques discussed so far, the \gls{instruction selector} operates
directly on the \gls{block DAG} when performing \gls{pattern selection}.
%
The same applies for most designs based on \gls{tree covering}.
%
But another approach is to indirectly solve the \gls{pattern selection} problem
by first transforming it into an instance of some other problem for which there
already exist efficient solving methods.
%
When that problem has been solved, the answer can be translated back into a
solution for the original \gls{pattern selection} problem.

%\begin{figure}[t]
%  \centering%
%  \begin{subfigure}[Matched DAG][mis-example-dag]
%    \begin{minipage}{3.3cm}
%      \centering%
%      % Copyright (c) 2017, Gabriel Hjort Blindell <ghb@kth.se>
%
% This work is licensed under a Creative Commons 4.0 International License (see
% LICENSE file or visit <http://creativecommons.org/licenses/by/4.0/> for a copy
% of the license).
%
\begingroup%
\figureFont%
\pgfdeclarelayer{background1}%
\pgfdeclarelayer{background2}%
\pgfdeclarelayer{background3}%
\pgfsetlayers{background3,background2,background1,main}%
\begin{tikzpicture}[%
    data node/.style={%
      op node,
    },
    outer match node/.style={%
      match node,
      draw=none,
      inner sep=0,
    },
    match label/.style={%
      minimum size=0,
      inner sep=0,
      outer sep=1pt,
      node distance=10pt,
      font=\footnotesize,
    },
  ]

  % Graph
  \node [op node] (add) {\opAdd};
  \node [data node, position=-135 degrees from add] (a) {\opVar{a}};
  \node [data node, position=- 45 degrees from add] (b) {\opVar{b}};
  \node [op node, node distance=1.2\opNodeDist, position=135 degrees from add]
        (mul) {\opMul};
  \node [data node, node distance=1.2\opNodeDist, position=45 degrees from add]
        (load) {\opLoad};
  \node [data node, position=-135 degrees from mul] (c) {\opVar{c}};
  \begin{scope}[data flow]
    \draw (a) -- (add);
    \draw (b) -- (add);
    \draw (add) -- (mul);
    \draw (add) -- (load);
    \draw (c) -- (mul);
  \end{scope}

  % Matches
  \begin{pgfonlayer}{background1}
    \node [match node, fill=shade2, fit=(add)] (m1) {};
    \node [match node, fill=shade2, fit=(mul)] (m2) {};
    \node [match node, fill=shade1, fit=(load)] (m4) {};
    % m3
    \begin{pgfonlayer}{background2}
      \node [outer match node, inner sep=-1pt, fit=(m2)] (m3a) {};
      \node [outer match node, inner sep=-1pt, fit=(m1)] (m3b) {};
      \draw [match line, bend left=45, fill=shade1]
            (m3a.south west)
            to
            (m3a.north west)
            to
            (m3a.north east)
            -- coordinate [pos=0] (m3)
            (m3b.north east)
            to
            (m3b.south east)
            to
            (m3b.south west)
            --
            cycle;
    \end{pgfonlayer}
    % m5
    \begin{pgfonlayer}{background3}
      \node [outer match node, inner sep=-1pt, fit=(m4)] (m5a) {};
      \node [outer match node, inner sep=1pt, fit=(m1)] (m5b) {};
      \draw [match line, bend left=45, fill=shade3]
            (m5a.north west)
            to
            (m5a.north east)
            to
            (m5a.south east)
            -- coordinate [pos=.25] (m5)
            (m5b.south east)
            to
            (m5b.south west)
            to
            (m5b.north west)
            --
            cycle;
    \end{pgfonlayer}
    % Fill in overdrawn edges of m5
    \begin{pgfonlayer}{background2}
      \draw [match line, bend left=45]
            (m5a.north west)
            to
            (m5a.north east)
            to
            (m5a.south east)
            --
            (m5b.south east)
            to
            (m5b.south west)
            to
            (m5b.north west)
            --
            cycle;
    \end{pgfonlayer}
  \end{pgfonlayer}

  % Match labels
  \node [match label, below=of m1] (m1l) {$\strut m_1$};
  \node [match label, left=of m2] (m2l) {$\strut m_2$};
  \node [match label, position=45 degrees from m3] (m3l) {$m_3$};
  \node [match label, right=of m4] (m4l) {$\strut m_4$};
  \node [match label, position=-45 degrees from m5] (m5l) {$m_5$};
  \foreach \i in {1, ..., 5} {
    \draw [thick] (m\i) -- (m\i l);
  }
\end{tikzpicture}%
\endgroup%
%
%    \end{minipage}%
%  \end{subfigure}
%  \hspace{40pt}%
%  \begin{subfigure}[Conflict graph][mis-example-conflict]
%    \begin{minipage}{3cm}
%      \centering%
%      \input{\figurePath/dag-covering/mis-example-conflict}%
%    \end{minipage}
%  \end{subfigure}
%
%  \figCaption[A conflict graph example]%
%    {Example of a pattern-matched \gls{block DAG} and its corresponding
%      conflict graph}
%  \labelFigure{mis-example}
%\end{figure}

One such problem is the \gls!{MIS}[ problem], where the task is to select the
largest set of \glspl{node} from a \gls{graph} such that no pairs of selected
\glspl{node} have an \gls{edge} between them.
%
In the general case, finding such a solution is
NP-complete~\cite{GareyJohnson:1979}, and the \gls{pattern selection} problem is
transformed into an \gls{MIS} problem as follows.
%
From the \glspl{match set} found by \gls{pattern matching}, a corresponding
\gls!{conflict graph} -- or \gls!{interference graph}, as it is sometimes called
-- is formed.
%
Each \gls{node} in the \gls{conflict graph} represents a \gls{match}, and there
exists an \gls{edge} between two \glspl{node} if and only if the corresponding
\glspl{match} overlap.
%
An example of this is given in \refFigure{mis-example}.
%
By solving the \gls{MIS} problem for the \gls{conflict graph}, we obtain a
selection of \glspl{match} such that every \gls{node} in the \gls{block DAG} is
covered by exactly one \gls{match}.

But a solution to the \gls{MIS} problem does not necessarily yield an
\gls{optimal.ps} solution to the \gls{pattern selection} problem, as the former
does not incorporate costs.
%
We address this limitation by transforming the \gls{MIS} problem into a
\gls!{MWIS}[ problem], where the task is to find a solution to the \gls{MIS}
problem that maximizes \mbox{$\sum_{p} \mWeight{p}$}, and assign as weights the
costs of the \glspl{pattern}.
%
We can get the solution with minimal total cost simply by negating the weights.
%
Note that although the \gls{MWIS}-based techniques discussed in this
dissertation have all been limited to \glspl{block DAG}, the approach can just
as well be applied in \gls{graph covering}, which will be introduced in
\refAppendix{graph-covering}.


\subsection{Applications}

In 2007, \textcite{ScharwaechterEtAl:2007} introduced what appears to be the first
\gls{instruction selection} technique to use the \gls{MWIS} approach for
selecting \glspl{pattern}.
%
But despite this novelty, the most cited contribution of their design is its
extensions to \glspl{machine grammar} to support \gls{multi-output.ic}
\glspl{instruction}.


\subsubsection{Machine Grammars with Multiple Left-Hand Side Nonterminals}

\labelPage{rules-multiple-productions}

To begin with, \citeauthor{ScharwaechterEtAl:2007} distinguishes between \glspl{rule}
having only one left-hand side \gls{nonterminal} in their \glspl{production}
from \glspl{rule} containing multiple left-hand side \glspl{nonterminal} by
referring to them as \glspl!{rule} and \gls!{complex.r}[ \glspl{rule}],
respectively.
%
In addition, we will refer to the right-hand side \glspl{symbol} appearing in
\glspl{rule} as \gls!{simple.p}[ \glspl{pattern}], and to the right-hand side
\glspl{symbol} appearing in \gls{complex.r} \glspl{rule} as \gls!{split.p}[
  \glspl{pattern}].\!%
%
\footnote{%
  In their paper, \citeauthor{ScharwaechterEtAl:2007} call these \gls!{simple.r}[
    \glspl{rule}] and \gls!{split.r}[ \glspl{rule}], respectively, but to
  conform with the terminology established on \refPage{rule-terms}, we will
  refer to them as \glspl{pattern}.%
}
%
A combination of \gls{split.p} \glspl{pattern} is thus known as a
\gls!{complex.p}[ \glspl{pattern}].
%
The aforementioned terms are illustrated more clearly below:
%
%\begingroup%
%\def\tPadded#1{%
%  \vbox to 1.5\baselineskip {%
%    \vfil%
%    \hbox{#1}%
%    \vfil%
%  }%
%}%
%\def\tPadded#1{%
%  \vbox to 1.5\baselineskip {%
%    \vfil%
%    \hbox{#1}%
%    \vfil%
%  }%
%}%
%\NewDocumentCommand{\pgT}{mo}{%
%  \tPadded{\text{%
%    \IfValueTF{#2}{%
%      \gT{#1}[#2]%
%    }{%
%      \gT{#1}%
%    }%
%  }}%
%}
%\NewDocumentCommand{\pgNT}{mo}{%
%  \tPadded{\text{%
%    \IfValueTF{#2}{%
%      \gNT{#1}[#2]%
%    }{%
%      \gNT{#1}%
%    }%
%  }}%
%}
%\NewDocumentCommand{\pgA}{mo}{%
%  \tPadded{\text{%
%    \IfValueTF{#2}{%
%      \gA{#1}[#2]%
%    }{%
%      \gA{#1}%
%    }%
%  }}%
%}
%\def\tText#1{\text{\figureFontStyle#1}}%
%\begin{displaymath}
%  \boolfalse{TinRunningText}%
%  \small%
%  \underbrace{
%    \pgNT{nt} \; \tPadded{$\rightarrow$} \;
%    \overbrace{%
%      \pgT{Op} \; \pgT{I}[1] \; \tPadded{\ldots}%
%    }^{\tText{simple pattern}}
%    \qquad
%    \tPadded{\tText{cost}}
%    \qquad
%    \tPadded{\tText{action}}
%  }_{\tText{rule}}
%\end{displaymath}
%\begin{displaymath}
%  \boolfalse{TinRunningText}%
%  \small%
%  \underbrace{
%    \pgNT{nt}[1] \; \pgNT{nt}[2] \; \tPadded{\ldots} \;
%    \tPadded{$\rightarrow$} \;
%    \overbrace{%
%      \overbrace{%
%        \pgT{Op}[1] \; \pgT{I}[1,1] \; \tPadded{\ldots}%
%      }^{\tText{split pattern}}
%      \tPadded{\raisebox{0pt}[1.5ex]{\tText{,}}} \;
%      \overbrace{%
%        \pgT{Op}[2] \; \pgT{I}[2,1] \; \tPadded{\ldots}%
%      }^{\tText{split pattern}}
%      \tPadded{\raisebox{0pt}[1.5ex]{\tText{,}}} \; \tPadded{\ldots}
%    }^{\tText{complex pattern}}
%    \qquad
%    \tPadded{\tText{cost}}
%    \qquad
%    \tPadded{\tText{action}}
%  }_{\text{complex rule}}
%\end{displaymath}%
%\endgroup

\Gls{pattern matching} is a two-step process.
%
First, the \glspl{match set} are found for the \gls{simple.p} and \gls{split.p}
\glspl{pattern}, using conventional \gls{tree}-based \gls{pattern matching}
techniques.
%
Second, the \glspl{match set} for the \gls{complex.p} \glspl{pattern} are found
by combining the \glspl{match} of \gls{split.p} \glspl{pattern} into
\glspl{match} of \gls{complex.p} \glspl{pattern} where appropriate.
%
The \gls{pattern selector} then checks whether it is worth applying a
\gls{complex.p} \gls{pattern} for covering a certain set of \glspl{node}, or if
they should be covered using the \gls{simple.p} \glspl{pattern} instead.
%
Since the intermediate results of \glspl{node} within \gls{complex.p}
\glspl{pattern} cannot be reused for other \glspl{pattern}, selecting a
\gls{complex.p} \gls{pattern} can incur an additional overhead cost as
\glspl{node} in the \gls{block DAG} may need to be covered using multiple
\glspl{pattern}.
%
Consequently, a \gls{complex.p} \gls{pattern} will only be selected if the cost
reduced by replacing a set of \gls{simple.p} \glspl{pattern} with this
\gls{pattern} is greater than the cost incurred by code duplication.
%
After these decisions have been taken, the next step is to perform \gls{pattern
  selection}.
%
For this, \citeauthor{ScharwaechterEtAl:2007} solve the corresponding \gls{MWIS}
problem in order to limit solutions to those of exact covering only.
%
The weights are calculated as the negated sum of the \gls{split.p} \gls{pattern}
costs, but the paper is ambiguous on how these costs are calculated.
%
Since the \gls{MWIS} problem is known to be NP-complete,
\citeauthor{ScharwaechterEtAl:2007} employed a greedy heuristic called \gls!{Gwmin2}
by \textcite{SakaiEtAl:2003}.
%
Lastly, \gls{split.p} \glspl{pattern} which have not been merged into
\gls{complex.p} \glspl{pattern} are replaced by corresponding \gls{simple.p}
\glspl{pattern} before \gls{assembly code} emission.

\citeauthor{ScharwaechterEtAl:2007} implemented a prototype called \gls{CBurg} as an
extension of \gls{Olive} (see \refSection{olive}), and then ran some experiments
by targeting a \gls{MIPS}-like architecture.
%
In these experiments \gls{CBurg} generated \gls{assembly code} which improved
performance by almost 25\%, and reduced code size by nearly 22\%, compared to
\gls{assembly code} which was only allowed to make use of \gls{single-output.ic}
\glspl{instruction}.
%
Measurements of \gls{CBurg} also indicate that this technique exhibits
near-linear time complexity.
%
\textcite{AhnEtAl:2009} later broadened this work by including scheduling dependency
conflicts between \gls{complex.p} \glspl{pattern}, and incorporating a feedback
loop with the \gls{register allocator} to facilitate \gls{register allocation}.

A shortcoming of both designs by \citeauthor{ScharwaechterEtAl:2007} and
\citeauthor{AhnEtAl:2009} is that \gls{complex.r} \glspl{rule} can only consist of
disconnected \glspl{pattern tree} (hence there can be no sharing of \glspl{node}
between the \gls{split.p} \glspl{pattern}).
%
\textcite{YounEtAl:2011} address this problem in a 2011~paper -- which is a revised
and extended version of the original paper by \citeauthor{ScharwaechterEtAl:2007} --
by introducing index subscripts for the operand specification of the
\gls{complex.r} \glspl{rule}; but the subscripts are restricted to the input
\glspl{node} of the \gls{pattern}, still hindering support for completely
arbitrary \glspl{pattern DAG}.


\subsubsection{Targeting Machines with Echo Instructions}

In 2004, \textcite{BriskEtAl:2004} introduced a technique to perform \gls{instruction
  selection} for \glspl{target machine} with special \glspl!{echo instruction},
which are small markers that refer back to an earlier portion in the
\gls{assembly code} for re-execution.
%
This allows the \gls{assembly code} to be compressed by basically using the same
idea that is applied in the LZ77 algorithm~\cite{ZivLampel:1977}.\!%
%
\footnote{%
  The algorithm performs string compression by replacing recurring substrings
  that appear earlier in the string with pointers, allowing the original string
  to be reconstructed by ``copy-pasting.''%
}
%
Since \glspl{echo instruction} do not incur a branch or a procedure call, the
\gls{assembly code} can be reduced in size without sacrificing performance.
%
Consequently, unlike for traditional \glspl{target machine}, the \gls{pattern
  set} is not fixed in this case but must be determined as a precursor to
\gls{pattern matching} (this is closely related to the \gls!{ISE}[ problem],
which we will discuss in \refAppendix{conclusions}).

The intuition behind this design is to use \glspl{echo instruction} where code
duplication is most prominent.
%
To find these cases in a given \gls{program}, \citeauthor{BriskEtAl:2004} first
enumerate all \glspl{subgraph} from the \glspl{block DAG}, and then match each
\gls{subgraph} over the \glspl{block DAG}.
%
\Gls{pattern matching} is done using \gls{VF2}, which is a generic \gls{subgraph
  isomorphism} algorithm that we will describe in \refAppendix{graph-covering}.
%
Summing the sizes of the resulting \glspl{match set} gives a measure of code
duplication for each \gls{subgraph}, but this value will be an overestimation as
the \glspl{match set} may contain overlapping \glspl{match}.
%
\citeauthor{BriskEtAl:2004} addressed this by first solving the \gls{MIS} problem on
the \gls{conflict graph} for each \gls{match set}, and then adding up the sizes
of \emph{these} sets.
%
After selecting the most beneficial \gls{subgraph}, the covered \glspl{node} in
the \glspl{block DAG} are collapsed into single \glspl{node} to reflect the use
of \glspl{echo instruction}.
%
This process of matching and collapsing is then repeated until no new
\gls{subgraph} better than some user-defined value criterion can be found.
%
\citeauthor{BriskEtAl:2004} performed experiments on a prototype using a selected set
of benchmark applications, which showed code size reductions of 25\% to 36\% on
average.


\section{Transforming Pattern Selection to a Unate/Binate Covering Problem}

Another approach to solving \gls{pattern selection} is to translate it to a
corresponding \glsshort!{unate covering} or \gls!{binate covering}[ problem].
%
The concepts behind the two are identical with the exception of one detail, and
both \glsshort{unate covering} and \gls{binate covering} can be used directly
for covering \glspl{graph} even though the designs discussed in this
dissertation have only been applied on \glspl{block DAG}.

Although \gls{binate covering}-based techniques actually appeared first, we will
begin with explaining \gls{unate covering}, as \gls{binate covering} is an
extension of \gls{unate covering}.


\subsubsection{Unate Covering}

The idea of \gls{unate covering} is to create a Boolean matrix~$\mMatrix{M}$,
where each row represents a \gls{node} in the \gls{block DAG} and each column
represents a \gls{match} covering one or more \glspl{node} in the \gls{block
  DAG}.
%
If we denote $m_{ij}$ as row~$i$ and column~$j$ in~$\mMatrix{M}$, then
\mbox{$m_{ij} = 1$} indicates that \gls{node}~$i$ is covered by
\gls{pattern}~$j$.
%
Hence the \gls{pattern selection} problem is equivalent to finding a valid
configuration of $\mMatrix{M}$ such that the sum of every row is at least~1.
%
An example is given in \refFigure{unate-covering-example}.
%
\Gls{unate covering} is an NP-complete problem, but as with the \gls{MIS} and
\gls{MWIS} problems there exist several efficient techniques for solving it
heuristically (see for example~\cite{CordoneEtAl:2000, GoldbergEtAl:2006} for an
overview).

%\begin{figure}[b]
%  \centering%
%
%  \begin{subfigure}[Graph to cover][unate-covering-example-graph]
%    \begin{minipage}{6cm}%
%      \centering%
%      \input{\figurePath/graph-covering/unate-covering-example-graph}%
%    \end{minipage}%
%  \end{subfigure}
%  \hfill%
%  \begin{subfigure}[Boolean matrix][unate-covering-example-matrix]
%    \begin{minipage}{4cm}%
%      \centering%
%      \begin{framedBoxWI}{\textwidth}
%        \centering%
%        \begin{tabular}{c@{~~}|@{~~}ccccccc}
%                & \tT{p}[1] & \tT{p}[2] & \tT{p}[3] & \tT{p}[4] & \tT{p}[5]
%                & \tT{p}[6] & \tT{p}[7]\\
%          \hline
%          % LAYOUT FIX: prevent first row from touching the line above
%          \rule{0pt}{2ex}%
%          \tT{n}[1] & 1* & 1* &  0 &  0 &  0 &  0 &  0\\
%          \tT{n}[2] &  0 & 1* &  1 &  0 &  0 &  0 &  0\\
%          \tT{n}[3] &  0 &  0 &  1 & 1* &  0 &  0 &  0\\
%          \tT{n}[4] &  0 &  0 &  0 & 1* &  1 &  1 &  0\\
%          \tT{n}[5] &  0 &  0 &  0 &  0 &  0 &  0 & 1*\\
%        \end{tabular}
%      \end{framedBoxWI}
%    \end{minipage}
%  \end{subfigure}
%
%  \figCaption[A unate covering example]%
%    {Example of unate covering.
%%
%Unmarked 1s in the matrix represent potential
%      but not selected covers, while the 1s marked with a star (1*) indicate a
%      selection that is optimal (assuming all patterns have the same cost)}
%  \labelFigure{unate-covering-example}
%\end{figure}

\Gls{unate covering} alone, however, does not incorporate all necessary
\glspl{constraint} of \gls{pattern selection} since some \glspl{pattern} require
-- and prevent -- the selection of other \glspl{pattern} in order to yield
correct \gls{assembly code}.
%
For example, assume that \gls{pattern}~$p_3$ in
\refFigure{unate-covering-example-graph} requires that \gls{pattern}~$p_6$ be
selected to cover $n_4$ instead of \gls{pattern}~$p_5$.
%
Using \glspl{machine grammar} this can be enforced with the appropriate use of
\glspl{nonterminal}, but for \gls{unate covering} we have no means of expressing
this \gls{constraint}.
%
We therefore turn to \gls{binate covering}, where this is possible.


\subsubsection{Binate Covering}

We first rewrite the Boolean matrix from the \gls{unate covering} problem into
Boolean formulas consisting of conjunctions of non-negated disjunctions.
%
The Boolean matrix in \refFigure{unate-covering-example-matrix} can thus be
rewritten as
%
\begin{displaymath}
  f =
  (\mVar{p}_1 \mOr \mVar{p}_2) \mAnd (\mVar{p}_2 \mOr \mVar{p}_3) \mAnd
  (\mVar{p}_3 \mOr \mVar{p}_4) \mAnd (\mVar{p}_4 \mOr \mVar{p}_5 \mOr \mVar{p}_6)
  \mAnd \mVar{p}_7.
\end{displaymath}
%
Now, the difference between \gls{unate covering} and \gls{binate covering} is
that all variables must be non-negated in the former, but may be negated in the
latter.
%
Hence, the aforementioned \gls{constraint} regarding the compulsory selection of
$p_6$ if $p_4$ is selected can now be expressed as
%
\begin{displaymath}
  \mNot\mVar{p}_4 \mOr \mVar{p}_6,
\end{displaymath}
%
which is called an \gls!{implication clause} as it is logically equivalent to
\mbox{$\mVar{p}_4 \mImp \mVar{p}_6$}.
%
This is then simply appended to the Boolean formula~$f$ using the $\mAnd$
operator.


\subsubsection{Applications}

According to \citeauthor{LiaoEtAl:1995}~\cite{LiaoEtAl:1995, LiaoEtAl:1998} and
\textcite{CongEtAl:2004}, the pioneering use of \gls{binate covering} to solve
\gls{DAG covering} was done by \textcite{Rudell:1989} in 1989 as a part of a
\gls{VLSI} synthesis design.
%
\citeauthor{LiaoEtAl:1995}~\cite{LiaoEtAl:1995, LiaoEtAl:1998} later adapted it to
\gls{instruction selection} in a method that optimizes code size for
one-\gls{register} \glspl{target machine}.
%
To prune the search space, \citeauthor{LiaoEtAl:1995} perform \gls{pattern selection}
in two iterations.
%
In the first iteration, \glspl{pattern} are selected such that the \gls{block
  DAG} is covered but the costs of necessary data transfers are ignored.
%
After this step the \glspl{node} covered by the same \gls{pattern} are collapsed
into single \glspl{node}, and a second \gls{binate covering} problem is
constructed to minimize the costs of data transfers.
%
Although these two problems can be solved simultaneously, \citeauthor{LiaoEtAl:1995}
chose not to do so as the number of necessary \glspl{implication clause} would
become very large.
%
Recently, \textcite{CongEtAl:2004} also applied \gls{binate covering} as part of
generating application-specific \glspl{instruction} for configurable processor
architectures.

\Gls{unate covering} was applied by \textcite{ClarkEtAl:2006} in generating
\gls{assembly code} for acyclic computation accelerators, which can be partially
customized in order to increase performance of the currently executed
\gls{program}.
%
Described in a paper from~2006, the \glspl{target machine} are presumably
homogeneous enough that \glspl{implication clause} are unnecessary.
%
The work by \citeauthor{ClarkEtAl:2006} was later expanded by \textcite{HormatiEtAl:2007}
to reduce the number of interconnects as well as data-centered latencies in
accelerator designs.
%
\citeauthor{MartinEtAl:2009}~\cite{MartinEtAl:2009, MartinEtAl:2012} also applied \gls{unate
  covering} to solve a similar problem concerning reconfigurable processor
extensions, but combined the \gls{instruction selection} problem with
\gls{instruction scheduling} and solved both in tandem using a method called
\glsdesc{CP} -- we will discuss this approach later in this appendix -- which
they also applied for solving the \gls{pattern matching} problem.
%
Unlike in the cases of \citeauthor{ClarkEtAl:2006} and \citeauthor{HormatiEtAl:2007}, who
solved their \gls{unate covering} problems using heuristics, the \gls{assembly
  code} generated by \citeauthor{MartinEtAl:2009} is potentially optimal.


\section{Modeling Instruction Selection with IP}

As explained in \refAppendix{introduction}, performing \gls{instruction
  selection}, \gls{instruction scheduling}, or \gls{register allocation} in
isolation will typically always yield suboptimal \gls{assembly code}.
%
But since each subproblem is already NP-complete on its own, attaining
\gls!{integrated.cg}[ \gls{code generation}] -- where all these problems are
solved simultaneously -- is an even more difficult problem.

These challenges notwithstanding, \textcite{WilsonEtAl:1994} introduced in 1994 what
appears to be the first design that could be said to yield truly optimal
\gls{assembly code}.
%
\citeauthor{WilsonEtAl:1994} accomplished this by using \gls{IP}, which is a method
for solving combinatorial optimization problems (sometimes \gls{IP} is also
called \gls{ILP}).
%
In \gls{IP} a problem is expressed using sets of integer \glspl{variable} and
linear equations (see for example~\cite{Wolsey:1998} for an overview), and a
\gls{solution} to an \gls{IP}~model is an assignment to all variables such that
all equations are fulfilled.
%
In general, solving an \gls{IP}~model is NP-complete, but extensive research in
this field has made many problem instances tractable.

In their seminal paper, \citeauthor{WilsonEtAl:1994} describe that the \gls{pattern
  selection} problem can be expressed as the following linear inequality:
%
\begin{displaymath}
  \forall n \in N : \sum_{p \in P_n} x_p \leq 1.
\end{displaymath}
%
This reads: for every \gls{node}~$n$ in the \gls{block DAG}~\mbox{$\mTuple{N,
    E}$}, at most one \gls{pattern}~$p$ from the \gls{match set} involving~$n$
(represented by $P_n$) may be selected.
%
The decision is represented by $x_p$, which is a Boolean \mbox{$0/1$}
variable.\!%
%
\footnote{%
  The more common \gls{constraint} is that \emph{exactly one} \gls{pattern} must
  be selected, but in the design by \citeauthor{WilsonEtAl:1994}, \glspl{node} are
  allowed to become inactive and thus need not be covered.%
}
%
Similar linear equations can be formulated for modeling \gls{instruction
  scheduling} and \gls{register allocation} -- which \citeauthor{WilsonEtAl:1994}
also included in their model -- but these are out of scope for this
dissertation.
%
In fact, any \gls{constraint} that can be formulated in this way can be added to
an existing \gls{IP}~model, making this approach a suitable \gls{code
  generation} method for targeting irregular architectures.
%
Furthermore, this is the first design we have seen that could potentially
support \gls{interdependent.ic} \glspl{instruction} (although this was not the
main focus of \citeauthor{WilsonEtAl:1994}).

Solving this monolithic \gls{IP}~model, however, typically requires considerably
more time compared to the previously discussed techniques of \gls{instruction
  selection}.
%
But the trade-off for longer compilation time is higher code quality;
\citeauthor{WilsonEtAl:1994} reported that experiments showed that the generated
\gls{assembly code} was of comparable code quality to that of manually optimized
\gls{assembly code}.
%
In theory, optimal \gls{assembly code} can be generated, although this is in
practice only feasible for small enough \glspl{program}.
%
Another much-valued feature is the ability to extend the model with additional
\glspl{constraint} in order to support complicated \glspl{target machine}, which
cannot be properly handled by the conventional designs as that typically
violates assumptions made by the underlying heuristics.


\subsection{Approaching Linear Solving Time with Horn Clauses}

\newcommand{\mCmd}[1]{\textsf{#1}}

Although \gls{IP}~models are NP-complete to solve in general, it was discovered
that for a certain class of problem instances -- namely those based on
\glspl{Horn clause} -- an optimal solution can be found in linear
time~\cite{Hooker:1988}.
%
A \gls!{Horn clause} is a disjunctive Boolean formula which contains at most one
non-negated term.
%
This can also be phrased as a logical statement that has at most one conclusion.
%
For example, the statement
%
\begin{displaymath}
  \mCmd{if}~\mVar{p}_1~\mCmd{and}~\mVar{p}_2~\mCmd{then}~\mVar{p}_3
\end{displaymath}
%
can be expressed as \mbox{$\mNot\mVar{p}_1 \mOr \mNot\mVar{p}_2 \mOr
  \mVar{p}_3$}, which is a \gls{Horn clause}, as only $\mVar{p}_3$ is not
negated.
%
This can then easily be rewritten into the linear inequality
%
\begin{displaymath}
(1 - \mVar{p}_1) + (1 - \mVar{p}_2) + \mVar{p}_3 \ge 1.
\end{displaymath}
%
Moreover, statements that do not yield \glspl{Horn clause} in their current form
can often be rewritten so that they do.
%
For example,
%
\begin{displaymath}
  \mCmd{if}~\mVar{a}~\mCmd{then}~\mVar{b}~\mCmd{and}~\mVar{c}
\end{displaymath}
%
can be expressed as \mbox{$\mNot\mVar{a} \mOr \mVar{b} \mOr \mVar{c}$} and is
thus not a \gls{Horn clause} because it has more than one non-negated term.
%
But by rewriting it into
%
\begin{displaymath}
  \begin{array}{c}
    \mCmd{if}~\mVar{a}~\mCmd{then}~\mVar{b}\\
    \mCmd{if}~\mVar{a}~\mCmd{then}~\mVar{c}
  \end{array}
\end{displaymath}
%
the statement can now be expressed as \mbox{$\mNot\mVar{a} \mOr \mVar{b}$} and
\mbox{$\mNot\mVar{a} \mOr \mVar{c}$}, which are two valid \glspl{Horn clause}.

\textcite{Gebotys:1997} exploited this property in 1997 by developing an
\gls{IP}~model for \gls{TMS320C2x} -- a common \gls{DSP} at the time -- where
many of the \glspl{constraint} imposed by the target architecture,
\gls{instruction selection}, and \gls{register allocation}, and a part of the
\gls{instruction scheduling} problem, are expressed as \glspl{Horn clause}.
%
Using only \glspl{Horn clause} may require a larger number of \glspl{constraint}
than are otherwise needed, but \citeauthor{Gebotys:1997} claims that the number
is still manageable.
%
When compared against a then-contemporary industrial \gls{DSP} \gls{compiler},
\citeauthor{Gebotys:1997} demonstrated that an implementation based on \gls{IP}
yielded a performance improvement mean of 44\% for a select set of functions,
while attaining reasonable compilation times.
%
However, the solving time increased by orders of magnitude when
\citeauthor{Gebotys:1997} augmented the \gls{IP}~model with the complete set of
\glspl{constraint} for \gls{instruction scheduling}, which cannot be expressed
entirely as \glspl{Horn clause}.


\subsection{IP-Based Designs with Multi-output Instruction Support}

\citeauthor{LeupersMarwedel:1996}~\cite{LeupersMarwedel:1995,
  LeupersMarwedel:1996} expanded the work of \citeauthor{WilsonEtAl:1994} --
whose design is restricted to \glspl{pattern tree} -- by developing an
\gls{IP}-based \gls{instruction selector} which also supports
\gls{multi-output.ic} \glspl{instruction}.
%
In a paper from~:1996, \citeauthor{LeupersMarwedel:1996} describe a scheme where the
\glspl{pattern DAG} of \gls{multi-output.ic} \glspl{instruction} --
\citeauthor{LeupersMarwedel:1996} refer to these as \gls{complex.p} \glspl{pattern} --
are first decomposed into multiple \glspl{pattern tree} according to their
\glspl{RT}.
%
\glspl{RT}~are akin to \citeauthor{Fraser:1979}'s \glspl{RTL}~\cite{Fraser:1979}
(see \refSection{register-transfer-lists}), and essentially mean that each
observable effect gets its own \gls{pattern tree}.
%
Each individual \gls{RT} may in turn correspond to one or more
\glspl{instruction}, but unlike in \citeauthor{Fraser:1979}'s design this is not
strictly required.

Assuming the \gls{block DAG} has already been \glsshort{undagging}[ged], each
\gls{expression tree} is first optimally covered using \gls{IBurg}.
%
The \glspl{RT} are expressed as \glspl{rule} in an \gls{machine grammar} that
has been automatically generated from a \gls{machine description} written in
\glsunset{MIMOLA}\gls{MIMOLA}\glsreset{MIMOLA} (we will come back to this in
\refSection{modeling-entire-target-machines}).
%
Once \glspl{RT} have been selected, the \gls{expression tree} is reduced to a
\gls{tree} of \glspl{super node}, where each \gls{super node} represents a set
of \glspl{node} covered by some \gls{RT} that have been collapsed into a single
\gls{node}.
%
Since \gls{multi-output.ic} and \gls{disjoint-output.ic} \glspl{instruction}
implement more than one~\gls{RT}, the goal is now to cover the \gls{super node}
\gls{graph} using the \glspl{pattern} which are formed when the
\glspl{instruction} are modeled as~\glspl{RT}.
%
\citeauthor{LeupersMarwedel:1996} addressed this problem by applying a modified version
of the \gls{IP}~model by \citeauthor{WilsonEtAl:1994}.

But because the step of selecting \glspl{RT} to cover the \gls{expression tree}
is separate from the step which implements them with \glspl{instruction}, the
generated \gls{assembly code} is not necessarily optimal for the whole
\gls{expression tree}.
%
To achieve this property, the covering of \glspl{RT} and selection of
\glspl{instruction} must be done in tandem.


\subsection{IP-Based Designs with Disjoint-output Instruction Support}

\textcite{Leupers:2000:SIMD} later made a more direct extension of the \gls{IP}~model
by \citeauthor{WilsonEtAl:1994} in order to support \gls{SIMD.i} \glspl{instruction},
which belong to the class of \gls{disjoint-output.ic} \glspl{instruction}.
%
Described in a paper from~2000, \citeauthor{Leupers:2000:SIMD}'s design assumes every
\gls{SIMD.i} \gls{instruction} performs two operations, each of which takes a
disjoint set of input operands.
%
This is collectively called a \gls!{SIMD pair}, and \citeauthor{Leupers:2000:SIMD}
then extended the \gls{IP}~model with linear equations for combining \glspl{SIMD
  pair} into \gls{SIMD.i} \glspl{instruction} and defined the objective function
so as to maximize the use of \gls{SIMD.i} \glspl{instruction}.

In the paper, \citeauthor{Leupers:2000:SIMD} reports experiments where the use of
\gls{SIMD.i} \glspl{instruction} reduced code size by up to 75\% for the
selected test cases and \glspl{target machine}.
%
But since this technique assumes that each individual operation of the
\gls{SIMD.i} \glspl{instruction} is expressed as a single \gls{node} in the
\gls{block DAG}, it is unclear whether the method can be extended to more
complex \gls{SIMD.i} \glspl{instruction}, and whether it scales to larger
\glspl{program}.
%
\textcite{TanakaEtal:2003} later expanded \citeauthor{Leupers:2000:SIMD}'s work
for selecting \gls{SIMD.i} \glspl{instruction} while also taking the cost of
data transfers into account by introducing auxiliary transfer \glspl{node} and
transfer \glspl{pattern} into the \gls{block DAG}.


\subsection{Modeling the Pattern Matching Problem with IP}

In 2006, \textcite{BednarskiKessler:2006} developed an \gls{integrated.cg}
\gls{code generation} design where both \gls{pattern matching} and \gls{pattern
  selection} are solved using \glsdesc{IP}.
%
The scheme -- which later was applied by \textcite{ErikssonEtAl:2008}, and is
also described in an article by \textcite{ErikssonKessler:2012} -- is an
extension of their earlier work where \gls{instruction selection} had previously
more or less been ignored (see~\cite{KesslerBednarski:2001,
  KesslerBednarski:2002}).

In broad outline, the \gls{IP}~model assumes that a sufficient number of
\glspl{match} has been generated for a given \gls{block DAG}~$G$.
%
This is done using a \gls{pattern matching} heuristic that computes an upper
bound.
%
For each match~$m$, the \gls{IP}~model contains integer \glspl{variable} that:
%
\begin{itemize}
  \item map a \gls{pattern} \gls{node} in $m$ to a \gls{node} in $G$;
  \item map a \gls{pattern} \gls{edge} in $m$ to an \gls{edge} in $G$; and
  \item decide whether $m$ is used in the solution.
    %
    Remember that we may have an excess of \glspl{match}, so they cannot all be
    selected.
\end{itemize}
%
Hence, in addition to the typical linear equations we have seen previously for
enforcing coverage, this \gls{IP}~model also includes equations to ensure that
the selected \glspl{match} are valid \glspl{match}.

Implemented in a framework called \gls!{Optimist},
\citeauthor{BednarskiKessler:2006} compared their \gls{IP}~model against another
\gls{integrated.cg} \gls{code generation} design based on \glsdesc{DP}, which
was developed by the same authors (see \cite{KesslerBednarski:2001}) and has
nothing to do with the conventional \gls{DP}~algorithm by
\textcite{AhoEtAl:1989}).
%
\citeauthor{BednarskiKessler:2006} found that \gls{Optimist} substantially reduced
compilation time while retaining code quality, but for several test cases -- the
largest \gls{block DAG} containing only 33~\glspl{node} -- \gls{Optimist} failed
to generate any \gls{assembly code} whatsoever within the set time limit.
%
One reasonable cause could be that the \gls{IP}~model also attempts to solve
\gls{pattern matching} -- a problem which we have seen can be solved externally
-- and thus further exacerbates an already computationally difficult problem.


\section{Modeling Instruction Selection with CP}

\todo{replace use of ``CP model''}

Although \glsdesc{IP} allows auxiliary \glspl{constraint} to be included into
the \gls{IP}~model, they may be cumbersome to express as linear equations.
%
This issue can be alleviated by using \gls{CP}, which is another method for
solving combinatorial optimization problems (see for
example~\cite{RossiEtAl:2006} for an overview) but has more flexible modeling
capabilities compared to \gls{IP}.

In 1990, \textcite{BashfordLeupers:1999} pioneered the use of \glsdesc{CP} in \gls{code
  generation} by developing a \gls{CP}~model for \gls{integrated.cg} \gls{code
  generation} that targets \glspl{DSP} with highly irregular architectures (the
work is also discussed in~\cite{Leupers:2000:SIMD, LeupersBashford:2000}).
%
Like \citeauthor{LeupersMarwedel:1996}'s \gls{IP}-based design,
\citeauthor{BashfordLeupers:1999}'s first breaks down the \gls{instruction set} of the
\gls{target machine} into a set of \glspl!{RT} which are used to cover
individual \glspl{node} in the \gls{block DAG}.
%
As each \gls{RT} concerns specific \glspl{register} on the \gls{target machine},
the covering problem essentially also incorporates \gls{register allocation}.
%
The goal is then to minimize the cost of covering by combining multiple
\glspl{RT} that can be executed in parallel as part of some \gls{instruction}.

For each \gls{node} in the \gls{block DAG} a \gls{FRT} is introduced, which
basically embodies all \glspl{RT} that match a particular \gls{node} and is
formally defined as the following tuple:
%
\begin{displaymath}
  \mTuple*{
    \mathit{op},
    \mVar{d},
    [\mVar{u}_1, \ldots, \mVar{u}_n],
    \mVar{f},
    \mVar{c},
    \mVar{t},
    \mathit{CS}
  }.
\end{displaymath}
%
$\mathit{op}$ is the operation of the \gls{node}.
%
$\mVar{d}$ and \mbox{$\mVar{u}_1, \ldots, \mVar{u}_n$} are \glspl{variable}
representing the \glspl{storage location} of the result and the respective
inputs to the operation.
%
These are typically the \glspl{register} that can be used for the operation, but
also include \gls!{virtual.sl}[ \glspl{storage location}] which convey that the
value is produced as an intermediate result in a chain of operations (for
example, the multiplication term in a multiply-accumulate instruction is such a
result).
%
Then, for every pair of operations that are \gls{adjacent.n} in the \gls{block
  DAG}, a set of \glspl{constraint} is added to ensure that there exists a valid
data transfer between the \glspl{storage location} of $\mVar{d}$ and
$\mVar{u}_i$ if these are assigned to different \glspl{register}, or that both
are identical if one is a \gls{virtual.sl} \gls{storage location}.
%
$\mVar{f}$,~$\mVar{c}$, and $\mVar{t}$ are all \glspl{variable} which
collectively represent the \gls{ERI} that specifies at which functional unit the
operation will be executed~($\mVar{f}$); at what cost~($\mVar{c}$), which is the
number of execution cycles; and by which \gls{instruction} type~($\mVar{t}$).
%
A combination of a functional unit and an \gls{instruction} type is later mapped
to a particular \gls{instruction}.
%
Multiple \glspl{RT} can be combined into the same \gls{instruction} when the
destination of the result is a \gls{virtual.sl} \gls{storage location} by
setting $\mVar{c} = 0$ and letting the last \gls{node} in the operation chain
account for the required number of execution cycles.
%
The last entity, $\mathit{CS}$, is the set of \glspl{constraint} for defining
the range of values for the \glspl{variable} and the dependencies between
$\mVar{d}$ and~$\mVar{u}_i$, as well as other auxiliary \glspl{constraint} that
may be required for the \gls{target machine}.
%
For example, if the set of \glspl{RT} matching a \gls{node} consists of
%
\mbox{$%
  \mSet{
    \cVar{r}[c] = \cVar{r}[a] + \cVar{r}[b],
    \cVar{r}[a] = \cVar{r}[c] + \cVar{r}[b]
  }
$}, then the corresponding \gls!{FRT} becomes
%
\begin{displaymath}
  \mTuple*{
    +,
    \mVar{d},
    [\mVar{u}_1, \mVar{u}_2],
    \mVar{f},
    \mVar{c},
    \mVar{t},
    \mSet*{
      \mVar{d} \in \mSet{\cVar{r}[c], \cVar{r}[a]},
      \mVar{u}_1 \in \mSet{\cVar{r}[a], \cVar{r}[c]},
      \mVar{u}_2 = \cVar{r}[b],
      \mVar{d} = \cVar{r}[c] \mImp \mVar{u}_1 = \cVar{r}[a]
    }
  }\!.
\end{displaymath}
%
For brevity, I have omitted several details such as the \glspl{constraint}
concerning the~\gls{ERI}.

This \gls{CP}~model is then solved to optimality using a \gls{constraint
  solver}.
%
But since \gls{optimal.ps} \gls{cover}[ing] using \glspl{FRT} is NP-complete,
\citeauthor{BashfordLeupers:1999} applied heuristics to curb the complexity by splitting
the \gls{block DAG} into smaller pieces along \glspl{edge} where intermediate
results are shared, and then performing \gls{instruction selection} on each
\gls{expression tree} in isolation.

Although the \glspl{constraint} in \citeauthor{BashfordLeupers:1999}'s \gls{CP}~model
appear to be limited to involving only a single~\gls{FRT} at a time -- thus
hindering support for \gls{interdependent.ic} \glspl{instruction} --
\glsdesc{CP} in general seems like a promising tool for performing
\gls{instruction selection}.
%
As with \glsdesc{IP}, \glsdesc{CP} facilitates \gls{integrated.cg} and
potentially optimal \gls{code generation}.
%
In addition, it allows additional restrictions of the \gls{target machine} to be
included in the \gls{CP}~model, but without the need of expressing these
\glspl{constraint} as linear equations.
%
At the time of writing, however, the existing techniques for solving
\gls{IP}~models are more mature than those for solving \gls{CP}~models, which
potentially makes \glsdesc{IP} a more powerful method than \glsdesc{CP} for
solving \gls{instruction selection}.
%
Having said that, it is still unclear which technique of combinatorial
optimization -- which also includes \gls{SAT} and other methods -- is best
suited for \gls{instruction selection} (and code generation in general).


\subsection{Taking Advantage of Global Constraints}

So far we have discussed several techniques that apply \glsdesc{CP} for solving
the problems of \gls{pattern matching} and \gls{pattern selection} -- namely
those by \citeauthor{BashfordLeupers:1999} and \citeauthor{MartinEtAl:2009}.
%
Recently, \textcite{Beg:2013} introduced another \gls{CP}~model for
\gls{instruction selection} as well as new methods for improving solving.
%
For example, in order to reduce the search space, \citeauthor{Beg:2013} applied
conventional \gls{DP}-based techniques to compute an upper bound on the cost.
%
However, the \gls{CP}~model mainly deals with the problem of \gls{pattern
  matching} rather than \gls{pattern selection}.
%
Moreover, \citeauthor{Beg:2013} noticed only a negligible improvement (less than
1\%) in code quality compared to \gls{LLVM}, mainly because the \glspl{target
  machine} (\gls{MIPS} and \gls{ARM}) were simple enough that greedy heuristics
generate near-optimal \gls{assembly code}.  In addition, the \glspl{block DAG}
of the benchmark \glspl{program} were fairly
\gls{tree}-shaped~\cite{VanBeek:2014}, for which \gls{optimal.ps} code can be
generated in linear time.
%
In any case, none of these designs take advantage of a key feature of
\glsdesc{CP}, which is the use of \gls{global.c} \glspl{constraint}.
%
A \gls{global.c} \gls{constraint} is a restriction that is enforced
simultaneously over multiple \glspl{variable} and results in more search space
pruning than if it had been expressed using multiple \glspl{constraint} over
only a subset of the variables at a time.

Hence, when \textcite{FlochEtAl:2010} in 2010 adapted the \gls{CP}~model by
\citeauthor{MartinEtAl:2009} to support processors with reconfigurable cell fabric,
they replaced the method of \gls{pattern selection} with \glspl{constraint} that
are radically different from those incurred by \gls{unate covering}.
%
In addition, unlike in the case of \citeauthor{BashfordLeupers:1999}, the design by
\citeauthor{FlochEtAl:2010} applies the more direct form of \gls{pattern matching}
instead of first breaking down the \glspl{pattern} into \glspl{RT} and then
selecting \glspl{instruction} that combine as many \glspl{RT} as possible.

As described in their 2010~paper, \citeauthor{FlochEtAl:2010} use the \gls{global
  cardinality constraint} to enforce the requirement that every \gls{node} in
the \gls{block DAG} must be covered by exactly one \gls{pattern}.\!%
%
\footnote{%
  It is also possible to enforce \gls{pattern selection} through a global set
  covering developed by \textcite{Mouthuy:2007}, but I have not seen any
  implementation do so.%
}
%
\todo{rewrite}
%
This \gls{constraint}, which we will refer to as \mbox{$\mGCC(\mVar{i},
  \mVar{var}, \mVar{val})$}, enforces that exactly $\mVar{val}$ number of
\glspl{variable} from the set~$\mVar{var}$ assume the value~$\mVar{i}$, where
$\mVar{val}$ can either be a fixed value or represent another \gls{variable}.
%
Let us assume that every \gls{node} in the \gls{block DAG} has an associated
\gls{match set} containing all the \glspl{pattern} that may cover that
\gls{node}, and that each \gls{match}~$m$ appearing in that \gls{match set} has
been assigned a unique integer value~$i_m$ (if $m$ appears in multiple
\glspl{match set}, it is still given the same value).
%
We introduce a \gls{variable}~$\mathit{match}_n$ for each \gls{node}~$n$ in the
\gls{block DAG} to represent the \gls{match} selected to cover~$n$.
%
For each \gls{match}~$m$, we also introduce a \gls{variable}
\mbox{$\mVar{nodecount}[m] \in \mSet{0, \mCovers(m)}$}, where
\mbox{$\mCovers{m}$} is the number of \gls{pattern} \glspl{node} in~$m$.
%
We also define \mbox{$\mVar{mset}[m] = \bigcup_{n \in \mFunFont{nodes}(m)}
  \mathit{match}_n$} as the set of $\mathit{match}_n$ variables in which
\gls{match}~$m$ may appear, where $\mFunFont{nodes}(m)$ is the set of
\glspl{node} matched by~$m$.
%
With this we can express the \gls{constraint} that every \gls{node} in the
\gls{block DAG} must be covered exactly once as
%
\begin{displaymath}
  \forall m \in M :
  \mGCC(\mVar{i}[m], \mVar{mset}[m], \mVar{nodecount}[m]),
\end{displaymath}
%
where $M$ is the total set of \glspl{match}.
%
This may appear convoluted at first glance, but it is actually rather simple.
%
$\mVar{mset}[m]$ essentially provides the \glspl{node} in the \gls{block DAG}
that may be covered by \gls{match}~$m$, $\mVar{nodecount}[m]$ provides the
number of \glspl{node} covered by~$m$, and $\mGCC$ ensures that the relation
between the two holds.
%
But since the domain of $\mVar{nodecount}[m]$ is initially restricted to only
two values -- zero and the size of the \gls{pattern} -- it must be so that
either \emph{no} \glspl{node} are covered by~$m$, or \emph{all} \glspl{node} are
covered by~$m$.
%
To identify which \glspl{match} have been selected, we simply check whether
\mbox{$\mVar{nodecount}[m] \neq 0$} for every \gls{pattern}~$m$.
%
Since a \gls{variable} cannot be assigned more than one value, each \gls{node}
can only be covered by exactly one \gls{match} (remember that the
$\mathit{match}_n$ variable for \gls{node}~$n$ can appear in multiple
$\mVar{mset}[m]$ sets).
%
Hence this \gls{constraint} is more restrictive than that of \gls{unate
  covering}, which results in more propagation and thus reduces the search
space.

This \gls{CP}~model was also further extended by
\citeauthor{ArslanKuchcinski:2013}~\cite{ArslanKuchcinski:2013, ArslanKuchcinski:2014, Arslan:2016} to
accommodate \gls{VLIW} architectures and \gls{disjoint-output.ic}
\glspl{instruction}.
%
First, every \gls{disjoint-output.ic} \glspl{instruction} is split into multiple
\glspl!{subinstruction}, each modeled by a disjoint \gls{pattern}.
%
A generic \gls{subgraph isomorphism} algorithm is used to find all \glspl{match
  set}, and \gls{pattern selection} is then modeled as an instance of the
\gls{CP}~model with the additional \glspl{constraint} to schedule the
\glspl{subinstruction} such that they can be replaced by the original
\gls{disjoint-output.ic} \gls{instruction}.
%
\citeauthor{ArslanKuchcinski:2013}'s design therefore differs from the previous techniques
that we have seen before, where \glspl{match} of \gls{partial.p} \glspl{pattern}
are recombined into \glspl{match} of \gls{complex.p} \glspl{pattern} prior to
\gls{pattern selection} (see for example \textcite{ScharwaechterEtAl:2007},
\textcite{AhnEtAl:2009}, \citeauthor{ArnoldCorporaal:1999}~\cite{Arnold:1999,
  ArnoldCorporaal:1999, ArnoldCorporaal:2001}), as it allows these two problems to be
solved in tandem.
%
The design is also capable of accepting multiple, disconnected \glspl{block DAG}
as a single input.

However, a limitation inherent to the \gls{CP}~models applied by
\citeauthor{MartinEtAl:2009}, \citeauthor{FlochEtAl:2010}, and
\citeauthor{ArslanKuchcinski:2013} is that they do not model the necessary data transfers
between different \glspl{register class}.
%
This in turn means that the cost model is only accurate for \glspl{target
  machine} equipped with a homogeneous \gls{register} architecture, which could
compromise code quality for more complicated \glspl{target machine}.


\section{Other DAG-Based Approaches}

\subsection{More Genetic Algorithms}

Seemingly independently from the earlier work by \textcite{ShuEtAl:1996}
(discussed in \refAppendix{tree-covering}),
\citeauthor{LorenzEtAl:2001}~\cite{LorenzEtAl:2001, LorenzMarwedel:2004}
introduced in 2001 another technique where \glsdesc{GA}s are applied to
\gls{code generation}.
%
But unlike the design by \citeauthor{ShuEtal:1996}, the one by
\citeauthor{LorenzEtAl:2001} takes \glspl{block DAG} instead of \glspl{tree} as
input and also incorporates \gls{instruction scheduling} and \gls{register
  allocation}.
%
\citeauthor{LorenzEtAl:2001} recognized that contemporary \glspl{compiler} struggled
with generating efficient \gls{assembly code} for \glspl{DSP} equipped with very
few \glspl{register} and typically always spill the results of common
subexpressions to memory and reload them when needed.
%
Compared to optimal \gls{assembly code}, this may incur more memory accesses
than needed.

The design by \citeauthor{LorenzEtAl:2001} is basically an iterative process.
%
First, the operations within a \gls{block} are scheduled using \gls!{list
  scheduling}, which is a traditional method of scheduling (see for
example~\cite{RauFisher:1993}).
%
For every scheduled operation, a \gls{gene} is formulated to encode all the
possible decisions to take in order to solve the problems of \gls{instruction
  selection} and \gls{register allocation}.
%
These decisions are then taken over multiple steps using standard
\gls{GA}~operations, where the values are selected probabilistically.
%
In each step the \gls{gene} is mutated and crossed over in order to produce new,
hopefully better \glspl{gene}, and a \gls{fitness function} is applied to
evaluate each \gls{gene} in terms of expected execution time.
%
After a certain number of generations, the process stops and the best \gls{gene}
is selected.
%
Certain steps are also followed by a routine based on \glsdesc{CP} that prunes
the search space for the subsequent decisions by removing values which will
never appear in any valid \gls{gene}.
%
Although every \gls{gene} represents a single \gls{node} in the \gls{block DAG},
\gls{complex.p} \glspl{pattern} can still be supported through an additional
variable for selecting the \gls{instruction} type for the \gls{node}.
%
If \glspl{node} with the same \gls{instruction} type have been scheduled to be
executed on the same cycle, then they can be implemented using the same
\gls{instruction} during \gls{assembly code} emission.

\citeauthor{LorenzEtAl:2001} originally developed this technique in order to reduce
power usage of \gls{assembly code} generated for constrained \glspl{DSP}, and
later extended the design to also incorporate \gls{instruction compaction} and
\gls{address generation}.
%
Experiments indicate that the technique for a selected set of test cases
resulted in energy savings of 18 to 36\% compared to a traditional \gls{tree
  covering}-based \gls{compiler}, and reduced execution time by up to 51\%.
%
According to \citeauthor{LorenzEtAl:2001}, the major contribution to this reduction
is due to improved usage of \glspl{register} for common subexpression values,
which in turn leads to less use of power-hungry and long-executing memory
operations.
%
But due to the probabilistic nature of \gls{GA}, optimality cannot be
guaranteed, making it unclear how this technique would fare against other
\gls{DAG covering}-based designs which allow a more exhaustive exploration of
the search space.


\subsection{Extending Trellis Diagrams to DAGs}
\labelSection{trellis-diagrams-dags}

In 1998, \citeauthor{HanonoDevadas:1998}~\cite{HanonoDevadas:1998, Hanono:1999} proposed a
technique that is similar to \citeauthor{Wess:1992}'s use of \glspl{trellis
  diagram}, which we discussed in \refAppendix{tree-covering}.
%
Implemented in a system called \gls{Aviv}, \citeauthor{HanonoDevadas:1998}'s
\gls{instruction selector} takes a \gls{block DAG} as input and duplicates each
operation \gls{node} according to the number of functional units in the
\gls{target machine} on which that operation can run.
%
Special \gls!{split.n} and \gls!{transfer.n}[ \glspl{node}] are inserted before
and after each duplicated operation \gls{node} to allows the data flow to
diverge and then reconverge before passing to the next operation \gls{node} in
the \gls{block DAG}.
%
The use of \gls{transfer.n} \glspl{node} also allow the cost of transferring
data from one functional unit to another to be taken into account.
%
Similarly to the \gls{trellis diagram}, \gls{instruction selection} is thus
transformed to finding a path from the \gls{leaf} \glspl{node} in the \gls{block
  DAG} to its \gls{root} \gls{node}.
%
But differently from the \gls{optimal.ps}, \mbox{\gls{DP}-oriented} design of
\citeauthor{Wess:1992}, \citeauthor{HanonoDevadas:1998} applied a greedy
heuristic that starts from the \gls{root} \gls{node} and makes it way towards
the \glspl{leaf}.

Unfortunately, as in \citeauthor{Wess:1992}'s design, this technique assumes a
\mbox{1-to-1} mapping between the \glspl{node} in the \gls{block DAG} and the
\glspl{instruction} in order to generate efficient \gls{assembly code}.
%
In fact, the main purpose behind \gls{Aviv} was to generate efficient
\gls{assembly code} for \gls{VLIW} architectures, where the focus is on
executing as many \glspl{instruction} as possible in parallel.


\subsection{Hardware Modeling Techniques}
\labelSection{modeling-entire-target-machines}

In 1984, \textcite{Marwedel:1984} developed a retargetable system called
\gls!{MSS} for \gls{microcode generation},\!%
%
\footnote{%
  \Gls{microcode} is essentially the hardware language that processors use
  internally for executing \glspl{instruction}.
  %
  For example, \gls{microcode} controls how the \glspl{register} and program
  counter should be updated for a given \gls{instruction}.%
}
%
where a \gls{machine description} written in \gls!{MIMOLA}~\cite{Zimmermann:1979}
is used for modeling the entire data path of the processor, instead of just the
\gls{instruction set} as we have commonly seen.
%
This is commonly used for \glspl{DSP} where the processor is small but highly
irregular.
%
Although \gls{MSS} consists of several tools, we will concentrate on the
\gls{compiler} -- \gls!{MSSQ} -- as its purpose is most aligned with
\gls{instruction selection}.
%
\gls{MSSQ} was developed by \textcite{LeupersMarwedel:1998} as a faster version
of \gls!{MSSC}~\cite{NowakMarwedel:1989}, which in turn is an extension of the
\gls{tree}-based \gls!{MSSV}~\cite{Marwedel:1993}.

%\begin{figure}[b]
%  \centering%
%  % Copyright (c) 2018, Gabriel Hjort Blindell <ghb@kth.se>
%
% This work is licensed under a Creative Commons 4.0 International License (see
% LICENSE file or visit <http://creativecommons.org/licenses/by/4.0/> for a copy
% of the license).
%
\begingroup%
\figureFont\figureFontSize%
\setlength{\nodeSize}{4pt}%
\setlength{\nodeDist}{16pt}%
\begin{tikzpicture}[
    node/.append style={
      fill=black,
    },
    label/.style={
      nothing,
      outer sep=2pt,
      node distance=1mm,
    },
    dots label/.style={
      label,
      outer sep=1mm,
      node distance=0.5\nodeDist,
    },
    flow/.style={
      data flow,
      -
    },
  ]

  % Nodes
  \node [node] (root) {};
  \node [node, node distance=3.5\nodeDist, position=165 degrees from root] (r0)
        {};
  \node [node, node distance=3.5\nodeDist, position=15  degrees from root] (r1)
        {};
  \foreach \i in {0, 1} {%
    \node [node, position=120 degrees from r\i] (r\i load) {};
    \node [node, position=60  degrees from r\i] (r\i noload) {};
    \node [node, position=120 degrees from r\i load] (r\i loadi) {};
    \node [node, position=60  degrees from r\i load] (r\i loadct) {};
    \node [node, position=60  degrees from r\i noload] (r\i noloadct) {};
    \node [node, above=1.5\nodeDist of r\i loadct] (r\i loadctif) {};
    \node [node, right=2\nodeDist of r\i loadctif] (r\i noloadctif) {};
  }
  \node [node] at ($(r0) !.5! (r1)$) (r2) {};
  \node [node, above=2\nodeDist of r2 |- r0noloadctif] (alu) {};
  \node [node, node distance=8\nodeDist, position=170 degrees from alu]
        (viaplus) {};
  \node [node, node distance=8\nodeDist, position=10  degrees from alu]
        (dat) {};
  \node [node] at ($(viaplus) !.33! (dat)$) (plus) {};
  \node [node] at ($(viaplus) !.66! (dat)$) (sub) {};
  \foreach \i in {viaplus, plus, sub, dat} {%
    \node [node, node distance=1.5\nodeDist, position=40  degrees from \i]
          (\i ct) {};
    \ifthenelse{\equal{\i}{sub}}{%
      \node [node, node distance=1.5\nodeDist, position=140 degrees from \i]
            (\i i2) {};
    }{%
      \node [node, node distance=1.5\nodeDist, position=140 degrees from \i]
            (\i i1) {};
    }
    \ifthenelse{\equal{\i}{dat}}{%
      % Add nothing
    }{%
      \ifthenelse{\equal{\i}{sub}}{%
        \node [node] at ($(\i i2) !.5! (\i ct)$) (\i i1) {};
      }{%
        \node [node] at ($(\i i1) !.5! (\i ct)$) (\i i2) {};
      }
    }
    \node [node, above=1.5\nodeDist of \i ct] (\i ctif) {};
  }
  \node [node, above=2.5\nodeDist of plusi2] (top r1) {};
  \node [node, above=3.5\nodeDist of plusi1] (mux) {};

  % Labels
  \node [dots label, above=of r2] (r2 dots) {$\ldots$};
  \node [dots label, above=of mux] (mux dots) {$\ldots$};
  \node [label, left=of r0] {R0};
  \node [label, right=of r2] {R2};
  \node [label, right=of r1] {R1};
  \foreach \i in {r0, r1} {%
    \node [label, left=of \i load] {load};
    \node [label, right=of \i noload] {noload};
    \node [label, left=of \i loadi] {i};
    \node [label, right=of \i loadct] {ct};
    \node [label, right=of \i noloadct] {ct};

  }
  \foreach \i in {0, 1} {%
    \node [label, right=of r\i loadctif] {I.f.R\i};
    \node [label, right=of r\i noloadctif] {I.f.R\i};
  }
  \node [label, right=of alu, yshift=-1mm] {ALU};
  \node [label, left=of viaplus] {via($+$)};
  \node [label, left=of plus] {$+$};
  \node [label, right=of sub] {$-$};
  \node [label, right=of dat] {dat};
  \foreach \i in {viaplus, plus, sub, dat} {%
    \node [label, right=of \i i1] {i1};
    \ifthenelse{\equal{\i}{dat}}{%
      % Add nothing
    }{%
      \node [label, right=of \i i2] {i2};
    }
    \node [label, right=of \i ct] {ct};
    \node [label, right=of \i ctif] {I.f.ALU};
  }
  \node [label, above=of top r1] {R1};
  \node [label, left=of mux, yshift=-2mm] {MUX};

  % Edges
  \begin{scope}[flow]
    \foreach \i in {0, 1, 2} {%
      \draw (root) -- (r\i);
    }
    \foreach \i in {0, 1} {%
      \draw (r\i) -- (r\i load);
      \draw (r\i) -- (r\i noload);
      \draw (r\i load) -- (r\i loadi);
      \draw (r\i load) -- (r\i loadct);
      \draw (r\i noload) -- (r\i noloadct);
      \draw (r\i loadct) -- (r\i loadctif);
      \draw (r\i noloadct) -- (r\i noloadctif);
    }
    \draw (r0loadi) -- (r0loadi |- r0loadctif) -- (alu);
    \draw (r1loadi) -- (alu);
    \foreach \i in {viaplus, plus, sub, dat} {%
      \draw (alu) -- (\i);
      \draw (\i) -- (\i i1);
      \draw (\i) -- (\i ct);
      \ifthenelse{\equal{\i}{dat}}{%
        % Draw nothing
      }{%
        \draw (\i) -- (\i i2);
      }
      \draw (\i ct) -- (\i ctif);
    }
    \foreach \i in {viaplus, plus, sub, dat} {%
      \ifthenelse{\equal{\i}{viaplus}}{%
        \coordinate (halfmark) at ([xshift=\nodeDist] \i i1 |- mux);
      }{%
        \ifthenelse{\equal{\i}{plus}}{%
          \coordinate (halfmark) at (mux);
        }{%
          \coordinate (halfmark) at ([xshift=-\nodeDist] \i i1 |- mux);
        }
      }
      \draw (\i i1) -- ([yshift=\nodeDist] \i i1 |- viaplusctif) -- (halfmark)
        -- (mux);
    }
    \foreach \i in {viaplus, plus, sub} {%
      \ifthenelse{\equal{\i}{viaplus}}{%
        \coordinate (halfmark) at ([xshift=\nodeDist] \i i2 |- top r1);
      }{%
        \ifthenelse{\equal{\i}{plus}}{%
          \coordinate (halfmark) at (top r1);
        }{%
          \coordinate (halfmark) at ([xshift=-\nodeDist] \i i2 |- top r1);
        }
      }
      \draw (\i i2) -- (\i i2 |- viaplusctif) -- (halfmark) -- (top r1);
    }
    \draw (r2) -- (r2 dots);
    \draw (mux) -- (mux dots);
  \end{scope}
\end{tikzpicture}%
\endgroup%

%  \figCaption[The \glsentrytext{CO graph} of a simple processor]%
%    {The \glsentrytext{CO graph} of a simple processor containing an arithmetic
%     logic unit,
%     two data registers, a program counter, and a control store}[NowakMarwedel:1989]
%  \labelFigure{co-graph-example}
%\end{figure}

The \gls{MIMOLA} specification contains the processor \glspl{register} as well
as all the operations that can be performed on these \glspl{register} within a
single cycle.
%
From this specification, a hardware~\gls{DAG} called the \gls{CO graph} is
automatically derived.
%
An example is given in \refFigure{co-graph-example}.
%
A \gls{pattern matcher} then attempts to find \glspl{subgraph} within the
\gls{CO graph} to cover the \glspl{expression tree}.
%
Because the \gls{CO graph} contains explicit \glspl{node} for every
\gls{register}, a \gls{match} found on this \gls{graph} -- called a
\gls!{version} -- is also an assignment of \gls{program} variables (and
\glspl{temporary}) to \glspl{register}.
%
If a \gls{match} cannot be found (due to a lack of \glspl{register}), the
\gls{expression tree} will be rewritten by splitting assignments and inserting
additional \glspl{temporary}.
%
The process then backtracks and repeats in a recursive fashion until the entire
\gls{expression tree} is covered.
%
A subsequent process then selects a specific \gls{version} from each \gls{match
  set} and tries to schedule them so that they can be combined into
\glspl{bundle} for parallel execution.

Although \gls{microcode generation} is at a lower hardware level than
\gls{assembly code} generation -- which is usually what we refer to with
\gls{instruction selection} -- we see several similarities between the problems
that must be solved in each, and that is why it is included in this dissertation
(further examples include~\cite{BalakrishnanEtAl:1986, MahmoodEtAl:1990,
  LangevinCerny:1993}).
%
In \refAppendix{graph-covering} we will see another design that also models the
entire processor but applies a more powerful technique.


\section{Summary}

In this appendix, we have investigated several methods that rely on the
\gls{principle} of \gls{DAG covering}, which is a more general form of \gls{tree
  covering}.
%
Operating on \glspl{DAG} instead of \glspl{tree} has several advantages.
%
Most importantly, common subexpressions can be directly modeled, and a larger
set of \glspl{instruction} -- including \gls{multi-output.ic} and
\gls{disjoint-output.ic} \glspl{instruction} -- can be supported and exploited
during \gls{instruction selection}, leading to improved performance and reduced
code size.
%
Consequently, techniques based on \gls{DAG covering} are today one of the most
widely applied methods for \gls{instruction selection} in modern
\glspl{compiler}.

The ultimate cost of transitioning from \glspl{tree} to \glspl{DAG}, however, is
that \gls{optimal.ps} \gls{pattern selection} can no longer be achieved in
linear time as it is NP-complete.
%
At the same time, \glspl{DAG} are not expressive enough to allow the proper
modeling of all aspects featured in the \glspl{program} and \glspl{instruction}.
%
For example, statements such as \mbox{for loops} incur \glspl{loop} in the
\gls{graph} representing the \gls{program}, restricting \gls{DAG covering} to
the scope of \glspl{block} and excluding the modeling of \gls{inter-block.ic}
\glspl{instruction}.
%
Another disadvantage is that optimization opportunities for storing
\gls{program} variables and \glspl{temporary} in different forms and at
different locations across the \gls{function} are forfeited.

In the next appendix, we will discuss the last and most general \gls{principle}
of \gls{instruction selection}, which addresses some of the aforementioned
deficiencies of \gls{DAG covering}.
