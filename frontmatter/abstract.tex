% Copyright (c) 2017, Gabriel Hjort Blindell <ghb@kth.se>
%
% This work is licensed under a Creative Commons 4.0 International License (see
% LICENSE file or visit <http://creativecommons.org/licenses/by/4.0/> for a copy
% of the license).

\begin{abstract}
  In \gls{code generation}, \gls{instruction selection} chooses
  \glspl{instruction} to implement a given \gls{program} under compilation,
  \gls{global code motion} moves computations from one part of the \gls{program}
  to another, and \gls{block ordering} places \gls{program} \glspl{block} in a
  consecutive sequence.
  %
  \Gls{local.is} \gls{instruction selection} chooses \glspl{instruction} one
  \gls{program} \gls{block} at a time while \gls{global.is} \gls{instruction
    selection} does so for the entire \gls{function}.
  %
  This dissertation introduces a new approach called \gls!{universal instruction
    selection} that integrates \gls{global.is} \gls{instruction selection} with
  \gls{global code motion} and \gls{block ordering}.
  %
  By doing so, it addresses limitations of existing \gls{instruction selection}
  techniques that fail to exploit many of the \glspl{instruction} provided by
  modern processors.

  To handle the combinatorial nature of these problems, the approach is based on
  \glsdesc{CP}, a combinatorial optimization method.
  %
  It relies on a novel model that is simpler and more flexible compared to the
  techniques used in modern \glspl{compiler} and that captures crucial features
  ignored by other combinatorial approaches.
  %
  The dissertation also proposes extensions to the model for integrating
  \gls{instruction scheduling} and \gls{register allocation}, two other
  important problems of \gls{code generation}.

  The model is enabled by a novel, \gls{graph}-based representation that unifies
  data and control flow for entire \glspl{function}.
  %
  The representation is crucial for integrating \gls{instruction selection} with
  \gls{global code motion} and for modeling sophisticated \glspl{instruction},
  whose behavior contains both data and control flow, as \glspl{graph}.

  Through experimental evaluation, \gls{universal instruction selection} is
  demonstrated to handle architectures with a rich \gls{instruction set} and
  scales up to \glspl{function} with hundreds of \glspl{operation}.
  %
  For these \glspl{function}, it generates code of equal or better quality
  compared to the state of the art.
  %
  The dissertation also demonstrates that there is sufficient data parallelism
  to be exploited through selection of
  \glsunset{SIMD.i}\gls{SIMD.i}\glsreset{SIMD.i} \glspl{instruction} and that
  this exploitation benefits from \gls{global code motion}.
  %
  With these results, it is argued that \glsdesc{CP} is a flexible, practical,
  competitive, and extensible approach for combining \gls{global.is}
  \gls{instruction selection}, \gls{global code motion}, and \gls{block
    ordering}.
\end{abstract}
