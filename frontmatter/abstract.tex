% Copyright (c) 2017, Gabriel Hjort Blindell <ghb@kth.se>
%
% This work is licensed under a Creative Commons 4.0 International License (see
% LICENSE file or visit <http://creativecommons.org/licenses/by/4.0/> for a copy
% of the license).

\begin{abstract}
  In code generation, instruction selection chooses instructions to implement a
  given program under compilation, global code motion moves computations from
  one part of the program to another, and block ordering places program blocks
  in a consecutive sequence.
  %
  Local instruction selection chooses instructions one program block at a
  time while global instruction selection does so for the entire function.
  %
  This dissertation introduces a new approach called \emph{universal instruction
    selection} that integrates global instruction selection with global code
  motion and block ordering.
  %
  By doing so, it addresses limitations of existing instruction selection
  techniques that fail to exploit many of the instructions provided by modern
  processors.

  To handle the combinatorial nature of these problems, the approach is based on
  constraint programming, a combinatorial optimization method.
  %
  It relies on a novel model that is simpler and more flexible compared to the
  techniques used in modern compilers and that captures crucial features ignored
  by other combinatorial approaches.
  %
  The dissertation also proposes extensions to the model for integrating
  instruction scheduling and register allocation, two other important problems
  of code generation.

  The model is enabled by a novel, graph-based representation that unifies data
  and control flow for entire functions.
  %
  The representation is crucial for integrating instruction selection with
  global code motion and for modeling sophisticated instructions, whose behavior
  contains both data and control flow, as graphs.

  Through experimental evaluation, universal instruction selection is
  demonstrated to handle architectures with a rich instruction set and to scale
  up to medium-sized functions.
  %
  For these functions, it generates code of equal or better quality compared to
  the state of the art.
  %
  The dissertation also demonstrates that there is sufficient data parallelism
  to be exploited through selection of SIMD instructions and that this
  exploitation benefits from global code motion.
  %
  With these results, it is argued that constraint programming is a flexible,
  practical, competitive, and extensible approach for combining global
  instruction selection, global code motion, and block ordering.
\end{abstract}
