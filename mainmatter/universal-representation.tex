% Copyright (c) 2017, Gabriel Hjort Blindell <ghb@kth.se>
%
% This work is licensed under a Creative Commons 4.0 International License (see
% LICENSE file or visit <http://creativecommons.org/licenses/by/4.0/> for a copy
% of the license).

\chapter{Universal Representation}
\labelChapter{universal-representation}

This chapter introduces the \gls{universal representation}, which is used in our
approach for modeling \glspl{function} and \glspl{instruction}.
%
Based on the conclusions drawn in
\refChapter{existing-isel-techniques-and-reps}, we begin in
\refSection{uni-rep-design-requirements} with listing the design requirements
that must be fulfilled by a graph-based representation.
%
From these requirements, we design the \gls{universal representation} in
\refSection{uni-rep-program-rep} and describe how it is used for modeling
\glspl{function}.
%
We then do the same for \glspl{instruction} in \refSection{uni-rep-instr-rep}.
%
In \refSection{uni-rep-pattern-matching}, we describe how to perform
\gls{pattern matching} with the \gls{universal representation}.
%
In \refSection{uni-rep-ir-comparison}, we compare \gls{universal representation}
with other, exising \glspl{sea-of-nodes IR}.
%
Lastly, a summary is given in \refSection{uni-rep-summary}.


\section{Design Requirements}
\labelSection{uni-rep-design-requirements}

As discussed in \refChapter{existing-isel-techniques-and-reps}, in order to
address the limitations of existing approaches we necessitate a \gls{constraint
  model} capable of capturing the problems described in
\refSection{intro-motivation}.
%
To this end, we require a \gls{graph}-based representation that fulfills the
following requirements:
%
\def\typesetReq#1{\emph{#1}}%
\begin{requirements}
  \item \labelRequirement{uni-rep-capture-data-control}
    \typesetReq{It must capture the data and control flow of an entire
      \gls{function}.}
    %
    This is needed for modeling \gls{global.is} \gls{instruction selection},
    which demands access to the entire \gls{function} under compilation.
    %
    This is also needed for uniform selection of data and control
    \glspl{instruction}, which requires both data and control flow must be
    captured in a single \gls{graph}.
  \item \labelRequirement{uni-rep-explicit-blocks}
    \typesetReq{\Glspl{block} must be explicitly represented as \glspl{node}.}
    %
    This is needed for \gls{pattern matching}, where we must not be allowed to
    match \glspl{pattern} whose control flow is inconsistent with the
    \gls{function graph}.
    %
    For example, assume the \gls{pattern} derived from a saturated addition
    \gls{instruction}.
    %
    Such a \gls{pattern} will consist of three \glspl{block}~$b_1$, $b_2$, and
    $b_3$, where $b_1$ represents the \gls{instruction}'s point of entry, $b_2$
    represents the part where clamping is performed, and $b_3$ represents the
    \gls{instruction}'s point of exit.
    %
    The control flow in this \gls{pattern} will be such that there two
    conditional jumps from $b_1$ to either $b_2$ or $b_3$, and an unconditional
    jump from $b_2$ to $b_3$.
    %
    To be matched, part of the \gls{function graph} must exhibit the same
    control flow structure.
    %
    This is also useful for modeling \gls{global code motion} as each such
    \gls{node} will incur a \gls{variable} in the \gls{constraint model}.
  \item \labelRequirement{uni-rep-explicit-ops}
    \typesetReq{Data and control \glspl{operation} must be explicitly
      represented as \glspl{node}.}
    %
    This is to retain the notion of \gls{cover}[age] and to uniformly treat
    \glspl{instruction} that operate on data or control flow.
  \item \labelRequirement{uni-rep-explicit-values}
    \typesetReq{Values produced and used by the \glspl{operation} must also be
      explicitly represented as \glspl{node}}
    %
    This is useful for modeling \gls{global code motion} and \gls{data copying}
    as every such \gls{node} will incur a \gls{variable} in the \gls{constraint
      model}.
  \item \labelRequirement{uni-rep-exactly-one-inbound-edge-for-values}
    \typesetReq{In a \glspl{function graph}, every \gls{node} representing a
      value must have exactly one inbound \gls{data-flow edge}.}
    %
    This ensures that every value has exactly one \gls{match} defining that
    value, which is helpful when modeling \gls{global code motion}.
  \item \labelRequirement{uni-rep-fixed-operations}
    \typesetReq{The \gls{block} in which a particular \gls{operation} in the
      \gls{function} is to be performed must not be fixed.}
    %
    Without this, \gls{global code motion} is not possible.
  \item \labelRequirement{uni-rep-preserve-semantics}
    \typesetReq{Data \gls{operation} must be forbidden to be placed in
      \glspl{block} that will break \gls{program} semantics.}
    %
    This is needed to ensure correctness when performing \gls{global code
      motion}.
  \item \labelRequirement{uni-rep-ssa}
    \typesetReq{The representation must be based on \gls{SSA}.}
    %
    This is needed in modeling \gls{global code motion} as it explicitly
    states which values must be defined in which \glspl{block} in order to
    preserve \gls{program} semantics.
    %
    This is also useful for practical reasons as most \glspl{IR} used in modern
    \glspl{compiler} are already based on \gls{SSA}.
\end{requirements}

While there exists many \gls{graph}-based representations (see
\cite{StanierWatson:2013} for a survey), most fulfill only some of the
requirements but not all.
%
Consequently, a new representation has to be designed.


\section{Program Representation}
\labelSection{uni-rep-program-rep}

The new representation, called \gls!{universal representation}, is essentially a
combination of two existing representations -- the \gls{SSA graph} and the
\gls{control-flow graph}%
%
\footnote{%
  A \gls!{control-flow graph} is a \gls{graph} where each \gls{node} represents
  a \gls{block} in the \gls{function} and each \gls{edge} represents a jump from
  one \gls{block} to another.
  %
  \Glspl{edge} representing conditional jumps are labeled with a Boolean
  value indicating under which conditions the jump is taken.
  %
  An example is given in \refFigure{function-example-cf-graph}.
}
%
-- which have been extended to fulfill the missing requirements and then merged
into a single \gls{graph}.
%
This makes for a simple construction process as the \glsshort{control-flow
  graph} and \glspl{SSA graph} are already used inside principally all modern
\glspl{compiler}.


\paragraph{Capturing Control Flow}

We start with the \gls{control-flow graph}.
%
As it captures the control flow for an entire \gls{function},
\refRequirement{uni-rep-capture-data-control} is already partially fulfilled.
%
\refRequirement{uni-rep-explicit-blocks} is also fulfilled since \glspl{block}
in the \gls{control-flow graph} are already represented as \glspl{node}, which
are called \glspl!{block node}.
%
In contexts where is no risk of confusion, the terms \emph{\glspl{block}} and
\emph{\glspl{block node}} can be interchanged.

To partially achieve \refRequirement{uni-rep-explicit-ops}, we insert
\glspl!{control node} representing \glspl{operation} that change the control
flow from on \gls{block} to another and redirect the \glspl{edge} such that
control flows through these \glspl{node}.
%
For example, in the \gls{control-flow graph} shown in
\refFigure{function-example-cf-graph}, \glspl{control node} representing
unconditional branches are inserted along the \glspl{edge} between the
\irBlock{entry} and \irBlock{head} \glspl{node} and between the \irBlock{body}
and \irBlock{head} \glspl{node}.
%
\begin{filecontents*}{uf-example-ssa.c}
int factorial(int $\irVar{n}[1]$) {
  entry:
    int $\irVar{f}[1]$ = 1;
  head:
    int $\irVar{f}[2]$ = $\mPhi$($\irVar{f}[1]$:entry, $\irVar{f}[3]$:body);
    int $\irVar{n}[2]$ = $\mPhi$($\irVar{n}[1]$:entry, $\irVar{n}[3]$:body);
    bool b = $\irVar{n}[2]$ <= 1;
    if b goto end;
  body:
    int $\irVar{f}[3]$ = $\irVar{f}[2]$ * $\irVar{n}[2]$;
    int $\irVar{n}[3]$ = $\irVar{n}[2]$ - 1;
    goto head;
  end:
    return $\irVar{f}[2]$;
}
\end{filecontents*}%
%
\begin{figure}
  \centering

  \begin{minipage}{55mm}
    \centering%
    \subcaptionbox{Function in SSA form\labelFigure{function-example-ssa-c}}%
                  {%
                    \lstinputlisting[language=c,mathescape,morekeywords={bool}]%
                                    {uf-example-ssa.c}%
                  }%
  \end{minipage}%
  \hfill%
  \begin{minipage}{67mm}
    \centering%
    \subcaptionbox{%
                    Control-flow graph%
                    \labelFigure{function-example-cf-graph}%
                  }%
                  [32mm]%
                  {%
                    \input{figures/universal-representation/%
                      control-flow-graph-example}%
                  }

    \vspace{\baselineskip}

    \subcaptionbox{SSA graph\labelFigure{function-example-ssa-graph}}%
                  {../existing-isel-techniques-and-reps/sea-of-nodes-example-ssa-graph.tex}
  \end{minipage}

  \caption[Function example used in describing the program representation]%
          {%
            Running example of a function and its corresponding control-flow and
            SSA graph, which will be used in describing the program
            representation%
          }
  \labelFigure{function-example}
\end{figure}%
%
\begin{figure}
  \centering

  \mbox{}%
  \hfill%
  \subcaptionbox{%
                  Extended control-flow graph%
                  \labelFigure{uf-graph-example-extended-cf-graph}%
                }%
                [46mm]%
                {%
                  \input{figures/universal-representation/%
                    uf-graph-example-control-part}%
                }%
  \hfill%
  \subcaptionbox{%
                  Extended SSA graph%
                  \labelFigure{uf-graph-example-extended-ssa-graph}%
                }%
                {%
                  \input{%
                    figures/universal-representation/%
                    uf-graph-example-data-part%
                  }%
                }%
  \hfill%
  \mbox{}

  \vspace{\betweensubfigures}

  \subcaptionbox{%
                  Universal function graph%
                  \labelFigure{uf-graph-example-full-graph}%
                }%
                {%
                  \parbox{\textwidth}{%
                    \parbox[b]{32mm}{%
                      \input{figures/universal-representation/%
                        uf-graph-example-control-part}%
                    }%
                    \hfill%
                    \parbox[b]{63mm}{%
                      \input{%
                        figures/universal-representation/%
                        uf-graph-example-data-part%
                      }%
                      \vspace*{1cm}%
                    }%
                    \input{%
                      figures/universal-representation/%
                      uf-graph-example-cross-edges%
                    }%
                  }%
                }%

  \caption[%
            Example of a universal function graph, built from
            \refFigure{function-example}%
          ]%
          {%
            Example of a universal function graph, built from the \gls{function}
            shown in \refFigure{function-example}.
            %
            Thick-lined diamonds, boxes, and arrows represent control nodes,
            block nodes, and control-flow edges, respectively.
            %
            Thin-lined circles, boxes, and arrows represent computation nodes,
            value nodes, and data-flow edges, respectively.
            %
            Dotted lines represent definition edges%
          }
  \labelFigure{uf-graph-example}
\end{figure}%
%
For the conditional control flow originating from the \irBlock{head}
\gls{block}, a \gls{control node} representing a conditional branch is inserted
and connected to the \irBlock{head} \gls{node}, and the labeled \glspl{edge} are
redirected to the new \gls{node}.
%
Lastly, a \gls{control node} representing \gls{function} return is inserted and
connected to the \irBlock{end} \gls{node}.
%
This results in the \gls{graph} shown in
\refFigure{uf-graph-example-extended-cf-graph}.

An invariant here is that each \gls{control node} has exactly one \gls{edge}
flowing \emph{from} a \gls{block node}, and each \gls{block node} has exactly
one \gls{edge} flowing \emph{to} a \gls{control node}.
%
In other words, every control \gls{operation} belongs to exactly one
\gls{block}, and every \gls{block} has exactly one point where changes in
control occur.
%
This also means that the extended \gls{control-flow graph} forms a bipartite
\gls{graph}, with \glspl{block node} on one end and \glspl{control node} on the
other.


\paragraph{Capturing Data Flow}

We continue with the \gls{SSA graph}.
%
As it captures the data flow for an entire \gls{function} and represents data
\glspl{operation} as \glspl{node} -- we call these \glspl!{computation node} --
the remaining parts of \refRequirement{uni-rep-capture-data-control} and
\refRequirement{uni-rep-explicit-ops} are fulfilled.
%
\refRequirement{uni-rep-ssa} is inherently fulfilled as the \gls{SSA graph}
requires the \gls{function} to be in \gls{SSA}~form.

To achieve \refRequirement{uni-rep-explicit-values}, we insert \glspl!{value
  node} representing the entities produced and used by the data
\glspl{operation} and redirect the \glspl{edge} in same fashion as when
extending the \gls{control-flow graph}.
%
\Glspl{node} representing \gls{function} return are removed as these are already
represented in the extended \gls{control-flow graph}.
%
Using the \gls{SSA graph} shown in \refFigure{function-example-ssa-graph} as
example, this results in the \gls{graph} shown in
\refFigure{uf-graph-example-extended-ssa-graph}.

Note that at this point, the invariant specified in
\refRequirement{uni-rep-exactly-one-inbound-edge-for-values} that every
\gls{value node} has exactly one inbound \gls{data-flow edge} is broken, but
this will be addressed shortly.


\paragraph{Combining The Graphs}

We now connect the two extended \glspl{graph} together.
%
First, \glspl{data-flow edge} are inserted to connect control \glspl{operation}
with the values used by these \glspl{operation}.
%
In the case of our running example, such \glspl{edge} are added from
values~\irVar{b} and~\irVar{f}[2] to the {\irFont\irCondBrText} and
{\irFont\irRetText} \glspl{operation}, respectively.

To achieve the invariant specified in
\refRequirement{uni-rep-exactly-one-inbound-edge-for-values}, \glspl{data-flow
  edge} are also inserted from the \glsshort{entry block} \gls{block node} to
each \gls{value node} representing constants and function arguments.
%
Intuitively, this means that such values are produced at the point of entry to
the \gls{function}.
%
Like with the extended \gls{control-flow graph}, the extended \gls{SSA graph}
also forms a bipartite \gls{graph}, with \glspl{value node} on one end and
\glspl{computation node} on the other.

Since there are no \glspl{edge} connecting \glspl{computation node} with
\glspl{block node}, the assignment of data \glspl{operation} to \glspl{block} is
free, thus fulfilling \refRequirement{uni-rep-fixed-operations}.
%
This alone, however, permits \glspl{operation} to be moved to \glspl{block} that
will break \gls{program} semantics.
%
For example, assume the code snippet and corresponding, extended
\glsshort{control-flow graph} and \glspl{SSA graph} shown in
\refFigure{preserving-semantics-example}.
%
\begin{filecontents*}{preserving-semantics-code.c}
  $\ldots$
  int $\irVar{x}[1]$ = $\ldots$;
check:
  bool b = $\ldots$;
  if b goto dec;
inc:
  int $\irVar{x}[2]$ = $\irVar{x}[1]$ + 1;
  goto merge;
dec:
  int $\irVar{x}[3]$ = $\irVar{x}[1]$ - 1;
join:
  int $\irVar{x}[4]$ = $\mPhi$($\irVar{x}[2]$:inc, $\irVar{x}[3]$:dec);
\end{filecontents*}
%
\begin{figure}
  \centering

  \mbox{}%
  \hfill%
  \subcaptionbox{%
                  Code snippet%
                  \labelFigure{preserving-semantics-example-code-snippet}%
                }{%
                  \lstinputlisting[language=c,mathescape]%
                                  {preserving-semantics-code.c}%
                }%
  \hfill\hfill\hfill%
  \subcaptionbox{%
                  UF subgraph%
                  \labelFigure{preserving-semantics-graph}%
                }{%
                  \input{figures/universal-representation/%
                    preserving-semantics-example-graph}%
                }%
  \hfill%
  \mbox{}

  \caption[Example illustrating the need for definition edges]%
          {%
            Example illustrating the need for definition edges to prevent
            certain operations from being moved into blocks that will break
            program semantics%
          }
  \labelFigure{preserving-semantics-example}
\end{figure}
%
In the original code, the addition should be performed in the \irBlock{inc}
\gls{block} while the subtraction should be performed in the \irBlock{dec}
\gls{block}.
%
But according to the \gls{graph}, swapping the placement of these
\glspl{operation} would be considered a valid move, which clearly results in a
different \gls{program}.
%
We recognize that such problems occur exactly in situations where a value is
expected to be produce in a particular \gls{block}, which are captured by the
\glspl{phi-function}.
%
Consequently, for each value-\gls{block} pair \mbox{$\mPair{v}{b}$} appearing as
argument to a \gls{phi-function}, we add a \gls!{definition edge} between the
corresponding \glsshort{value node} and \gls{block node}.
%
This forces $v$ to be produced in $b$, which in turn forbids the \gls{operation}
producing $v$ from being moved out of $b$.
%
Hence \refRequirement{uni-rep-preserve-semantics} is achieved.

Lastly, for convenience we prevent \glspl{phi-function} from being moved by
inserting a \gls{definition edge} between the value produced by the
\gls{phi-function} and the \gls{block} wherein the \gls{phi-function} resides in
the \gls{IR}.
%
This results in the \gls{graph} shown in
\refFigure{uf-graph-example-full-graph},\!%
%
\footnote{%
  In this case, the \glspl{definition edge} from the \irBlock{entry} \gls{node}
  to the \irVar{n}[1] and \irVar{1} \glspl{node} are redundant since the
  \glspl{data-flow edge} are sufficient to force these values to be produced in
  the \irBlock{entry} \gls{block}.
}
%
which is called \gls!{UF graph}.


\paragraph{Refining the Notion of Coverage}

Since new \glspl{node} have been introduced, we must refine the definition of
\gls{cover}[age] to apply for \glspl{UF graph}.
%
Hence, a subset~\mbox{$C \subseteq M$}, where $M$ is a \gls{match set},
\gls!{cover}[s] a \gls{UF graph}~$G$ if every \gls{operation} in $G$, where an
\gls!{operation} denotes either a \glsshort{computation node} or \gls{control
  node}, appears in exactly one match from~$C$.


\subsection{Representing Constants As Single Or Multiple Nodes}

Duplicated constants may either be represented using individual \glspl{value
  node} (as in \refFigure{uf-graph-example-full-graph}) or through a single
\gls{value node} (as in \refFigure{preserving-semantics-graph}).
%
The former is simpler from a \gls{code generation} perspective, but may result
in redundant \glspl{instruction} where the same constant is needlessly reloaded.
%
This can be avoided by using the latter together with a technique to be
described in \refSection{modeling-value-reuse} in the next chapter, but this
also necessitates the \gls{UF graph} to be transformed and extended with
additional \glspl{operation} in order to guarantee correctness.
%
For example, assume a \gls{function} containing two \glspl{phi-function}
\mbox{\irFont \irPhi{$\ldots$, 1:\irBlock{a}, $\ldots$}} and \mbox{\irFont
  \irPhi{$\ldots$, 1:\irBlock{b}, $\ldots$}}.
%
If the constant~\irVar{1} is represented using a single \gls{value node}, then
the \gls{value node} will have two \glspl{definition edge}, one from
\gls{block}~\irBlock{a} and another from \gls{block}~\irBlock{b}.
%
Since a value cannot be defined in two \gls{block} simultaneously, there exist
no \gls{solution} for this \gls{UF graph}.
%
This problem is fixed by applying \gls{copy extension}, which will be described
in \refChapter{constraint-model}.


\subsection{Data Types of Values}

Both \glspl{program} and \glspl{instruction} expect their argument values to be
represented in a specific representation.
%
For example, in integer arithmetics the values are typically represented using
signed or unsigned two's complement values of a certain width.
%
It is also common that the result is given using the same representation as the
argument values.
%
As this could lead to overflow, mechanisms are often in place to detect such
occurrances.
%
However, if the selected \gls{instruction} produces results of wider bit width
than specified in the \gls{function graph}, the expected overflow may not occur.
%
Consequently, for both \glsshort{function graph} and \glspl{pattern graph} the
data type of a value is also specified in the corresponding \gls{value node}.
%
For integers this entails the value's bit width, and two values are compatible
if they have the same bit width.
%
\Glspl{value node} representing constants also have a value range, which is a
singleton for constants appearing in the \gls{UF graph}.
%
A constant~$c$ is compatible with another constant~$d$ if the value range of~$d$
is a subset of the value range of~$c$.


\subsection{Common Subexpression Elimination}

Common subexpressions may appear as part of legalizing the \gls{function} before
passing it to the \gls{instruction selector}.
%
For example, in \gls{LLVM} there is an operation called {\instrFont
  getelementptr} which takes care of computing the address when accessing an
array element or object field.
%
To construct the corresponding \gls{UF graph}, these operations first need to be
lowered into a series of additions and multiplications.
%
However, if two or more {\instrFont getelementptr}s compute the same address
then these expressions will be duplicated, thus resulting in redundant
\glspl{instruction}.
%
Consequently, such common subexpressions should be eliminated after having
performed legalization but before constructing the \gls{UF graph}.


\subsection{Handling Side Effects}

The representation described thus far is sufficient for handling
\glspl{operation} without side effects, such as arithmetic computations.
%
\Glspl{operation} with side effects, such as memory \glspl{operation} and
function calls, require additional \gls{graph} structures as they may implicitly
depend on one another.
%
These dependencies are captured using \glspl!{state node}.
%
A side-effect \gls{operation} takes, in addition to \glspl{value node}, exactly
one \gls{state node} as input and produces another \gls{state node}.
%
Hence, if an \glspl{operation}~$o$ implicitly depends on another
\gls{operation}~$p$, then $o$ takes as input the \gls{state node} produced by
$p$.

We create these dependencies as follows.
%
For each \gls{block} in the \gls{function}, a new \gls{state node} is created
and remembered as the \emph{last} \gls{state node}.
%
When a \gls{computation node} with side effects is created, a \gls{state-flow
  edge} is also inserted from the last \gls{state node} to this \gls{node}.
%
A new \gls{state node} is then created and set as last, and another
\gls{state-flow edge} is inserted from the \gls{computation node} to the new
\gls{state node}.
%
Once all \glspl{operation} in the \gls{block} has been processed, the first and
last \glspl{state node} are connected to the \gls{block node} through a
\gls{state-flow edge} and a \gls{definition edge}, respectively (note that,
unlike with \glspl{value node}, the \gls{state-flow edge} must not be drawn from
the \glsshort{entry block} \gls{block node}).
%
Hence the same invariant for \glspl{value node} also apply for \glspl{state
  node}.
%
Because the first and the last \gls{state node} must both be defined within the
\gls{block}, this effectively forbids \glspl{operation} with side effects from
being moved out of the \gls{block}, which could break \gls{program} semantics.
%
An example is shown in \refFigure{uni-rep-side-effects-example}.
%
\begin{filecontents*}{side-effects-example-ir.c}
block:
  $\ldots$
  store p, $\ldots$
  call foo, p
  store p, $\ldots$
\end{filecontents*}
%
\begin{figure}
  \centering%
  \mbox{}%
  \hfill%
  \subcaptionbox{Code snippet\labelFigure{uni-rep-side-effects-example-ir}}%
                [25mm]%
                {%
                  \lstinputlisting[language=c,mathescape]%
                                  {side-effects-example-ir.c}%
                }%
  \hfill%
  \subcaptionbox{%
                  UF subgraph.
                  %
                  Gray boxes represent state nodes, and dashed lines represent
                  state-flow edges%
                  \labelFigure{uni-rep-side-effects-example-graph}%
                }%
                [70mm]%
                {%
                  \input{figures/universal-representation/%
                    side-effects-example-graph}%
                }%
  \hfill%
  \mbox{}

  \caption{Example illustrating how to handle side effects in UF graphs}%
  \labelFigure{uni-rep-side-effects-example}
\end{figure}


\subsection{Edge Numbers}

During \gls{graph} construction, every \gls{edge} is given two \glspl!{edge
  number}, which allows the \glspl{edge} to be ordered \wrt to a given
\gls{node}.
%
This is needed for \gls{pattern matching}, which will be described in
\refSection{uni-rep-pattern-matching}.
%
For a given \gls{edge}~\mbox{$e = \mEdge{n}{m}$} of type~$t$, the
\gls!{inbound.en} \gls{edge number}, denoted \mbox{$\mInEdgeNr(e)$}, indicates
that $e$ is the $i$th ingoing \gls{edge} of type~$t$ connected to $m$, where
\mbox{$i = \mInEdgeNr(e)$}.
%
Similarly, the \gls!{outbound.en} \gls{edge number}, denoted
\mbox{$\mOutEdgeNr(e)$}, indicates that $e$ is the $i$th outgoing \gls{edge} of
type~$t$ connected to $n$, where \mbox{$i = \mOutEdgeNr(e)$}.
%
Consequently, all \glspl{edge} which are of the same type -- that is, they all
represent either control flow, data flow, or state flow -- and have the same
\gls{node} as source must be given distinct \gls{outbound.en} \glspl{edge
  number} such that they form a contiguous sequence.
%
The same naturally also applies for the \gls{inbound.en} \glspl{edge number} of
all \glspl{edge} with the same type and target \gls{node}.
%
An example is given in \refFigure{edge-numbers-example}.
%
\begin{figure}
  \centering%
  % Copyright (c) 2018, Gabriel Hjort Blindell <ghb@kth.se>
%
% This work is licensed under a Creative Commons 4.0 International License (see
% LICENSE file or visit <http://creativecommons.org/licenses/by/4.0/> for a copy
% of the license).
%
\begingroup%
\figureFont\figureFontSize%
\pgfdeclarelayer{background}%
\pgfsetlayers{background,main}%
\setlength{\opNodeDist}{18pt}%
\begin{tikzpicture}[
    control-flow label/.append style={
      inner xsep=.5\controlFlowLabelXSep,
      pos=.45,
    },
    outbound edge number/.style={
      edge number,
      inner ysep=0,
      inner xsep=.75\edgeNrXSep,
      pos=.25,
      swap,
    },
    inbound edge number/.style={
      edge number,
      inner ysep=0,
      pos=.8,
    },
  ]

  % Control-flow graph
  \node [block node] (check) {\irBlock{check}};
  \node [control node, below=of check] (check-c) {\opCondBr};
  \node [block node, node distance=1.5\opNodeDist,
         position=-135 degrees from check-c] (inc) {\irBlock{inc}};
  \node [block node, node distance=1.5\opNodeDist,
         position=-45 degrees from check-c] (dec) {\irBlock{dec}};
  \node [control node, below=of inc] (inc-c) {\opBr};
  \node [control node, below=of dec] (dec-c) {\opBr};
  \node [block node, node distance=.5\opNodeDist,
         below=of $(inc-c.south) !.5! (dec-c.south)$] (join)
        {\irBlock{join}};

  \begin{pgfonlayer}{background}
    \begin{scope}[control flow]
      \foreach \n in {check, inc, dec} {
        \draw (\n)
              -- node [outbound edge number] {0}
                 node [inbound edge number] {0}
              (\n-c);
      }
      \draw (check-c)
            -- node [outbound edge number, swap] {0}
               node [inbound edge number] {0}
               node [control-flow label, swap, pos=.25] {T}
            (inc);
      \draw (check-c)
            -- node [outbound edge number] {1}
               node [inbound edge number] {0}
               node [control-flow label, pos=.25] {F}
               coordinate (between-check-c-and-dec)
            (dec);
      \draw (inc-c)
            -- node [outbound edge number, pos=.3] {0}
               node [inbound edge number] {0}
            (join);
      \draw (dec-c)
            -- node [outbound edge number, pos=.1] {0}
               node [inbound edge number, pos=.75] {1}
            (join);
    \end{scope}
  \end{pgfonlayer}

  % Data-flow graph
  \node [value node, below right=-.2\opNodeDist and 4\opNodeDist of check]
        (add-1) {\opVar{1}};
  \node [computation node, position=-45 degrees from add-1] (add) {\opAdd};
  \node [value node, position=45 degrees from add] (x-1) {\opVar{x}[1]};
  \node [value node, below=of add] (x-2) {\opVar{x}[2]};
  \node [computation node, position=-45 degrees from x-1] (sub) {\opSub};
  \coordinate [node distance=\opNodeDist, position=45 degrees from sub] (sub-1);
  \node [value node, below=of sub] (x-3) {\opVar{x}[3]};
  \node [computation node, position=-45 degrees from x-2] (phi) {\opPhi};
  \node [value node, below=of phi] (x-4) {\opVar{x}[4]};
  \node [value node, below right=.5\opNodeDist and .9\opNodeDist of check]
        (b) {\opVar{b}};


  \begin{scope}[
      data flow,
      outbound edge number/.append style={
        pos=.2,
      },
      inbound edge number/.append style={
        pos=.7,
      },
    ]

    \draw (add-1)
          -- node [outbound edge number] {0}
             node [inbound edge number] {0}
          (add);
    \draw (x-1)
          -- node [outbound edge number, pos=.05] {0}
             node [inbound edge number, pos=.8] {1}
          (add);
    \draw (add)
          -- node [outbound edge number] {0}
             node [inbound edge number, pos=.72] {0}
          (x-2);
    \draw (x-1)
          -- node [outbound edge number] {1}
             node [inbound edge number] {0}
          (sub);
    \draw (sub)
          -- node [outbound edge number] {0}
             node [inbound edge number, pos=.72] {0}
          (x-3);
    \draw (x-2)
          -- node [outbound edge number] {0}
             node [inbound edge number] {0}
             coordinate (between-x-2-and-phi)
          (phi);
    \draw (x-3)
          -- node [outbound edge number, swap] {0}
             node [inbound edge number, pos=.8] {1}
          (phi);
    \draw (phi)
          -- node [outbound edge number] {0}
             node [inbound edge number, pos=.72] {0}
          (x-4);

    \coordinate (above-add-1) at ([shift=(45:.5\opNodeDist)] add-1.north east);
    \draw [-, rounded corners=3pt]
          (add-1)
          -- node [outbound edge number] {1}
          (above-add-1)
          [rounded corners=8pt]
          -| coordinate [pos=1] (tmp)
          ([yshift=3pt] sub-1);
    \draw [rounded corners=3pt]
          (tmp)
          --
          (sub-1)
          -- node [inbound edge number, pos=.8] {1}
          (sub);
  \end{scope}
  \begin{scope}[data flow]
    \path [name path=from-check-c]
          (check-c)
          --
          +(45:2\opNodeDist);
    \path [name path=from-b]
          (b)
          --
          +(180:3\opNodeDist);
    \draw [name intersections={of=from-check-c and from-b},
           rounded corners=3pt,
          ]
          (b)
          -- node [outbound edge number, inner ysep=\edgeNrYSep] {0}
          (intersection-1)
          -- node [inbound edge number] {0}
          (check-c);
  \end{scope}

  \begin{scope}[definition edge, ->]
    \path [overlay, name path=from-dec]
          (dec)
          --
          (dec -| x-2);
    \path [overlay, name path=left-from-x-2]
          ([xshift=-.5\opNodeDist] x-2.south west)
          --
          +(-45:2cm);
    \path [overlay, name path=right-from-x-2]
          ([shift=(-135:2cm)] between-x-2-and-phi)
          --
          +(45:4cm);
    \path [overlay, name path=from-x-3]
          (x-3.east)
          --
          +(180:2cm);
    \draw [name intersections={of=from-dec and left-from-x-2,
                               name=1-intersection,
                              },
           name intersections={of=left-from-x-2 and right-from-x-2,
                               name=2-intersection,
                              },
           name intersections={of=right-from-x-2 and from-x-3,
                               name=3-intersection,
                              },
           rounded corners=3pt,
          ]
          (x-3)
          -- node [outbound edge number, inner ysep=\edgeNrYSep] {0}
          (3-intersection-1)
          [rounded corners=8pt]
          --
          (2-intersection-1)
          [rounded corners=3pt]
          --
          (1-intersection-1)
          -- node [inbound edge number, pos=.9, inner ysep=\edgeNrYSep] {0}
          (dec);

    \path [overlay, name path=from-inc]
          (inc)
          --
          (inc -| dec);
    \path [overlay, name path=left-from-dec]
          ([shift=(-135:2cm)] between-check-c-and-dec)
          --
          +(45:4cm);
    \path [overlay, name path=above-dec]
          ([yshift=.75\opNodeDist, xshift=-1cm] dec.north)
          --
          +(0:4cm);
    \path [overlay, name path=from-x-2]
          (x-2.north west)
          --
          +(135:2cm);
    \draw [name intersections={of=from-inc and left-from-dec,
                               name=1-intersection,
                              },
           name intersections={of=left-from-dec and above-dec,
                               name=2-intersection,
                              },
           name intersections={of=above-dec and from-x-2,
                               name=3-intersection,
                              },
           rounded corners=3pt,
          ]
          (x-2)
          -- node [outbound edge number, swap, pos=.05] {0}
          (3-intersection-1)
          --
          (2-intersection-1)
          --
          (1-intersection-1)
          -- node [inbound edge number, pos=.7, inner ysep=\edgeNrYSep] {0}
          (inc);

    \draw [rounded corners=3pt]
          (join)
          -| node [outbound edge number, pos=.02, inner ysep=\edgeNrYSep] {0}
             node [inbound edge number, swap, pos=.65] {0}
          (x-4);
  \end{scope}
\end{tikzpicture}%
\endgroup%


  \caption[Example of edge numbers]%
          {%
            Example of edge numbers.
            %
            The inbound and outbound edge numbers are attached to the edges'
            arrows and tails, respectively%
          }%
  \labelFigure{edge-numbers-example}
\end{figure}
%
Note that the \glspl{definition edge} in the example are oriented and labeled
with the same \glspl{edge number} as the \glspl{data-flow edge} connected to the
same \gls{value node}.
%
This is to be able to identify which \gls{definition edge} belongs to which
\gls{data-flow edge} when there exist multiple such \gls{edge}, which is needed
when performing \gls{copy extension} (to be described in
\refChapter{constraint-model}).
%
In terms of restricting the definition placements, however, the orientation of
the \glspl{definition edge} does not matter and is thus skipped in cases where
this information is not needed.


\section{Instruction Representation}
\labelSection{uni-rep-instr-rep}

Modeling \glspl{instruction} as \glspl{pattern} is nearly identical to modeling
\glspl{function}.
%
The exceptions are that the \gls{control-flow graph} becomes empty if the output
is not dependent on control flow -- hence a \gls{pattern} has either none or
exactly one \gls{entry block} -- and that no additional \glspl{data-flow edge}
are added for \glspl{value node} representing constants and \gls{instruction}
input.
%
In other words, the invariant specified in
\refRequirement{uni-rep-exactly-one-inbound-edge-for-values} that every
\gls{value node} has exactly one inbound \gls{data-flow edge} does not need to
(and should not) hold for such values.
%
This results in a \gls{graph} called \gls!{UP graph}, and two examples are given
in \refFigure{up-graph-examples}.

\begin{filecontents*}{add-code.c}
int add(int $\irVar{s}$, int $\irVar{t}$) {
  entry:
    int $\irVar{d}$ = $\irVar{s}$ + $\irVar{t}$;
    return $\irVar{d}$;
}
\end{filecontents*}

\begin{filecontents*}{satadd-code.c}
int satadd(int $\irVar{s}$, int $\irVar{t}$) {
  entry:
    int $\irVar{d}[1]$ = $\irVar{s}$ + $\irVar{t}$;
    if ($\irVar{d}[1]$ > MAX) goto clamp;
  clamp:
    int $\irVar{d}[2]$ = MAX;
  end:
    int $\irVar{d}[3]$ = $\mPhi$($\irVar{d}[3]$:entry, $\irVar{d}[2]$:clamp);
    return $\irVar{d}[3]$;
}
\end{filecontents*}

\begin{figure}
  \centering%
  \setlength{\tmpLength}{2\baselineskip + \betweensubfigures}%
  \begin{tabular}{@{}c@{\quad}c@{}}
    \subcaptionbox{%
                    Semantic behavior of an \mbox{\instrFont add \$d, \$s, \$t}
                    instruction%
                    \labelFigure{up-graph-examples-add-code}%
                  }%
                  [40mm]%
                  {%
                    \lstinputlisting[language=c,mathescape,morekeywords={bool}]%
                                    {add-code.c}%
                  }%
    &
    \subcaptionbox{%
                    UP graph of {\instrFont add}%
                    \labelFigure{up-graph-examples-add-graph}%
                  }%
                  [28mm]%
                  {% Copyright (c) 2017, Gabriel Hjort Blindell <ghb@kth.se>
%
% This work is licensed under a Creative Commons 4.0 International License (see
% LICENSE file or visit <http://creativecommons.org/licenses/by/4.0/> for a copy
% of the license).
%
\begingroup%
\figureFont\figureFontSize%
\setlength{\opNodeDist}{12pt}%
\begin{tikzpicture}
  % Data flow
  \node [value node] (s) {\opVar{s}};
  \node [computation node, position=-45 degrees from s] (add) {\opAdd};
  \node [value node, position=45 degrees from add] (t) {\opVar{t}};
  \node [value node, below=of add] (d) {\opVar{d}};
  \begin{scope}[data flow]
    \draw (s) -- (add);
    \draw (t) -- (add);
    \draw (add) -- (d);
  \end{scope}
\end{tikzpicture}%
\endgroup%
}%
    \\[\tmpLength]
    \subcaptionbox{%
                    Semantic behavior of a
                    \mbox{\instrFont satadd \$d, \$s, \$t} instruction%
                    \labelFigure{up-graph-examples-satadd-code}%
                  }%
                  [64mm]%
                  {%
                    \lstinputlisting[language=c,mathescape,morekeywords={bool}]%
                                    {satadd-code.c}%
                  }%
    &
    \subcaptionbox{%
                    UP graph of {\instrFont satadd}%
                    \labelFigure{up-graph-examples-satadd}%
                  }{% Copyright (c) 2018, Gabriel Hjort Blindell <ghb@kth.se>
%
% This work is licensed under a Creative Commons 4.0 International License (see
% LICENSE file or visit <http://creativecommons.org/licenses/by/4.0/> for a copy
% of the license).
%
\begingroup%
\figureFont\figureFontSize%
\setlength{\nodeDist}{12pt}%
\pgfdeclarelayer{background}%
\pgfsetlayers{background,main}%
\begin{tikzpicture}[
    control-flow label/.append style={
      inner ysep=.75\controlFlowLabelXSep,
    },
  ]

  % Control flow
  \node [block node] (entry) {\nBlock{entry}};
  \node [control node, below=of entry] (entry-c) {\nCondBr};
  \node [block node, right=1.25\nodeDist of entry-c]
        (clamp) {\nBlock{clamp}};
  \node [control node, below=of clamp] (clamp-c) {\nBr};
  \node [block node, below=of clamp-c] (end) {\nBlock{end}};
  \begin{pgfonlayer}{background}
    \begin{scope}[control flow]
      \foreach \b in {entry, clamp} {
        \draw (\b) -- (\b-c);
      }
      \draw (entry-c)
            -- node [control-flow label, swap, xshift=-2pt] {T}
            (clamp);
      \draw [rounded corners=6pt]
            (entry-c)
            -- node [control-flow label, pos=.125, swap] {F}
            (entry-c |- end)
            --
            (end);
      \draw (clamp-c) -- (end);
    \end{scope}
  \end{pgfonlayer}

  % Data flow
  \coordinate (cf-north-east) at (entry.north -| clamp-c.east);
  \node [value node, right=3\nodeDist of cf-north-east,
         yshift=-.225\nodeSize]
        (s) {\nVar{s}};
  \node [computation node, position=-45 degrees from s] (add) {\nAdd};
  \node [value node, position= 45 degrees from add] (t) {\nVar{t}};
  \node [value node, below=of add] (d1) {\nVar{d}[1]};
  \node [computation node, position=- 45 degrees from d1] (phi) {\nPhi};
  \node [computation node, position=-135 degrees from d1] (cmp) {\nLT};
  \node [value node, position=135 degrees from cmp] (MAX-1) {\nVar{MAX}};
  \node [value node, position=45 degrees from phi] (MAX-2) {\nVar{MAX}};
  \node [value node, below=of cmp] (bool) {};
  \node [value node, below=of phi] (d3) {\nVar{d}[3]};
  \begin{scope}[data flow]
    \draw (s) -- (add);
    \draw (t) -- (add);
    \draw (add) -- (d1);
    \draw (MAX-1) -- (cmp);
    \draw (d1) -- (cmp);
    \draw (cmp) -- (bool);
    \draw (d1) -- (phi);
    \draw (MAX-2) -- (phi);
    \draw (phi) -- (d3);

    \coordinate (above-entry-c) at ([shift=(45:\nodeDist)] entry-c.45);
    \coordinate (between-clamp-and-MAX-1) at ($(clamp.east) !.5! (MAX-1.west)$);
    \draw [rounded corners=8pt]
          (bool)
          -|
          (between-clamp-and-MAX-1 |- above-entry-c)
          [rounded corners=3pt]
          --
          (above-entry-c)
          --
          (entry-c);
  \end{scope}

  \begin{scope}[definition edge]
    \path [name path=from-clamp]
          (clamp) -- (clamp -| MAX-2);
    \path [name path=from-MAX-2]
          (MAX-2.north west)
          --
          +(135:1.5\nodeDist);
    \draw [rounded corners=3pt,
           name intersections={of=from-clamp and from-MAX-2}]
          (clamp)
          --
          (intersection-1)
          --
          (MAX-2);

    \path [name path=from-entry]
          (entry) -- (entry -| d1);
    \path [name path=from-d1]
          (d1.north west)
          --
          ([shift=(135:2cm)] d1.135);
    \draw [rounded corners=6pt,
           name intersections={of=from-entry and from-d1}]
          (d1)
          --
          (intersection-1)
          --
          (entry);

    \path [name path=from-end]
          (end.south east)
          --
          +(-45:.5\nodeDist);
    \path [name path=from-d3]
          (d3.south west)
          --
          +(-135:\nodeDist);
    \path [name path=right-from-below-end]
          ([yshift=-.25\nodeDist] end.south)
          --
          +(0:8\nodeDist);
    \draw [name intersections={
             of=right-from-below-end and from-end, name=c1-intersection,
           },
           name intersections={
             of=right-from-below-end and from-d3, name=c2-intersection,
           },
           rounded corners=3pt]
          (end)
          --
          (c1-intersection-1)
          --
          (c2-intersection-1)
          --
          (d3);
  \end{scope}
\end{tikzpicture}%
\endgroup%
}
  \end{tabular}

  \caption{Examples of universal pattern graphs}
  \labelFigure{up-graph-examples}
\end{figure}


\subsection{Covering $\mPhi$-Nodes}

Since covering of a \gls{UF graph} also includes the \glspl!{phi-node} -- that
is, the \glspl{node} representing \glspl{phi-function} -- which typically do not
correspond to any \gls{instruction} on the \gls{target machine}, we need a
special \gls{pattern} for covering such \glspl{node}.
%
Consequently, it is assumed that the \gls{pattern set} always includes a
\gls!{phi-pattern}, which is illustrated in \refFigure{phi-pattern}.
%
\begin{figure}
  \centering%
  % Copyright (c) 2017-2018, Gabriel Hjort Blindell <ghb@kth.se>
%
% This work is licensed under a Creative Commons Attribution-NoDerivatives 4.0
% International License (see LICENSE file or visit
% <http://creativecommons.org/licenses/by-nc-nd/4.0/> for details).
%
\begingroup%
\figureFont\figureFontSize%
\begin{tikzpicture}
  \node [computation node] (phi) {\nPhi};
  \node [value node, position=135 degrees from phi] (i1) {\strut\nVar{i}[$1$]};
  \node [value node, position= 45 degrees from phi] (ik) {\strut\nVar{i}[$k$]};
  \node [value node, below=of phi] (d) {};

  \begin{scope}[data flow]
    \draw (i1) -- (phi);
    \draw (ik) -- (phi);
    \draw (phi) -- (d);
  \end{scope}

  \node [nothing] at ($(i1) !.5! (ik)$) {$\ldots$};
\end{tikzpicture}%
\endgroup%


  \caption{The $\mPhi$-pattern}
  \labelFigure{phi-pattern}
\end{figure}
%
The \gls{phi-pattern} has a variable number of input values since a
\gls{phi-function} may take an arbitrary number of arguments.
%
In our experiments, it proved sufficient to duplicate the \gls{phi-pattern} for
\mbox{$k = 2, \ldots, 15$}.
%
A \gls{match} derived from a \gls{phi-pattern} is called a \gls!{phi-match},
which has zero cost and emits nothing if selected.


\section{Pattern Matching}
\labelSection{uni-rep-pattern-matching}

Because neither \glsshort{UF graph} nor \glspl{UP graph} are necessarily
tree-shaped, a \gls{subgraph isomorphism} algorithm is needed for performing
\gls{pattern matching}.
%
In our approach, we apply the \gls{VF2}~algorithm~\cite{CordellaEtAl:2001} due
to its simplicity and ease of implementation.\!%
%
\footnote{%
  In his dissertation, \textcite{McCreesh:2017} cautions against using \gls{VF2}
  after showing that, for a large body of problem instances, it exhibits
  considerably worse performance compared to two other \gls{subgraph
    isomorphism} algorithms (\gls{LAD}~\cite{Solnon:2010} and
  \gls{Glasgow}~\cite{McCreeshProsser:2015}).
  %
  In case of \gls{instruction selection}, however, the problem instances are
  typically small enough that the time to find all \glspl{match} is negligible
  compared to the time to solve the \gls{constraint model}.
}
%
The algorithm is described in detail in
\refChapter{existing-isel-techniques-and-reps} on
\refPageOfSection{ex-isel-rep-vf2-algorithm}.

To adapt it for the \gls{universal representation}, we need to customize the
semantic rules (see \refDefinition{pm-semantic-rules} on
\refPageOfDefinition{pm-semantic-rules}).
%
For two \glspl{node}~$n$ and~$m$, \mbox{$n \mVFTwoAttrCmp m$} holds if $n$ and
$m$ are of the same type.
%
That is, the \glspl{node} must either be two \glspl{block node}, two
\glspl{computation node}, two \glspl{value node}, \etc (for \glspl{value node}
their data types must also be compatible).

The same applies for two \glspl{edge}~$e$ and~$d$, with the additional condition
that, for certain \glspl{node}, the order of compatible \glspl{edge} must also
match.
%
Consequently, in a \gls{match} where $e$ and $d$ are connected to \glspl{node}
representing non-commutative \glspl{operation} -- such as subtraction and
division, but also memory stores and function calls -- the \gls{inbound.en}
\glspl{edge number} of $e$ and $d$ must be identical.
%
This prevents the arguments of such \glspl{instruction} from being swapped,
which would obviously break \gls{program} semantics.
%
Similarly, in a \gls{match} where $e$ and $d$ are connected to \glspl{node} with
multiple outgoing \gls{edge} of the same type -- such as conditional jump
\glspl{operation} -- the \gls{outbound.en} \glspl{edge number} of $e$ and $d$
must be identical, which prevents swapping of the target labels where
appropriate.


\subsection{Matching SIMD Instructions Efficiently}

Although the algorithm natively supports matching of \glspl{pattern} derived
from \gls{SIMD.instr} \glspl{instruction}, doing so directly will result in many
redundant \glspl{match}.
%
For example, if a \gls{SIMD.instr} \gls{pattern} consists of $k$ identical
\glspl{operation} and the function contains $n$ such \glspl{operation}, then
this \gls{pattern} alone will result in \mbox{$n! / (n - k)!$} \glspl{match}.
%
However, since all \gls{SIMD.instr} \glspl{instruction} consist of disjoint
\glspl{pattern} that are symmetric to one another, the order in which the
\gls{pattern} \glspl{node} are mapped to \gls{function} \glspl{node} does not
matter and should therefore yield at most $\binom{n}{k}$ \glspl{match}.

To this end, instead of matching the full \gls{SIMD.instr} \gls{pattern} over
the \gls{UF graph}, we do so only for one of the disjoint \glspl{subgraph} in
the \gls{pattern}.
%
After having found all \glspl{match} for the \gls{subgraph}, we then compute all
combinations of these \gls{match} and construct for each combination a
\gls{match} of the full \gls{SIMD.instr} \gls{pattern}.
%
We also ignore combinations that will lead to cyclic data dependencies by
applying a variant of the method to be described in
\refSection{forbidding-cyclic-data-dependencies} in the next chapter.

To curb the number of \glspl{match}, we prune away all \gls{SIMD.instr}
\glspl{match} where at least one of its arguments is a constant value.
%
This is such values will typically always require the \gls{SIMD.instr}
\gls{instruction} to be preceeded by a copy \gls{instruction}, thereby
nullifying the benefit of the \gls{SIMD.instr} \gls{instruction}.


\section{Comparison with Other Sea-of-Nodes IRs}
\labelSection{uni-rep-ir-comparison}

The \gls{UF graph} is similar to the \gls{Click-Paleczny graph} (compare for
example \refFigure{uf-graph-example-full-graph} on
\refPageOfFigure{uf-graph-example-full-graph} with
\refFigure{sea-of-nodes-example-click-paleczny-graph} on
\refPageOfFigure{sea-of-nodes-example-click-paleczny-graph}).
%
Both are combinations of \glsshort{control-flow graph} and \glspl{SSA graph},
they represent (some) of the control \glspl{operation} as \glspl{node}, and they
can restrict the placement of \glspl{operation} to \glspl{block} through
auxiliary edges.
%
However, the \gls{Click-Paleczny graph} does not completely fulfill
\refRequirement{uni-rep-explicit-ops} as unconditional branches are not
represented as \glspl{node}, nor does it fulfill
\refRequirement{uni-rep-exactly-one-inbound-edge-for-values}.

\textcite{BraunEtAl:2011} introduced another graph-based \gls{IR}, called
\gls!{Firm}, that is based on the \gls{Click-Paleczny graph}.
%
Consequently, \gls{Firm} also does not fulfill
\refRequirement{uni-rep-explicit-ops}, and neither does it fulfill
\refRequirement{uni-rep-explicit-values} nor \refRequirement{uni-rep-ssa}.


\section{Summary}
\labelSection{uni-rep-summary}

In this chapter, we have introduced a novel graph-based representation, called
\gls{universal representation}, that models both data and control flow on a
global scope.
%
This means that entire \gls{function} can be captured as a single graph and that
complex \glspl{instruction}, with or without control flow, can be modeled as
\glspl{pattern}.
%
The \gls{universal representation} also provides enough freedom to move
computations from one \gls{block} to another without breaking \gls{program}
semantics.
%
These features jointly enable the problems of \gls{global.is} \gls{instruction
  selection}, \gls{global code motion}, and \gls{data copying} to be modeled as
a \gls{constraint model}, which is introduced in the next chapter.
%
We have also described an algorithm for performing \gls{pattern matching} using
the \gls{universal representation}.
