% Copyright (c) 2017, Gabriel Hjort Blindell <ghb@kth.se>
%
% This work is licensed under a Creative Commons 4.0 International License (see
% LICENSE file or visit <http://creativecommons.org/licenses/by/4.0/> for a copy
% of the license).

\chapter{Introduction}
\labelChapter{introduction}

\enlargethispage{2pt}

Processors are built to execute a vast range of \glspl{program}, from tiny
\emph{Hello, world!} samples to large-scale Earth simulations.
%
Most importantly, the processors are built to minimize the execution time for
these \glspl{program}.
%
To this end, CPU manufacturers continuously extend their processors with new,
sophisticated \glspl{instruction} that allow complex computations to be executed
using fewer \glspl{instruction}.
%
Such \glspl{instruction} are especially common in \glspl{DSP} that appear in
most contemporary mobile phones.
%
But while the technology behind modern processors continues to advance, the
techniques for automatically making use of the \glspl{instruction} have not.
%
In fact, the state-of-the-art \glspl{compiler} -- these are tools for
translating \glspl{program} into \gls{assembly code} -- essentially apply the
same techniques for selecting \glspl{instruction} that were used in the 1980s.

Due to underlying assumptions about the \gls{instruction set}, many of the
\glspl{instruction} currently available in modern processors cannot be handled
by these techniques.
%
Instead, \gls{compiler} developers are forced to implement hand-written routines
for checking whether a specific \gls{instruction} is applicable and, if so,
greedily selecting it.
%
With over \num{100}~million microprocessors being shipped every
quarter~\cite{Intel:2014:NewsRelease}, through release cycles that become
shorter and shorter, there is a growing need for new and improved
\gls{instruction selection} techniques.

Furthermore, the impact and availability of selecting a particular
\gls{instruction} is highly dependent on other \gls{compiler} tasks.
%
One such task is \gls{global code motion}, which involves moving computations
from one part of the \gls{program} to another.
%
Integrating \gls{global code motion} with \gls{instruction selection} enables a
larger set of combinations of computations, some of which may be implemented
using sophisticated \glspl{instruction}.
%
Consequently, in order to generate high-quality code, these tasks must be
performed in unison.

This dissertation introduces \gls!{universal instruction selection}, a new
approach based on \glsdesc{CP} that addresses the problems described above.
%
\begin{figure}
  \centering%
  % Copyright (c) 2017, Gabriel Hjort Blindell <ghb@kth.se>
%
% This work is licensed under a Creative Commons 4.0 International License (see
% LICENSE file or visit <http://creativecommons.org/licenses/by/4.0/> for a copy
% of the license).
%
\begingroup%
\figureFont\figureFontSize\relsize{-0.5}%
\def\nodeDist{14mm}%
\def\rowSep{2.5pt}%
\begin{tikzpicture}[
    input/.style={
      nothing,
      inner sep=2pt,
      node distance=\nodeDist,
    },
    output/.style={
      input,
    },
    tool/.style={
      draw,
      thick,
      minimum height=5mm,
      inner xsep=3pt,
      inner ysep=1.5pt,
      node distance=\nodeDist,
      fill=shade1,
      font=\strut,
    },
    flow/.style={
      ->,
      thick,
    },
    item/.style={
      nothing,
      node distance=1pt,
    },
    sep line/.style={
      dotted,
    },
    sep label/.style={
      nothing,
    },
  ]

  \node [input] (ir)
        {%
          \begin{tabular}{@{}c@{}}
            IR
          \end{tabular}%
        };
  \node [tool, right=0.5*\nodeDist of ir] (graph-builder)
        {\begin{tabular}{@{}c@{}}
          graph\\
          builder
         \end{tabular}%
        };
  \node [tool, below=0.5*\nodeDist of graph-builder] (g-transformations)
        {transformations};
  \node [tool, right=0.8*\nodeDist of g-transformations] (matcher) {matcher};
  \node [tool, right=1.1*\nodeDist of matcher] (modeler) {modeler};
  \node [tool, right=1.1*\nodeDist of modeler] (solver) {solver};
  \node [tool, right=0.9*\nodeDist of solver] (code-emitter)
        {\begin{tabular}{@{}c@{}}
          code\\
          emitter
         \end{tabular}%
        };
  \node [output, inner sep=0, below=0.25*\nodeDist of code-emitter] (code)
        {%
          \begin{tabular}{@{}c@{}}
            code
          \end{tabular}%
        };

  \node [input, above right=2.25*\nodeDist and 0 of ir.west]
        (machine-description)
        {%
          \begin{tabular}{@{}c@{}}
            machine\\
            description
          \end{tabular}%
        };
  \node [tool] at (machine-description -| matcher) (pattern-set-builder)
        {\begin{tabular}{@{}c@{}}
          pattern set\\
          builder
         \end{tabular}%
        };
  \node [tool, below=0.5*\nodeDist of pattern-set-builder] (ps-transformations)
        {transformations};

  \begin{scope}[flow]
    \foreach \i [remember=\i as \previ (initially ir)]
    in {%
         graph-builder, g-transformations, matcher, modeler, solver,
         code-emitter, code%
       }
    {%
      \draw (\previ)
            -- coordinate (between-\previ-and-\i)
            (\i);
    }
    \draw [rounded corners=4pt]
          (between-g-transformations-and-matcher)
          |-
          ([yshift=-0.2*\nodeDist] matcher.south)
          -|
          (modeler);

    \foreach \i [remember=\i as \previ (initially machine-description)]
    in {pattern-set-builder, ps-transformations}
    {%
      \draw (\previ)
            -- coordinate (between-\previ-and-\i)
            (\i);
    }
    \draw (ps-transformations)
          --
          (matcher);
    \coordinate (below-ps-transformations)
                at ([yshift=-0.5*\nodeDist] ps-transformations.south);
    \draw [rounded corners=4pt]
          (below-ps-transformations)
          -|
          ($(modeler.north west) !.33! (modeler.north east)$);

    \draw [rounded corners=4pt]
          (between-machine-description-and-pattern-set-builder)
          |-
          ([yshift=0.2*\nodeDist] pattern-set-builder.north)
          [rounded corners=8pt]
          -|
          ($(modeler.north west) !.66! (modeler.north east)$);
  \end{scope}

  \node [item, right=2pt of between-graph-builder-and-g-transformations]
        {%
          \begin{tabular}{@{}c@{}}
            graph
          \end{tabular}%
        };
  \node [item, above=of between-g-transformations-and-matcher]
        {%
          \begin{tabular}{@{}c@{}}
            graph
          \end{tabular}%
        };
  \node [item, above=of between-matcher-and-modeler]
        {%
          \begin{tabular}{@{}c@{}}
            match set
          \end{tabular}%
        };
  \node [item] at (between-modeler-and-solver)
        {%
          \begin{tabular}{@{}c@{}}
            constraint\\[\rowSep]
            model
          \end{tabular}%
        };
  \node [item, above=of between-solver-and-code-emitter]
        {%
          \begin{tabular}{@{}c@{}}
            solution
          \end{tabular}%
        };
  \node [item, right=2pt of between-pattern-set-builder-and-ps-transformations]
        {%
          \begin{tabular}{@{}c@{}}
            pattern set
          \end{tabular}%
        };
  \node [%
          item,
          right=2pt of
                $(ps-transformations.south) !.5! (below-ps-transformations)$,
        ]%
        {%
          \begin{tabular}{@{}c@{}}
            pattern set
          \end{tabular}%
        };

  \coordinate (between-md-and-ir)
              at ($(below-ps-transformations)
                   !.5!
                   (graph-builder.north -| below-ps-transformations)$);
  \coordinate (sep-line-east) at (code-emitter.east |- between-md-and-ir);
  \draw [sep line]
        (ir.west |- between-md-and-ir)
        --
        (sep-line-east);
  \node [sep label, above left=2pt and 0 of sep-line-east]
        {at compiler build time};
  \node [sep label, below left=2pt and 0 of sep-line-east]
        {at program compile time};
\end{tikzpicture}%
\endgroup%


  \caption{Overview of the approach}
  \labelFigure{approach-overview}
\end{figure}
%
Outlined in \refFigure{approach-overview}, the approach is the first to combine
\gls{instruction selection} with \gls{global code motion} and \gls{block
  ordering}.
%
In doing so, our approach alleviates selection of sophisticated
\glspl{instruction} that would otherwise not have been selectable.
%
The approach relies on a novel combinatorial model that is simpler and more
flexible compared to non-combinatorial approaches.
%
It also captures crucial features ignored by other, existing combinatorial
approaches.
%
Extensions for integrating \gls{instruction scheduling} and \gls{register
  allocation} -- these are two other \gls{code generation} tasks known to impact
\gls{instruction selection} and vice versa -- are also proposed.

The model is enabled by a novel, \gls{graph}-based representation that unifies
data flow and control flow for entire \glspl{function}.
%
Not only is this representation crucial for combining \gls{instruction
  selection} with \gls{global code motion}, it also enables \glspl{instruction}
whose behavior contains both data and control flow to be modeled as
\glspl{graph}.
%
Hence there is no longer any need for hand-written routines to handle
\glspl{instruction} that violate underlying assumptions about the
\gls{instruction set}.


\section{Thesis Statement}
\labelSection{intro-thesis-statement}

\begin{statement}
  \Glsdesc{CP} is a flexible, practical, competitive, and
  extensible approach for combining \gls{global.is} \gls{instruction selection},
  \gls{global code motion}, and \gls{block ordering}.
\end{statement}
%
By \emph{flexible}, it means that our approach can handle hardware architectures
with a rich \gls{instruction set}.
%
By \emph{practical}, it means that our approach can select \glspl{instruction}
for \glspl{program} of sufficient complexity and scales to medium-sized
\glspl{function} (measured in hundreds of \glspl{operation}).
%
By \emph{competitive}, it means that our approach generates code of equal or
better quality compared to the state of the art.
%
By \emph{extensible}, it means that our approach can be extended to integrate
other \gls{code generation} tasks.


\section{Motivation}
\labelSection{intro-motivation}

A \gls!{compiler} is a tool that takes a \gls{program}, written in some
programming language, as input and produces equivalent \gls{assembly code} for a
specific processor, called the \gls!{target machine}, as output.
%
\begin{figure}
  \centering%
  % Copyright (c) 2017, Gabriel Hjort Blindell <ghb@kth.se>
%
% This work is licensed under a Creative Commons 4.0 International License (see
% LICENSE file or visit <http://creativecommons.org/licenses/by/4.0/> for a copy
% of the license).
%
\begingroup%
\figureFont\figureFontSize\relsize{-0.5}%
\def\compStageWidth{1.5cm}%
\def\compStageHeight{8mm}%
\def\compStageDist{0.6*\compStageWidth}%
\def\compWrapperInnerSep{0.5*\compStageDist}%
\def\optStageWidth{\compStageWidth}%
\def\optStageHeight{\compStageHeight}%
\def\optStageDist{0.5*\compStageDist}%
\def\optWrapperInnerSep{3mm}%
\def\backStageWidth{\compStageWidth}%
\def\backStageHeight{\compStageHeight}%
\def\backStageDist{0.5*\compStageDist}%
\def\backWrapperInnerSep{3mm}%
\def\wrapperOuterSep{9mm}%
\pgfdeclarelayer{background}%
\pgfsetlayers{background,main}%
\begin{tikzpicture}[%
    box/.style={%
      draw,
    },
    box/.append style={%
      thick,
    },
    compiler stage/.style={%
      box,
      inner sep=0,
      outer sep=0,
      minimum width=\compStageWidth,
      minimum height=\compStageHeight,
      node distance=\compStageDist,
      fill=shade0,
    },
    compiler wrapper/.style={%
      box,
      inner sep=\compWrapperInnerSep,
      rounded corners=6pt,
      fill=shade2,
    },
    compiler input/.style={%
      inner sep=0,
      outer sep=2pt,
      node distance=\compStageDist,
      font=\strut,
    },
    compiler output/.style={%
      compiler input,
    },
    backend stage/.style={%
      compiler stage,
      minimum width=\backStageWidth,
      minimum height=\backStageHeight,
      node distance=\backStageDist,
    },
    backend wrapper/.style={%
      box,
      inner sep=\backWrapperInnerSep,
      rounded corners=3pt,
      fill=shade1,
    },
    backend input/.style={%
      compiler input,
      node distance=\backStageDist,
      font=\rule[-.25\baselineskip]{0pt}{\baselineskip},
    },
    backend output/.style={%
      backend input,
    },
    optimizer stage/.style={%
      backend stage,
      minimum width=\optStageWidth,
      minimum height=\optStageHeight,
      node distance=\optStageDist,
    },
    optimizer intermediate stage/.style={%
      optimizer stage,
      minimum width=0,
      outer xsep=3pt,
      draw=none,
      fill=none,
    },
    optimizer wrapper/.style={%
      backend wrapper,
      inner xsep=\optWrapperInnerSep,
      inner ysep=\optWrapperInnerSep,
    },
    optimizer input/.style={%
      compiler input,
      node distance=\optStageDist,
      font=\rule[-.25\baselineskip]{0pt}{\baselineskip},
    },
    optimizer output/.style={%
      optimizer input,
    },
    flow/.style={%
      ->,
      thick,
    },
    label/.style={%
      inner sep=0,
      outer sep=3pt,
      node distance=0,
    },
    wrapper label/.style={%
      label,
      outer sep=0,
      font=\strut\bfseries,
    },
  ]

  % Compiler stages
  \node [compiler input] (compiler-input) {program};
  \node [compiler stage, right=of compiler-input] (frontend) {frontend};
  \node [compiler stage, right=of frontend] (optimizer) {optimizer};
  \node [compiler stage, right=of optimizer] (backend) {backend};
  \node [compiler output, right=of backend] (compiler-output) {assembly code};
  \begin{pgfonlayer}{background}
    \node [compiler wrapper, fit=(frontend) (optimizer) (backend)]
          (compiler-wrapper) {};
  \end{pgfonlayer}
  \node [wrapper label, above=of compiler-wrapper] {compiler};

  % Compiler execution flow
  \begin{scope}[flow]
    \draw (compiler-input) -- (frontend);
    \draw (frontend) -- node [label, above] {IR} (optimizer);
    \draw (optimizer) -- node [label, above] {IR} (backend);
    \draw (backend) -- (compiler-output);
  \end{scope}

  % Optimizer stages
  \node [optimizer stage, below=\wrapperOuterSep of compiler-wrapper]
        (gcmotion) {%
    \begin{tabular}{c}
      global\\
      code mover
    \end{tabular}
  };
  \node [optimizer intermediate stage, left=of gcmotion] (pre-gcmotion)
        {$\ldots$};
  \node [optimizer intermediate stage, right=of gcmotion] (post-gcmotion)
        {$\ldots$};
  \begin{pgfonlayer}{background}
    \node [optimizer wrapper, fit=(pre-gcmotion) (gcmotion) (post-gcmotion)]
          (optimizer-wrapper) {};
  \end{pgfonlayer}
  \node [optimizer input, left=of optimizer-wrapper.west] (opt-input) {IR};
  \node [optimizer output, right=of optimizer-wrapper.east] (opt-output) {IR};
  \node [wrapper label, above=of optimizer-wrapper] {optimizer};

  % Optimizer execution flow
  \begin{scope}[flow]
    \draw (opt-input) -- (pre-gcmotion);
    \draw (pre-gcmotion) -- (gcmotion);
    \draw (gcmotion) -- (post-gcmotion);
    \draw (post-gcmotion) -- (opt-output);
  \end{scope}

  % Backend stages
  \node [backend stage,
         below left=\wrapperOuterSep and 0pt of optimizer-wrapper]
        (isel)
  {%
    \begin{tabular}{c}
      instruction\\
      selector
    \end{tabular}
  };
  \node [backend stage, right=of isel] (regalloc) {%
    \begin{tabular}{c}
      register\\
      allocator
    \end{tabular}
  };
  \node [backend stage, right=of regalloc] (isched) {%
    \begin{tabular}{c}
      instruction\\
      scheduler
    \end{tabular}
  };
  \node [backend stage, right=of isched] (ordering) {%
    \begin{tabular}{c}
      block\\
      orderer
    \end{tabular}
  };
  \begin{pgfonlayer}{background}
    \node [backend wrapper, fit=(isel) (regalloc) (isched) (ordering)]
          (backend-wrapper) {};
  \end{pgfonlayer}
  \node [backend input, left=of backend-wrapper.west] (back-input) {IR};
  \node [backend output, right=of backend-wrapper.east] (back-output)
        {assembly code};
  \node [wrapper label, above=of backend-wrapper] {backend};

  % Backend execution flow
  \begin{scope}[flow]
    \draw ([xshift=-\backStageDist] backend-wrapper.west) -- (isel);
    \draw (isel) -- (regalloc);
    \draw (regalloc) -- (isched);
    \draw (isched) -- (ordering);
    \draw (ordering) -- ([xshift=\backStageDist] backend-wrapper.east);
  \end{scope}
\end{tikzpicture}%
\endgroup%


  \caption{Overview of a typical compiler}
  \labelFigure{compiler-overview}
\end{figure}
%
As shown in \refFigure{compiler-overview}, a \gls{compiler} typically consists
of three parts:%
%
\begin{inlinelist}[itemjoin={; }, itemjoin*={; and}]
  \item a \gls!{frontend}, which performs syntactic and semantic analysis and
    transforms the program into an \gls!{IR}
  \item an \gls!{optimizer} (sometimes called \gls!{middle-end}), which performs
    target-independent optimizations
  \item a \gls!{backend}, which performs \gls{code generation}
\end{inlinelist}.

The \gls{optimizer} is arguably the largest component of any \gls{compiler},
consisting of many tasks such as \gls{constant folding}, \gls{dead code
  elimination}, and \gls{loop unrolling}.
%
Another such optimization is \gls!{global code motion}, where \glspl{operation}
are moved from one \gls{basic block} to another, which is done mainly to move
expensive \glspl{operation} into \glspl{block} with lower execution frequency.

The \gls{backend} also consists of several tasks, of which three
typically are most promiment:%
%
\begin{inlinelist}[itemjoin={; }, itemjoin*={; and}]
  \item \gls!{instruction selection}, where \glspl{instruction} implementing the
    given \gls{program} are selected
  \item \gls!{register allocation}, where \gls{virtual.temp} \glspl{temporary}
    are assigned to \glspl{register}
  \item \gls!{instruction scheduling}, where \glspl{instruction} are reordered
    to increase instruction-level parallelism
\end{inlinelist}.
%
Another code generation task of interest is \gls!{block ordering}, where the
\glspl{basic block} are rearranged in order to minimize the number of jump
\glspl{instruction}.


\subsubsection{%
  The Need for New Instruction Selection Techniques and Representations%
}

\begin{filecontents*}{isel-gcmotion-example.c}
int i = 0;
while (i < N) {
  int a = A[i];
  int b = B[i];
  int c = a + b;
  if (MAX < c) c = MAX;
  C[i] = c;
  i++;
}
\end{filecontents*}

\begin{figure}
  \centering%
  \subcaptionbox{C code\labelFigure{isel-gcmotion-example-c}}%
                {\lstinputlisting[language=c]{isel-gcmotion-example.c}}%
  \hspace{5mm}%
  \subcaptionbox{Corresponding IR and control-flow graph%
                 \labelFigure{isel-gmotion-example-ir}}%
                [64mm]%
                {% Copyright (c) 2017-2018, Gabriel Hjort Blindell <ghb@kth.se>
%
% This work is licensed under a Creative Commons Attribution-NoDerivatives 4.0
% International License (see LICENSE file or visit
% <http://creativecommons.org/licenses/by-nc-nd/4.0/> for details).
%
\begingroup%
\figureFont\figureFontSize\relsize{-0.5}%
\def\nodeDist{16pt}%
\def\jumpOffset{8pt}%
\pgfdeclarelayer{background}%
\pgfsetlayers{background,main}%
\begin{tikzpicture}[
    block node/.style={
      draw,
      fill=white,
      line width=\controlFlowLineWidth,
      minimum size=0,
      inner xsep=4pt,
      inner ysep=2.5pt,
      node distance=\nodeDist,
    },
    control-flow label/.append style={
      inner ysep=2.5pt,
    },
  ]

  \node [block node] (b1) {%
               \begin{tabular}{@{}l@{}}
                 \irAssign{\irVar{i}}{\irVar{0}}\\
                 \irBr{b2}
               \end{tabular}%
             };
  \node [block node, below=of b1] (b2)
        {%
          \begin{tabular}{@{}l@{}}
            \irAssign{\irTemp{1}}{\irLE{\irVar{i}}{\irVar{N}}}\\
            \irCondBr{\irTemp{1}}{b3}{end}
          \end{tabular}%
        };
  \node [block node, below=of b2] (b3)
        {%
          \begin{tabular}{@{}l@{}}
            \irAssign{\irTemp{2}}{\irMul{\irVar{i}}{\irVar{4}}}\\
            \irAssign{\irTemp{3}}{\irAdd{\irVar{A}}{\irTemp{2}}}\\
            \irAssign{\irVar{a}}{\irLoad{\irTemp{3}}}\\
            \irAssign{\irTemp{4}}{\irAdd{\irVar{B}}{\irTemp{2}}}\\
            \irAssign{\irVar{b}}{\irLoad{\irTemp{4}}}\\
            \irAssign{\irVar{c}}{\irAdd{\irVar{a}}{\irVar{b}}}\\
            \irAssign{\irTemp{5}}{\irLE{\irVar{MAX}}{\irVar{c}}}\\
            \irCondBr{\irTemp{5}}{b4}{b5}
          \end{tabular}%
        };
  \node [block node, below right=\nodeDist and 0 of b3.south west] (b4)
        {%
          \begin{tabular}{@{}l@{}}
            \irAssign{\irVar{c}}{\irVar{MAX}}\\
            \irBr{b5}
          \end{tabular}%
        };
  \coordinate (from-b4) at ([yshift=\jumpOffset] b4.south east);
  \node [block node, below right=-\jumpOffset and \nodeDist of from-b4] (b5)
        {%
          \begin{tabular}{@{}l@{}}
            \irAssign{\irTemp{6}}{\irAdd{\irVar{C}}{\irTemp{2}}}\\
            \irStore{\irTemp{6}}{\irVar{c}}\\
            \irAssign{\irVar{i}}{\irAdd{\irVar{i}}{\irVar{1}}}\\
            \irBr{b2}
          \end{tabular}%
        };

  \begin{pgfonlayer}{background}
    \begin{scope}[control flow]
      \draw (b1) -- (b2);
      \draw (b2)
             -- node [control-flow label,
                      inner xsep=.75\controlFlowLabelXSep] {T}
             (b3);
      \draw (b3)
             -- node [control-flow label, inner xsep=.5\controlFlowLabelXSep,
                      pos=0.1] {T}
             (b4);
      \draw (b3)
            -- node [control-flow label, inner xsep=.5\controlFlowLabelXSep,
                     pos=0.06, swap] {F}
            (b5);
      \draw (from-b4) -- (from-b4 -| b5.west);
      \draw [rounded corners=10pt]
            ([yshift=\jumpOffset] b5.south west)
            -|
            ([xshift=-\nodeDist] b3.west)
            |-
            ([yshift=-\jumpOffset] b2.north west);
      \coordinate (from-b2) at ([yshift=\jumpOffset] b2.south east);
      \draw (from-b2)
            -- node [control-flow label] {F}
            ([xshift=2*\nodeDist] from-b2);
    \end{scope}
  \end{pgfonlayer}

  \node [block label, below left=0 and 0 of b1.north west, inner xsep=2pt] {b1};
  \foreach \i in {2, 3, 4} {
    \node [block label, above right=0 and 0 of b\i.north west] {b\i};
  }
  \node [block label, above left=0 and 0 of b5.north east] {b5};
\end{tikzpicture}%
\endgroup%
}%

  \caption[%
            Example illustrating the need for new techniques and the interaction
            between instruction selection and global code motion%
          ]{%
            Example to illustrate the need for new techniques and the
            interaction between instruction selection and global code motion.
            %
            The program computes the saturated sums of two arrays \irVar{A} and
            \irVar{B} as a new array~\irVar{C}, all of which are assumed to be
            of equal lengths and stored in memory.
            %
            The variables \irVar{N} and \irVar{MAX} are constants representing
            the array length and the upper limit, respectively.
            %
            An integer is assumed to be 4~bytes%
          }
  \labelFigure{isel-gcmotion-example}%
\end{figure}

\labelPage{saturated-arithmetic}

\RefFigure{isel-gcmotion-example} shows a program that computes the saturated
sums of two integer arrays.
%
In \gls!{saturation arithmetic}, the result of an arithmetic \gls{operation}
will always stay within a range fixed by a minimum and maximum value.
%
If the \gls{operation} would produce a value outside of this range, then the
value is set (``clamped'') to the closest limit, thus becoming ``saturated''.

Assume a \gls{target machine} that has an instruction capable of implementing
the saturated-add \gls{operation} used in the \gls{program} shown in
\refFigure{isel-gcmotion-example}.
%
Hence the \gls{instruction} would implement the following five operations:%
%
\begin{inlinelist}[itemjoin={, }, itemjoin*={, and}]
  \item the \irAdd{\irVar{a}}{\irVar{b}} addition
  \item the \irLE{\irVar{MAX}}{\irVar{c}} comparison
  \item the conditional jump to either of blocks~\irBlock{b4} and~\irBlock{b5}
  \item the \irAssign{\irVar{c}}{\irVar{MAX}} assignment
  \item the unconditional jump to~\irBlock{b5}
\end{inlinelist}.
%
Selecting this \gls{instruction} can have tremendous impact on performance.
%
Assume, for example, that each \gls{operation} can be implemented using an
\gls{instruction} that takes one cycle to execute. Hence executing one iteration
of the loop takes \num{16}~cycles, and selecting the saturated-add
\gls{instruction} reduces the execution time by \SI{25}{\percent}.

Existing \gls{instruction selection} techniques and representations, however, do
not support selection of such \glspl{instruction}.
%
Since the \glspl{operation} above reside in separate \glspl{block} (\irBlock{b3}
and \irBlock{b4}), making use of the saturated-add \gls{instruction} requries an
\gls{instruction selector} that is capable of processing multiple \glspl{basic
  block} simultaneously.
%
In comparison, \glsshort{traditional approach} \gls{instruction selection}
techniques only consider one \gls{basic block} at a time.
%
Morever, most approaches represent the \glspl{instruction} as \glspl{graph}.
%
As the saturated-add \gls{instruction} contains \glspl{operation} for both data
and control flow, modeling it as a \gls{graph} requires a representation that
captures both data and control flow.
%
In comparison, existing representations only capture data flow.


\subsubsection{For Combining Instruction Selection and Global Code Motion}

Assume that the \gls{target machine} also has a \gls{SIMD.instr}
\gls{instruction} for addition.\!%
%
\footnote{%
  A \gls!{SIMD.instr}[ \gls{instruction}] is an \gls{instruction} that executes
  the same \gls{operation} over multiple sets of input data.%
}
%
Revisiting the example shown in \refFigure{isel-gcmotion-example}, there are
four additions in the \gls{program} (\irAdd{\irVar{A}}{\irTemp{2}},
\irAdd{\irVar{B}}{\irTemp{2}}, \irAdd{\irVar{C}}{\irTemp{2}}, and
\irAdd{\irVar{i}}{\irVar{1}}) which are independent from one another and can
therefore be executed in parallel.
%
Assuming again that all \glspl{instruction} takes one cycle to execute,
implementing these four additions using a single \gls{SIMD.instr}
\gls{instruction} would improve performance by almost \SI{19}{\percent}.
%
This requires that the two additions in block~\irBlock{b5} be moved to
block~\irBlock{b4}, which is the task of \gls{global code motion}.
%
However, as \gls{global code motion} is commonly considered to be a
target-independent optimization, it is often done before \gls{code generation}.
%
Consequently, the \gls{global code mover} may take decisions which are
detrimental for the \gls{instruction selector}.


\subsubsection{For Taking Cost of Data Copying Into Account}

Although selecting \gls{SIMD.instr} \glspl{instruction} may significantly
improve code quality -- like in the previous example -- doing so carelessly may
also have the opposite effect.
%
Assume, for example, that the \gls{SIMD.instr} \gls{instruction} uses a limited
set of registers.
%
If the other selected \glspl{instruction} cannot directly write to and read from
these registers, then additional \glspl{instruction} must be emitted to copy the
values between the general registers and the \gls{SIMD.instr} registers.
%
In the case of the program shown in \refFigure{isel-gcmotion-example}, eight
such \glspl{instruction} would be needed, leading to a slowdown of over
\SI{31}{\percent}.
%
These restrictions must be known by the \gls{instruction selector} to make
effective use of \gls{SIMD.instr} \glspl{instruction}.


\subsubsection{For Combining Instruction Selection and Block Ordering}

\begin{filecontents*}{isel-blorder-example.c}
int f() {
  int a;
  do {
    a = g();
  } while (a == 0);
  return a;
}
\end{filecontents*}

\begin{figure}
  \centering%
  \subcaptionbox{C code\labelFigure{isel-blorder-example-c}}%
                {\lstinputlisting[language=c]{isel-blorder-example.c}}%
  \hspace{5mm}%
  \subcaptionbox{Corresponding IR and control-flow graph%
                 \labelFigure{isel-blorder-example-ir}}%
                [64mm]%
                {% Copyright (c) 2017, Gabriel Hjort Blindell <ghb@kth.se>
%
% This work is licensed under a Creative Commons 4.0 International License (see
% LICENSE file or visit <http://creativecommons.org/licenses/by/4.0/> for a copy
% of the license).
%
\begingroup%
\figureFont\figureFontSize\relsize{-0.5}%
\def\nodeDist{16pt}%
\def\jumpOffset{8pt}%
\pgfdeclarelayer{background}%
\pgfsetlayers{background,main}%
\begin{tikzpicture}[
    every node/.style={
      draw,
      fill=white,
      line width=\controlFlowLineWidth,
      minimum size=0,
      inner xsep=4pt,
      inner ysep=2.5pt,
      node distance=\nodeDist,
    },
  ]

  \node (b1) {%
               \begin{tabular}{@{}l@{}}
                 \irAssign{\irVar{a}}{\irCall{g}}\\
                 \irAssign{\irTemp{1}}{\irEQ{\irVar{a}}{\irVar{0}}}\\
                 \irCondBr{\irTemp{1}}{b1}{b2}
               \end{tabular}%
             };
  \node [below left=\nodeDist and 0 of b1.south east] (b2)
        {%
          \begin{tabular}{@{}l@{}}
            \irRet{\irVar{a}}
          \end{tabular}%
        };

  \begin{pgfonlayer}{background}
    \begin{scope}[control flow]
      \draw (b2 |- b1.south) -- node [control-flow label, pos=0.4] {F} (b2);
      \draw [rounded corners=8pt]
            ($(b1.south west) !.75! (b1.south)$)
            |- node [control-flow label, pos=0.2, swap] {T}
            ([xshift=-\nodeDist, yshift=-\nodeDist] b1.south west)
            |-
            ([yshift=-\jumpOffset] b1.north west);
    \end{scope}
  \end{pgfonlayer}


  \node [block label, above right=0 and 0 of b1.north west] {b1};
  \node [block label, left=of b2, inner xsep=2pt] {b2};
\end{tikzpicture}%
\endgroup%
}%

  \vspace{\betweensubfigures}

  \subcaptionbox{%
                  Selecting basic jump instruction, after block ordering.
                  %
                  Cycle count: 6%
                  \labelFigure{isel-blorder-example-code-1}%
                }%
                [34mm]%
                {%
                  \figureFontSize%
                  \begin{tabular}{%
                                   @{}>{\instrFont}l@{\hspace{4pt}}%
                                   >{\instrFont}l@{\hspace{4pt}}%
                                   >{\instrFont}l@{}%
                                 }
                    b1: & call & a $\leftarrow$ g()\\
                        & cmp  & \instrTemp{1} $\leftarrow$
                                 \instrEQ{\instrVar{a}}{\instrVar{0}}\\
                        & jmp  & \instrTemp{1}, \instrBlock{b1}\\
                    b2: & ret  & \instrVar{a}
                  \end{tabular}%
                }%
  \hfill%
  \subcaptionbox{%
                  Selecting complex jump instruction, before block ordering.
                  %
                  Cycle count: 5%
                  \labelFigure{isel-blorder-example-code-2}%
                }%
                [39mm]%
                {%
                  \figureFontSize%
                  \begin{tabular}{%
                                   @{}>{\instrFont}l@{\hspace{4pt}}%
                                   >{\instrFont}l@{\hspace{4pt}}%
                                   >{\instrFont}l@{}%
                                 }
                    b1: & call & a $\leftarrow$ g()\\
                        & jmp  & \instrNE{\instrVar{a}}{\instrVar{0}},
                                 \instrBlock{b2}\\
                    \\
                    b2: & ret  & \instrVar{a}
                  \end{tabular}%
                }
  \hfill%
  \subcaptionbox{%
                  Selecting complex jump instruction, after block ordering.
                  %
                  Cycle count: 8%
                  \labelFigure{isel-blorder-example-code-3}%
                }%
                [40mm]%
                {%
                  \figureFontSize%
                  \begin{tabular}{%
                                   @{}>{\instrFont}l@{\hspace{4pt}}%
                                   >{\instrFont}l@{\hspace{4pt}}%
                                   >{\instrFont}l@{}%
                                 }
                    b1: & call & a $\leftarrow$ g()\\
                        & jmp  & \instrNE{\instrVar{a}}{\instrVar{0}},
                                 \instrBlock{b2}\\
                        & jmp  & \instrBlock{b1}\\
                    b2: & ret  & \instrVar{a}
                  \end{tabular}%
                }

  \caption[%
            Example illustrating the interaction between instruction
            selection and block ordering%
          ]{%
            Example to illustrate the interaction between instruction
            selection and block ordering.
            %
            The function \irCode{f} calls another function \irCode{g} until it
            returns a non-zero value, and then returns that value%
          }
  \labelFigure{isel-blorder-example}%
\end{figure}

\RefFigure{isel-blorder-example} shows a \gls{function} that keeps calling
another \gls{function} (with side effects) until it returns a non-zero value.
%
Assume that the \gls{target machine} has three \glspl{instruction} for handling
control flow:%
%
\begin{inlinelist}[itemjoin={; }, itemjoin*={; and}]
  \item a \mbox{{\instrFont jmp} $p$, $b$} \gls{instruction}, which
    branches to block~$b$ if the value in register~$p$ corresponds the Boolean
    value~$\mathit{true}$
  \item a \mbox{{\instrFont jmp} $r \neq 0$, $b$} \gls{instruction}, which
    branches to block~$b$ if the condition \mbox{$r \neq 0$} holds, where $r$
    is a register
  \item a \mbox{{\instrFont jmp} $b$} \gls{instruction}, which
    unconditionally branches to block~$b$
\end{inlinelist}.
%
Assume also that these branch \glspl{instruction} take three cycles compared to
the other \glspl{instruction} in the \gls{target machine}, which take one cycle.

At first glance it appears that only the first jump \gls{instruction} is
selectable for implementing the conditional branch (see
\refFigure{isel-blorder-example-code-1}), leading to a total of six cycles for
the entire \gls{function}.
%
But by flipping the condition and swapping block labels (conditionally jumping
to \irBlock{b2} instead of \irBlock{b1}), the more complex jump
\gls{instruction} becomes selectable (see
\refFigure{isel-blorder-example-code-2}), bringing the cycle count to five
cycles and thus reducing the execution time by almost \SI{17}{\percent}.
%
However, although this decision may appear better at the point of
\gls{instruction selection}, it necessitates an additional jump
\gls{instruction} when ordering the \glspl{block} (because block~\irBlock{b1}
cannot fall-through to the top of itself; see
\refFigure{isel-blorder-example-code-3}).
%
This code takes eight cycles to execute, thus leading to a slowdown of
\SI{33}{\percent}.
%
The \gls{instruction selector} must therefore be aware of additional jump
\glspl{instruction} that may be required when making such decisions.


\section{Contributions}
\labelSection{intro-contributions}

The dissertation makes six contributions to the areas of \gls{code generation}
and \glsdesc{CP}:
%
\begin{contributions}
  \item \labelContribution{survey}
    It presents a comprehensive survey that
    \begin{contributions}
      \item \labelContribution{survey-spanning}
        examines over four decades of research on \gls{instruction selection},
        covering a significantly wider scope and timespan compared to existing
        surveys \cite{Cattell:1977, GanapathiEtAl:1982:Survey, Lunell:1983,
          Leupers:2000:Survey, BoulytchevLomov:2001} which are either too old
        or incomplete.
    \end{contributions}
    The survey structures the approaches according to two dimensions:
    \begin{contributions}[resume]
      \item \labelContribution{survey-principles}
        the fundamental \gls{principle} -- \gls{macro expansion}, \gls{tree
          covering}, \gls{DAG covering}, or \gls{graph covering} -- that they
        apply, and
    \end{contributions}
    \begin{contributions}[resume]
      \item \labelContribution{survey-instruction-characteristics}
        the \glspl{instruction characteristic} -- \gls{single-output.ic},
        \gls{multi-output.ic}, \gls{disjoint-output.ic}, \gls{inter-block.ic},
        and \gls{interdependent.ic} -- that they support.
    \end{contributions}
    %
    In addition, the survey
    \begin{contributions}[resume]
      \item \labelContribution{survey-problem-identification}
        identifies connections between \gls{instruction selection} and other
        \gls{code generation} problems that have yet to be investigated.
    \end{contributions}
  \item \labelContribution{representations}
    It introduces a novel \gls{program} and \gls{instruction} representation
    that
    \begin{contributions}
      \item \labelContribution{rep-data-and-control-flow}
        captures both data and control flow for entire \glspl{function} and
        \glspl{instruction},
    \end{contributions}
    which enables
    \begin{contributions}[resume]
      \item \labelContribution{rep-complex-instructions}
        an unprecedented range of instruction behaviors to be captured and
        modeled as \glspl{pattern}.
    \end{contributions}
    In addition, the representation is crucial for
    \begin{contributions}[resume]
      \item \labelContribution{rep-combining-problems}
        combining \gls{instruction selection} and \gls{global code motion} --
        two problems that have until now been considered separately from one
        another -- and solving these two problems in unison.
    \end{contributions}
  \item \labelContribution{constraint-model}
    It introduces a \gls{constraint model} and related transformations for
    \gls{universal instruction selection} which, for the first time, enables
    \begin{contributions}
      \item \labelContribution{cp-uniform-selection}
        uniform selection of data and control \glspl{instruction},
    \end{contributions}
    and integration of
    \begin{contributions}[resume]
      \item \labelContribution{cp-global-instruction-selection}
        \gls{global.is} \gls{instruction selection} with
      \item \labelContribution{cp-global-code-motion}
        \gls{global code motion}.
    \end{contributions}
    In addition, the \gls{constraint model} integrates
    \begin{contributions}[resume]
      \item \labelContribution{cp-data-copying}
        \gls{data copying},
      \item \labelContribution{cp-value-reuse}
        \gls{value reuse}, and
      \item \labelContribution{cp-block-ordering}
        \gls{block ordering}.
    \end{contributions}
  \item \labelContribution{solving-techniques}
    It introduces techniques to improve solving of the \gls{constraint model}.
  \item \labelContribution{experiments}
    It presents thorough experiments demonstrating that the approach scales to
    medium-sized \glspl{program} and yields equal or better code than
    \glspl{traditional approach}.
  \item \labelContribution{integration}
    It outlines how the \gls{constraint model} can be extended to integrate
    other \gls{code generation} tasks, such as \gls{instruction scheduling} and
    \gls{register allocation}.
\end{contributions}
%
\refTable{contributions-per-chapter} shows in which chapters each contribution
is manifested and discussed further.

\begin{table}
  \centering%
  \begin{tabular}{c@{\qquad}*{6}{c}}
    \toprule
      \tabhead chapter
    & \tabhead\refContribution{survey}
    & \tabhead\refContribution{representations}
    & \tabhead\refContribution{constraint-model}
    & \tabhead\refContribution{solving-techniques}
    & \tabhead\refContribution{experiments}
    & \tabhead\refContribution{integration}\\
    \midrule
    \refChapter*{existing-isel-techniques-and-reps}
    & \supportYes
    & \supportNo
    & \supportNo
    & \supportNo
    & \supportNo
    & \supportNo\\
    \refChapter*{universal-representation}
    & \supportNo
    & \supportYes
    & \supportNo
    & \supportNo
    & \supportNo
    & \supportNo\\
    \refChapter*{constraint-model}
    & \supportNo
    & \supportNo
    & \supportYes
    & \supportNo
    & \supportNo
    & \supportNo\\
    \refChapter*{solving-techniques}
    & \supportNo
    & \supportNo
    & \supportNo
    & \supportYes
    & \supportNo
    & \supportNo\\
    \refChapter*{comparison-with-the-state-of-the-art}
    & \supportNo
    & \supportNo
    & \supportNo
    & \supportNo
    & \supportYes
    & \supportNo\\
    \refChapter*{future-work}
    & \supportNo
    & \supportNo
    & \supportNo
    & \supportNo
    & \supportNo
    & \supportYes\\
    \refAppendix*{macro-expansion}
    & \supportYes
    & \supportNo
    & \supportNo
    & \supportNo
    & \supportNo
    & \supportNo\\
    \refAppendix*{tree-covering}
    & \supportYes
    & \supportNo
    & \supportNo
    & \supportNo
    & \supportNo
    & \supportNo\\
    \refAppendix*{dag-covering}
    & \supportYes
    & \supportNo
    & \supportNo
    & \supportNo
    & \supportNo
    & \supportNo\\
    \refAppendix*{graph-covering}
    & \supportYes
    & \supportNo
    & \supportNo
    & \supportNo
    & \supportNo
    & \supportNo\\
    \bottomrule
  \end{tabular}

  \caption{Contributions per chapter}
  \labelTable{contributions-per-chapter}
\end{table}


\section{Publications}
\labelSection{intro-publications}

This dissertation is based on material presented in the following publications:


\subsubsection{Books}

\begin{publications}
  \item \labelPublication{survey-book}
    \fullcite{HjortBlindell:2016:Survey}.
\end{publications}


\subsubsection{Conference Papers}

\begin{publications}[resume]
  \item \labelPublication{cp-paper}
    \fullcite{HjortBlindellEtAl:2015:CP}.
    %
    \begin{authorsContribution}
      The author of this dissertation conceived, designed, and implemented the
      work presented in the paper, oversaw the writing of the paper, wrote the
      majority of the text, designed the figures, and assisted in data gathering
      and analysis.
    \end{authorsContribution}
\end{publications}


\subsubsection{Articles}

\begin{publications}[resume]
  \item \labelPublication{cases-paper}
    \fullcite{HjortBlindellEtAl:2017:CASES}.
    %
    \begin{authorsContribution}
      The author conceived, designed, and implemented the work presented in the
      paper, gathered and analyzed the experimental data, oversaw the writing of
      the paper, wrote the majority of the text, and designed the figures.
    \end{authorsContribution}
\end{publications}
%
\refTable{contributions-per-publication} shows the relation between the
contributions and the publications above.

\begin{table}
  \centering%
  \begin{tabular}{c@{\qquad}*{11}{c}}
    \toprule
      \tabhead publication
    & \tabhead\refContribution{survey}
    & \tabhead\refContribution{representations}
    & \multicolumn{6}{c}{\tabhead\refContribution{constraint-model}}
    & \tabhead\refContribution{solving-techniques}
    & \tabhead\refContribution{experiments}
    & \tabhead\refContribution{integration}\\
    \cmidrule(lr){4-9}%
    &
    &
    & \tabhead\refContribution{cp-uniform-selection}
    & \tabhead\refContribution{cp-global-instruction-selection}
    & \tabhead\refContribution{cp-global-code-motion}
    & \tabhead\refContribution{cp-data-copying}
    & \tabhead\refContribution{cp-block-ordering}
    & \tabhead\refContribution{cp-value-reuse}
    &
    &
    &\\
    \midrule
    \refPublication{survey-book}
    & \supportYes
    & \supportNo
    & \supportNo
    & \supportNo
    & \supportNo
    & \supportNo
    & \supportNo
    & \supportNo
    & \supportNo
    & \supportNo
    & \supportNo\\
    \refPublication{cp-paper}
    & \supportNo
    & \supportYes
    & \supportYes
    & \supportYes
    & \supportYes
    & \supportYes
    & \supportYes
    & \supportNo
    & \supportNo
    & \supportYes
    & \supportNo\\
    \refPublication{cases-paper}
    & \supportNo
    & \supportNo
    & \supportNo
    & \supportNo
    & \supportNo
    & \supportNo
    & \supportNo
    & \supportYes
    & \supportYes
    & \supportYes
    & \supportNo\\
    \bottomrule
  \end{tabular}

  \caption{Contributions per publication}
  \labelTable{contributions-per-publication}
\end{table}

The author also participated in the following publications, which are out of
scope for the dissertation:


\subsubsection{Book Chapters, Conference Papers, and Workshop Papers}

\begin{publications}[resume]
  \item \labelPublication{survey-report}
    \fullcite{HjortBlindell:2013:Survey}.
  \item \labelPublication{scopes}
    \fullcite{CastanedaLozanoEtAl:2013:M-SCOPES}.
    %
    \begin{authorsContribution}
      The author assisted in writing the paper.
    \end{authorsContribution}
  \item \labelPublication{lctes}
    \fullcite{CastanedaLozanoEtAl:2014:LCTES}.
    %
    \begin{authorsContribution}
      The author assisted in gathering and analyzing the experimental data, and
      to writing the paper.
    \end{authorsContribution}
  \item \labelPublication{cc}
    \fullcite{CastanedaLozanoEtAl:2016:CC}.
    %
    \begin{authorsContribution}
      The author assisted in writing the paper.
    \end{authorsContribution}
  \item \labelPublication{fdl-2016}
    \fullcite{HjortBlindellEtAl:2016:FDL}.
    %
    \begin{authorsContribution}
      The author conceived, designed, and implemented the work presented in the
      paper, gathered and analyzed the experimental data, oversaw the writing of
      the paper, wrote the majority of the text, and designed the figures.
    \end{authorsContribution}
\end{publications}
%
\refPublication{survey-report} is excluded as it is subsumed and extended by
\refPublication{survey-book}. \refPublication{scopes}--\refPublication{cc} are
excluded as they are only partially related to the dissertation (they apply
\glsdesc{CP} to solve \gls{register allocation} and \gls{instruction scheduling}
without considering \gls{instruction selection}). \refPublication{fdl-2016} is
excluded as it belongs to a different topic entirely (high-level code generation
for graphics processors).


\section{Research Methodology}
\labelSection{intro-research-methodology}

We begin with a thorough and systematic literature review to identify the
strengths and limitations of existing \gls{instruction selection} techniques and
common denominators among them.
%
As part of this study, four \gls{instruction selection} \glspl{principle} and
five \glspl{instruction characteristic} are identified and the techniques are
systematically classified accordingly.
%
This classification enables us to recognize that certain classes of
\glspl{instruction} are poorly supported by existing \gls{instruction selection}
techniques.
%
In particular this is due to lack of appropriate \gls{program} and
\gls{instruction} representations.

Having established the need for new representations, we identify a set of
requirements that such a representation must fulfill.
%
We then build a new representation by unifying two existing, well-established
representations -- one for capturing data flow and another for capturing control
flow -- and then augment the result as needed until all requirements are met.
%
As is common, we then apply a traditional \gls{subgraph isomorphism} algorithm
for doing \gls{pattern matching} on this new representation.

With the new representation at hand, we proceed with building the
\gls{constraint model}.
%
For each task to be integrated, we first identify what constitutes a
\gls{solution} to this task and then add the necessary \glspl{variable} and
\glspl{constraint} to enforce such \glspl{solution}.
%
If more than one design choice exists for integrating the same task, then we
implement both as separate \glsplshort{constraint model} and evaluate which is
better before proceeding.
%
This is because the tasks are orthogonal from one another and can therefore be
evaluated in isolation.

We evaluate a \gls{constraint model} by applying it on a set of \glspl{function}
which are sampled from a well-established benchmark suite.
%
The sampling is done using a sound clustering method to ensure diversity among
the selected \glspl{function}.
%
All measurements are averaged over multiple runs to mitigate deviations, and
well-established methods from statistics are applied in order to draw sound
conclusions.

Once all tasks are integrated, we apply a range of solving techniques in order
to increase scalability and robustness.
%
Because these techniques influence one another, we first evaluate the usefulness
of each solving technique individually in order to form groups of solving
techniques and then evaluate each group as a whole.
%
This is to avoid unreasonably long experiment runtimes.

Using the \gls{constraint model} with the best design choices and solving
techniques, we then evaluate the significance of our approach by comparing its
\glspl{solution} with those produced by the state of the art.


\section{Outline}
\labelSection{intro-outline}

\begin{figure}
  \centering%
  % Copyright (c) 2017, Gabriel Hjort Blindell <ghb@kth.se>
%
% This work is licensed under a Creative Commons 4.0 International License (see
% LICENSE file or visit <http://creativecommons.org/licenses/by/4.0/> for a copy
% of the license).
%
\begingroup%
\figureFont\figureFontSize%
\def\nodeSep{4mm}%
\def\belowstrut{\vrule height 0pt depth 2pt width 0pt}%
\begin{tikzpicture}[%
    every node/.style={
      nothing,
      node distance=\nodeSep,
      inner sep=1mm,
      font=\belowstrut,
    },
    reading order/.style={
      ->,
      line width=1.5\normalLineWidth,
    },
    category line/.style={
      line width=2\normalLineWidth,
      draw=shade2,
      dashed,
    },
    category label/.style={
      nothing,
      node distance=0,
      inner xsep=1mm,
      inner ysep=2mm,
      font=\bfseries\scshape\smaller,
    },
  ]

  \node (graphs) {\namerefAppendix{graph-definitions}};

  \node [below=2.5*\nodeSep of graphs] (ex-isel-techs)
        {%
          \parbox{4.3cm}{%
            \centering%
            \namerefChapter{existing-isel-techniques-and-reps}%
          }%
        };
  \node [below=of ex-isel-techs]
        (macro-expansion) {\namerefAppendix{macro-expansion}};
  \node [below=of macro-expansion]
        (tree-covering)   {\namerefAppendix{tree-covering}};
  \node [below=of tree-covering]
        (dag-covering)    {\namerefAppendix{dag-covering}};
  \node [below=of dag-covering]
        (graph-covering)  {\namerefAppendix{graph-covering}};

  \node [right=3*\nodeSep of ex-isel-techs.east |- macro-expansion] (uni-rep)
        {%
          \parbox{2.1cm}{%
            \centering%
            \namerefChapter{universal-representation}%
          }%
        };
  \node [below=2.*\nodeSep of uni-rep]
        (model) {\namerefChapter{constraint-model}};
  \node [below right=1.5*\nodeSep and 1.5*\nodeSep of model.south]
        (solv-techs) {\namerefChapter{solving-techniques}};
  \node [below right=\nodeSep and \nodeSep of solv-techs.south]
        (minizinc)
        {%
          \parbox{2.1cm}{%
            \centering%
            \namerefAppendix{minizinc-implementation}%
          }%
        };
  \node [below=of minizinc.south -| model] (comparison)
        {%
          \parbox{2.5cm}{%
            \centering%
            \namerefChapter{comparison-with-the-state-of-the-art}%
          }%
        };

  \node [below=2*\nodeSep of comparison.south west, xshift=-1.5*\nodeSep]
        (future-work) {\namerefChapter{future-work}};

  \node [below=1.5*\nodeSep of future-work.south -| comparison]
        (conclusions) {\namerefChapter*{conclusions}};

  \coordinate (next-to-uni-rep) at ([xshift=1.5*\nodeSep] uni-rep.east);
  \node at (graphs -| next-to-uni-rep)
        (cp) {\namerefChapter{constraint-programming}};

  \coordinate (left-border) at ([xshift=-6mm] ex-isel-techs.west);
  \coordinate (right-border) at (minizinc.east);

  \begin{scope}[category line]
    \coordinate (background-y) at
                ($(graphs.south) !.33! (ex-isel-techs.north)$);
    \draw (background-y -| left-border)
          --
          (background-y -| right-border);
    \coordinate (tmp-a) at (background-y -| left-border);
    \coordinate (tmp-b) at (tmp-a |- graphs.north);
    \coordinate (tmp-c) at ([yshift=\nodeSep] tmp-b);
    \node [category label, right=of tmp-c, inner ysep=0] {background};

    \node [category label, below right=of tmp-a] {literature survey};

    \coordinate (survey-y) at ([yshift=-\nodeSep] graph-covering.south);
    \coordinate (survey-x) at ([xshift=\nodeSep] ex-isel-techs.east);
    \coordinate (survey-x-y) at (survey-x |- survey-y);
    \draw [rounded corners=6pt]
          (survey-y -| left-border)
          --
          ([xshift=-3*\nodeSep] survey-x-y)
          --
          ([yshift= 3*\nodeSep] survey-x-y)
          --
          (survey-x |- background-y);

    \coordinate (tmp-a) at (survey-y -| left-border);
    \node [category label, below right=of tmp-a]
          {universal instruction selection};

    \coordinate (ending-y) at
                ($(comparison.south) !.5! (future-work.north -| comparison)$);
    \draw (ending-y -| left-border)
          --
          (ending-y -| right-border);
    \coordinate (tmp-a) at (ending-y -| left-border);
    \node [category label, below right=of tmp-a]
          {ending};
  \end{scope}

  \begin{scope}[reading order]
    \draw (graphs) -- (ex-isel-techs);
    \draw (ex-isel-techs) -- (macro-expansion);
    \draw (macro-expansion) -- (tree-covering);
    \draw (tree-covering) -- (dag-covering);
    \draw (dag-covering) -- (graph-covering);

    \draw [rounded corners=12pt]
          (ex-isel-techs) -| (uni-rep);
    \draw (cp.south)
          .. controls +(down:2.5cm) and +(40:1.5cm) ..
          (model.20);
    \draw (uni-rep) -- (model);
    \draw (model) -- (comparison);
    \draw (model) -- (solv-techs);
    \draw (model.-155)
          .. controls +(-140:1.8cm) and +(up:2cm) ..
          (future-work.north);
    \draw (solv-techs) -- (minizinc);
    \draw (comparison) -- (conclusions);
    \draw (future-work) -- (conclusions);
    \draw (solv-techs.south)
          .. controls +(down:2.5cm) and +(40:2.2cm) ..
          (conclusions.27);
  \end{scope}
\end{tikzpicture}%
\endgroup%


  \caption{Structure of the dissertation}
  \labelFigure{dissertation-structure}
\end{figure}

As shown in \refFigure{dissertation-structure}, the dissertation is structured
into four parts:
%
\begin{description}
  \item[Background]
    Provides necessary background material:
    %
    \begin{inlinelist}[itemjoin={; }, itemjoin*={; and}]
      \item \refChapter{constraint-programming} describes \glsdesc{CP}, which is
        a combinatorial optimization method on which our approach is based
      \item \refAppendix{graph-definitions} contains exact definitions of
        \glspl{graph} and related terms which are used throughout the
        dissertation
    \end{inlinelist}.
  \item[Literature survey]
    Discusses existing \gls{instruction selection} techniques:
    %
    \begin{inlinelist}[itemjoin={; }, itemjoin*={; and}]
      \item \refChapter{existing-isel-techniques-and-reps} covers the techniques
        relevant for \gls{universal instruction selection}
      \item \refAppendixList{macro-expansion, tree-covering, dag-covering,
        graph-covering} cover the remaining techniques
    \end{inlinelist}.
  \item[Universal instruction selection]
    Presents our approach:
    %
    \begin{inlinelist}[itemjoin={; }, itemjoin*={; and}]
      \item \refChapter{universal-representation} introduces the \gls{universal
        representation}
      \item \refChapter{constraint-model} introduces the \gls{constraint model}
      \item \refChapter{solving-techniques} introduces the solving techniques
      \item \refChapter{comparison-with-the-state-of-the-art} compares our
        approach with the state of the art
      \item \refAppendix{minizinc-implementation} provides an implementation of
        the \gls{constraint model}, written in \gls{MiniZinc} (a high-level
        modeling language)
    \end{inlinelist}.
  \item[Ending]
    Closes the dissertation:
    %
    \begin{inlinelist}[itemjoin={; }, itemjoin*={; and}]
      \item \refChapter{future-work} describes future work
      \item \refChapter{conclusions} delivers the conclusions
    \end{inlinelist}.
\end{description}
