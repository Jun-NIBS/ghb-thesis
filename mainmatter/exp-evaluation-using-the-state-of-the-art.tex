% Copyright (c) 2017-2018, Gabriel Hjort Blindell <ghb@kth.se>
%
% This work is licensed under a Creative Commons Attribution-NoDerivatives 4.0
% International License (see LICENSE file or visit
% <http://creativecommons.org/licenses/by-nc-nd/4.0/> for details).

\chapter[Experimental Evaluation Using the State of the Art]%
        {Experimental Evaluation\\ Using the State of the Art}
\labelChapter{exp-evaluation-using-the-state-of-the-art}

This chapter evaluates how \gls{universal instruction selection} compares
against a state-of-the-art \gls{compiler}.
%
This is done in \refSection{evaluation-unison-vs-llvm}.
%
Since the approach is able to leverage selection of \gls{SIMD.i}
\glspl{instruction}, we also evaluate this impact in
\refSection{evaluation-with-or-without-simds}.


\section{Unison \versus LLVM}
\labelSection{evaluation-unison-vs-llvm}

We evaluate the impact of the approach by comparing the cost (that is, the total
number of cycles, as described in \refChapter{constraint-model} on
\refPageOfSection{cm-objective-function}) of \glspl{solution} produced by the
approach with the \glspl{solution} produced by \mbox{\gls{LLVM} 3.8} -- a
state-of-the-art \gls{compiler}.


\paragraph{Setup}

When filtering, we remove all \glspl{function} that have fewer than
\num{50}~\gls{LLVM} \gls{IR} \glspl{instruction} and more than
\num{200}~\glspl{instruction}.
%
Anything smaller will most likely not show any gain using the approach, and
anything larger will lead to unreasonably long experiment runtimes.
%
This leaves a pool of \num{284}~\glspl{function} up to medium size, from which
we then draw \num{20}~samples.

To curb experiment runtimes, we apply a time limit of \SI{600}{\s} to the
\gls{constraint solver}.
%
For any given \gls{function}, the last \gls{solution} found is considered
optimal if and only if the \glsshort{constraint solver} has finished its
execution within the time limit.
%
When using an upper cost bound, we take the cost for the \gls{solution} computed
by \gls{LLVM} for the given \gls{function}.


\paragraph{Results}

\RefFigure{unison-vs-llvm-cycles-plot} shows the normalized \gls{solution}
costs, with \gls{LLVM} as \gls{baseline} and \gls{universal instruction
  selection} as \gls{subject}.
%
\begin{figure}
  \centering%
  \maxsizebox{\textwidth}{!}{%
    \trimBarchartPlot{%
      \begin{tikzpicture}[gnuplot]
%% generated with GNUPLOT 5.0p4 (Lua 5.2; terminal rev. 99, script rev. 100)
%% lör  3 feb 2018 16:36:32
\path (0.000,0.000) rectangle (12.500,8.750);
\gpcolor{rgb color={0.753,0.753,0.753}}
\gpsetlinetype{gp lt axes}
\gpsetdashtype{gp dt axes}
\gpsetlinewidth{0.50}
\draw[gp path] (1.380,1.779)--(36.945,1.779);
\gpcolor{color=gp lt color border}
\node[gp node right] at (1.380,1.779) {\plotZCNormTics{0}};
\gpcolor{rgb color={0.753,0.753,0.753}}
\draw[gp path] (1.380,2.439)--(36.945,2.439);
\gpcolor{color=gp lt color border}
\node[gp node right] at (1.380,2.439) {\plotZCNormTics{0.02}};
\gpcolor{rgb color={0.753,0.753,0.753}}
\draw[gp path] (1.380,3.099)--(36.945,3.099);
\gpcolor{color=gp lt color border}
\node[gp node right] at (1.380,3.099) {\plotZCNormTics{0.04}};
\gpcolor{rgb color={0.753,0.753,0.753}}
\draw[gp path] (1.380,3.760)--(36.945,3.760);
\gpcolor{color=gp lt color border}
\node[gp node right] at (1.380,3.760) {\plotZCNormTics{0.06}};
\gpcolor{rgb color={0.753,0.753,0.753}}
\draw[gp path] (1.380,4.420)--(36.945,4.420);
\gpcolor{color=gp lt color border}
\node[gp node right] at (1.380,4.420) {\plotZCNormTics{0.08}};
\gpcolor{rgb color={0.753,0.753,0.753}}
\draw[gp path] (1.380,5.080)--(36.945,5.080);
\gpcolor{color=gp lt color border}
\node[gp node right] at (1.380,5.080) {\plotZCNormTics{0.1}};
\gpcolor{rgb color={0.753,0.753,0.753}}
\draw[gp path] (1.380,5.740)--(36.945,5.740);
\gpcolor{color=gp lt color border}
\node[gp node right] at (1.380,5.740) {\plotZCNormTics{0.12}};
\gpcolor{rgb color={0.753,0.753,0.753}}
\draw[gp path] (1.380,6.400)--(36.945,6.400);
\gpcolor{color=gp lt color border}
\node[gp node right] at (1.380,6.400) {\plotZCNormTics{0.14}};
\gpcolor{rgb color={0.753,0.753,0.753}}
\draw[gp path] (1.380,7.061)--(36.945,7.061);
\gpcolor{color=gp lt color border}
\node[gp node right] at (1.380,7.061) {\plotZCNormTics{0.16}};
\gpcolor{rgb color={0.753,0.753,0.753}}
\draw[gp path] (1.380,7.721)--(36.945,7.721);
\gpcolor{color=gp lt color border}
\node[gp node right] at (1.380,7.721) {\plotZCNormTics{0.18}};
\gpcolor{rgb color={0.753,0.753,0.753}}
\draw[gp path] (1.380,8.381)--(36.945,8.381);
\gpcolor{color=gp lt color border}
\node[gp node right] at (1.380,8.381) {\plotZCNormTics{0.2}};
\node[gp node left,rotate=-30] at (3.110,1.534) {\functionName{alloc_name_is_s.}};
\node[gp node left,rotate=-30] at (4.803,1.534) {\functionName{alloc_save_spac.}};
\node[gp node left,rotate=-30] at (6.497,1.534) {\functionName{build_ycc_rgb_t.}};
\node[gp node left,rotate=-30] at (8.190,1.534) {\functionName{checksum}};
\node[gp node left,rotate=-30] at (9.884,1.534) {\functionName{debug_dump_byte.}};
\node[gp node left,rotate=-30] at (11.577,1.534) {\functionName{gl_EnableClient.}};
\node[gp node left,rotate=-30] at (13.271,1.534) {\functionName{gl_init_lists}};
\node[gp node left,rotate=-30] at (14.965,1.534) {\functionName{gl_save_Color4u.}};
\node[gp node left,rotate=-30] at (16.658,1.534) {\functionName{gl_save_EvalPoi.}};
\node[gp node left,rotate=-30] at (18.352,1.534) {\functionName{gl_save_PushMat.}};
\node[gp node left,rotate=-30] at (20.045,1.534) {\functionName{gl_swap4}};
\node[gp node left,rotate=-30] at (21.739,1.534) {\functionName{gl_TexImage3DEX.}};
\node[gp node left,rotate=-30] at (23.432,1.534) {\functionName{gp_enumerate_fi.}};
\node[gp node left,rotate=-30] at (25.126,1.534) {\functionName{gpk_open}};
\node[gp node left,rotate=-30] at (26.820,1.534) {\functionName{gs_interp_init}};
\node[gp node left,rotate=-30] at (28.513,1.534) {\functionName{jinit_forward_d.}};
\node[gp node left,rotate=-30] at (30.207,1.534) {\functionName{jpeg_read_heade.}};
\node[gp node left,rotate=-30] at (31.900,1.534) {\functionName{trueRandAccum}};
\node[gp node left,rotate=-30] at (33.594,1.534) {\functionName{write_file_trai.}};
\node[gp node left,rotate=-30] at (35.287,1.534) {\functionName{zero}};
\gpsetlinetype{gp lt border}
\gpsetdashtype{gp dt solid}
\gpsetlinewidth{1.00}
\draw[gp path] (1.380,8.381)--(1.380,1.779)--(36.945,1.779)--(36.945,8.381)--cycle;
\gpcolor{rgb color={0.000,0.000,0.000}}
\draw[gp path] (1.380,1.779)--(1.739,1.779)--(2.098,1.779)--(2.458,1.779)--(2.817,1.779)%
  --(3.176,1.779)--(3.535,1.779)--(3.895,1.779)--(4.254,1.779)--(4.613,1.779)--(4.972,1.779)%
  --(5.332,1.779)--(5.691,1.779)--(6.050,1.779)--(6.409,1.779)--(6.769,1.779)--(7.128,1.779)%
  --(7.487,1.779)--(7.846,1.779)--(8.206,1.779)--(8.565,1.779)--(8.924,1.779)--(9.283,1.779)%
  --(9.643,1.779)--(10.002,1.779)--(10.361,1.779)--(10.720,1.779)--(11.080,1.779)--(11.439,1.779)%
  --(11.798,1.779)--(12.157,1.779)--(12.517,1.779)--(12.876,1.779)--(13.235,1.779)--(13.594,1.779)%
  --(13.953,1.779)--(14.313,1.779)--(14.672,1.779)--(15.031,1.779)--(15.390,1.779)--(15.750,1.779)%
  --(16.109,1.779)--(16.468,1.779)--(16.827,1.779)--(17.187,1.779)--(17.546,1.779)--(17.905,1.779)%
  --(18.264,1.779)--(18.624,1.779)--(18.983,1.779)--(19.342,1.779)--(19.701,1.779)--(20.061,1.779)%
  --(20.420,1.779)--(20.779,1.779)--(21.138,1.779)--(21.498,1.779)--(21.857,1.779)--(22.216,1.779)%
  --(22.575,1.779)--(22.935,1.779)--(23.294,1.779)--(23.653,1.779)--(24.012,1.779)--(24.372,1.779)%
  --(24.731,1.779)--(25.090,1.779)--(25.449,1.779)--(25.808,1.779)--(26.168,1.779)--(26.527,1.779)%
  --(26.886,1.779)--(27.245,1.779)--(27.605,1.779)--(27.964,1.779)--(28.323,1.779)--(28.682,1.779)%
  --(29.042,1.779)--(29.401,1.779)--(29.760,1.779)--(30.119,1.779)--(30.479,1.779)--(30.838,1.779)%
  --(31.197,1.779)--(31.556,1.779)--(31.916,1.779)--(32.275,1.779)--(32.634,1.779)--(32.993,1.779)%
  --(33.353,1.779)--(33.712,1.779)--(34.071,1.779)--(34.430,1.779)--(34.790,1.779)--(35.149,1.779)%
  --(35.508,1.779)--(35.867,1.779)--(36.227,1.779)--(36.586,1.779)--(36.945,1.779);
\gpfill{rgb color={0.333,0.333,0.333}} (4.697,1.779)--(5.403,1.779)--(5.403,5.435)--(4.697,5.435)--cycle;
\gpcolor{color=gp lt color border}
\draw[gp path] (4.697,1.779)--(4.697,5.434)--(5.402,5.434)--(5.402,1.779)--cycle;
\gpfill{rgb color={0.333,0.333,0.333}} (8.084,1.779)--(8.790,1.779)--(8.790,1.981)--(8.084,1.981)--cycle;
\draw[gp path] (8.084,1.779)--(8.084,1.980)--(8.789,1.980)--(8.789,1.779)--cycle;
\gpfill{rgb color={0.333,0.333,0.333}} (9.777,1.779)--(10.484,1.779)--(10.484,2.185)--(9.777,2.185)--cycle;
\draw[gp path] (9.777,1.779)--(9.777,2.184)--(10.483,2.184)--(10.483,1.779)--cycle;
\gpfill{rgb color={0.333,0.333,0.333}} (11.471,1.779)--(12.178,1.779)--(12.178,2.349)--(11.471,2.349)--cycle;
\draw[gp path] (11.471,1.779)--(11.471,2.348)--(12.177,2.348)--(12.177,1.779)--cycle;
\gpfill{rgb color={0.333,0.333,0.333}} (14.858,1.779)--(15.565,1.779)--(15.565,2.011)--(14.858,2.011)--cycle;
\draw[gp path] (14.858,1.779)--(14.858,2.010)--(15.564,2.010)--(15.564,1.779)--cycle;
\gpfill{rgb color={0.333,0.333,0.333}} (16.552,1.779)--(17.258,1.779)--(17.258,2.025)--(16.552,2.025)--cycle;
\draw[gp path] (16.552,1.779)--(16.552,2.024)--(17.257,2.024)--(17.257,1.779)--cycle;
\gpfill{rgb color={0.333,0.333,0.333}} (18.245,1.779)--(18.952,1.779)--(18.952,2.058)--(18.245,2.058)--cycle;
\draw[gp path] (18.245,1.779)--(18.245,2.057)--(18.951,2.057)--(18.951,1.779)--cycle;
\gpfill{rgb color={0.333,0.333,0.333}} (19.939,1.779)--(20.645,1.779)--(20.645,1.837)--(19.939,1.837)--cycle;
\draw[gp path] (19.939,1.779)--(19.939,1.836)--(20.644,1.836)--(20.644,1.779)--cycle;
\gpfill{rgb color={0.333,0.333,0.333}} (21.632,1.779)--(22.339,1.779)--(22.339,1.810)--(21.632,1.810)--cycle;
\draw[gp path] (21.632,1.779)--(21.632,1.809)--(22.338,1.809)--(22.338,1.779)--cycle;
\gpfill{rgb color={0.333,0.333,0.333}} (23.326,1.779)--(24.033,1.779)--(24.033,4.369)--(23.326,4.369)--cycle;
\draw[gp path] (23.326,1.779)--(23.326,4.368)--(24.032,4.368)--(24.032,1.779)--cycle;
\gpfill{rgb color={0.333,0.333,0.333}} (25.019,1.779)--(25.726,1.779)--(25.726,7.744)--(25.019,7.744)--cycle;
\draw[gp path] (25.019,1.779)--(25.019,7.743)--(25.725,7.743)--(25.725,1.779)--cycle;
\gpfill{rgb color={0.333,0.333,0.333}} (26.713,1.779)--(27.420,1.779)--(27.420,2.568)--(26.713,2.568)--cycle;
\draw[gp path] (26.713,1.779)--(26.713,2.567)--(27.419,2.567)--(27.419,1.779)--cycle;
\gpfill{rgb color={0.333,0.333,0.333}} (30.100,1.779)--(30.807,1.779)--(30.807,1.994)--(30.100,1.994)--cycle;
\draw[gp path] (30.100,1.779)--(30.100,1.993)--(30.806,1.993)--(30.806,1.779)--cycle;
\gpfill{rgb color={0.333,0.333,0.333}} (31.794,1.779)--(32.500,1.779)--(32.500,3.668)--(31.794,3.668)--cycle;
\draw[gp path] (31.794,1.779)--(31.794,3.667)--(32.499,3.667)--(32.499,1.779)--cycle;
\gpfill{rgb color={0.333,0.333,0.333}} (35.181,1.779)--(35.888,1.779)--(35.888,1.930)--(35.181,1.930)--cycle;
\draw[gp path] (35.181,1.779)--(35.181,1.929)--(35.887,1.929)--(35.887,1.779)--cycle;
\node[gp node center] at (3.350,1.963) {\plotBarNormValueNoSolution};
\node[gp node center] at (5.043,5.618) {\plotBarNormValue{0.110717}};
\node[gp node center] at (6.737,1.963) {\plotBarNormValueNoSolution};
\node[gp node center] at (8.430,2.164) {\plotBarNormValue{0.006095}};
\node[gp node center] at (10.124,2.368) {\plotBarNormValue{0.012263}};
\node[gp node center] at (11.817,2.532) {\plotBarNormValue{0.017241}};
\node[gp node center] at (13.511,1.963) {\plotBarNormValueNoSolution};
\node[gp node center] at (15.205,2.194) {\plotBarNormValue{0.007012}};
\node[gp node center] at (16.898,2.208) {\plotBarNormValue{0.007437}};
\node[gp node center] at (18.592,2.241) {\plotBarNormValue{0.008424}};
\node[gp node center] at (20.285,2.020) {\plotBarNormValue{0.001725}};
\node[gp node center] at (21.979,1.993) {\plotBarNormValue{0.000908}};
\node[gp node center] at (23.672,4.552) {\plotBarNormValue{0.078442}};
\node[gp node center] at (25.366,7.927) {\plotBarNormValue{0.180670}};
\node[gp node center] at (27.060,2.751) {\plotBarNormValue{0.023877}};
\node[gp node center] at (28.753,1.963) {\plotBarNormValueNoSolution};
\node[gp node center] at (30.447,2.177) {\plotBarNormValue{0.006472}};
\node[gp node center] at (32.140,3.851) {\plotBarNormValue{0.057194}};
\node[gp node center] at (33.834,1.963) {\plotBarNormValueNoSolution};
\node[gp node center] at (35.527,2.113) {\plotBarNormValue{0.004532}};
\node[gp node center] at (27.060,2.782) {\plotSubOptSymbol};
\draw[gp path] (1.380,8.381)--(1.380,1.779)--(36.945,1.779)--(36.945,8.381)--cycle;
%% coordinates of the plot area
\gpdefrectangularnode{gp plot 1}{\pgfpoint{1.380cm}{1.779cm}}{\pgfpoint{36.945cm}{8.381cm}}
\end{tikzpicture}
%% gnuplot variables
%
    }%
  }

  \caption[%
            Plot for evaluating universal instruction selection's impact on code
            quality in comparison with LLVM%
          ]%
          {%
            Solution costs produced by universal instruction selection,
            normalized to those produced by LLVM.
            %
            GMI:~\printGMI{%
              \UnisonVsLlvmHexagonFiveCyclesSpeedupCyclesRegularSpeedupGmean%
            },
            CI:~\printGMICI{%
              \UnisonVsLlvmHexagonFiveCyclesSpeedupCyclesRegularSpeedupCiMin%
            }{%
              \UnisonVsLlvmHexagonFiveCyclesSpeedupCyclesRegularSpeedupCiMax%
            }.
            %
            \Glspl{function} whose bars are marked with two dots are those for
            which the \gls{subject} does not find the optimal solution, and
            \glspl{function} marked with \barNormValueNoSolution{} are those
            where the solution produced by \gls{LLVM} is already optimal \wrt
            the model%
          }
  \labelFigure{unison-vs-llvm-cycles-plot}
\end{figure}
%
The size of the \glspl{UF graph} range from \num{189} to
\num{1524}~\glspl{node}.
%
The costs range from
\printCycles{\UnisonVsLlvmHexagonFiveCyclesSpeedupCyclesAvgMin} to
\printCycles{\UnisonVsLlvmHexagonFiveCyclesSpeedupCyclesAvgMax}, with a maximum
coefficient of variation of
\num{\UnisonVsLlvmHexagonFiveCyclesSpeedupCyclesCvMax}.
%
The solving times range from
\printSolvingTime{\UnisonVsLlvmHexagonFiveCyclesSpeedupPrePlusSolvingTimeAvgMin}
to
\printSolvingTime{\UnisonVsLlvmHexagonFiveCyclesSpeedupPrePlusSolvingTimeAvgMax}
with a \gls{CV} of
\num{\UnisonVsLlvmHexagonFiveCyclesSpeedupPrePlusSolvingTimeCvMax}.
%
The \gls{GMI} is \printGMI{%
  \UnisonVsLlvmHexagonFiveCyclesSpeedupCyclesRegularSpeedupGmean%
} with \gls{CI}~\printGMICI{%
  \UnisonVsLlvmHexagonFiveCyclesSpeedupCyclesRegularSpeedupCiMin%
}{%
  \UnisonVsLlvmHexagonFiveCyclesSpeedupCyclesRegularSpeedupCiMax%
}.

We see that \gls{universal instruction selection} produces \glspl{solution} with
significantly less cost than those produced by \gls{LLVM} (up to~\printZCNorm{%
  \UnisonVsLlvmHexagonFiveCyclesSpeedupCyclesZeroCenteredSpeedupMax%
} improvement).
%
This is predominantly due to the combination of \gls{global.is}
\gls{instruction selection}, \gls{global code motion}, and \gls{block ordering}.
%
In three cases (\cCode*{alloc\_save\_spac}, \cCode*{gp\_enumerate\_fi}, and
\cCode*{gpk\_open}), for example, the approach is able to reduce cost by lifting
computations, in particular constant loads, out of \glspl{block} with high
execution frequency into \glspl{block} with lower frequency.
%
In another case (\cCode*{checksum}), the approach is able to move an addition
and memory \gls{operation} into the same \gls{block} and implement both using a
single auto-increment memory \gls{instruction}, whereas \gls{LLVM} must
implement these computations using two \glspl{instruction}.
%
Such improvements are only possible when integrating \gls{global.is}
\gls{instruction selection} with \gls{global code motion}.

In three other cases (\cCode*{jpeg\_read\_header}, \cCode*{gl\_TexImage3DEX},
and \cCode*{gl\_EnableClient}), the approach is able to reorder the
\glspl{block} to remove one to two jump \glspl{instruction}.
%
Such improvements are only possible when integrating \gls{global.is}
\gls{instruction selection} with \gls{block ordering}.


\paragraph{Conclusions}

From the results for these experiments, we conclude that \gls{universal
  instruction selection} generates code of equal or better quality compared to
the state of the art for up to medium-sized \glspl{function}.


\section{Impact of SIMD instructions}
\labelSection{evaluation-with-or-without-simds}

We evaluate the impact of selecting \gls{SIMD.i} \glspl{instruction}
by comparing the cost of \glspl{solution} produced from two \glspl{pattern set}
derived from \gls{Hexagon}:
%
\begin{patternList}
  \item \labelPattern{no-simd}
    one with no \gls{SIMD.i} \glspl{instruction}
  \item \labelPattern{with-simd}
    one with \num{2}- and \num{4}-way \cCode*{add}, \cCode*{sub}, \cCode*{and},
    and \cCode*{or} \glspl{instruction}
\end{patternList}.


\paragraph{Setup}

When filtering, we again all \glspl{function} that have fewer than
\num{50}~\gls{LLVM} \gls{IR} \glspl{instruction} and more than
\num{150}~\glspl{instruction}.
%
Anything smaller will most likely not have enough data parallelism for selection
of \gls{SIMD.i} \glspl{instruction}, and anything larger will lead to
unreasonably long experiment runtimes.
%
To increase the chance of data parallelism, we also remove all \glspl{function}
not containing at least two addition, subtraction, logical-and, or logical-or
\glspl{instruction}.
%
This leaves a pool of \num{221}~\glspl{function}, from which we then draw
\num{20}~samples.

In this experiment we do not apply an upper bound in this case as that may
prevent interesting \gls{solution} that make use of \gls{SIMD.i}
\gls{instruction}.
%
Note that no \gls{loop unrolling}%
%
\footnote{%
  \Gls!{loop unrolling} is the task of duplicating the body of a loop in order
  to increase data parallelism at the cost of increasing code size.%
}
%
is performed on any of the \glspl{function} prior to \gls{instruction
  selection}.


\paragraph{Results}

\RefFigure{simd-vs-without-cycles-plot} shows the normalized \gls{solution}
costs for the two \glspl{pattern set} describe above, with \gls{pattern
  set}~\refPattern{no-simd} as \gls{baseline} and \gls{pattern
  set}~\refPattern{with-simd} as \gls{subject}.
%
\begin{figure}
  \centering%
  \maxsizebox{\textwidth}{!}{%
    \trimBarchartPlot{%
      \begin{tikzpicture}[gnuplot]
%% generated with GNUPLOT 5.0p4 (Lua 5.2; terminal rev. 99, script rev. 100)
%% lör  3 feb 2018 16:43:20
\path (0.000,0.000) rectangle (12.500,8.750);
\gpcolor{rgb color={0.753,0.753,0.753}}
\gpsetlinetype{gp lt axes}
\gpsetdashtype{gp dt axes}
\gpsetlinewidth{0.50}
\draw[gp path] (1.380,1.779)--(36.945,1.779);
\gpcolor{color=gp lt color border}
\node[gp node right] at (1.380,1.779) {\plotZCNormTics{0}};
\gpcolor{rgb color={0.753,0.753,0.753}}
\draw[gp path] (1.380,2.879)--(36.945,2.879);
\gpcolor{color=gp lt color border}
\node[gp node right] at (1.380,2.879) {\plotZCNormTics{0.02}};
\gpcolor{rgb color={0.753,0.753,0.753}}
\draw[gp path] (1.380,3.980)--(36.945,3.980);
\gpcolor{color=gp lt color border}
\node[gp node right] at (1.380,3.980) {\plotZCNormTics{0.04}};
\gpcolor{rgb color={0.753,0.753,0.753}}
\draw[gp path] (1.380,5.080)--(36.945,5.080);
\gpcolor{color=gp lt color border}
\node[gp node right] at (1.380,5.080) {\plotZCNormTics{0.06}};
\gpcolor{rgb color={0.753,0.753,0.753}}
\draw[gp path] (1.380,6.180)--(36.945,6.180);
\gpcolor{color=gp lt color border}
\node[gp node right] at (1.380,6.180) {\plotZCNormTics{0.08}};
\gpcolor{rgb color={0.753,0.753,0.753}}
\draw[gp path] (1.380,7.281)--(36.945,7.281);
\gpcolor{color=gp lt color border}
\node[gp node right] at (1.380,7.281) {\plotZCNormTics{0.1}};
\gpcolor{rgb color={0.753,0.753,0.753}}
\draw[gp path] (1.380,8.381)--(36.945,8.381);
\gpcolor{color=gp lt color border}
\node[gp node right] at (1.380,8.381) {\plotZCNormTics{0.12}};
\node[gp node left,rotate=-30] at (3.110,1.534) {\functionName{color_cmyk_to_r.}};
\node[gp node left,rotate=-30] at (4.803,1.534) {\functionName{debug_print_str.}};
\node[gp node left,rotate=-30] at (6.497,1.534) {\functionName{delete_contours}};
\node[gp node left,rotate=-30] at (8.190,1.534) {\functionName{fill_input_buff.}};
\node[gp node left,rotate=-30] at (9.884,1.534) {\functionName{free_new_ctrl}};
\node[gp node left,rotate=-30] at (11.577,1.534) {\functionName{gl_read_alpha_s.}};
\node[gp node left,rotate=-30] at (13.271,1.534) {\functionName{gluBeginPolygon}};
\node[gp node left,rotate=-30] at (14.965,1.534) {\functionName{gx_curve_cursor.}};
\node[gp node left,rotate=-30] at (16.658,1.534) {\functionName{inflate_block}};
\node[gp node left,rotate=-30] at (18.352,1.534) {\functionName{is_compromised}};
\node[gp node left,rotate=-30] at (20.045,1.534) {\functionName{jinit_inverse_d.}};
\node[gp node left,rotate=-30] at (21.739,1.534) {\functionName{jpeg_finish_out.}};
\node[gp node left,rotate=-30] at (23.432,1.534) {\functionName{jpeg_stdio_src}};
\node[gp node left,rotate=-30] at (25.126,1.534) {\functionName{motion_vector}};
\node[gp node left,rotate=-30] at (26.820,1.534) {\functionName{mp_dmul}};
\node[gp node left,rotate=-30] at (28.513,1.534) {\functionName{mp_shortmod}};
\node[gp node left,rotate=-30] at (30.207,1.534) {\functionName{pack_tree_iter}};
\node[gp node left,rotate=-30] at (31.900,1.534) {\functionName{pbm_getint}};
\node[gp node left,rotate=-30] at (33.594,1.534) {\functionName{post_process_pr.}};
\node[gp node left,rotate=-30] at (35.287,1.534) {\functionName{zero}};
\gpsetlinetype{gp lt border}
\gpsetdashtype{gp dt solid}
\gpsetlinewidth{1.00}
\draw[gp path] (1.380,8.381)--(1.380,1.779)--(36.945,1.779)--(36.945,8.381)--cycle;
\gpcolor{rgb color={0.000,0.000,0.000}}
\draw[gp path] (1.380,1.779)--(1.739,1.779)--(2.098,1.779)--(2.458,1.779)--(2.817,1.779)%
  --(3.176,1.779)--(3.535,1.779)--(3.895,1.779)--(4.254,1.779)--(4.613,1.779)--(4.972,1.779)%
  --(5.332,1.779)--(5.691,1.779)--(6.050,1.779)--(6.409,1.779)--(6.769,1.779)--(7.128,1.779)%
  --(7.487,1.779)--(7.846,1.779)--(8.206,1.779)--(8.565,1.779)--(8.924,1.779)--(9.283,1.779)%
  --(9.643,1.779)--(10.002,1.779)--(10.361,1.779)--(10.720,1.779)--(11.080,1.779)--(11.439,1.779)%
  --(11.798,1.779)--(12.157,1.779)--(12.517,1.779)--(12.876,1.779)--(13.235,1.779)--(13.594,1.779)%
  --(13.953,1.779)--(14.313,1.779)--(14.672,1.779)--(15.031,1.779)--(15.390,1.779)--(15.750,1.779)%
  --(16.109,1.779)--(16.468,1.779)--(16.827,1.779)--(17.187,1.779)--(17.546,1.779)--(17.905,1.779)%
  --(18.264,1.779)--(18.624,1.779)--(18.983,1.779)--(19.342,1.779)--(19.701,1.779)--(20.061,1.779)%
  --(20.420,1.779)--(20.779,1.779)--(21.138,1.779)--(21.498,1.779)--(21.857,1.779)--(22.216,1.779)%
  --(22.575,1.779)--(22.935,1.779)--(23.294,1.779)--(23.653,1.779)--(24.012,1.779)--(24.372,1.779)%
  --(24.731,1.779)--(25.090,1.779)--(25.449,1.779)--(25.808,1.779)--(26.168,1.779)--(26.527,1.779)%
  --(26.886,1.779)--(27.245,1.779)--(27.605,1.779)--(27.964,1.779)--(28.323,1.779)--(28.682,1.779)%
  --(29.042,1.779)--(29.401,1.779)--(29.760,1.779)--(30.119,1.779)--(30.479,1.779)--(30.838,1.779)%
  --(31.197,1.779)--(31.556,1.779)--(31.916,1.779)--(32.275,1.779)--(32.634,1.779)--(32.993,1.779)%
  --(33.353,1.779)--(33.712,1.779)--(34.071,1.779)--(34.430,1.779)--(34.790,1.779)--(35.149,1.779)%
  --(35.508,1.779)--(35.867,1.779)--(36.227,1.779)--(36.586,1.779)--(36.945,1.779);
\gpfill{rgb color={0.333,0.333,0.333}} (4.697,1.779)--(5.403,1.779)--(5.403,3.233)--(4.697,3.233)--cycle;
\gpcolor{color=gp lt color border}
\draw[gp path] (4.697,1.779)--(4.697,3.232)--(5.402,3.232)--(5.402,1.779)--cycle;
\gpfill{rgb color={0.333,0.333,0.333}} (11.471,1.779)--(12.178,1.779)--(12.178,3.707)--(11.471,3.707)--cycle;
\draw[gp path] (11.471,1.779)--(11.471,3.706)--(12.177,3.706)--(12.177,1.779)--cycle;
\gpfill{rgb color={0.333,0.333,0.333}} (14.858,1.779)--(15.565,1.779)--(15.565,4.409)--(14.858,4.409)--cycle;
\draw[gp path] (14.858,1.779)--(14.858,4.408)--(15.564,4.408)--(15.564,1.779)--cycle;
\gpfill{rgb color={0.333,0.333,0.333}} (26.713,1.779)--(27.420,1.779)--(27.420,8.267)--(26.713,8.267)--cycle;
\draw[gp path] (26.713,1.779)--(26.713,8.266)--(27.419,8.266)--(27.419,1.779)--cycle;
\gpfill{rgb color={0.333,0.333,0.333}} (35.181,1.779)--(35.888,1.779)--(35.888,7.310)--(35.181,7.310)--cycle;
\draw[gp path] (35.181,1.779)--(35.181,7.309)--(35.887,7.309)--(35.887,1.779)--cycle;
\node[gp node center] at (3.350,1.963) {\plotBarNormValue{-0.000000}};
\node[gp node center] at (5.043,3.416) {\plotBarNormValue{0.026414}};
\node[gp node center] at (6.737,1.963) {\plotBarNormValue{-0.000000}};
\node[gp node center] at (8.430,1.963) {\plotBarNormValue{-0.000000}};
\node[gp node center] at (10.124,1.963) {\plotBarNormValue{-0.000000}};
\node[gp node center] at (11.817,3.890) {\plotBarNormValue{0.035018}};
\node[gp node center] at (13.511,1.963) {\plotBarNormValue{-0.000000}};
\node[gp node center] at (15.205,4.592) {\plotBarNormValue{0.047793}};
\node[gp node center] at (16.898,1.963) {\plotBarNormValue{-0.000000}};
\node[gp node center] at (18.592,1.963) {\plotBarNormValue{-0.000000}};
\node[gp node center] at (20.285,1.963) {\plotBarNormValue{-0.000000}};
\node[gp node center] at (21.979,1.963) {\plotBarNormValue{-0.000000}};
\node[gp node center] at (23.672,1.963) {\plotBarNormValue{-0.000000}};
\node[gp node center] at (25.366,1.963) {\plotBarNormValue{-0.000000}};
\node[gp node center] at (27.060,8.450) {\plotBarNormValue{0.117904}};
\node[gp node center] at (28.753,1.963) {\plotBarNormValue{-0.000000}};
\node[gp node center] at (30.447,1.963) {\plotBarNormValue{-0.000000}};
\node[gp node center] at (32.140,1.963) {\plotBarNormValue{-0.000000}};
\node[gp node center] at (33.834,1.963) {\plotBarNormValue{-0.000000}};
\node[gp node center] at (35.527,7.493) {\plotBarNormValue{0.100507}};
\node[gp node center] at (18.592,1.994) {\plotSubOptSymbol};
\draw[gp path] (1.380,8.381)--(1.380,1.779)--(36.945,1.779)--(36.945,8.381)--cycle;
%% coordinates of the plot area
\gpdefrectangularnode{gp plot 1}{\pgfpoint{1.380cm}{1.779cm}}{\pgfpoint{36.945cm}{8.381cm}}
\end{tikzpicture}
%% gnuplot variables
%
    }%
  }

  \caption[Plot for evaluating SIMD instructions' impact on code quality]%
          {%
            Normalized solution costs for two pattern sets:
            %
            \begin{inlinelist}[itemjoin={, }, itemjoin*={, and}]
              \item one without SIMD instructions (baseline)
              \item one with such instruction (subject)
            \end{inlinelist}.
            %
            GMI:~\printGMI{%
              \SimdVsWithoutCyclesSpeedupCyclesRegularSpeedupGmean%
            },
            CI:~\printGMICI[round-precision=3]{%
              \SimdVsWithoutCyclesSpeedupCyclesRegularSpeedupCiMin%
            }{%
              \SimdVsWithoutCyclesSpeedupCyclesRegularSpeedupCiMax%
            }.
            %
            \Glspl{function} whose bars are marked with two dots are those
            for which the \gls{subject} does not find the optimal solution%
          }
  \labelFigure{simd-vs-without-cycles-plot}
\end{figure}
%
The costs range from
\printMinCycles{%
  \SimdVsWithoutCyclesSpeedupCyclesAvgMin,
  \SimdVsWithoutCyclesSpeedupBaselineCyclesAvgMin
} to
\printMaxCycles{%
  \SimdVsWithoutCyclesSpeedupCyclesAvgMax,
  \SimdVsWithoutCyclesSpeedupBaselineCyclesAvgMax
}, with a \gls{CV} of
\numMaxOf{
  \SimdVsWithoutCyclesSpeedupCyclesCvMax,
  \SimdVsWithoutCyclesSpeedupBaselineCyclesCvMax
}.
%
The solving times from
\printMinSolvingTime{%
  \SimdVsWithoutCyclesSpeedupPrePlusSolvingTimeAvgMin,
  \SimdVsWithoutCyclesSpeedupBaselinePrePlusSolvingTimeAvgMin
} to
\printMaxSolvingTime{%
  \SimdVsWithoutCyclesSpeedupPrePlusSolvingTimeAvgMax,
  \SimdVsWithoutCyclesSpeedupBaselinePrePlusSolvingTimeAvgMax
}, with a \gls{CV} of
\numMaxOf{
  \SimdVsWithoutCyclesSpeedupPrePlusSolvingTimeCvMax,
  \SimdVsWithoutCyclesSpeedupBaselinePrePlusSolvingTimeCvMax
}.
%
The \gls{GMI} is \printGMI{%
  \SimdVsWithoutCyclesSpeedupCyclesRegularSpeedupGmean%
} with \gls{CI}~\printGMICI[round-precision=3]{%
  \SimdVsWithoutCyclesSpeedupCyclesRegularSpeedupCiMin%
}{%
  \SimdVsWithoutCyclesSpeedupCyclesRegularSpeedupCiMax%
}.

We see that the \gls{pattern set}~\refPattern{with-simd} yields \glspl{solution}
with significantly less cost than those yielded by \gls{pattern
  set}~\refPattern{no-simd} (up to~\printZCNorm{%
  \SimdVsWithoutCyclesSpeedupCyclesZeroCenteredSpeedupMax%
} improvement).
%
The five cases with less cost (\cCode*{debug\_print\_str},
\cCode*{gl\_read\_alpha\_s}, \cCode*{gx\_curve\_cursor}, \cCode*{mp\_dmul}, and
\cCode*{zero}), \gls{universal instruction selection} is able to combine pairs
of additions or subtractions into \num{2}-way \gls{SIMD.i} \glspl{instruction}.
%
In addition, in one of these cases (\cCode*{gl\_read\_alpha\_s}) the additions
originally reside in different \glspl{block}, but due to \gls{global code
  motion} the approach is able to move the computations to the same \gls{block}
and implement these using a single \gls{instruction}.


\paragraph{Conclusions}

From the results for these experiments, we conclude that there is sufficient
data parallelism to be exploited through selection of \gls{SIMD.i}
\glspl{instruction} without having to resort to \gls{loop unrolling}.
%
In addition, this exploitation benefits from \gls{global code motion} as that
allows computations to be gathered from different \glspl{block} and implemented
using a single \gls{SIMD.i} \gls{instruction}.
