% Copyright (c) 2017, Gabriel Hjort Blindell <ghb@kth.se>
%
% This work is licensed under a Creative Commons 4.0 International License (see
% LICENSE file or visit <http://creativecommons.org/licenses/by/4.0/> for a copy
% of the license).

\chapter{Constraint Model}
\labelChapter{constraint-model}

This chapter introduces the \gls{constraint model} for \gls{universal
  instruction selection}.
%
We build the \glsshort{constraint model} by integrating one task a time.
%
To this end,
\refSectionRange{modeling-global-instruction-selection}{modeling-block-ordering}
describe the \glspl{variable} and \gls{constraint} for modeling \gls{global.is}
\gls{instruction selection}, \gls{global code motion}, \gls{data copying},
\gls{value reuse}, and \gls{block ordering}, respectively.
%
We then add the \gls{objective function}, which is described in
\refSection{cm-objective-function}.
%
With all crucial components in place, we discuss the limitations of this
\gls{constraint model} in \refSection{cm-limitations}.
%
Lastly, a summary is given in \refSection{model-summary}.


\section{Modeling Global Instruction Selection}
\labelSection{modeling-global-instruction-selection}

Modeling \gls{global.is} \gls{instruction selection} entails that all
\glspl{operation} must be \gls{cover}[ed] and all \glspl{datum} must be
\gls{define.d}[d].
%
This could, however, lead to situations resulting in cyclic data dependencies,
which must be forbidden.


\subsection{Covering Operations and Defining Data}

In \gls{global.is}[ \gls{instruction selection}], a set of \glspl{match} must be
selected such that every \gls{operation} in a given \gls{UF graph} is covered.
%
There are two variants of this problem:
%
\begin{enumerate*}[label=(\arabic*)]
  \item each \gls{operation} must appear in \emph{exactly} one selected
    \gls{match}; and
%
  \item each \gls{operation} must appear in \emph{at least} one selected
    \gls{match}, hence allowing matches to \gls{overlap}.
\end{enumerate*}
%
The former problem is more common as it is stricter, resulting in simpler models
with smaller \glspl{solution space}.
%
It also allows use of \glspl{constraint} that enable strong \gls{propagation},
which is essential for curbing solving time and increasing scalability.

Depending on the \gls{instruction set}, the latter problem permits
\glspl{solution} with potentially higher code quality.
%
For example, assume a \gls{UF graph} where a sum is used as address in two
memory operations and a \gls{target machine} where the address can be computed
as part of the memory instructions.
%
A \gls{solution} to the latter problem would therefore only need two
instructions, whereas the former problem would require three instructions -- one
to compute the sum and two to perform the memory operations -- since the
addition is not allowed to be covered by both memory instructions.
%
In certain conditions, however, an add instruction may still be required.
%
For example, assume a \gls{UF graph} where the sum is also used in a
subtraction.
%
For this \gls{UF graph}, unless the \gls{target machine} has an instruction that
performs both an addition and a subtraction, a \gls{solution} to either problem
requires an add instruction to compute the sum.
%
Due to the increased complexity of the relaxed version of the problem, we model
exact \gls{cover}[age] in this dissertation.

Similarly, every value and state must be produced by exactly one selected match.
%
If a \gls!{datum}~$d$ denotes either a \glsshort{state node} or \gls{value node}
in the \gls{UF graph}, then we say that a \gls{match}~$m$ \gls!{define.d}[s] $d$
if there exists an inbound \glsshort{state-flow edge} or \gls{data-flow edge} to
$d$ in the \gls{UP graph} from which $m$ was derived.
%
Similarly, $m$ \gls!{use.d}[s] $d$ if there exists an outbound
\glsshort{state-flow edge} or \gls{data-flow edge} to $d$ in the \gls{UP graph}
of $m$.


\paragraph{Variables}

Given a \gls{UF graph}~$\mUFGraph$ and a set~$\mMatchSet$ of \glspl{match} found
for $\mUFGraph$, the set of \glspl{variable} \mbox{$\mVar{sel}[m] \in \mSet{0,
    1}$} models whether \gls{match}~\mbox{$m \in \mMatchSet$} is selected.
%
Hence $m$ is selected if \mbox{$\mVar{sel}[m] = 1$}, abbreviated
$\mVar{sel}[m]$, and not selected if \mbox{$\mVar{sel}[m] = 0$}, abbreviated
\mbox{$\neg\mVar{sel}[m]$}.

The set of \glspl{variable} \mbox{$\mVar{omatch}[o] \in \mMatchSet[o]$} models
which selected \gls{match} covers \gls{operation}~\mbox{$o \in
  \mOpSet$\hspace{-1pt}}, where $\mOpSet$ denotes the set of \glspl{operation}
in $\mUFGraph$, and \mbox{$\mMatchSet[o] \subseteq \mMatchSet$} denotes the set
of \glspl{match} that can cover~$o$.
%
Similarly, the set of \glspl{variable} \mbox{$\mVar{dmatch}[d] \in
  \mMatchSet[o]$} models which selected \gls{match} \gls{define.d}[s]
\gls{datum}~\mbox{$d \in \mDataSet$\hspace{-1.5pt}}, where $\mDataSet$ denotes
the set of \glspl{datum} in $\mUFGraph$, and \mbox{$\mMatchSet[d] \subseteq
  \mMatchSet$} denotes the set of \glspl{match} that can \gls{define.d}~$d$.


\paragraph{Constraints}

If an \gls{operation}~$o$ is covered by a \gls{match}~$m$, then that entails
selection of $m$, and vice versa.
%
Hence the \gls{constraint} that every \gls{operation} must be covered is modeled
as
%
\begin{equation}
  \forall o \in \mOpSet \hspace{-1pt},
  \forall m \in \mMatchSet[o] :
  \mVar{omatch}[o] = m \mEq \mVar{sel}[m].
  \labelEquation{operation-coverage}
\end{equation}
%
This \gls{constraint} gives equally strong \gls{propagation} as
\refEquation{pattern-selection-using-gcc}, making it redundant to add the latter
as an \gls{implied.c} \gls{constraint} to the \glsshort{constraint model}.

Likewise, if a \gls{datum}~$d$ is \gls{define.d}[d] by a \gls{match}~$m$, then
that entails selection of $m$, and vice versa.
%
Hence the \gls{constraint} that every \gls{datum} must be \gls{define.d}[d]
is modeled as
%
\begin{equation}
  \forall d \in \mDataSet \hspace{-1.5pt},
  \forall m \in \mMatchSet[d] :
  \mVar{dmatch}[d] = m \mEq \mVar{sel}[m].
  \labelEquation{data-definitions}
\end{equation}
%
We assume the \gls{pattern set} has been extended with a special \gls!{null-def
  pattern}, with \gls{graph} structure \mbox{$\mEdge{b}{d}$} where $b$ is a
\gls{entry block} and $d$ is a \gls{datum}, that defines $d$ at zero cost.
%
This \gls{pattern} is needed for defining \glspl{datum} representing
\gls{function} arguments and constants since these are not produced by any
\gls{operation}.


\subsection{Forbidding Cyclic Data Dependencies}
\labelSection{forbidding-cyclic-data-dependencies}

In certain cases, selecting \glspl{match} of \glspl{instruction} producing
multiple results -- for example, many modern processors provide memory
\glspl{instruction} that automatically increment or decrement the address value
-- could lead to cyclic data dependencies~\cite{EbnerEtAl:2008}.
%
\begin{filecontents*}{cyclic-data-deps-example-ir.c}
$\ldots$
$\irAssign{\irVar{p}[2]}{\irAdd{\irVar{p}[1]}{\irVar{4}}}$
$\irStore{\irVar{q}[1]}{\irVar{p}[2]}$
$\irAssign{\irVar{q}[2]}{\irAdd{\irVar{q}[1]}{\irVar{4}}}$
$\irStore{\irVar{p}[1]}{\irVar{q}[2]}$
\end{filecontents*}%
%
\begin{figure}
  \subcaptionbox{IR\labelFigure{cyclic-data-deps-example-ir}}%
                {%
                  \lstinputlisting[language=c,mathescape]%
                                  {cyclic-data-deps-example-ir.c}%
                }%
  \hfill%
  \subcaptionbox{%
                  UF graph, covered by two matches derived from an
                  auto-increment store instruction.
                  %
                  For brevity, the state nodes are not included%
                  \labelFigure{cyclic-data-deps-example-uf-graph}%
                }%
                [62mm]%
                {%
                  \input{%
                    figures/constraint-model/cyclic-data-deps-example-uf-graph%
                  }%
                }%
  \hfill%
  \subcaptionbox{%
                  Dependency graph%
                  \labelFigure{cyclic-data-deps-example-dep-graph}%
                }%
                [32mm]%
                {%
                  \input{%
                    figures/constraint-model/cyclic-data-deps-example-dep-graph%
                  }%
                }

  \caption{Example illustrating cyclic data dependencies}%
  \labelFigure{cyclic-data-deps-example}%
\end{figure}
%
An example of such a situation is given in \refFigure{cyclic-data-deps-example}.
%
If both \glspl{match} are selected, then either value~\irVar*{p}[2] or
value~\irVar*{q}[2] will be \gls{use.d}[d] before it is available (depending on
the instruction order), thus resulting in incorrect code.
%
Consequently, such combinations must be identified and forbidden.

We detect such combinations -- which could involve more than two \glspl{match}
-- by first constructing a \gls!{dependency graph}, where each \gls{node}
represents a \gls{match} and each \gls{edge}~$\mEdge{n}{m}$ indicates
that \gls{match}~$m$ \gls{use.d}[s] \glspl{datum} produced by \gls{match}~$n$.
%
\Glspl{phi-match} are not taken into consideration as they always yield a
\gls{cycle} but is not a true data dependency.
%
A \gls{cycle} in this \gls{graph} corresponds a combination of \glspl{match}
which will lead to a cyclic data dependency if all \glspl{match} are selected.
%
Hence we find all \glspl{cycle} in the \gls{dependency graph} -- we applied
\citeauthor{Johnson:1975}'s algorithm for this task~\cite{Johnson:1975} -- and
add \glspl{constraint} forbidding selection of all \glspl{match} appearing in a
\gls{cycle}.


\paragraph{Constraints}

Given a set~\mbox{$\mForbiddenCombSet \subseteq \mPowerset{\mMatchSet}$} of
\glspl{cycle} found for the \gls{dependency graph} built from a \gls{UF graph}
and \gls{match set}, the \gls{constraint} forbidding cyclic data dependencies is
modeled as
%
\begin{equation}
  \forall f \in \mForbiddenCombSet :
  \sum_{\mathclap{m \in f}} \mVar{sel}[m] < \mCard{f}.
  \labelEquation{cyclic-data-deps}
\end{equation}


\section{Modeling Global Code Motion}
\labelSection{modeling-global-code-motion}

The \gls{global code motion} problem entails that \glspl{datum} must be placed
in \glspl{block} such that each definition of a \gls{datum}~$d$ precedes all
\gls{use.d}[s] of~$d$.
%
This condition can be expressed in terms of \gls{block} dominance.
%
Given a \gls{function}~$f$, a \gls{block}~$b$ in $f$ \gls!{dominate.b}[s]
another \gls{block}~$c$ in $f$ if $b$ appears on every control-flow path from
$f$'s \gls{entry block} to $c$ (see \refFigure{block-dominance-example} for an
example).
%
\begin{figure}
  \mbox{}%
  \hfill%
  \subcaptionbox{Control-flow graph\labelFigure{block-dominance-example-cfg}}%
                [34mm]%
                {%
                  \input{%
                    figures/constraint-model/block-dominance-example%
                  }%
                }%
  \hfill%
  \subcaptionbox{Dominance\labelFigure{block-dominance-example-doms}}%
                {%
                  \figureFont\figureFontSize%
                  \begin{tabular}{cc}
                    \toprule
                      \tabhead block
                    & \tabhead dominates\\
                    \midrule
                      \irBlock{entry}
                    & $\mSet{\irBlock{entry}, \irBlock{A}, \irBlock{B},
                        \irBlock{C}, \irBlock{D}, \irBlock{E}}$\\
                      \irBlock{A}
                    & $\mSet{\irBlock{A}, \irBlock{B}, \irBlock{C},
                        \irBlock{D}}$\\
                      \irBlock{B}
                    & $\mSet{\irBlock{B}}$\\
                      \irBlock{C}
                    & $\mSet{\irBlock{C}}$\\
                      \irBlock{D}
                    & $\mSet{\irBlock{D}}$\\
                      \irBlock{E}
                    & $\mSet{\irBlock{E}}$\\
                    \bottomrule
                  \end{tabular}%
                }%
  \hfill%
  \mbox{}

  \caption{Example of block dominance}
  \labelFigure{block-dominance-example}
\end{figure}%
%
By definition, a \gls{block} always \gls{dominate.b}[s] itself.
%
Hence a placement of \glspl{match} into \glspl{block} is a \gls{solution} to the
\gls{global code motion} problem if each \gls{datum}~$d$ is \gls{define.d}[d] by
some selected \gls{match} placed in a \gls{block}~$b$, and every
non-\gls{phi-match} \glsshort{use.d}[ing] $d$ is placed in a \gls{block}
\gls{dominate.b}[d] by~$b$.
%
The \glspl{phi-match} must be excluded since, due to the \glspl{definition
  edge}, at least one \gls{datum} used by such \glspl{match} must be
\gls{define.d}[d] in a \gls{block} that does not \gls{dominate.b} the
\gls{block} wherein the \gls{phi-match} must be placed.


\paragraph{Variables}

The set of \glspl{variable} \mbox{$\mVar{oplace}[o] \in \mBlockSet$} models in
which \gls{block} \gls{operation}~$o$ is placed, where $\mBlockSet$ denotes the
set of \glspl{block} in $\mUFGraph$.
%
Likewise, the set of \glspl{variable} \mbox{$\mVar{dplace}[d] \in \mBlockSet$}
models in which \gls{block} the definition of \gls{datum}~$d$ is placed.


\paragraph{Constraints}

Intuitively, all \glspl{operation} covered by a \gls{match}~$m$ must be placed
in the same \gls{block} wherein $m$ itself is placed.
%
Hence, if \mbox{$\mCovers(m) \subseteq \mOpSet$} denotes the set of
\glspl{operation} covered by \gls{match}~$m$, then this \gls{constraint} is
modeled as
%
\begin{equation}
  \forall m \in \mMatchSet,
  \forall o_1\hspace{-1pt}, o_2 \in \mCovers(m) :
  \mVar{sel}[m] \mImp \mVar{oplace}[o_1] = \mVar{oplace}[o_2].
  \labelEquation{operation-placement}
\end{equation}
%
This also enables the placement of $m$ to be deduced from any of the
corresponding $\mVar{oplace}$~\glspl{variable}.

We prevent control-flow \glspl{operation} from being moved to another
\gls{block}, which in all likelihood would break \gls{program} semantics, by
forcing selected \glspl{match} with control flow to be placed in the \gls{block}
matched by the \gls{UP graph}'s \gls{entry block}.
%
Hence, if \mbox{$\mEntry(m) \subseteq \mBlockSet$} returns either the empty set
or a set containing only the \gls{entry block} of match~$m$ (when the \gls{UP
  graph} of $m$ has such a node), then this \gls{constraint} is modeled as
%
\begin{equation}
  \forall m \in \mMatchSet,
  \forall o \in \mCovers(m),
  \forall b \in \mEntry(m) :
  \mVar{sel}[m] \mImp \mVar{oplace}[o] = b \hspace{-1pt}.
  \labelEquation{preventing-control-flow-op-moves}
\end{equation}

As stated previously, each \gls{datum}~$d$ must be \gls{define.d}[d] in some
\gls{block}~\mbox{$b \in \mBlockSet$} such that $b$ \gls{dominate.b}[s] every
\gls{block} wherein $d$ is \gls{use.d}[d], excluding \gls{use.d}[s] made by the
\glspl{phi-match}.
%
To this end, let \mbox{$\mDefines(m) \subseteq \mOperandSet$} and
\mbox{$\mUses(m) \subseteq \mOperandSet$} denote the set of \glspl{datum}
\gls{define.d}[d] respectively \gls{use.d}[d] by \gls{match}~$m$.
%
Let also \mbox{$\mDom(b) \subseteq \mBlockSet$} denote the set of \glspl{block}
\gls{dominate.b}[d] by \gls{block}~$b$.
%
With these definitions together with the fact that the placement of a
\gls{match} is inferred via its $\mVar{oplace}$~\glspl{variable}, the
\gls{constraint} can naively be modeled as
%
\begin{equation}
  \forall m \in \mPhiMatchCompSet,
  \forall d \in \mUses(m),
  \forall o \in \mCovers(m) :
  \mVar{dplace}[d] \in \mDom(\mVar{oplace}[o]),
  \labelEquation{naive-dom}
\end{equation}
%
where $\mPhiMatchCompSet$ denotes the set~$\mMatchSet$ without the
\glspl{phi-match}.
%
This implementation, however, has a number of flaws that will be explained and
addressed in \refChapter{solving-techniques}.
%
We therefore only use \refEquation{naive-dom} for sake of describing the
\glsshort{constraint model}, keeping in mind that a refined version is applied
in practice.

Next, we need to constrain the $\mVar{dplace}$~\glspl{variable} to depend on
where a selected \gls{match} is placed.
%
Intuitively, every \gls{datum} \gls{define.d}[d] by a \gls{match}~$m$ should be
placed in the same \gls{block} as $m$ together with all \glspl{operation}
covered by~$m$.
%
This alone, however, could result in an over-constrained \glsshort{constraint
  model} that prevents selection of certain \glspl{match}.
%
For example, assume a \gls{match}~$m$ of the \gls{UP graph} shown in
\refFigure{up-graph-examples-add-graph} on
\refPageOfFigure{up-graph-examples-add-graph}, where the \glspl{block node}
\irBlock*{entry}, \irBlock*{clamp}, and \irBlock*{end} are matched to
\glspl{block} in $\mUFGraph$ labeled \irBlock*{A}, \irBlock*{B}, and
\irBlock*{C}, respectively.
%
Because of \refEquation{preventing-control-flow-op-moves}, $m$ must be placed in
the \irBlock*{A}~\gls{block}.
%
But because of \refEquation{def-edges}, one of its \glspl{value node} must be
placed in the \irBlock*{B}~\gls{block}.
%
Consequently, for such conditions the \gls{constraint} is relaxed as follows.
%
We say that a \gls{match} \gls!{span.b}[s] the \glspl{block} matched by the
\gls{UP graph}'s \glspl{block node} (hence $m$ \gls{span.b}[s] \glspl{block}
\irBlock*{A}, \irBlock*{B}, and~\irBlock*{C}).
%
We also say that a \gls{match} \gls!{consume.b}[s] any matched \glspl{block}
where the corresponding \gls{block node} has both \gls{inbound.e} and
\gls{outbound.e} control-flow \glspl{edge} in the \gls{UP graph} (hence $m$
\gls{consume.b}[s] \gls{block}~\irBlock*{B}).
%
Consequently, every \gls{datum}~$d$ \gls{define.d}[d] by a \gls{match}~$m$ must
be placed in the same \gls{block} as $m$ only if $m$ \gls{span.b}[s] no
\glspl{block}, otherwise $d$ may be \gls{define.d}[d] in any of the
\glspl{block} \gls{span.b}[ned] by~$m$.
%
If \mbox{$\mSpans(m) \subseteq B$} denotes the set of \glspl{block}
\gls{span.b}[ned] by \gls{match}~$m$, then this constraint is modeled as
%
\begin{equation}
  \begin{array}{c}
    \forall m \in \mMatchSet,
    \forall d \in \mDefines(m),
    \forall o \in \mCovers(m) : \\
    \mVar{sel}[m]
    \mImp
    \mVar{dplace}[d] \in \mSet{\mVar{oplace}[o]} \cup \mSpans(m).
  \end{array}
  \labelEquation{spanning}
\end{equation}

\Glsshort{consume.b}[ing] a \gls{block} entails that the corresponding
\gls{instruction} assumes full control of the control flow to and from that
\gls{block}, which in turn means no \glspl{operation} covered by other
\glspl{match} may be placed in that \gls{block}.
%
Hence, if \mbox{$\mConsumes(m) \subseteq B$} denotes the set of \glspl{block}
\gls{consume.b}[d] by \gls{match}~$m$, then this constraint is modeled as
%
\begin{equation}
  \begin{array}{c}
    \forall m \in \mMatchSet,
    \forall o \in \mOpSet \setminus \mCovers(m),
    \forall b \in \mConsumes(m) : \\
    \mVar{sel}[m]
    \mImp
    \mVar{oplace}[o] \neq b \hspace{-1pt}.
  \end{array}
  \labelEquation{consumption}
\end{equation}

Lastly, the restrictions imposed by the \glspl{definition edge} are modeled as
%
\begin{equation}
  \forall \mEdge{d}{b} \in \mFunctionDefEdgeSet :
  \mVar{dplace}[d] = b \hspace{-1.2pt},
  \labelEquation{def-edges}
\end{equation}
%
where $\mFunctionDefEdgeSet$ denotes the set of \glspl{definition edge} in
$\mUFGraph$.
%
It is assumed that the \glspl{edge} in $\mFunctionDefEdgeSet$ have been
reoriented such that all \glspl{source} are either \glsshort{state node} or
\glspl{value node} and all \glspl{target} are \glspl{block node}.


\section{Modeling Data Copying}
\labelSection{modeling-data-copying}

The cost of \gls{data copying} is taken into account by keeping track of the
storage requirements for the data used and produced by the selected
\glspl{match}.
%
The idea is as follows.
%
For each value~$v$ in the \gls{UF graph}, let a \gls{variable}~$\mVar{x}$ decide
in which \gls{location} provided by the \gls{target machine} $v$ is stored.
%
In this context a \gls!{location} is an abstract representation, typically
representing a \gls{register} but it could also indicate that the value is for
example stored in memory.
%
A \gls{match}~$m$ that either \gls{use.d}[s] or \gls{define.d}[s] $v$ and
requires $v$ to be in one of a set~$L$ of \glspl{location} can then enforce, if
selected, that \mbox{$\mVar{x} \in L$}.


\paragraph{Variables}

The set of \glspl{variable} \mbox{$\mVar{loc}[d] \in \mLocationSet \cup
  \mSet{\mIntLocation}$} models in which \gls{location} \gls{datum}~$d$ is
available, where $\mLocationSet$ denotes the \gls!{location set} provided by the
\gls{target machine} and $\mIntLocation$ denotes a special \gls{location} for
values that cannot be reused across \glspl{instruction}.
%
The special \gls{location} is used for \glspl!{intermediate value}, which are
values produced within an \gls{instruction} and can only be accessed by this
very \gls{instruction}.
%
For example, the address computed by a memory load \gls{instruction} with a
sophisticated addressing mode is produced within the pipeline and thus cannot be
reused by other \glspl{instruction}.
%
A value which is not an \gls{intermediate value} is called an \gls!{exterior
  value}, meaning it can be accessed by other \glspl{instruction}.


\paragraph{Constraints}

Every \gls{datum} must be made available in a \gls{location} that is compatible
for all selected \glspl{match}.
%
If \mbox{$\mStores(m, d) \subseteq \mLocationSet \cup \mSet{\mIntLocation}$}
denotes the set of compatible \glspl{location} (including the special
\gls{location} for intermediate values) for a \gls{datum}~$d$ which is either
\gls{define.d}[d] or \gls{use.d}[d] by a \gls{match}~$m$, then this
\gls{constraint} is modeled as
%
\begin{equation}
  \forall m \in \mMatchSet,
  \forall d \in \mDefines(m) \cup \mUses(m) : \\
  \mVar{sel}[m]
  \mImp
  \mVar{loc}[d] \in \mStores(m, d).
  \labelEquation{compatible-locations}
\end{equation}

As expected, \glspl{phi-match} require the \glspl{location} of all its
\glspl{datum} to be the same.
%
This \gls{constraint} is modeled as
%
\begin{equation}
  \forall m \in \mPhiMatchSet \hspace{-.8pt},
  \forall d_1 \hspace{-.8pt}, d_2 \in \mDefines(m) \cup \mUses(m) :
  \mVar{sel}[m]
  \mImp
  \mVar{loc}[d_1] = \mVar{loc}[d_2].
  \labelEquation{phi-match-locations}
\end{equation}


\subsection{Copy Extension}

Depending on the \gls{target machine}, \refEquation{compatible-locations} can
result in an over-constrained \glsshort{constraint model}.
%
For example, in many \glspl{target machine} the \gls{SIMD.instr}
\glspl{instruction} use a different set of \glspl{register} than the other,
general \glspl{instruction}.
%
In such situations, the \glspl{match} derived from the \gls{SIMD.instr}
\glspl{instruction} and the \glspl{match} derived from the general
\glspl{instruction} will have non-overlapping \glspl{location} on the same
\glspl{datum} (that is, \mbox{$\mStores(m_1, d) \cap \mStores(m_2, d) =
  \mEmptySet$}), preventing selection of such \glspl{match}.

Since non-overlapping \glspl{location} entails the need for copy
\glspl{instruction}, we extend the \gls{UF graph} with \glspl!{copy node}
through a process called \gls!{copy extension}.
%
For each \gls{data-flow edge}~$\mEdge{v}{o}$, where $v$ is a \gls{value node}
and $o$ is an \gls{operation}, we remove this \gls{edge} and insert a new
\gls{copy node}~$c$, \gls{value node}~$v'$, and \glspl{data-flow edge} such that
\mbox{$\mEdge{v}{\mEdge{c}{\mEdge{v'}{o}}}$}.
%
\Glspl{match} covering exactly one \gls{copy node} are called \glspl!{copy
  match}.
%
If $o$ is a \gls{phi-node} then the corresponding \gls{definition edge}
connected to $v$ -- this is the \gls{edge} with the same \gls{outbound.en}
\gls{edge number} as the \gls{data-flow edge} -- is also moved to $v'$.
%
This is to ensure that the placement restrictions are applied only on the
\glspl{datum} actually used by the \glspl{phi-function} (that is, the copied
value and not the original value).
%
We also extend the \gls{pattern set} with a special \gls!{null-copy pattern},
with \gls{graph} structure \mbox{$\mEdge{v}{\mEdge{c}{v'}}$}, that covers $c$ at
zero cost provided that \mbox{$\mVar{loc}[v] = \mVar{loc}[v']$}.
%
Obviously, \glspl{match} derived from the \gls{null-copy pattern} -- we call
these \glspl!{null-copy match} -- emit nothing if selected.
%
If the \gls{null-copy match} cannot be selected for covering a particular
\gls{copy node}, then this means an appropriate copy \gls{instruction} must be
emitted whose cost will be accounted for.

In order to retain matching of \glspl{pattern}, we also need to perform
\gls{copy extension} on every \gls{UP graph} in the \gls{pattern set}.
%
\begin{figure}
  \setlength{\opNodeDist}{12pt}%

  \mbox{}%
  \hfill%
  \subcaptionbox{%
                  Original UP graph%
                  \labelFigure{copy-extending-pattern-example-before}%
                }%
                [50mm]%
                {%
                  \input{%
                    figures/constraint-model/%
                    copy-extending-pattern-example-before%
                  }%
                }%
  \hfill%
  \subcaptionbox{%
                  After copy extension%
                  \labelFigure{copy-extending-pattern-example-after}%
                }%
                [50mm]%
                {%
                  \input{%
                    figures/constraint-model/%
                    copy-extending-pattern-example-after%
                  }%
                }%
  \hfill%
  \mbox{}

  \caption{Example of copy-extending a pattern}
  \labelFigure{copy-extending-pattern-example}
\end{figure}
%
See for example \refFigure{copy-extending-pattern-example}.
%
The \gls{UP graph} captures the behavior of an \gls{instruction} that adds two
values~\irVar*{r} and~\irVar*{s} and then shifts the result by one bit to the
right (\refFigure{copy-extending-pattern-example-before}).
%
Since we want to preserve the ability of selecting copy \glspl{instruction} for
moving \glspl{datum} between \glspl{instruction}, we only copy-extend the values
in a \gls{UP graph} which are both \gls{define.d}[d] and \gls{use.d}[d] by the
\gls{pattern}.
%
We also copy-extend any constant values since these obviously do not require a
separate copy \gls{instruction} to be used by the \gls{pattern}.
%
The resulting \gls{UP graph} will now yield the same \glspl{match} as before
\gls{copy extension} (\refFigure{copy-extending-pattern-example-after}).


\subsection{Handling Calling Conventions}

The method for handling \gls{data copying} can also be used for handling
\glspl{calling convention} of the specific \gls{target machine}.\!%
%
\footnote{%
  A \gls!{calling convention} is an implementation scheme that defines how
  arguments and return values are passed and received when making a
  \gls{function} call.
  %
  The \gls{function} from which the call is made is typically called the
  \gls!{caller}, and the \gls{function} to which the call is made is called the
  \gls!{callee}.
  %
  A \gls{calling convention} is specific for the \gls{target machine} and may
  therefore differ from one \glsshort{target machine} to another.
}
%
\Glspl{constraint} that \gls{callee} arguments must reside in a specific
\gls{location} are modeled as
%
\begin{equation}
  \forall d \in \mArgSet :
  \mVar{loc}[d] \in \mArgLoc(d).
  \labelEquation{function-args}
\end{equation}
%
where \mbox{$\mArgSet \subseteq \mDataSet$} denotes the set of \gls{function}
arguments in $\mUFGraph$ and \mbox{$\mArgLoc(d) \subseteq \mLocationSet$}
denotes the set of \glspl{location} in which argument~$d$ resides.
%
Arguments residing on the stack can be signified by introducing another special
\gls{location}, thus requiring a memory load \gls{instruction} in order to be
used by other \glspl{instruction}.

\Glspl{location} for caller arguments can be enforced either through
\refEquation{function-args} or through \refEquation{compatible-locations}.
%
If the \gls{calling convention} depends on the \gls{instruction} selected then
the former is needed, otherwise the latter is more suitable as the restrictions
can be enforced before a \gls{match} is selected.\!%
%
\footnote{%
  If exactly one \gls{match}~$m$ can cover a given \gls{function} call
  \gls{node}, then both \glspl{constraint} provide the same amount of
  \gls{propagation} as \mbox{$\mVar{sel}[m] = 1$} will always hold for the
  implication in \refEquation{compatible-locations}.
}


\section{Modeling Value Reuse}
\labelSection{modeling-value-reuse}

Code quality can be increased if \glspl{instruction} are allowed to reuse copies
of values, which is a crucial feature to be expected in the code emitted by any
modern \gls{instruction selector}.
%
\begin{figure}
  \setlength{\opNodeDist}{12pt}%

  \mbox{}%
  \hfill%
  \subcaptionbox{%
                  A UF graph with two matches%
                  \labelFigure{value-reuse-example-uf-graph}%
                }%
                [50mm]%
                {% Copyright (c) 2017-2018, Gabriel Hjort Blindell <ghb@kth.se>
%
% This work is licensed under a Creative Commons Attribution-NoDerivatives 4.0
% International License (see LICENSE file or visit
% <http://creativecommons.org/licenses/by-nc-nd/4.0/> for details).
%
\begingroup%
\figureFont\figureFontSize%
\pgfdeclarelayer{background}%
\pgfsetlayers{background,main}%
\begin{tikzpicture}[
    outer match node/.style={%
      match node,
      draw=none,
      inner sep=0,
    },
  ]

  \node [value node] (v) {\strut\nVar{v}};
  \node [computation node, position=-135 degrees from v] (op1) {};
  \node [computation node, position=- 45 degrees from v] (op2) {};

  \begin{scope}[data flow]
    \draw (v) -- (op1);
    \draw (v) -- (op2);
  \end{scope}

  \begin{pgfonlayer}{background}
    % m1
    \node [outer match node, inner sep=3pt, fit=(v)] (m1a) {};
    \node [outer match node, fit=(op1)] (m1b) {};
    \def\pathMI{
      [bend left=45]
      (m1a.north west)
      to
      (m1a.north east)
      to
      (m1a.south east)
      --
      (m1b.south east)
      to
      (m1b.south west)
      to
      (m1b.north west)
      -- coordinate (m1)
      cycle
    }
    \path [fill=shade1]
          \pathMI;

    % m2
    \node [outer match node, inner sep=1.5pt, fit=(v)] (m2a) {};
    \node [outer match node, fit=(op2)] (m2b) {};
    \def\pathMII{
      [bend left=45]
      (m2a.south west)
      to
      (m2a.north west)
      to
      (m2a.north east)
      -- coordinate (m2)
      (m2b.north east)
      to
      (m2b.south east)
      to
      (m2b.south west)
      --
      cycle
    }
    \path [fill=shade1]
          \pathMII;

    \begin{scope}
      \path [clip] \pathMI;
      \path [fill=shade2]
            \pathMII;
    \end{scope}

    \draw [match line]
          \pathMI;
    \draw [match line]
          \pathMII;
  \end{pgfonlayer}

  % Match labels
  \begin{scope}[overlay]
    \node [match label, position=135 degrees from m1] (m1l) {$\strut m_1$};
    \node [match label, position= 45 degrees from m2] (m2l) {$\strut m_2$};
    \foreach \i in {1, 2} {
      \draw [match attachment line] (m\i) -- (m\i l);
    }
  \end{scope}
\end{tikzpicture}%
\endgroup%
}%
  \hfill%
  \subcaptionbox{%
                  UF graph after copy extension%
                  \labelFigure{value-reuse-example-after-ce}%
                }%
                [50mm]%
                {% Copyright (c) 2018, Gabriel Hjort Blindell <ghb@kth.se>
%
% This work is licensed under a Creative Commons 4.0 International License (see
% LICENSE file or visit <http://creativecommons.org/licenses/by/4.0/> for a copy
% of the license).
%
\begingroup%
\figureFont\figureFontSize%
\pgfdeclarelayer{background}%
\pgfsetlayers{background,main}%
\begin{tikzpicture}[
    outer match node/.style={%
      match node,
      draw=none,
      inner sep=0,
    },
  ]

  \node [value node] (v) {\strut\nVar{v}};
  \node [computation node, position=-135 degrees from v] (cp1) {\nCopy};
  \node [computation node, position=- 45 degrees from v] (cp2) {\nCopy};
  \node [value node, below=of cp1] (v1) {\strut\nVar{v}[\hspace{-1pt}1]};
  \node [value node, below=of cp2] (v2) {\strut\nVar{v}[\hspace{-1pt}2]};
  \node [computation node, below=of v1] (op1) {};
  \node [computation node, below=of v2] (op2) {};

  \begin{scope}[data flow]
    \draw (v) -- (cp1);
    \draw (v) -- (cp2);
    \draw (cp1) -- (v1);
    \draw (cp2) -- (v2);
    \draw (v1) -- (op1);
    \draw (v2) -- (op2);
  \end{scope}

  \begin{pgfonlayer}{background}
    % m1
    \node [outer match node, inner sep=2pt, fit=(v1)] (m1a) {};
    \node [outer match node, fit=(op1)] (m1b) {};
    \def\pathMI{
      [bend left=45]
      (m1a.west)
      to
      (m1a.north)
      to
      (m1a.east)
      --
      (m1b.east)
      to
      (m1b.south)
      to
      (m1b.west)
      -- coordinate (m1)
      cycle
    }
    \draw [match line, fill=shade1]
          \pathMI;

    % m2
    \node [outer match node, inner sep=2pt, fit=(v2)] (m2a) {};
    \node [outer match node, fit=(op2)] (m2b) {};
    \def\pathMII{
      [bend left=45]
      (m2a.west)
      to
      (m2a.north)
      to
      (m2a.east)
      -- coordinate (m2)
      (m2b.east)
      to
      (m2b.south)
      to
      (m2b.west)
      --
      cycle
    }
    \draw [match line, fill=shade1]
          \pathMII;
  \end{pgfonlayer}

  % Match labels
  \begin{scope}[overlay]
    \node [match label, left=of m1] (m1l) {$\strut m_1$};
    \node [match label, right=of m2] (m2l) {$\strut m_2$};
    \foreach \i in {1, 2} {
      \draw [match attachment line] (m\i) -- (m\i l);
    }
  \end{scope}
\end{tikzpicture}%
\endgroup%
}%
  \hfill%
  \mbox{}

  \vspace{\betweensubfigures}

  \mbox{}%
  \hfill%
  \subcaptionbox{%
                  Redundant copying of values%
                  \labelFigure{value-reuse-example-redundant-copies}%
                }%
                [50mm]%
                {%
                  \input{%
                    figures/constraint-model/%
                    value-reuse-example-redundant-copies%
                  }%
                }%
  \hfill%
  \subcaptionbox{%
                  Reuse of copied value%
                  \labelFigure{value-reuse-example-one-copy}%
                }%
                [50mm]%
                {%
                  \input{%
                    figures/constraint-model/value-reuse-example-one-copy%
                  }%
                }%
  \hfill%
  \mbox{}

  \caption{Example of value reuse}
  \labelFigure{value-reuse-example}
\end{figure}
%
See for example \refFigure{value-reuse-example}.
%
Originally, the \gls{UF graph} has a value~\irVar*{v} which is used by two
\glspl{operation} coverable by \glspl{match}~$m_1$ and~$m_2$
(\refFigure{value-reuse-example-uf-graph}).
%
After \gls{copy extension}, $m_1$ and $m_2$ use their own private copy --
\irVar*{v}[1] and \irVar*{v}[2], respectively -- of \irVar*{v}
(\refFigure{value-reuse-example-after-ce}).
%
Assume that both $m_1$ and $m_2$ require its copy of \irVar*{v} to reside in a
\gls{location} different from \irVar*{v} -- \irVar*{v} may reside on the stack,
for instance -- which means that selection of $m_1$ or $m_2$ also entails
emission of a copy \gls{instruction}.
%
With the \glsshort{constraint model} introduced thus far, two copy
\glspl{instruction} will be emitted if both $m_1$ and $m_2$ are selected
(\refFigure{value-reuse-example-redundant-copies}).
%
However, if \irVar*{v}[1] and \irVar*{v}[2] could reside in the same
\gls{location} then either of the values could be reused by either \gls{match},
thus necessitating only one copy \gls{instruction}
(\refFigure{value-reuse-example-one-copy}).
%
We call this notion \gls!{value reuse}.

In this dissertation, we discuss two methods for reusing values: \gls{match
  duplication} and \glspl{alternative value}.
%
We first introduce each in turn and then present experiments showing that one is
superior to the other.


\subsection{Match Duplication}

The idea behind \gls!{match duplication} is to duplicating appropriate
\glspl{match} in the \gls{match set} where \gls{value reuse} is possible.
%
We first say that two values~$v_1$ and~$v_2$ are \gls!{copy-related.d} if and
only if they are copies of the same value and $v_1$ and $v_2$ have the same data
type (the second clause is necessary as copies of constants may be of different
data types and are therefore not interchangeable).
%
Then, for each \gls{match}~$m$, we create a new \gls{match} for each permutation
of \glspl{datum} that are \gls{copy-related.d} to the \glspl{input datum} in
$m$, where an \gls!{input datum} is a \gls{datum} \gls{use.d}[d] but not
\gls{define.d}[d] by~$m$.
%
\begin{figure}
  \setlength{\opNodeDist}{12pt}%

  \mbox{}%
  \hfill%
  \subcaptionbox{%
                  Original match set%
                  \labelFigure{match-duplication-example-before}%
                }%
                [50mm]%
                {%
                  \input{%
                    figures/constraint-model/match-duplication-example-before%
                  }%
                }%
  \hfill%
  \subcaptionbox{%
                  After match duplication%
                  \labelFigure{match-duplication-example-after}%
                }%
                [50mm]%
                {%
                  \input{%
                    figures/constraint-model/match-duplication-example-after%
                  }%
                }%
  \hfill%
  \mbox{}

  \caption{Example of match duplication}
  \labelFigure{match-duplication-example}
\end{figure}
%
See for example \refFigure{match-duplication-example}.
%
The \gls{match set} contains a \gls{match}~$m$ that uses value~\irVar*{v}[1],
which is \gls{copy-related.d} with value~\irVar*{v}[2]
(\refFigure{match-duplication-example-before}).
%
Because \irVar*{v}[1] is an \gls{input datum} in~$m$, we duplicate $m$ to
instead use \irVar*{v}[2], resulting in match~$m'$
(\refFigure{match-duplication-example-after}).

The main advantage of \gls{match duplication} is that no changes need to be done
for the \gls{constraint model}; the decision of which value to use (and reuse)
depends entirely on the selection of \glspl{match}.
%
However, this comes at a cost of inflating the \gls{match set}, which in turn
inflates the \gls{search space}.
%
If a \gls{match} has $k$~\glspl{input datum}, each with $n$~\gls{copy-related.d}
values, then $\mBigO(n^k)$ new \glspl{match} will be created.
%
Although the decision of \gls{value reuse} is intuitively orthogonal to the
decisions of selecting a \gls{match} and placing it into a \glspl{block}, these
decisions must be remade for each new \gls{match}, thereby enlarging the
\gls{search space} with many symmetric \glspl{solution}.


\subsection{Alternative Values}

Instead of expanding the \gls{match set} (like in \gls{match duplication}), we
can postpone the decision of which \gls{input datum} to use for a particular
\gls{match} and integrate it as part of the \gls{constraint model}.
%
The idea is as follows.
%
For each \gls{match}~$m$, let every \glspl{input datum}~$d$ in $m$ be mapped to
any \gls{datum} that is \gls{copy-related.d} to~$d$.
%
In other words, unlike before when a \gls{match} was \mbox{1-to-1} mapping
between \glspl{node} in the \gls{UP graph} and \glspl{node} in the \gls{UF
  graph}, we now allow certain mappings to be a \mbox{1-to-$n$} mapping.
%
\begin{figure}
  \setlength{\opNodeDist}{12pt}%

  \mbox{}%
  \hfill%
  \subcaptionbox{%
                  Original match set%
                  \labelFigure{alt-values-example-before}%
                }%
                [50mm]%
                {%
                  \input{%
                    figures/constraint-model/alt-values-example-without%
                  }%
                }%
  \hfill%
  \subcaptionbox{%
                  With alternative values%
                  \labelFigure{alt-values-example-after}%
                }%
                [50mm]%
                {%
                  \input{%
                    figures/constraint-model/alt-values-example-with%
                  }%
                }%
  \hfill%
  \mbox{}

  \caption{Example of alternative values}
  \labelFigure{alt-values-example}
\end{figure}
%
See for example \refFigure{alt-values-example}.
%
Again, the \gls{match set} contains a \gls{match}~$m$ that uses
value~\irVar*{v}[1], which is \gls{copy-related.d} with value~\irVar*{v}[2]
(\refFigure{alt-values-example-before}).
%
Because \irVar*{v}[1] is an \gls{input datum} in~$m$, we extend the mapping from
$n$ to \irVar*{v}[1] to include \irVar*{v}[2], where $n$ is the corresponding
\gls{value node} in the \gls{UP graph} of~$m$
(\refFigure{alt-values-example-after}).
%
In this context, \irVar*{v}[1] and \irVar*{v}[2] are said to be
\glspl!{alternative value} to~$m$.
%
For convenience, we assume that the set of \glspl{alternative value} for each
non-\gls{input datum}~$d$ in a \gls{match} contains only the \glsshort{state
  node} or \gls{value node} to which $d$ is already mapped.

Special care must be taken to \glspl{match} derived from the \gls{phi-pattern}.
%
Assume, for example, that the \gls{match}~$m$ using value~\irVar*{v}[1] in
\refFigure{alt-values-example-before} is a \gls{phi-match}.
%
This means that \irVar*{v}[1] will participate in a \gls{definition edge},
forcing \irVar*{v}[1] to be \gls{define.d}[d] in some \gls{block}.
%
If $m$ is extended as in \refFigure{alt-values-example-after}, then $m$ is
allowed to make use of value~\irVar*{v}[2], which does not participate in the
\gls{definition edge} and hence could break \gls{program} semantics.
%
There are two approaches to fixing this problem:
%
\begin{figure}
  \centering%
  % Copyright (c) 2018, Gabriel Hjort Blindell <ghb@kth.se>
%
% This work is licensed under a Creative Commons 4.0 International License (see
% LICENSE file or visit <http://creativecommons.org/licenses/by/4.0/> for a copy
% of the license).
%
\begingroup%
\figureFont\figureFontSize%
\begin{tikzpicture}
  \node [computation node] (phi) {\nPhi};
  \node [value node, position=135 degrees from phi] (i1) {\strut\nVar{i}[$1$]};
  \node [value node, position= 45 degrees from phi] (ik) {\strut\nVar{i}[$k$]};
  \node [value node, below=of phi] (d) {};
  \node [block node, above=of i1] (b1) {\strut$\nBlock{b}_{1}$};
  \node [block node, above=of ik] (bk) {\strut$\nBlock{b}_{k}$};

  \begin{scope}[data flow]
    \draw (i1) -- (phi);
    \draw (ik) -- (phi);
    \draw (phi) -- (d);
  \end{scope}
  \begin{scope}[definition edge]
    \draw (b1) -- (i1);
    \draw (bk) -- (ik);
  \end{scope}

  \node [nothing] at ($(i1) !.5! (ik)$) {$\ldots$};
  \node [nothing] at ($(b1) !.5! (bk)$) {$\ldots$};
\end{tikzpicture}%
\endgroup%


  \caption{Extended $\mPhi$-pattern}
  \labelFigure{ext-phi-pattern}
\end{figure}
%
\begin{inlinelist}[label=(\roman*), itemjoin={; }, itemjoin*={; or\ }]
  \item either all \glspl{phi-match} are excluded from being extended with
    \glspl{alternative value}
  \item the \gls{phi-pattern} is extended with \glspl{block node} and
    \glspl{definition edge}, as shown in \refFigure{ext-phi-pattern}, which
    allows the value placement restrictions to be captured as part of the
    \gls{match}
\end{inlinelist}.
%
The former is simpler but interferes with an \gls{implied.c} \gls{constraint}
(\refEquation{dom-cons-kill-match-selection}) to be introduced in
\refChapter{solving-techniques}, which may remove potentially optimal
\glspl{solution}.
%
Consequently, we apply the latter.
%
The extended \gls{phi-pattern} assumes that no value is used more than once by
the same \gls{phi-node}, and that no pair of values used by the same
\gls{phi-node} have \glspl{definition edge} to the same \gls{block}.
%
These invariants can be achieved by transforming the \gls{function} before
\gls{pattern matching}.

After having extended the \gls{match set} with \glspl{alternative value}, the
next step is to extend the \gls{constraint model} with an another level of
indirection wherever a \gls{constraint} refers to a \gls{datum}.


\paragraph{Variables}

Assume first that each \glsshort{define.d}[ition] or \glsshort{use.d}[age] of
\glspl{datum} within each \gls{match} incurs a unique \gls!{operand}.
%
Consequently, instead of \glsshort{define.d}[ing] and \glsshort{use.d}[ing]
\glspl{datum}, we now assume that all \glspl{match} \gls{define.d} and
\gls{use.d} \glspl{operand}.
%
Hence the set of \glspl{variable} \mbox{$\mVar{alt}[p] \in \mDataSet[p]$} models
to which \gls{datum} \gls{operand}~\mbox{$p \in \mOperandSet$} is mapped, where
\mbox{$\mDataSet[p] \subseteq \mDataSet$} denotes the set of \glspl{alternative
  value} for~$p$, and $\mOperandSet$ denotes the set of \glspl{operand} incurred
by~$\mMatchSet$.


\paragraph{Constraints}

As stated previously, the aim is to add another level of indirection whenever a
\gls{constraint} refers to a \gls{datum}.
%
To this end, let \mbox{$\mDefines(m) \subseteq \mOperandSet$} and
\mbox{$\mUses(m) \subseteq \mOperandSet$} now denote the set of \glspl{operand}
(instead of \glspl{datum}) \gls{define.d}[d] respectively \gls{use.d}[d] by
\gls{match}~$m$.
%
With these new definitions, \refEquationList{naive-dom, spanning,
  compatible-locations} are adjusted accordingly (the changes are highlighted in
grey):
%
\begin{equation}
  \forall m \in \mPhiMatchCompSet,
  \forall \hlDiff{p} \in \mUses(m),
  \forall o \in \mCovers(m) :
  \mVar{dplace}[\hlDiff{\mVar{alt}[p]}[1pt]] \in \mDom(\mVar{oplace}[o]),
  \labelEquation{naive-dom-alt}
\end{equation}
%
\begin{equation}
  \begin{array}{c}
    \forall m \in \mMatchSet,
    \forall \hlDiff{p} \in \mDefines(m),
    \forall o \in \mCovers(m) : \\
    \mVar{sel}[m]
    \mImp
    \mVar{dplace}[\hlDiff{\mVar{alt}[p]}[1pt]] \in
      \mSet{\mVar{oplace}[o]} \cup \mSpans(m),
  \end{array}
  \labelEquation{spanning-alt}
\end{equation}
%
\begin{equation}
  \forall m \in \mMatchSet,
  \forall \hlDiff{p} \in \mDefines(m) \cup \mUses(m) :
  \mVar{sel}[m]
  \mImp
  \mVar{loc}[\hlDiff{\mVar{alt}[p]}[1pt]] \in \mStores(m, \hlDiff{p}),
  \labelEquation{compatible-locations-alt}
\end{equation}
%
\begin{equation}
  \begin{array}{c}
    \forall m \in \mPhiMatchSet,
    \forall \hlDiff{p}_1 \hspace{-.8pt}, \hlDiff{p}_2 \in
      \mDefines(m) \cup \mUses(m) : \\
    \mVar{sel}[m]
    \mImp
    \mVar{loc}[\hlDiff{\mVar{alt}[p_1]}] = \mVar{loc}[\hlDiff{\mVar{alt}[p_2]}].
  \end{array}
  \labelEquation{phi-match-locations-alt}
\end{equation}

Due to these changes there may be \glspl{datum} that is not \gls{use.d}[d] by
any \gls{match}, yet \refEquation{data-definitions} still requires that every
\gls{datum} must be \gls{define.d}[d] by some \gls{match}.
%
We address this by extending the \gls{pattern set} with a \gls!{kill pattern}
(shaped like \mbox{$\mEdge{v}{\mEdge{c}{v'}}$}, where $v$ and $v'$ are
\glspl{value node} and $c$ is a \gls{copy node}).
%
\Glspl{match} derived from this \gls{pattern} are called \glspl!{kill match},
which have zero cost and emit nothing if selected.
%
A \gls{datum} is said to be \gls!{killed.d} if and only if it is
\gls{define.d}[d] by a \gls{kill match}, and non-\glspl{kill match} are only
allowed to make use of not \gls{killed.d} \glspl{datum}.

To model whether a \gls{datum} is \gls{killed.d}, we extend the \gls{location
  set} with a special \gls{location}~$\mKilledLocation$ and require that a
\gls{kill match}~$m$ is selected if and only if the \gls{location} of the
\gls{datum} \gls{define.d}[d] by $m$ is $\mKilledLocation$.
%
This is modeled as
%
\begin{equation}
  \forall m \in \mKillMatchSet,
  \forall p \in \mDefines(m) :
  \mVar{sel}[m] \mEq \mVar{loc}[\mVar{alt}[p]] = \mKilledLocation,
  \labelEquation{killed-data}
\end{equation}
%
where $\mKillMatchSet$ denotes the set of \glspl{kill match}.

Lastly, we need to enforce the value placements appearing in the
\glspl{phi-match}.
%
Let $\mMatchDefEdgeSet$ denote this set of value placements, encoded as a
tuple~\mbox{$\mTuple{m, b, p}$} for each \gls{definition edge} between a
\gls{block}~$b$ and an operand~$p$ appearing in \gls{match}~$m$.
%
These \glspl{constraint} are then modeled as
%
\begin{equation}
  \forall \mTuple{m, b, p} \in \mMatchDefEdgeSet :
  \mVar{sel}[m] \mImp \mVar{dplace}[\mVar{alt}[p]] = b \hspace{-1pt}.
  \labelEquation{match-def-edges}
\end{equation}


\subsection{Experimental Evaluation}

We first evaluate the two different methods for \gls{value reuse}.
%
Based on the results, we then evaluate the impact of \gls{value reuse} using the
superior method.

When filtering, we remove all \glspl{function} that have less than less
than ten~\gls{LLVM} \gls{IR} \glspl{instruction} -- anything smaller will most
likely not be benefitted by \gls{value reuse} -- and greater than
\num{50}~\glspl{instruction} -- anything larger will lead to unreasonably long
solving times.
%
This leaves a pool of \num{453}~\glspl{function}, on which we then perform
sampling.


\paragraph{Match Duplication \versus Alternative Values}

We evaluate the different methods of \gls{value reuse} by comparing the solving
time exhibited by two \glsplshort{constraint model}:
%
\begin{modelList}
  \item \labelModel{match-dup}
    one based on \gls{match duplication}
  \item \labelModel{alt-values}
    one based on \glspl{alternative value}
\end{modelList}.
%
Since \gls{match duplication} yields an exponential increase in number of
\glspl{match} compared to \glspl{alternative value}, we expect
\glsshort{constraint model}~\refModel{alt-values} to perform better than
\glsshort{constraint model}~\refModel{match-dup}.

\newcommand{\AltValuesVsMatchDupPrePlusSolvingTimeSpeedupSimpleNameAmean}{}
\newcommand{\AltValuesVsMatchDupPrePlusSolvingTimeSpeedupSimpleNameGmean}{}
\newcommand{\AltValuesVsMatchDupPrePlusSolvingTimeSpeedupSimpleNameMedian}{}
\newcommand{\AltValuesVsMatchDupPrePlusSolvingTimeSpeedupSimpleNameMin}{}
\newcommand{\AltValuesVsMatchDupPrePlusSolvingTimeSpeedupSimpleNameMax}{}
\newcommand{\AltValuesVsMatchDupPrePlusSolvingTimeSpeedupSolutionFoundAvgAmean}{1.0}
\newcommand{\AltValuesVsMatchDupPrePlusSolvingTimeSpeedupSolutionFoundAvgGmean}{}
\newcommand{\AltValuesVsMatchDupPrePlusSolvingTimeSpeedupSolutionFoundAvgMedian}{1.0}
\newcommand{\AltValuesVsMatchDupPrePlusSolvingTimeSpeedupSolutionFoundAvgMin}{1.0}
\newcommand{\AltValuesVsMatchDupPrePlusSolvingTimeSpeedupSolutionFoundAvgMax}{1.0}
\newcommand{\AltValuesVsMatchDupPrePlusSolvingTimeSpeedupCyclesAvgAmean}{792.64999999999998}
\newcommand{\AltValuesVsMatchDupPrePlusSolvingTimeSpeedupCyclesAvgGmean}{}
\newcommand{\AltValuesVsMatchDupPrePlusSolvingTimeSpeedupCyclesAvgMedian}{209.0}
\newcommand{\AltValuesVsMatchDupPrePlusSolvingTimeSpeedupCyclesAvgMin}{56.0}
\newcommand{\AltValuesVsMatchDupPrePlusSolvingTimeSpeedupCyclesAvgMax}{4476.0}
\newcommand{\AltValuesVsMatchDupPrePlusSolvingTimeSpeedupOptimalAvgAmean}{1.0}
\newcommand{\AltValuesVsMatchDupPrePlusSolvingTimeSpeedupOptimalAvgGmean}{}
\newcommand{\AltValuesVsMatchDupPrePlusSolvingTimeSpeedupOptimalAvgMedian}{1.0}
\newcommand{\AltValuesVsMatchDupPrePlusSolvingTimeSpeedupOptimalAvgMin}{1.0}
\newcommand{\AltValuesVsMatchDupPrePlusSolvingTimeSpeedupOptimalAvgMax}{1.0}
\newcommand{\AltValuesVsMatchDupPrePlusSolvingTimeSpeedupMatchingTimeAvgAmean}{0.347681900135}
\newcommand{\AltValuesVsMatchDupPrePlusSolvingTimeSpeedupMatchingTimeAvgGmean}{}
\newcommand{\AltValuesVsMatchDupPrePlusSolvingTimeSpeedupMatchingTimeAvgMedian}{0.27548791525000005}
\newcommand{\AltValuesVsMatchDupPrePlusSolvingTimeSpeedupMatchingTimeAvgMin}{0.055165400699999999}
\newcommand{\AltValuesVsMatchDupPrePlusSolvingTimeSpeedupMatchingTimeAvgMax}{0.84551443630000001}
\newcommand{\AltValuesVsMatchDupPrePlusSolvingTimeSpeedupLbCompTimeAvgAmean}{0.0}
\newcommand{\AltValuesVsMatchDupPrePlusSolvingTimeSpeedupLbCompTimeAvgGmean}{}
\newcommand{\AltValuesVsMatchDupPrePlusSolvingTimeSpeedupLbCompTimeAvgMedian}{0.0}
\newcommand{\AltValuesVsMatchDupPrePlusSolvingTimeSpeedupLbCompTimeAvgMin}{0.0}
\newcommand{\AltValuesVsMatchDupPrePlusSolvingTimeSpeedupLbCompTimeAvgMax}{0.0}
\newcommand{\AltValuesVsMatchDupPrePlusSolvingTimeSpeedupDomProcTimeAvgAmean}{0.03622652530670166}
\newcommand{\AltValuesVsMatchDupPrePlusSolvingTimeSpeedupDomProcTimeAvgGmean}{}
\newcommand{\AltValuesVsMatchDupPrePlusSolvingTimeSpeedupDomProcTimeAvgMedian}{0.023510563373565673}
\newcommand{\AltValuesVsMatchDupPrePlusSolvingTimeSpeedupDomProcTimeAvgMin}{0.0037126779556274415}
\newcommand{\AltValuesVsMatchDupPrePlusSolvingTimeSpeedupDomProcTimeAvgMax}{0.11369328498840332}
\newcommand{\AltValuesVsMatchDupPrePlusSolvingTimeSpeedupIllProcTimeAvgAmean}{0.064399623870849604}
\newcommand{\AltValuesVsMatchDupPrePlusSolvingTimeSpeedupIllProcTimeAvgGmean}{}
\newcommand{\AltValuesVsMatchDupPrePlusSolvingTimeSpeedupIllProcTimeAvgMedian}{0.042168414592742919}
\newcommand{\AltValuesVsMatchDupPrePlusSolvingTimeSpeedupIllProcTimeAvgMin}{0.0093542098999023441}
\newcommand{\AltValuesVsMatchDupPrePlusSolvingTimeSpeedupIllProcTimeAvgMax}{0.21914119720458985}
\newcommand{\AltValuesVsMatchDupPrePlusSolvingTimeSpeedupRedunProcTimeAvgAmean}{0.02617197632789612}
\newcommand{\AltValuesVsMatchDupPrePlusSolvingTimeSpeedupRedunProcTimeAvgGmean}{}
\newcommand{\AltValuesVsMatchDupPrePlusSolvingTimeSpeedupRedunProcTimeAvgMedian}{0.019291234016418458}
\newcommand{\AltValuesVsMatchDupPrePlusSolvingTimeSpeedupRedunProcTimeAvgMin}{0.0061443567276000975}
\newcommand{\AltValuesVsMatchDupPrePlusSolvingTimeSpeedupRedunProcTimeAvgMax}{0.07725059986114502}
\newcommand{\AltValuesVsMatchDupPrePlusSolvingTimeSpeedupModelPrepTimeAvgAmean}{2.7993999999999999}
\newcommand{\AltValuesVsMatchDupPrePlusSolvingTimeSpeedupModelPrepTimeAvgGmean}{}
\newcommand{\AltValuesVsMatchDupPrePlusSolvingTimeSpeedupModelPrepTimeAvgMedian}{1.5409999999999999}
\newcommand{\AltValuesVsMatchDupPrePlusSolvingTimeSpeedupModelPrepTimeAvgMin}{0.50299999999999989}
\newcommand{\AltValuesVsMatchDupPrePlusSolvingTimeSpeedupModelPrepTimeAvgMax}{8.0560000000000009}
\newcommand{\AltValuesVsMatchDupPrePlusSolvingTimeSpeedupSolvingTimeAvgAmean}{0.16219999999999998}
\newcommand{\AltValuesVsMatchDupPrePlusSolvingTimeSpeedupSolvingTimeAvgGmean}{}
\newcommand{\AltValuesVsMatchDupPrePlusSolvingTimeSpeedupSolvingTimeAvgMedian}{0.070000000000000021}
\newcommand{\AltValuesVsMatchDupPrePlusSolvingTimeSpeedupSolvingTimeAvgMin}{0.0099999999999999985}
\newcommand{\AltValuesVsMatchDupPrePlusSolvingTimeSpeedupSolvingTimeAvgMax}{1.048}
\newcommand{\AltValuesVsMatchDupPrePlusSolvingTimeSpeedupPrePlusSolvingTimeAvgAmean}{0.2889981255054474}
\newcommand{\AltValuesVsMatchDupPrePlusSolvingTimeSpeedupPrePlusSolvingTimeAvgGmean}{}
\newcommand{\AltValuesVsMatchDupPrePlusSolvingTimeSpeedupPrePlusSolvingTimeAvgMedian}{0.15711308097839355}
\newcommand{\AltValuesVsMatchDupPrePlusSolvingTimeSpeedupPrePlusSolvingTimeAvgMin}{0.029211244583129885}
\newcommand{\AltValuesVsMatchDupPrePlusSolvingTimeSpeedupPrePlusSolvingTimeAvgMax}{1.282584641456604}
\newcommand{\AltValuesVsMatchDupPrePlusSolvingTimeSpeedupTotalTimeAvgAmean}{0.63668002564044746}
\newcommand{\AltValuesVsMatchDupPrePlusSolvingTimeSpeedupTotalTimeAvgGmean}{}
\newcommand{\AltValuesVsMatchDupPrePlusSolvingTimeSpeedupTotalTimeAvgMedian}{0.4938628999030944}
\newcommand{\AltValuesVsMatchDupPrePlusSolvingTimeSpeedupTotalTimeAvgMin}{0.10221239109459228}
\newcommand{\AltValuesVsMatchDupPrePlusSolvingTimeSpeedupTotalTimeAvgMax}{1.5277213663566043}
\newcommand{\AltValuesVsMatchDupPrePlusSolvingTimeSpeedupCyclesCvAmean}{0.0}
\newcommand{\AltValuesVsMatchDupPrePlusSolvingTimeSpeedupCyclesCvGmean}{}
\newcommand{\AltValuesVsMatchDupPrePlusSolvingTimeSpeedupCyclesCvMedian}{0.0}
\newcommand{\AltValuesVsMatchDupPrePlusSolvingTimeSpeedupCyclesCvMin}{0.0}
\newcommand{\AltValuesVsMatchDupPrePlusSolvingTimeSpeedupCyclesCvMax}{0.0}
\newcommand{\AltValuesVsMatchDupPrePlusSolvingTimeSpeedupLbCompTimeCvAmean}{0.0}
\newcommand{\AltValuesVsMatchDupPrePlusSolvingTimeSpeedupLbCompTimeCvGmean}{}
\newcommand{\AltValuesVsMatchDupPrePlusSolvingTimeSpeedupLbCompTimeCvMedian}{0.0}
\newcommand{\AltValuesVsMatchDupPrePlusSolvingTimeSpeedupLbCompTimeCvMin}{0.0}
\newcommand{\AltValuesVsMatchDupPrePlusSolvingTimeSpeedupLbCompTimeCvMax}{0.0}
\newcommand{\AltValuesVsMatchDupPrePlusSolvingTimeSpeedupDomProcTimeCvAmean}{0.063443583084763347}
\newcommand{\AltValuesVsMatchDupPrePlusSolvingTimeSpeedupDomProcTimeCvGmean}{}
\newcommand{\AltValuesVsMatchDupPrePlusSolvingTimeSpeedupDomProcTimeCvMedian}{0.014649587132410491}
\newcommand{\AltValuesVsMatchDupPrePlusSolvingTimeSpeedupDomProcTimeCvMin}{0.0054009361579640596}
\newcommand{\AltValuesVsMatchDupPrePlusSolvingTimeSpeedupDomProcTimeCvMax}{0.40087342390784292}
\newcommand{\AltValuesVsMatchDupPrePlusSolvingTimeSpeedupIllProcTimeCvAmean}{0.037588921632898599}
\newcommand{\AltValuesVsMatchDupPrePlusSolvingTimeSpeedupIllProcTimeCvGmean}{}
\newcommand{\AltValuesVsMatchDupPrePlusSolvingTimeSpeedupIllProcTimeCvMedian}{0.011133521578967825}
\newcommand{\AltValuesVsMatchDupPrePlusSolvingTimeSpeedupIllProcTimeCvMin}{0.0045808437529523066}
\newcommand{\AltValuesVsMatchDupPrePlusSolvingTimeSpeedupIllProcTimeCvMax}{0.12620978690811835}
\newcommand{\AltValuesVsMatchDupPrePlusSolvingTimeSpeedupRedunProcTimeCvAmean}{0.054532332709919994}
\newcommand{\AltValuesVsMatchDupPrePlusSolvingTimeSpeedupRedunProcTimeCvGmean}{}
\newcommand{\AltValuesVsMatchDupPrePlusSolvingTimeSpeedupRedunProcTimeCvMedian}{0.018706518851293302}
\newcommand{\AltValuesVsMatchDupPrePlusSolvingTimeSpeedupRedunProcTimeCvMin}{0.010458465379296563}
\newcommand{\AltValuesVsMatchDupPrePlusSolvingTimeSpeedupRedunProcTimeCvMax}{0.23077617017581448}
\newcommand{\AltValuesVsMatchDupPrePlusSolvingTimeSpeedupModelPrepTimeCvAmean}{0.0074496427893159271}
\newcommand{\AltValuesVsMatchDupPrePlusSolvingTimeSpeedupModelPrepTimeCvGmean}{}
\newcommand{\AltValuesVsMatchDupPrePlusSolvingTimeSpeedupModelPrepTimeCvMedian}{0.0071906965871146433}
\newcommand{\AltValuesVsMatchDupPrePlusSolvingTimeSpeedupModelPrepTimeCvMin}{0.0038552896163789514}
\newcommand{\AltValuesVsMatchDupPrePlusSolvingTimeSpeedupModelPrepTimeCvMax}{0.015527335339774679}
\newcommand{\AltValuesVsMatchDupPrePlusSolvingTimeSpeedupSolvingTimeCvAmean}{0.052022712571383786}
\newcommand{\AltValuesVsMatchDupPrePlusSolvingTimeSpeedupSolvingTimeCvGmean}{}
\newcommand{\AltValuesVsMatchDupPrePlusSolvingTimeSpeedupSolvingTimeCvMedian}{0.01712384255274757}
\newcommand{\AltValuesVsMatchDupPrePlusSolvingTimeSpeedupSolvingTimeCvMin}{0.0}
\newcommand{\AltValuesVsMatchDupPrePlusSolvingTimeSpeedupSolvingTimeCvMax}{0.33333333333333331}
\newcommand{\AltValuesVsMatchDupPrePlusSolvingTimeSpeedupPrePlusSolvingTimeCvAmean}{0.028465374178821872}
\newcommand{\AltValuesVsMatchDupPrePlusSolvingTimeSpeedupPrePlusSolvingTimeCvGmean}{}
\newcommand{\AltValuesVsMatchDupPrePlusSolvingTimeSpeedupPrePlusSolvingTimeCvMedian}{0.019683924864323413}
\newcommand{\AltValuesVsMatchDupPrePlusSolvingTimeSpeedupPrePlusSolvingTimeCvMin}{0.0015061849698106559}
\newcommand{\AltValuesVsMatchDupPrePlusSolvingTimeSpeedupPrePlusSolvingTimeCvMax}{0.083861020086850313}
\newcommand{\AltValuesVsMatchDupPrePlusSolvingTimeSpeedupTotalTimeCvAmean}{0.01070449793805953}
\newcommand{\AltValuesVsMatchDupPrePlusSolvingTimeSpeedupTotalTimeCvGmean}{}
\newcommand{\AltValuesVsMatchDupPrePlusSolvingTimeSpeedupTotalTimeCvMedian}{0.0078673555185165348}
\newcommand{\AltValuesVsMatchDupPrePlusSolvingTimeSpeedupTotalTimeCvMin}{0.0026270611769105643}
\newcommand{\AltValuesVsMatchDupPrePlusSolvingTimeSpeedupTotalTimeCvMax}{0.034374438078413373}
\newcommand{\AltValuesVsMatchDupPrePlusSolvingTimeSpeedupBaselineSimpleNameAmean}{}
\newcommand{\AltValuesVsMatchDupPrePlusSolvingTimeSpeedupBaselineSimpleNameGmean}{}
\newcommand{\AltValuesVsMatchDupPrePlusSolvingTimeSpeedupBaselineSimpleNameMedian}{}
\newcommand{\AltValuesVsMatchDupPrePlusSolvingTimeSpeedupBaselineSimpleNameMin}{}
\newcommand{\AltValuesVsMatchDupPrePlusSolvingTimeSpeedupBaselineSimpleNameMax}{}
\newcommand{\AltValuesVsMatchDupPrePlusSolvingTimeSpeedupBaselineSolutionFoundAvgAmean}{1.0}
\newcommand{\AltValuesVsMatchDupPrePlusSolvingTimeSpeedupBaselineSolutionFoundAvgGmean}{}
\newcommand{\AltValuesVsMatchDupPrePlusSolvingTimeSpeedupBaselineSolutionFoundAvgMedian}{1.0}
\newcommand{\AltValuesVsMatchDupPrePlusSolvingTimeSpeedupBaselineSolutionFoundAvgMin}{1.0}
\newcommand{\AltValuesVsMatchDupPrePlusSolvingTimeSpeedupBaselineSolutionFoundAvgMax}{1.0}
\newcommand{\AltValuesVsMatchDupPrePlusSolvingTimeSpeedupBaselineCyclesAvgAmean}{793.45000000000005}
\newcommand{\AltValuesVsMatchDupPrePlusSolvingTimeSpeedupBaselineCyclesAvgGmean}{}
\newcommand{\AltValuesVsMatchDupPrePlusSolvingTimeSpeedupBaselineCyclesAvgMedian}{217.0}
\newcommand{\AltValuesVsMatchDupPrePlusSolvingTimeSpeedupBaselineCyclesAvgMin}{56.0}
\newcommand{\AltValuesVsMatchDupPrePlusSolvingTimeSpeedupBaselineCyclesAvgMax}{4476.0}
\newcommand{\AltValuesVsMatchDupPrePlusSolvingTimeSpeedupBaselineOptimalAvgAmean}{1.0}
\newcommand{\AltValuesVsMatchDupPrePlusSolvingTimeSpeedupBaselineOptimalAvgGmean}{}
\newcommand{\AltValuesVsMatchDupPrePlusSolvingTimeSpeedupBaselineOptimalAvgMedian}{1.0}
\newcommand{\AltValuesVsMatchDupPrePlusSolvingTimeSpeedupBaselineOptimalAvgMin}{1.0}
\newcommand{\AltValuesVsMatchDupPrePlusSolvingTimeSpeedupBaselineOptimalAvgMax}{1.0}
\newcommand{\AltValuesVsMatchDupPrePlusSolvingTimeSpeedupBaselineMatchingTimeAvgAmean}{0.34875073656500005}
\newcommand{\AltValuesVsMatchDupPrePlusSolvingTimeSpeedupBaselineMatchingTimeAvgGmean}{}
\newcommand{\AltValuesVsMatchDupPrePlusSolvingTimeSpeedupBaselineMatchingTimeAvgMedian}{0.27613018969999997}
\newcommand{\AltValuesVsMatchDupPrePlusSolvingTimeSpeedupBaselineMatchingTimeAvgMin}{0.055342191599999993}
\newcommand{\AltValuesVsMatchDupPrePlusSolvingTimeSpeedupBaselineMatchingTimeAvgMax}{0.85120100070000004}
\newcommand{\AltValuesVsMatchDupPrePlusSolvingTimeSpeedupBaselineLbCompTimeAvgAmean}{0.0}
\newcommand{\AltValuesVsMatchDupPrePlusSolvingTimeSpeedupBaselineLbCompTimeAvgGmean}{}
\newcommand{\AltValuesVsMatchDupPrePlusSolvingTimeSpeedupBaselineLbCompTimeAvgMedian}{0.0}
\newcommand{\AltValuesVsMatchDupPrePlusSolvingTimeSpeedupBaselineLbCompTimeAvgMin}{0.0}
\newcommand{\AltValuesVsMatchDupPrePlusSolvingTimeSpeedupBaselineLbCompTimeAvgMax}{0.0}
\newcommand{\AltValuesVsMatchDupPrePlusSolvingTimeSpeedupBaselineDomProcTimeAvgAmean}{0.736092256307602}
\newcommand{\AltValuesVsMatchDupPrePlusSolvingTimeSpeedupBaselineDomProcTimeAvgGmean}{}
\newcommand{\AltValuesVsMatchDupPrePlusSolvingTimeSpeedupBaselineDomProcTimeAvgMedian}{0.21901499032974242}
\newcommand{\AltValuesVsMatchDupPrePlusSolvingTimeSpeedupBaselineDomProcTimeAvgMin}{0.009364676475524903}
\newcommand{\AltValuesVsMatchDupPrePlusSolvingTimeSpeedupBaselineDomProcTimeAvgMax}{7.8705319404602054}
\newcommand{\AltValuesVsMatchDupPrePlusSolvingTimeSpeedupBaselineIllProcTimeAvgAmean}{0.27205534219741828}
\newcommand{\AltValuesVsMatchDupPrePlusSolvingTimeSpeedupBaselineIllProcTimeAvgGmean}{}
\newcommand{\AltValuesVsMatchDupPrePlusSolvingTimeSpeedupBaselineIllProcTimeAvgMedian}{0.12335673570632934}
\newcommand{\AltValuesVsMatchDupPrePlusSolvingTimeSpeedupBaselineIllProcTimeAvgMin}{0.016881108283996582}
\newcommand{\AltValuesVsMatchDupPrePlusSolvingTimeSpeedupBaselineIllProcTimeAvgMax}{1.073728895187378}
\newcommand{\AltValuesVsMatchDupPrePlusSolvingTimeSpeedupBaselineRedunProcTimeAvgAmean}{0.060946557521820079}
\newcommand{\AltValuesVsMatchDupPrePlusSolvingTimeSpeedupBaselineRedunProcTimeAvgGmean}{}
\newcommand{\AltValuesVsMatchDupPrePlusSolvingTimeSpeedupBaselineRedunProcTimeAvgMedian}{0.038130486011505128}
\newcommand{\AltValuesVsMatchDupPrePlusSolvingTimeSpeedupBaselineRedunProcTimeAvgMin}{0.0083100557327270511}
\newcommand{\AltValuesVsMatchDupPrePlusSolvingTimeSpeedupBaselineRedunProcTimeAvgMax}{0.17907221317291261}
\newcommand{\AltValuesVsMatchDupPrePlusSolvingTimeSpeedupBaselineModelPrepTimeAvgAmean}{11.693149999999999}
\newcommand{\AltValuesVsMatchDupPrePlusSolvingTimeSpeedupBaselineModelPrepTimeAvgGmean}{}
\newcommand{\AltValuesVsMatchDupPrePlusSolvingTimeSpeedupBaselineModelPrepTimeAvgMedian}{3.7960000000000003}
\newcommand{\AltValuesVsMatchDupPrePlusSolvingTimeSpeedupBaselineModelPrepTimeAvgMin}{0.72899999999999998}
\newcommand{\AltValuesVsMatchDupPrePlusSolvingTimeSpeedupBaselineModelPrepTimeAvgMax}{67.159999999999997}
\newcommand{\AltValuesVsMatchDupPrePlusSolvingTimeSpeedupBaselineSolvingTimeAvgAmean}{0.37574999999999997}
\newcommand{\AltValuesVsMatchDupPrePlusSolvingTimeSpeedupBaselineSolvingTimeAvgGmean}{}
\newcommand{\AltValuesVsMatchDupPrePlusSolvingTimeSpeedupBaselineSolvingTimeAvgMedian}{0.11899999999999999}
\newcommand{\AltValuesVsMatchDupPrePlusSolvingTimeSpeedupBaselineSolvingTimeAvgMin}{0.0099999999999999985}
\newcommand{\AltValuesVsMatchDupPrePlusSolvingTimeSpeedupBaselineSolvingTimeAvgMax}{1.9479999999999997}
\newcommand{\AltValuesVsMatchDupPrePlusSolvingTimeSpeedupBaselinePrePlusSolvingTimeAvgAmean}{1.4448441560268399}
\newcommand{\AltValuesVsMatchDupPrePlusSolvingTimeSpeedupBaselinePrePlusSolvingTimeAvgGmean}{}
\newcommand{\AltValuesVsMatchDupPrePlusSolvingTimeSpeedupBaselinePrePlusSolvingTimeAvgMedian}{0.47401351070404052}
\newcommand{\AltValuesVsMatchDupPrePlusSolvingTimeSpeedupBaselinePrePlusSolvingTimeAvgMin}{0.054555840492248542}
\newcommand{\AltValuesVsMatchDupPrePlusSolvingTimeSpeedupBaselinePrePlusSolvingTimeAvgMax}{10.396333048820495}
\newcommand{\AltValuesVsMatchDupPrePlusSolvingTimeSpeedupBaselineTotalTimeAvgAmean}{1.7935948925918406}
\newcommand{\AltValuesVsMatchDupPrePlusSolvingTimeSpeedupBaselineTotalTimeAvgGmean}{}
\newcommand{\AltValuesVsMatchDupPrePlusSolvingTimeSpeedupBaselineTotalTimeAvgMedian}{0.84062555055404053}
\newcommand{\AltValuesVsMatchDupPrePlusSolvingTimeSpeedupBaselineTotalTimeAvgMin}{0.14372422999224851}
\newcommand{\AltValuesVsMatchDupPrePlusSolvingTimeSpeedupBaselineTotalTimeAvgMax}{10.767147976220496}
\newcommand{\AltValuesVsMatchDupPrePlusSolvingTimeSpeedupBaselineCyclesCvAmean}{0.0}
\newcommand{\AltValuesVsMatchDupPrePlusSolvingTimeSpeedupBaselineCyclesCvGmean}{}
\newcommand{\AltValuesVsMatchDupPrePlusSolvingTimeSpeedupBaselineCyclesCvMedian}{0.0}
\newcommand{\AltValuesVsMatchDupPrePlusSolvingTimeSpeedupBaselineCyclesCvMin}{0.0}
\newcommand{\AltValuesVsMatchDupPrePlusSolvingTimeSpeedupBaselineCyclesCvMax}{0.0}
\newcommand{\AltValuesVsMatchDupPrePlusSolvingTimeSpeedupBaselineLbCompTimeCvAmean}{0.0}
\newcommand{\AltValuesVsMatchDupPrePlusSolvingTimeSpeedupBaselineLbCompTimeCvGmean}{}
\newcommand{\AltValuesVsMatchDupPrePlusSolvingTimeSpeedupBaselineLbCompTimeCvMedian}{0.0}
\newcommand{\AltValuesVsMatchDupPrePlusSolvingTimeSpeedupBaselineLbCompTimeCvMin}{0.0}
\newcommand{\AltValuesVsMatchDupPrePlusSolvingTimeSpeedupBaselineLbCompTimeCvMax}{0.0}
\newcommand{\AltValuesVsMatchDupPrePlusSolvingTimeSpeedupBaselineDomProcTimeCvAmean}{0.033313692810768639}
\newcommand{\AltValuesVsMatchDupPrePlusSolvingTimeSpeedupBaselineDomProcTimeCvGmean}{}
\newcommand{\AltValuesVsMatchDupPrePlusSolvingTimeSpeedupBaselineDomProcTimeCvMedian}{0.0092155640169980946}
\newcommand{\AltValuesVsMatchDupPrePlusSolvingTimeSpeedupBaselineDomProcTimeCvMin}{0.0027404579133161427}
\newcommand{\AltValuesVsMatchDupPrePlusSolvingTimeSpeedupBaselineDomProcTimeCvMax}{0.16961824132605649}
\newcommand{\AltValuesVsMatchDupPrePlusSolvingTimeSpeedupBaselineIllProcTimeCvAmean}{0.014520474622784251}
\newcommand{\AltValuesVsMatchDupPrePlusSolvingTimeSpeedupBaselineIllProcTimeCvGmean}{}
\newcommand{\AltValuesVsMatchDupPrePlusSolvingTimeSpeedupBaselineIllProcTimeCvMedian}{0.0098797329008626124}
\newcommand{\AltValuesVsMatchDupPrePlusSolvingTimeSpeedupBaselineIllProcTimeCvMin}{0.0043162615408652164}
\newcommand{\AltValuesVsMatchDupPrePlusSolvingTimeSpeedupBaselineIllProcTimeCvMax}{0.10703860711133653}
\newcommand{\AltValuesVsMatchDupPrePlusSolvingTimeSpeedupBaselineRedunProcTimeCvAmean}{0.066556331835671251}
\newcommand{\AltValuesVsMatchDupPrePlusSolvingTimeSpeedupBaselineRedunProcTimeCvGmean}{}
\newcommand{\AltValuesVsMatchDupPrePlusSolvingTimeSpeedupBaselineRedunProcTimeCvMedian}{0.013390393317851957}
\newcommand{\AltValuesVsMatchDupPrePlusSolvingTimeSpeedupBaselineRedunProcTimeCvMin}{0.0067871909694950877}
\newcommand{\AltValuesVsMatchDupPrePlusSolvingTimeSpeedupBaselineRedunProcTimeCvMax}{0.44794610651948813}
\newcommand{\AltValuesVsMatchDupPrePlusSolvingTimeSpeedupBaselineModelPrepTimeCvAmean}{0.0058822368406538485}
\newcommand{\AltValuesVsMatchDupPrePlusSolvingTimeSpeedupBaselineModelPrepTimeCvGmean}{}
\newcommand{\AltValuesVsMatchDupPrePlusSolvingTimeSpeedupBaselineModelPrepTimeCvMedian}{0.0055487597722966974}
\newcommand{\AltValuesVsMatchDupPrePlusSolvingTimeSpeedupBaselineModelPrepTimeCvMin}{1.1686558153949016e-16}
\newcommand{\AltValuesVsMatchDupPrePlusSolvingTimeSpeedupBaselineModelPrepTimeCvMax}{0.012940989207210716}
\newcommand{\AltValuesVsMatchDupPrePlusSolvingTimeSpeedupBaselineSolvingTimeCvAmean}{0.031029170823734742}
\newcommand{\AltValuesVsMatchDupPrePlusSolvingTimeSpeedupBaselineSolvingTimeCvGmean}{}
\newcommand{\AltValuesVsMatchDupPrePlusSolvingTimeSpeedupBaselineSolvingTimeCvMedian}{0.009356021285118532}
\newcommand{\AltValuesVsMatchDupPrePlusSolvingTimeSpeedupBaselineSolvingTimeCvMin}{0.0}
\newcommand{\AltValuesVsMatchDupPrePlusSolvingTimeSpeedupBaselineSolvingTimeCvMax}{0.27272727272727271}
\newcommand{\AltValuesVsMatchDupPrePlusSolvingTimeSpeedupBaselinePrePlusSolvingTimeCvAmean}{0.024380333297092084}
\newcommand{\AltValuesVsMatchDupPrePlusSolvingTimeSpeedupBaselinePrePlusSolvingTimeCvGmean}{}
\newcommand{\AltValuesVsMatchDupPrePlusSolvingTimeSpeedupBaselinePrePlusSolvingTimeCvMedian}{0.0091186328536995431}
\newcommand{\AltValuesVsMatchDupPrePlusSolvingTimeSpeedupBaselinePrePlusSolvingTimeCvMin}{0.0034174839942242904}
\newcommand{\AltValuesVsMatchDupPrePlusSolvingTimeSpeedupBaselinePrePlusSolvingTimeCvMax}{0.11406873838626899}
\newcommand{\AltValuesVsMatchDupPrePlusSolvingTimeSpeedupBaselineTotalTimeCvAmean}{0.011647338374092736}
\newcommand{\AltValuesVsMatchDupPrePlusSolvingTimeSpeedupBaselineTotalTimeCvGmean}{}
\newcommand{\AltValuesVsMatchDupPrePlusSolvingTimeSpeedupBaselineTotalTimeCvMedian}{0.0068609476236861037}
\newcommand{\AltValuesVsMatchDupPrePlusSolvingTimeSpeedupBaselineTotalTimeCvMin}{0.002044183731041125}
\newcommand{\AltValuesVsMatchDupPrePlusSolvingTimeSpeedupBaselineTotalTimeCvMax}{0.037893885453466815}
\newcommand{\AltValuesVsMatchDupPrePlusSolvingTimeSpeedupPrePlusSolvingTimeZeroCenteredSpeedupAmean}{n/a}
\newcommand{\AltValuesVsMatchDupPrePlusSolvingTimeSpeedupPrePlusSolvingTimeZeroCenteredSpeedupGmean}{n/a}
\newcommand{\AltValuesVsMatchDupPrePlusSolvingTimeSpeedupPrePlusSolvingTimeZeroCenteredSpeedupMedian}{1.5696153151589773}
\newcommand{\AltValuesVsMatchDupPrePlusSolvingTimeSpeedupPrePlusSolvingTimeZeroCenteredSpeedupMin}{0.17111569577328262}
\newcommand{\AltValuesVsMatchDupPrePlusSolvingTimeSpeedupPrePlusSolvingTimeZeroCenteredSpeedupMax}{46.911686554191974}
\newcommand{\AltValuesVsMatchDupPrePlusSolvingTimeSpeedupPrePlusSolvingTimeRegularSpeedupAmean}{n/a}
\newcommand{\AltValuesVsMatchDupPrePlusSolvingTimeSpeedupPrePlusSolvingTimeRegularSpeedupGmean}{3.3474483077024209}
\newcommand{\AltValuesVsMatchDupPrePlusSolvingTimeSpeedupPrePlusSolvingTimeRegularSpeedupMedian}{2.5696153151589769}
\newcommand{\AltValuesVsMatchDupPrePlusSolvingTimeSpeedupPrePlusSolvingTimeRegularSpeedupMin}{1.1711156957732827}
\newcommand{\AltValuesVsMatchDupPrePlusSolvingTimeSpeedupPrePlusSolvingTimeRegularSpeedupMax}{47.911686554191974}
\newcommand{\AltValuesVsMatchDupPrePlusSolvingTimeSpeedupPrePlusSolvingTimeRegularSpeedupCiAmean}{n/a}
\newcommand{\AltValuesVsMatchDupPrePlusSolvingTimeSpeedupPrePlusSolvingTimeRegularSpeedupCiGmean}{n/a}
\newcommand{\AltValuesVsMatchDupPrePlusSolvingTimeSpeedupPrePlusSolvingTimeRegularSpeedupCiMedian}{n/a}
\newcommand{\AltValuesVsMatchDupPrePlusSolvingTimeSpeedupPrePlusSolvingTimeRegularSpeedupCiMin}{2.241264756328063}
\newcommand{\AltValuesVsMatchDupPrePlusSolvingTimeSpeedupPrePlusSolvingTimeRegularSpeedupCiMax}{4.6872396966756105}


\begin{figure}
  \centering%
  \maxsizebox{\textwidth}{!}{%
    \trimBarchartPlot{%
      \begin{tikzpicture}[gnuplot]
%% generated with GNUPLOT 5.0p4 (Lua 5.2; terminal rev. 99, script rev. 100)
%% lör  3 feb 2018 16:43:33
\path (0.000,0.000) rectangle (12.500,8.750);
\gpcolor{rgb color={0.753,0.753,0.753}}
\gpsetlinetype{gp lt axes}
\gpsetdashtype{gp dt axes}
\gpsetlinewidth{0.50}
\draw[gp path] (1.012,1.687)--(36.945,1.687);
\gpcolor{color=gp lt color border}
\node[gp node right] at (1.012,1.687) {\plotZCNormTics{0}};
\gpcolor{rgb color={0.753,0.753,0.753}}
\draw[gp path] (1.012,2.356)--(36.945,2.356);
\gpcolor{color=gp lt color border}
\node[gp node right] at (1.012,2.356) {\plotZCNormTics{5}};
\gpcolor{rgb color={0.753,0.753,0.753}}
\draw[gp path] (1.012,3.026)--(36.945,3.026);
\gpcolor{color=gp lt color border}
\node[gp node right] at (1.012,3.026) {\plotZCNormTics{10}};
\gpcolor{rgb color={0.753,0.753,0.753}}
\draw[gp path] (1.012,3.695)--(36.945,3.695);
\gpcolor{color=gp lt color border}
\node[gp node right] at (1.012,3.695) {\plotZCNormTics{15}};
\gpcolor{rgb color={0.753,0.753,0.753}}
\draw[gp path] (1.012,4.365)--(36.945,4.365);
\gpcolor{color=gp lt color border}
\node[gp node right] at (1.012,4.365) {\plotZCNormTics{20}};
\gpcolor{rgb color={0.753,0.753,0.753}}
\draw[gp path] (1.012,5.034)--(36.945,5.034);
\gpcolor{color=gp lt color border}
\node[gp node right] at (1.012,5.034) {\plotZCNormTics{25}};
\gpcolor{rgb color={0.753,0.753,0.753}}
\draw[gp path] (1.012,5.703)--(36.945,5.703);
\gpcolor{color=gp lt color border}
\node[gp node right] at (1.012,5.703) {\plotZCNormTics{30}};
\gpcolor{rgb color={0.753,0.753,0.753}}
\draw[gp path] (1.012,6.373)--(36.945,6.373);
\gpcolor{color=gp lt color border}
\node[gp node right] at (1.012,6.373) {\plotZCNormTics{35}};
\gpcolor{rgb color={0.753,0.753,0.753}}
\draw[gp path] (1.012,7.042)--(36.945,7.042);
\gpcolor{color=gp lt color border}
\node[gp node right] at (1.012,7.042) {\plotZCNormTics{40}};
\gpcolor{rgb color={0.753,0.753,0.753}}
\draw[gp path] (1.012,7.712)--(36.945,7.712);
\gpcolor{color=gp lt color border}
\node[gp node right] at (1.012,7.712) {\plotZCNormTics{45}};
\gpcolor{rgb color={0.753,0.753,0.753}}
\draw[gp path] (1.012,8.381)--(36.945,8.381);
\gpcolor{color=gp lt color border}
\node[gp node right] at (1.012,8.381) {\plotZCNormTics{50}};
\node[gp node left,rotate=-30] at (2.759,1.442) {\functionName{bi_reverse}};
\node[gp node left,rotate=-30] at (4.470,1.442) {\functionName{device_color_en.}};
\node[gp node left,rotate=-30] at (6.181,1.442) {\functionName{dict_put_string}};
\node[gp node left,rotate=-30] at (7.892,1.442) {\functionName{ecSub}};
\node[gp node left,rotate=-30] at (9.603,1.442) {\functionName{FORD1}};
\node[gp node left,rotate=-30] at (11.315,1.442) {\functionName{free_tree_nodes}};
\node[gp node left,rotate=-30] at (13.026,1.442) {\functionName{gl_flip_bytes}};
\node[gp node left,rotate=-30] at (14.737,1.442) {\functionName{gluNextContour}};
\node[gp node left,rotate=-30] at (16.448,1.442) {\functionName{gx_color_frac_m.}};
\node[gp node left,rotate=-30] at (18.159,1.442) {\functionName{hash_initial}};
\node[gp node left,rotate=-30] at (19.870,1.442) {\functionName{jinit_huff_deco.}};
\node[gp node left,rotate=-30] at (21.581,1.442) {\functionName{jinit_phuff_dec.}};
\node[gp node left,rotate=-30] at (23.292,1.442) {\functionName{jpeg_alloc_quan.}};
\node[gp node left,rotate=-30] at (25.003,1.442) {\functionName{jpeg_has_multip.}};
\node[gp node left,rotate=-30] at (26.714,1.442) {\functionName{mp_quo_digit}};
\node[gp node left,rotate=-30] at (28.426,1.442) {\functionName{name_ref_sub_ta.}};
\node[gp node left,rotate=-30] at (30.137,1.442) {\functionName{putACfirst}};
\node[gp node left,rotate=-30] at (31.848,1.442) {\functionName{putpicthdr}};
\node[gp node left,rotate=-30] at (33.559,1.442) {\functionName{putseqdispext}};
\node[gp node left,rotate=-30] at (35.270,1.442) {\functionName{reg2rsaref}};
\gpsetlinetype{gp lt border}
\gpsetdashtype{gp dt solid}
\gpsetlinewidth{1.00}
\draw[gp path] (1.012,8.381)--(1.012,1.687)--(36.945,1.687)--(36.945,8.381)--cycle;
\gpcolor{rgb color={0.000,0.000,0.000}}
\draw[gp path] (1.012,1.687)--(1.375,1.687)--(1.738,1.687)--(2.101,1.687)--(2.464,1.687)%
  --(2.827,1.687)--(3.190,1.687)--(3.553,1.687)--(3.916,1.687)--(4.279,1.687)--(4.642,1.687)%
  --(5.005,1.687)--(5.368,1.687)--(5.730,1.687)--(6.093,1.687)--(6.456,1.687)--(6.819,1.687)%
  --(7.182,1.687)--(7.545,1.687)--(7.908,1.687)--(8.271,1.687)--(8.634,1.687)--(8.997,1.687)%
  --(9.360,1.687)--(9.723,1.687)--(10.086,1.687)--(10.449,1.687)--(10.812,1.687)--(11.175,1.687)%
  --(11.538,1.687)--(11.901,1.687)--(12.264,1.687)--(12.627,1.687)--(12.990,1.687)--(13.353,1.687)%
  --(13.716,1.687)--(14.079,1.687)--(14.442,1.687)--(14.804,1.687)--(15.167,1.687)--(15.530,1.687)%
  --(15.893,1.687)--(16.256,1.687)--(16.619,1.687)--(16.982,1.687)--(17.345,1.687)--(17.708,1.687)%
  --(18.071,1.687)--(18.434,1.687)--(18.797,1.687)--(19.160,1.687)--(19.523,1.687)--(19.886,1.687)%
  --(20.249,1.687)--(20.612,1.687)--(20.975,1.687)--(21.338,1.687)--(21.701,1.687)--(22.064,1.687)%
  --(22.427,1.687)--(22.790,1.687)--(23.153,1.687)--(23.515,1.687)--(23.878,1.687)--(24.241,1.687)%
  --(24.604,1.687)--(24.967,1.687)--(25.330,1.687)--(25.693,1.687)--(26.056,1.687)--(26.419,1.687)%
  --(26.782,1.687)--(27.145,1.687)--(27.508,1.687)--(27.871,1.687)--(28.234,1.687)--(28.597,1.687)%
  --(28.960,1.687)--(29.323,1.687)--(29.686,1.687)--(30.049,1.687)--(30.412,1.687)--(30.775,1.687)%
  --(31.138,1.687)--(31.501,1.687)--(31.864,1.687)--(32.227,1.687)--(32.589,1.687)--(32.952,1.687)%
  --(33.315,1.687)--(33.678,1.687)--(34.041,1.687)--(34.404,1.687)--(34.767,1.687)--(35.130,1.687)%
  --(35.493,1.687)--(35.856,1.687)--(36.219,1.687)--(36.582,1.687)--(36.945,1.687);
\gpfill{rgb color={0.533,0.533,0.533}} (2.652,1.687)--(3.366,1.687)--(3.366,1.772)--(2.652,1.772)--cycle;
\gpcolor{color=gp lt color border}
\draw[gp path] (2.652,1.687)--(2.652,1.771)--(3.365,1.771)--(3.365,1.687)--cycle;
\gpfill{rgb color={0.533,0.533,0.533}} (4.363,1.687)--(5.077,1.687)--(5.077,1.719)--(4.363,1.719)--cycle;
\draw[gp path] (4.363,1.687)--(4.363,1.718)--(5.076,1.718)--(5.076,1.687)--cycle;
\gpfill{rgb color={0.533,0.533,0.533}} (6.074,1.687)--(6.788,1.687)--(6.788,1.907)--(6.074,1.907)--cycle;
\draw[gp path] (6.074,1.687)--(6.074,1.906)--(6.787,1.906)--(6.787,1.687)--cycle;
\gpfill{rgb color={0.533,0.533,0.533}} (7.785,1.687)--(8.499,1.687)--(8.499,2.013)--(7.785,2.013)--cycle;
\draw[gp path] (7.785,1.687)--(7.785,2.012)--(8.498,2.012)--(8.498,1.687)--cycle;
\gpfill{rgb color={0.533,0.533,0.533}} (9.496,1.687)--(10.210,1.687)--(10.210,1.969)--(9.496,1.969)--cycle;
\draw[gp path] (9.496,1.687)--(9.496,1.968)--(10.209,1.968)--(10.209,1.687)--cycle;
\gpfill{rgb color={0.533,0.533,0.533}} (11.207,1.687)--(11.921,1.687)--(11.921,1.857)--(11.207,1.857)--cycle;
\draw[gp path] (11.207,1.687)--(11.207,1.856)--(11.920,1.856)--(11.920,1.687)--cycle;
\gpfill{rgb color={0.533,0.533,0.533}} (12.918,1.687)--(13.632,1.687)--(13.632,2.062)--(12.918,2.062)--cycle;
\draw[gp path] (12.918,1.687)--(12.918,2.061)--(13.631,2.061)--(13.631,1.687)--cycle;
\gpfill{rgb color={0.533,0.533,0.533}} (14.629,1.687)--(15.343,1.687)--(15.343,1.928)--(14.629,1.928)--cycle;
\draw[gp path] (14.629,1.687)--(14.629,1.927)--(15.342,1.927)--(15.342,1.687)--cycle;
\gpfill{rgb color={0.533,0.533,0.533}} (16.341,1.687)--(17.055,1.687)--(17.055,1.884)--(16.341,1.884)--cycle;
\draw[gp path] (16.341,1.687)--(16.341,1.883)--(17.054,1.883)--(17.054,1.687)--cycle;
\gpfill{rgb color={0.533,0.533,0.533}} (18.052,1.687)--(18.766,1.687)--(18.766,3.543)--(18.052,3.543)--cycle;
\draw[gp path] (18.052,1.687)--(18.052,3.542)--(18.765,3.542)--(18.765,1.687)--cycle;
\gpfill{rgb color={0.533,0.533,0.533}} (19.763,1.687)--(20.477,1.687)--(20.477,2.234)--(19.763,2.234)--cycle;
\draw[gp path] (19.763,1.687)--(19.763,2.233)--(20.476,2.233)--(20.476,1.687)--cycle;
\gpfill{rgb color={0.533,0.533,0.533}} (21.474,1.687)--(22.188,1.687)--(22.188,2.426)--(21.474,2.426)--cycle;
\draw[gp path] (21.474,1.687)--(21.474,2.425)--(22.187,2.425)--(22.187,1.687)--cycle;
\gpfill{rgb color={0.533,0.533,0.533}} (23.185,1.687)--(23.899,1.687)--(23.899,1.824)--(23.185,1.824)--cycle;
\draw[gp path] (23.185,1.687)--(23.185,1.823)--(23.898,1.823)--(23.898,1.687)--cycle;
\gpfill{rgb color={0.533,0.533,0.533}} (24.896,1.687)--(25.610,1.687)--(25.610,1.923)--(24.896,1.923)--cycle;
\draw[gp path] (24.896,1.687)--(24.896,1.922)--(25.609,1.922)--(25.609,1.687)--cycle;
\gpfill{rgb color={0.533,0.533,0.533}} (26.607,1.687)--(27.321,1.687)--(27.321,1.889)--(26.607,1.889)--cycle;
\draw[gp path] (26.607,1.687)--(26.607,1.888)--(27.320,1.888)--(27.320,1.687)--cycle;
\gpfill{rgb color={0.533,0.533,0.533}} (28.318,1.687)--(29.032,1.687)--(29.032,1.721)--(28.318,1.721)--cycle;
\draw[gp path] (28.318,1.687)--(28.318,1.720)--(29.031,1.720)--(29.031,1.687)--cycle;
\gpfill{rgb color={0.533,0.533,0.533}} (30.029,1.687)--(30.743,1.687)--(30.743,1.888)--(30.029,1.888)--cycle;
\draw[gp path] (30.029,1.687)--(30.029,1.887)--(30.742,1.887)--(30.742,1.687)--cycle;
\gpfill{rgb color={0.533,0.533,0.533}} (31.740,1.687)--(32.454,1.687)--(32.454,1.829)--(31.740,1.829)--cycle;
\draw[gp path] (31.740,1.687)--(31.740,1.828)--(32.453,1.828)--(32.453,1.687)--cycle;
\gpfill{rgb color={0.533,0.533,0.533}} (33.452,1.687)--(34.165,1.687)--(34.165,1.711)--(33.452,1.711)--cycle;
\draw[gp path] (33.452,1.687)--(33.452,1.710)--(34.164,1.710)--(34.164,1.687)--cycle;
\gpfill{rgb color={0.533,0.533,0.533}} (35.163,1.687)--(35.877,1.687)--(35.877,7.969)--(35.163,7.969)--cycle;
\draw[gp path] (35.163,1.687)--(35.163,7.968)--(35.876,7.968)--(35.876,1.687)--cycle;
\node[gp node center] at (2.999,1.955) {\plotBarNormValue{0.625112}};
\node[gp node center] at (4.710,1.902) {\plotBarNormValue{0.229050}};
\node[gp node center] at (6.421,2.090) {\plotBarNormValue{1.637786}};
\node[gp node center] at (8.132,2.196) {\plotBarNormValue{2.425374}};
\node[gp node center] at (9.843,2.152) {\plotBarNormValue{2.101483}};
\node[gp node center] at (11.555,2.040) {\plotBarNormValue{1.262073}};
\node[gp node center] at (13.266,2.245) {\plotBarNormValue{2.796017}};
\node[gp node center] at (14.977,2.111) {\plotBarNormValue{1.795430}};
\node[gp node center] at (16.688,2.067) {\plotBarNormValue{1.462714}};
\node[gp node center] at (18.399,3.726) {\plotBarNormValue{13.854515}};
\node[gp node center] at (20.110,2.417) {\plotBarNormValue{4.076437}};
\node[gp node center] at (21.821,2.609) {\plotBarNormValue{5.515458}};
\node[gp node center] at (23.532,2.007) {\plotBarNormValue{1.013928}};
\node[gp node center] at (25.243,2.106) {\plotBarNormValue{1.755737}};
\node[gp node center] at (26.954,2.072) {\plotBarNormValue{1.501445}};
\node[gp node center] at (28.666,1.904) {\plotBarNormValue{0.249549}};
\node[gp node center] at (30.377,2.071) {\plotBarNormValue{1.494730}};
\node[gp node center] at (32.088,2.012) {\plotBarNormValue{1.056352}};
\node[gp node center] at (33.799,1.894) {\plotBarNormValue{0.171116}};
\node[gp node center] at (35.510,8.152) {\plotBarNormValue{46.911687}};
\draw[gp path] (1.012,8.381)--(1.012,1.687)--(36.945,1.687)--(36.945,8.381)--cycle;
%% coordinates of the plot area
\gpdefrectangularnode{gp plot 1}{\pgfpoint{1.012cm}{1.687cm}}{\pgfpoint{36.945cm}{8.381cm}}
\end{tikzpicture}
%% gnuplot variables
%
    }%
  }

  \caption[%
            Plot for evaluating match duplication's and alternative values'
            impact on solving time%
          ]%
          {%
            Normalized solving times (incl.\ presolving time) for two
            constraint models that supports value reuse: one based on match
            duplication (baseline), and one based on alternative values
            (subject).
            %
            GMI:~\printGMI{%
              \AltValuesVsMatchDupPrePlusSolvingTimeSpeedupPrePlusSolvingTimeRegularSpeedupGmean%
            },
            CI~\printGMICI{%
              \AltValuesVsMatchDupPrePlusSolvingTimeSpeedupPrePlusSolvingTimeRegularSpeedupCiMin%
            }{%
              \AltValuesVsMatchDupPrePlusSolvingTimeSpeedupPrePlusSolvingTimeRegularSpeedupCiMax%
            }%
          }
  \labelFigure{alt-values-vs-match-dup-solving-time-plot}
\end{figure}

\RefFigure{alt-values-vs-match-dup-solving-time-plot} shows the normalized
solving times (including \gls{presolving} time) for the two \glspl{constraint
  model} described above, with \glsshort{constraint model}~\refModel{match-dup}
as \gls{baseline} and \glsshort{constraint model}~\refModel{alt-values} as
\gls{subject}.
%
All \glspl{function} are solved to optimality.
%
The solving times range from
\printMinSolvingTime{
  \AltValuesVsMatchDupPrePlusSolvingTimeSpeedupSolvingTimeAvgMin,
  \AltValuesVsMatchDupPrePlusSolvingTimeSpeedupBaselineSolvingTimeAvgMin
}
to
\printMaxSolvingTime{
  \AltValuesVsMatchDupPrePlusSolvingTimeSpeedupSolvingTimeAvgMax,
  \AltValuesVsMatchDupPrePlusSolvingTimeSpeedupBaselineSolvingTimeAvgMax
}
with a \gls{CV} of
\numMaxOf{
  \AltValuesVsMatchDupPrePlusSolvingTimeSpeedupPrePlusSolvingTimeCvMax,
  \AltValuesVsMatchDupPrePlusSolvingTimeSpeedupBaselinePrePlusSolvingTimeCvMax
}.
%
The \gls{GMI} is \printGMI{%
  \AltValuesVsMatchDupPrePlusSolvingTimeSpeedupPrePlusSolvingTimeRegularSpeedupGmean%
} with \gls{CI}~\printGMICI{%
  \AltValuesVsMatchDupPrePlusSolvingTimeSpeedupPrePlusSolvingTimeRegularSpeedupCiMin%
}{%
  \AltValuesVsMatchDupPrePlusSolvingTimeSpeedupPrePlusSolvingTimeRegularSpeedupCiMax%
}.

We see clearly that \glsshort{constraint model}~\refModel{alt-values} results in
significantly shorter solving times than \glsshort{constraint
  model}~\refModel{match-dup}.
%
In one case (\cCode*{reg2rsaref}), for example, the solving time differs by
nearly a factor of~\num{50}, which is due to a high rate of \gls{copy-related.d}
values for which \glspl{alternative value} results in only
\num{94}~\glspl{match} whereas \gls{match duplication} results in
\num{537}~\glspl{match}.

Based on the observations above, we conclude that \glspl{alternative value} is a
better design choice over \gls{match duplication} when implementing \gls{value
  reuse}.


\paragraph{Impact of Value Reuse}

We evaluate the impact of \gls{value reuse} by comparing the cost (that is, the
total number of cycles, as described in \refSection{cm-objective-function}) of
the optimal \glspl{solution} produced by two \glsplshort{constraint model}:
%
\begin{modelList}
  \item \labelModel{wo-value-reuse}
    one without \gls{value reuse} support
  \item \labelModel{w-value-reuse}
    one with this support (based on \glspl{alternative value})
\end{modelList}.
%
Since \gls{value reuse} reduces the number of selected copy \glspl{instruction},
we expect \glsshort{constraint model}~\refModel{w-value-reuse} to produce
\glspl{solution} with less cost compared with \glsshort{constraint
  model}~\refModel{wo-value-reuse}.

\newcommand{\AltValuesVsWithoutCyclesSpeedupSimpleNameAmean}{}
\newcommand{\AltValuesVsWithoutCyclesSpeedupSimpleNameGmean}{}
\newcommand{\AltValuesVsWithoutCyclesSpeedupSimpleNameMedian}{}
\newcommand{\AltValuesVsWithoutCyclesSpeedupSimpleNameMin}{}
\newcommand{\AltValuesVsWithoutCyclesSpeedupSimpleNameMax}{}
\newcommand{\AltValuesVsWithoutCyclesSpeedupSolutionFoundAvgAmean}{1.0}
\newcommand{\AltValuesVsWithoutCyclesSpeedupSolutionFoundAvgGmean}{}
\newcommand{\AltValuesVsWithoutCyclesSpeedupSolutionFoundAvgMedian}{1.0}
\newcommand{\AltValuesVsWithoutCyclesSpeedupSolutionFoundAvgMin}{1.0}
\newcommand{\AltValuesVsWithoutCyclesSpeedupSolutionFoundAvgMax}{1.0}
\newcommand{\AltValuesVsWithoutCyclesSpeedupCyclesAvgAmean}{844.0}
\newcommand{\AltValuesVsWithoutCyclesSpeedupCyclesAvgGmean}{}
\newcommand{\AltValuesVsWithoutCyclesSpeedupCyclesAvgMedian}{209.0}
\newcommand{\AltValuesVsWithoutCyclesSpeedupCyclesAvgMin}{56.0}
\newcommand{\AltValuesVsWithoutCyclesSpeedupCyclesAvgMax}{4579.0}
\newcommand{\AltValuesVsWithoutCyclesSpeedupOptimalAvgAmean}{1.0}
\newcommand{\AltValuesVsWithoutCyclesSpeedupOptimalAvgGmean}{}
\newcommand{\AltValuesVsWithoutCyclesSpeedupOptimalAvgMedian}{1.0}
\newcommand{\AltValuesVsWithoutCyclesSpeedupOptimalAvgMin}{1.0}
\newcommand{\AltValuesVsWithoutCyclesSpeedupOptimalAvgMax}{1.0}
\newcommand{\AltValuesVsWithoutCyclesSpeedupMatchingTimeAvgAmean}{0.34967712798}
\newcommand{\AltValuesVsWithoutCyclesSpeedupMatchingTimeAvgGmean}{}
\newcommand{\AltValuesVsWithoutCyclesSpeedupMatchingTimeAvgMedian}{0.27725442950000001}
\newcommand{\AltValuesVsWithoutCyclesSpeedupMatchingTimeAvgMin}{0.055402152199999999}
\newcommand{\AltValuesVsWithoutCyclesSpeedupMatchingTimeAvgMax}{0.85017893280000012}
\newcommand{\AltValuesVsWithoutCyclesSpeedupLbCompTimeAvgAmean}{0.0}
\newcommand{\AltValuesVsWithoutCyclesSpeedupLbCompTimeAvgGmean}{}
\newcommand{\AltValuesVsWithoutCyclesSpeedupLbCompTimeAvgMedian}{0.0}
\newcommand{\AltValuesVsWithoutCyclesSpeedupLbCompTimeAvgMin}{0.0}
\newcommand{\AltValuesVsWithoutCyclesSpeedupLbCompTimeAvgMax}{0.0}
\newcommand{\AltValuesVsWithoutCyclesSpeedupDomProcTimeAvgAmean}{0.03591879844665527}
\newcommand{\AltValuesVsWithoutCyclesSpeedupDomProcTimeAvgGmean}{}
\newcommand{\AltValuesVsWithoutCyclesSpeedupDomProcTimeAvgMedian}{0.022027969360351562}
\newcommand{\AltValuesVsWithoutCyclesSpeedupDomProcTimeAvgMin}{0.0053460121154785155}
\newcommand{\AltValuesVsWithoutCyclesSpeedupDomProcTimeAvgMax}{0.11321244239807129}
\newcommand{\AltValuesVsWithoutCyclesSpeedupIllProcTimeAvgAmean}{0.064540748596191411}
\newcommand{\AltValuesVsWithoutCyclesSpeedupIllProcTimeAvgGmean}{}
\newcommand{\AltValuesVsWithoutCyclesSpeedupIllProcTimeAvgMedian}{0.042433738708496094}
\newcommand{\AltValuesVsWithoutCyclesSpeedupIllProcTimeAvgMin}{0.0096159934997558597}
\newcommand{\AltValuesVsWithoutCyclesSpeedupIllProcTimeAvgMax}{0.21834096908569336}
\newcommand{\AltValuesVsWithoutCyclesSpeedupRedunProcTimeAvgAmean}{0.025688898563385004}
\newcommand{\AltValuesVsWithoutCyclesSpeedupRedunProcTimeAvgGmean}{}
\newcommand{\AltValuesVsWithoutCyclesSpeedupRedunProcTimeAvgMedian}{0.019464659690856933}
\newcommand{\AltValuesVsWithoutCyclesSpeedupRedunProcTimeAvgMin}{0.0060076236724853514}
\newcommand{\AltValuesVsWithoutCyclesSpeedupRedunProcTimeAvgMax}{0.075340604782104498}
\newcommand{\AltValuesVsWithoutCyclesSpeedupModelPrepTimeAvgAmean}{2.8004999999999995}
\newcommand{\AltValuesVsWithoutCyclesSpeedupModelPrepTimeAvgGmean}{}
\newcommand{\AltValuesVsWithoutCyclesSpeedupModelPrepTimeAvgMedian}{1.5510000000000002}
\newcommand{\AltValuesVsWithoutCyclesSpeedupModelPrepTimeAvgMin}{0.504}
\newcommand{\AltValuesVsWithoutCyclesSpeedupModelPrepTimeAvgMax}{8.0639999999999983}
\newcommand{\AltValuesVsWithoutCyclesSpeedupSolvingTimeAvgAmean}{0.16850000000000001}
\newcommand{\AltValuesVsWithoutCyclesSpeedupSolvingTimeAvgGmean}{}
\newcommand{\AltValuesVsWithoutCyclesSpeedupSolvingTimeAvgMedian}{0.070000000000000007}
\newcommand{\AltValuesVsWithoutCyclesSpeedupSolvingTimeAvgMin}{0.01}
\newcommand{\AltValuesVsWithoutCyclesSpeedupSolvingTimeAvgMax}{1.0840000000000001}
\newcommand{\AltValuesVsWithoutCyclesSpeedupPrePlusSolvingTimeAvgAmean}{0.29464844560623166}
\newcommand{\AltValuesVsWithoutCyclesSpeedupPrePlusSolvingTimeAvgGmean}{}
\newcommand{\AltValuesVsWithoutCyclesSpeedupPrePlusSolvingTimeAvgMedian}{0.15729642391204834}
\newcommand{\AltValuesVsWithoutCyclesSpeedupPrePlusSolvingTimeAvgMin}{0.030969629287719729}
\newcommand{\AltValuesVsWithoutCyclesSpeedupPrePlusSolvingTimeAvgMax}{1.3183576908111572}
\newcommand{\AltValuesVsWithoutCyclesSpeedupTotalTimeAvgAmean}{0.64432557358623166}
\newcommand{\AltValuesVsWithoutCyclesSpeedupTotalTimeAvgGmean}{}
\newcommand{\AltValuesVsWithoutCyclesSpeedupTotalTimeAvgMedian}{0.49498851253255616}
\newcommand{\AltValuesVsWithoutCyclesSpeedupTotalTimeAvgMin}{0.10385340418364257}
\newcommand{\AltValuesVsWithoutCyclesSpeedupTotalTimeAvgMax}{1.6087699042658692}
\newcommand{\AltValuesVsWithoutCyclesSpeedupCyclesCvAmean}{0.0}
\newcommand{\AltValuesVsWithoutCyclesSpeedupCyclesCvGmean}{}
\newcommand{\AltValuesVsWithoutCyclesSpeedupCyclesCvMedian}{0.0}
\newcommand{\AltValuesVsWithoutCyclesSpeedupCyclesCvMin}{0.0}
\newcommand{\AltValuesVsWithoutCyclesSpeedupCyclesCvMax}{0.0}
\newcommand{\AltValuesVsWithoutCyclesSpeedupLbCompTimeCvAmean}{0.0}
\newcommand{\AltValuesVsWithoutCyclesSpeedupLbCompTimeCvGmean}{}
\newcommand{\AltValuesVsWithoutCyclesSpeedupLbCompTimeCvMedian}{0.0}
\newcommand{\AltValuesVsWithoutCyclesSpeedupLbCompTimeCvMin}{0.0}
\newcommand{\AltValuesVsWithoutCyclesSpeedupLbCompTimeCvMax}{0.0}
\newcommand{\AltValuesVsWithoutCyclesSpeedupDomProcTimeCvAmean}{0.049518428037234094}
\newcommand{\AltValuesVsWithoutCyclesSpeedupDomProcTimeCvGmean}{}
\newcommand{\AltValuesVsWithoutCyclesSpeedupDomProcTimeCvMedian}{0.011350449301566271}
\newcommand{\AltValuesVsWithoutCyclesSpeedupDomProcTimeCvMin}{0.0053408580936775791}
\newcommand{\AltValuesVsWithoutCyclesSpeedupDomProcTimeCvMax}{0.62384651985142725}
\newcommand{\AltValuesVsWithoutCyclesSpeedupIllProcTimeCvAmean}{0.047642760733462179}
\newcommand{\AltValuesVsWithoutCyclesSpeedupIllProcTimeCvGmean}{}
\newcommand{\AltValuesVsWithoutCyclesSpeedupIllProcTimeCvMedian}{0.0081630322218440854}
\newcommand{\AltValuesVsWithoutCyclesSpeedupIllProcTimeCvMin}{0.0018652509988406913}
\newcommand{\AltValuesVsWithoutCyclesSpeedupIllProcTimeCvMax}{0.27964899215913719}
\newcommand{\AltValuesVsWithoutCyclesSpeedupRedunProcTimeCvAmean}{0.037432042934795642}
\newcommand{\AltValuesVsWithoutCyclesSpeedupRedunProcTimeCvGmean}{}
\newcommand{\AltValuesVsWithoutCyclesSpeedupRedunProcTimeCvMedian}{0.011178914139401875}
\newcommand{\AltValuesVsWithoutCyclesSpeedupRedunProcTimeCvMin}{0.004707955214663561}
\newcommand{\AltValuesVsWithoutCyclesSpeedupRedunProcTimeCvMax}{0.26738690886592942}
\newcommand{\AltValuesVsWithoutCyclesSpeedupModelPrepTimeCvAmean}{0.0044872552724609781}
\newcommand{\AltValuesVsWithoutCyclesSpeedupModelPrepTimeCvGmean}{}
\newcommand{\AltValuesVsWithoutCyclesSpeedupModelPrepTimeCvMedian}{0.0035925622649660384}
\newcommand{\AltValuesVsWithoutCyclesSpeedupModelPrepTimeCvMin}{0.0}
\newcommand{\AltValuesVsWithoutCyclesSpeedupModelPrepTimeCvMax}{0.015873015873015886}
\newcommand{\AltValuesVsWithoutCyclesSpeedupSolvingTimeCvAmean}{0.035178050676744448}
\newcommand{\AltValuesVsWithoutCyclesSpeedupSolvingTimeCvGmean}{}
\newcommand{\AltValuesVsWithoutCyclesSpeedupSolvingTimeCvMedian}{0.009643660404658189}
\newcommand{\AltValuesVsWithoutCyclesSpeedupSolvingTimeCvMin}{0.0}
\newcommand{\AltValuesVsWithoutCyclesSpeedupSolvingTimeCvMax}{0.34992710611188255}
\newcommand{\AltValuesVsWithoutCyclesSpeedupPrePlusSolvingTimeCvAmean}{0.030832302605391854}
\newcommand{\AltValuesVsWithoutCyclesSpeedupPrePlusSolvingTimeCvGmean}{}
\newcommand{\AltValuesVsWithoutCyclesSpeedupPrePlusSolvingTimeCvMedian}{0.018864630569824872}
\newcommand{\AltValuesVsWithoutCyclesSpeedupPrePlusSolvingTimeCvMin}{0.00067335187056895274}
\newcommand{\AltValuesVsWithoutCyclesSpeedupPrePlusSolvingTimeCvMax}{0.10081372176061817}
\newcommand{\AltValuesVsWithoutCyclesSpeedupTotalTimeCvAmean}{0.012076544813327744}
\newcommand{\AltValuesVsWithoutCyclesSpeedupTotalTimeCvGmean}{}
\newcommand{\AltValuesVsWithoutCyclesSpeedupTotalTimeCvMedian}{0.0093217093647337181}
\newcommand{\AltValuesVsWithoutCyclesSpeedupTotalTimeCvMin}{0.00052251388876036202}
\newcommand{\AltValuesVsWithoutCyclesSpeedupTotalTimeCvMax}{0.035628743950326482}
\newcommand{\AltValuesVsWithoutCyclesSpeedupBaselineSimpleNameAmean}{}
\newcommand{\AltValuesVsWithoutCyclesSpeedupBaselineSimpleNameGmean}{}
\newcommand{\AltValuesVsWithoutCyclesSpeedupBaselineSimpleNameMedian}{}
\newcommand{\AltValuesVsWithoutCyclesSpeedupBaselineSimpleNameMin}{}
\newcommand{\AltValuesVsWithoutCyclesSpeedupBaselineSimpleNameMax}{}
\newcommand{\AltValuesVsWithoutCyclesSpeedupBaselineSolutionFoundAvgAmean}{1.0}
\newcommand{\AltValuesVsWithoutCyclesSpeedupBaselineSolutionFoundAvgGmean}{}
\newcommand{\AltValuesVsWithoutCyclesSpeedupBaselineSolutionFoundAvgMedian}{1.0}
\newcommand{\AltValuesVsWithoutCyclesSpeedupBaselineSolutionFoundAvgMin}{1.0}
\newcommand{\AltValuesVsWithoutCyclesSpeedupBaselineSolutionFoundAvgMax}{1.0}
\newcommand{\AltValuesVsWithoutCyclesSpeedupBaselineCyclesAvgAmean}{859.29999999999995}
\newcommand{\AltValuesVsWithoutCyclesSpeedupBaselineCyclesAvgGmean}{}
\newcommand{\AltValuesVsWithoutCyclesSpeedupBaselineCyclesAvgMedian}{240.0}
\newcommand{\AltValuesVsWithoutCyclesSpeedupBaselineCyclesAvgMin}{56.0}
\newcommand{\AltValuesVsWithoutCyclesSpeedupBaselineCyclesAvgMax}{4587.0}
\newcommand{\AltValuesVsWithoutCyclesSpeedupBaselineOptimalAvgAmean}{1.0}
\newcommand{\AltValuesVsWithoutCyclesSpeedupBaselineOptimalAvgGmean}{}
\newcommand{\AltValuesVsWithoutCyclesSpeedupBaselineOptimalAvgMedian}{1.0}
\newcommand{\AltValuesVsWithoutCyclesSpeedupBaselineOptimalAvgMin}{1.0}
\newcommand{\AltValuesVsWithoutCyclesSpeedupBaselineOptimalAvgMax}{1.0}
\newcommand{\AltValuesVsWithoutCyclesSpeedupBaselineMatchingTimeAvgAmean}{0.34874708341000005}
\newcommand{\AltValuesVsWithoutCyclesSpeedupBaselineMatchingTimeAvgGmean}{}
\newcommand{\AltValuesVsWithoutCyclesSpeedupBaselineMatchingTimeAvgMedian}{0.27691842529999999}
\newcommand{\AltValuesVsWithoutCyclesSpeedupBaselineMatchingTimeAvgMin}{0.055787963600000004}
\newcommand{\AltValuesVsWithoutCyclesSpeedupBaselineMatchingTimeAvgMax}{0.8463931024000001}
\newcommand{\AltValuesVsWithoutCyclesSpeedupBaselineLbCompTimeAvgAmean}{0.0}
\newcommand{\AltValuesVsWithoutCyclesSpeedupBaselineLbCompTimeAvgGmean}{}
\newcommand{\AltValuesVsWithoutCyclesSpeedupBaselineLbCompTimeAvgMedian}{0.0}
\newcommand{\AltValuesVsWithoutCyclesSpeedupBaselineLbCompTimeAvgMin}{0.0}
\newcommand{\AltValuesVsWithoutCyclesSpeedupBaselineLbCompTimeAvgMax}{0.0}
\newcommand{\AltValuesVsWithoutCyclesSpeedupBaselineDomProcTimeAvgAmean}{0.03524016857147217}
\newcommand{\AltValuesVsWithoutCyclesSpeedupBaselineDomProcTimeAvgGmean}{}
\newcommand{\AltValuesVsWithoutCyclesSpeedupBaselineDomProcTimeAvgMedian}{0.021691155433654786}
\newcommand{\AltValuesVsWithoutCyclesSpeedupBaselineDomProcTimeAvgMin}{0.003702211380004883}
\newcommand{\AltValuesVsWithoutCyclesSpeedupBaselineDomProcTimeAvgMax}{0.11030745506286621}
\newcommand{\AltValuesVsWithoutCyclesSpeedupBaselineIllProcTimeAvgAmean}{0.06415964126586915}
\newcommand{\AltValuesVsWithoutCyclesSpeedupBaselineIllProcTimeAvgGmean}{}
\newcommand{\AltValuesVsWithoutCyclesSpeedupBaselineIllProcTimeAvgMedian}{0.043309450149536133}
\newcommand{\AltValuesVsWithoutCyclesSpeedupBaselineIllProcTimeAvgMin}{0.0095195770263671875}
\newcommand{\AltValuesVsWithoutCyclesSpeedupBaselineIllProcTimeAvgMax}{0.21637024879455566}
\newcommand{\AltValuesVsWithoutCyclesSpeedupBaselineRedunProcTimeAvgAmean}{0.024435949325561524}
\newcommand{\AltValuesVsWithoutCyclesSpeedupBaselineRedunProcTimeAvgGmean}{}
\newcommand{\AltValuesVsWithoutCyclesSpeedupBaselineRedunProcTimeAvgMedian}{0.018855500221252444}
\newcommand{\AltValuesVsWithoutCyclesSpeedupBaselineRedunProcTimeAvgMin}{0.0056238174438476562}
\newcommand{\AltValuesVsWithoutCyclesSpeedupBaselineRedunProcTimeAvgMax}{0.073705816268920893}
\newcommand{\AltValuesVsWithoutCyclesSpeedupBaselineModelPrepTimeAvgAmean}{2.4803999999999999}
\newcommand{\AltValuesVsWithoutCyclesSpeedupBaselineModelPrepTimeAvgGmean}{}
\newcommand{\AltValuesVsWithoutCyclesSpeedupBaselineModelPrepTimeAvgMedian}{1.4219999999999997}
\newcommand{\AltValuesVsWithoutCyclesSpeedupBaselineModelPrepTimeAvgMin}{0.43200000000000005}
\newcommand{\AltValuesVsWithoutCyclesSpeedupBaselineModelPrepTimeAvgMax}{7.2479999999999993}
\newcommand{\AltValuesVsWithoutCyclesSpeedupBaselineSolvingTimeAvgAmean}{0.099599999999999994}
\newcommand{\AltValuesVsWithoutCyclesSpeedupBaselineSolvingTimeAvgGmean}{}
\newcommand{\AltValuesVsWithoutCyclesSpeedupBaselineSolvingTimeAvgMedian}{0.049000000000000002}
\newcommand{\AltValuesVsWithoutCyclesSpeedupBaselineSolvingTimeAvgMin}{0.0}
\newcommand{\AltValuesVsWithoutCyclesSpeedupBaselineSolvingTimeAvgMax}{0.43200000000000005}
\newcommand{\AltValuesVsWithoutCyclesSpeedupBaselinePrePlusSolvingTimeAvgAmean}{0.22343575916290281}
\newcommand{\AltValuesVsWithoutCyclesSpeedupBaselinePrePlusSolvingTimeAvgGmean}{}
\newcommand{\AltValuesVsWithoutCyclesSpeedupBaselinePrePlusSolvingTimeAvgMedian}{0.13691612243652346}
\newcommand{\AltValuesVsWithoutCyclesSpeedupBaselinePrePlusSolvingTimeAvgMin}{0.018845605850219726}
\newcommand{\AltValuesVsWithoutCyclesSpeedupBaselinePrePlusSolvingTimeAvgMax}{0.66482923698425278}
\newcommand{\AltValuesVsWithoutCyclesSpeedupBaselineTotalTimeAvgAmean}{0.57218284257290286}
\newcommand{\AltValuesVsWithoutCyclesSpeedupBaselineTotalTimeAvgGmean}{}
\newcommand{\AltValuesVsWithoutCyclesSpeedupBaselineTotalTimeAvgMedian}{0.46104910599174803}
\newcommand{\AltValuesVsWithoutCyclesSpeedupBaselineTotalTimeAvgMin}{0.092840270728906243}
\newcommand{\AltValuesVsWithoutCyclesSpeedupBaselineTotalTimeAvgMax}{1.4323679755263425}
\newcommand{\AltValuesVsWithoutCyclesSpeedupBaselineCyclesCvAmean}{0.0}
\newcommand{\AltValuesVsWithoutCyclesSpeedupBaselineCyclesCvGmean}{}
\newcommand{\AltValuesVsWithoutCyclesSpeedupBaselineCyclesCvMedian}{0.0}
\newcommand{\AltValuesVsWithoutCyclesSpeedupBaselineCyclesCvMin}{0.0}
\newcommand{\AltValuesVsWithoutCyclesSpeedupBaselineCyclesCvMax}{0.0}
\newcommand{\AltValuesVsWithoutCyclesSpeedupBaselineLbCompTimeCvAmean}{0.0}
\newcommand{\AltValuesVsWithoutCyclesSpeedupBaselineLbCompTimeCvGmean}{}
\newcommand{\AltValuesVsWithoutCyclesSpeedupBaselineLbCompTimeCvMedian}{0.0}
\newcommand{\AltValuesVsWithoutCyclesSpeedupBaselineLbCompTimeCvMin}{0.0}
\newcommand{\AltValuesVsWithoutCyclesSpeedupBaselineLbCompTimeCvMax}{0.0}
\newcommand{\AltValuesVsWithoutCyclesSpeedupBaselineDomProcTimeCvAmean}{0.016881349950741396}
\newcommand{\AltValuesVsWithoutCyclesSpeedupBaselineDomProcTimeCvGmean}{}
\newcommand{\AltValuesVsWithoutCyclesSpeedupBaselineDomProcTimeCvMedian}{0.008285453831358848}
\newcommand{\AltValuesVsWithoutCyclesSpeedupBaselineDomProcTimeCvMin}{0.0029999145005271521}
\newcommand{\AltValuesVsWithoutCyclesSpeedupBaselineDomProcTimeCvMax}{0.11781874602301859}
\newcommand{\AltValuesVsWithoutCyclesSpeedupBaselineIllProcTimeCvAmean}{0.034783626856487587}
\newcommand{\AltValuesVsWithoutCyclesSpeedupBaselineIllProcTimeCvGmean}{}
\newcommand{\AltValuesVsWithoutCyclesSpeedupBaselineIllProcTimeCvMedian}{0.011279765584824852}
\newcommand{\AltValuesVsWithoutCyclesSpeedupBaselineIllProcTimeCvMin}{0.0038844899653923187}
\newcommand{\AltValuesVsWithoutCyclesSpeedupBaselineIllProcTimeCvMax}{0.25556061638551392}
\newcommand{\AltValuesVsWithoutCyclesSpeedupBaselineRedunProcTimeCvAmean}{0.026824188788872584}
\newcommand{\AltValuesVsWithoutCyclesSpeedupBaselineRedunProcTimeCvGmean}{}
\newcommand{\AltValuesVsWithoutCyclesSpeedupBaselineRedunProcTimeCvMedian}{0.011363132643724242}
\newcommand{\AltValuesVsWithoutCyclesSpeedupBaselineRedunProcTimeCvMin}{0.0052115663727194091}
\newcommand{\AltValuesVsWithoutCyclesSpeedupBaselineRedunProcTimeCvMax}{0.25171753841371619}
\newcommand{\AltValuesVsWithoutCyclesSpeedupBaselineModelPrepTimeCvAmean}{0.0067927094715224557}
\newcommand{\AltValuesVsWithoutCyclesSpeedupBaselineModelPrepTimeCvGmean}{}
\newcommand{\AltValuesVsWithoutCyclesSpeedupBaselineModelPrepTimeCvMedian}{0.0062619554610103736}
\newcommand{\AltValuesVsWithoutCyclesSpeedupBaselineModelPrepTimeCvMin}{0.0}
\newcommand{\AltValuesVsWithoutCyclesSpeedupBaselineModelPrepTimeCvMax}{0.016806722689075598}
\newcommand{\AltValuesVsWithoutCyclesSpeedupBaselineSolvingTimeCvAmean}{0.088358823848871254}
\newcommand{\AltValuesVsWithoutCyclesSpeedupBaselineSolvingTimeCvGmean}{}
\newcommand{\AltValuesVsWithoutCyclesSpeedupBaselineSolvingTimeCvMedian}{0.017224530242096976}
\newcommand{\AltValuesVsWithoutCyclesSpeedupBaselineSolvingTimeCvMin}{0.0}
\newcommand{\AltValuesVsWithoutCyclesSpeedupBaselineSolvingTimeCvMax}{0.33333333333333331}
\newcommand{\AltValuesVsWithoutCyclesSpeedupBaselinePrePlusSolvingTimeCvAmean}{0.035858202022851213}
\newcommand{\AltValuesVsWithoutCyclesSpeedupBaselinePrePlusSolvingTimeCvGmean}{}
\newcommand{\AltValuesVsWithoutCyclesSpeedupBaselinePrePlusSolvingTimeCvMedian}{0.014879202699241329}
\newcommand{\AltValuesVsWithoutCyclesSpeedupBaselinePrePlusSolvingTimeCvMin}{0.0012431831189241033}
\newcommand{\AltValuesVsWithoutCyclesSpeedupBaselinePrePlusSolvingTimeCvMax}{0.1124681222180277}
\newcommand{\AltValuesVsWithoutCyclesSpeedupBaselineTotalTimeCvAmean}{0.01120699053583089}
\newcommand{\AltValuesVsWithoutCyclesSpeedupBaselineTotalTimeCvGmean}{}
\newcommand{\AltValuesVsWithoutCyclesSpeedupBaselineTotalTimeCvMedian}{0.0076970697836954323}
\newcommand{\AltValuesVsWithoutCyclesSpeedupBaselineTotalTimeCvMin}{0.00079346495313162215}
\newcommand{\AltValuesVsWithoutCyclesSpeedupBaselineTotalTimeCvMax}{0.042017587066423716}
\newcommand{\AltValuesVsWithoutCyclesSpeedupCyclesZeroCenteredSpeedupAmean}{n/a}
\newcommand{\AltValuesVsWithoutCyclesSpeedupCyclesZeroCenteredSpeedupGmean}{n/a}
\newcommand{\AltValuesVsWithoutCyclesSpeedupCyclesZeroCenteredSpeedupMedian}{0.00087355317754968329}
\newcommand{\AltValuesVsWithoutCyclesSpeedupCyclesZeroCenteredSpeedupMin}{-0.0}
\newcommand{\AltValuesVsWithoutCyclesSpeedupCyclesZeroCenteredSpeedupMax}{0.23076923076923078}
\newcommand{\AltValuesVsWithoutCyclesSpeedupCyclesRegularSpeedupAmean}{n/a}
\newcommand{\AltValuesVsWithoutCyclesSpeedupCyclesRegularSpeedupGmean}{1.0446381345032036}
\newcommand{\AltValuesVsWithoutCyclesSpeedupCyclesRegularSpeedupMedian}{1.0008735531775497}
\newcommand{\AltValuesVsWithoutCyclesSpeedupCyclesRegularSpeedupMin}{1.0}
\newcommand{\AltValuesVsWithoutCyclesSpeedupCyclesRegularSpeedupMax}{1.2307692307692308}
\newcommand{\AltValuesVsWithoutCyclesSpeedupCyclesRegularSpeedupCiAmean}{n/a}
\newcommand{\AltValuesVsWithoutCyclesSpeedupCyclesRegularSpeedupCiGmean}{n/a}
\newcommand{\AltValuesVsWithoutCyclesSpeedupCyclesRegularSpeedupCiMedian}{n/a}
\newcommand{\AltValuesVsWithoutCyclesSpeedupCyclesRegularSpeedupCiMin}{1.019425810412065}
\newcommand{\AltValuesVsWithoutCyclesSpeedupCyclesRegularSpeedupCiMax}{1.0727148869986509}


\begin{figure}
  \centering%
  \maxsizebox{\textwidth}{!}{%
    \trimBarchartPlot{%
      \begin{tikzpicture}[gnuplot]
%% generated with GNUPLOT 5.0p4 (Lua 5.2; terminal rev. 99, script rev. 100)
%% ons  3 jan 2018 09:44:02
\path (0.000,0.000) rectangle (12.500,8.750);
\gpcolor{rgb color={0.753,0.753,0.753}}
\gpsetlinetype{gp lt axes}
\gpsetdashtype{gp dt axes}
\gpsetlinewidth{0.50}
\draw[gp path] (1.380,1.687)--(36.945,1.687);
\gpcolor{color=gp lt color border}
\node[gp node right] at (1.380,1.687) {\plotSpeedupTics{0}};
\gpcolor{rgb color={0.753,0.753,0.753}}
\draw[gp path] (1.380,3.026)--(36.945,3.026);
\gpcolor{color=gp lt color border}
\node[gp node right] at (1.380,3.026) {\plotSpeedupTics{0.05}};
\gpcolor{rgb color={0.753,0.753,0.753}}
\draw[gp path] (1.380,4.365)--(36.945,4.365);
\gpcolor{color=gp lt color border}
\node[gp node right] at (1.380,4.365) {\plotSpeedupTics{0.1}};
\gpcolor{rgb color={0.753,0.753,0.753}}
\draw[gp path] (1.380,5.703)--(36.945,5.703);
\gpcolor{color=gp lt color border}
\node[gp node right] at (1.380,5.703) {\plotSpeedupTics{0.15}};
\gpcolor{rgb color={0.753,0.753,0.753}}
\draw[gp path] (1.380,7.042)--(36.945,7.042);
\gpcolor{color=gp lt color border}
\node[gp node right] at (1.380,7.042) {\plotSpeedupTics{0.2}};
\gpcolor{rgb color={0.753,0.753,0.753}}
\draw[gp path] (1.380,8.381)--(36.945,8.381);
\gpcolor{color=gp lt color border}
\node[gp node right] at (1.380,8.381) {\plotSpeedupTics{0.25}};
\node[gp node left,rotate=-30] at (3.110,1.442) {\functionName{bi_reverse}};
\node[gp node left,rotate=-30] at (4.803,1.442) {\functionName{device_color_en.}};
\node[gp node left,rotate=-30] at (6.497,1.442) {\functionName{dict_put_string}};
\node[gp node left,rotate=-30] at (8.190,1.442) {\functionName{ecSub}};
\node[gp node left,rotate=-30] at (9.884,1.442) {\functionName{FORD1}};
\node[gp node left,rotate=-30] at (11.577,1.442) {\functionName{free_tree_nodes}};
\node[gp node left,rotate=-30] at (13.271,1.442) {\functionName{gl_flip_bytes}};
\node[gp node left,rotate=-30] at (14.965,1.442) {\functionName{gluNextContour}};
\node[gp node left,rotate=-30] at (16.658,1.442) {\functionName{gx_color_frac_m.}};
\node[gp node left,rotate=-30] at (18.352,1.442) {\functionName{hash_initial}};
\node[gp node left,rotate=-30] at (20.045,1.442) {\functionName{jinit_huff_deco.}};
\node[gp node left,rotate=-30] at (21.739,1.442) {\functionName{jinit_phuff_dec.}};
\node[gp node left,rotate=-30] at (23.432,1.442) {\functionName{jpeg_alloc_quan.}};
\node[gp node left,rotate=-30] at (25.126,1.442) {\functionName{jpeg_has_multip.}};
\node[gp node left,rotate=-30] at (26.820,1.442) {\functionName{mp_quo_digit}};
\node[gp node left,rotate=-30] at (28.513,1.442) {\functionName{name_ref_sub_ta.}};
\node[gp node left,rotate=-30] at (30.207,1.442) {\functionName{putACfirst}};
\node[gp node left,rotate=-30] at (31.900,1.442) {\functionName{putpicthdr}};
\node[gp node left,rotate=-30] at (33.594,1.442) {\functionName{putseqdispext}};
\node[gp node left,rotate=-30] at (35.287,1.442) {\functionName{reg2rsaref}};
\gpsetlinetype{gp lt border}
\gpsetdashtype{gp dt solid}
\gpsetlinewidth{1.00}
\draw[gp path] (1.380,8.381)--(1.380,1.687)--(36.945,1.687)--(36.945,8.381)--cycle;
\gpcolor{rgb color={0.000,0.000,0.000}}
\draw[gp path] (1.380,1.687)--(1.739,1.687)--(2.098,1.687)--(2.458,1.687)--(2.817,1.687)%
  --(3.176,1.687)--(3.535,1.687)--(3.895,1.687)--(4.254,1.687)--(4.613,1.687)--(4.972,1.687)%
  --(5.332,1.687)--(5.691,1.687)--(6.050,1.687)--(6.409,1.687)--(6.769,1.687)--(7.128,1.687)%
  --(7.487,1.687)--(7.846,1.687)--(8.206,1.687)--(8.565,1.687)--(8.924,1.687)--(9.283,1.687)%
  --(9.643,1.687)--(10.002,1.687)--(10.361,1.687)--(10.720,1.687)--(11.080,1.687)--(11.439,1.687)%
  --(11.798,1.687)--(12.157,1.687)--(12.517,1.687)--(12.876,1.687)--(13.235,1.687)--(13.594,1.687)%
  --(13.953,1.687)--(14.313,1.687)--(14.672,1.687)--(15.031,1.687)--(15.390,1.687)--(15.750,1.687)%
  --(16.109,1.687)--(16.468,1.687)--(16.827,1.687)--(17.187,1.687)--(17.546,1.687)--(17.905,1.687)%
  --(18.264,1.687)--(18.624,1.687)--(18.983,1.687)--(19.342,1.687)--(19.701,1.687)--(20.061,1.687)%
  --(20.420,1.687)--(20.779,1.687)--(21.138,1.687)--(21.498,1.687)--(21.857,1.687)--(22.216,1.687)%
  --(22.575,1.687)--(22.935,1.687)--(23.294,1.687)--(23.653,1.687)--(24.012,1.687)--(24.372,1.687)%
  --(24.731,1.687)--(25.090,1.687)--(25.449,1.687)--(25.808,1.687)--(26.168,1.687)--(26.527,1.687)%
  --(26.886,1.687)--(27.245,1.687)--(27.605,1.687)--(27.964,1.687)--(28.323,1.687)--(28.682,1.687)%
  --(29.042,1.687)--(29.401,1.687)--(29.760,1.687)--(30.119,1.687)--(30.479,1.687)--(30.838,1.687)%
  --(31.197,1.687)--(31.556,1.687)--(31.916,1.687)--(32.275,1.687)--(32.634,1.687)--(32.993,1.687)%
  --(33.353,1.687)--(33.712,1.687)--(34.071,1.687)--(34.430,1.687)--(34.790,1.687)--(35.149,1.687)%
  --(35.508,1.687)--(35.867,1.687)--(36.227,1.687)--(36.586,1.687)--(36.945,1.687);
\gpfill{rgb color={0.333,0.333,0.333}} (6.390,1.687)--(7.097,1.687)--(7.097,4.458)--(6.390,4.458)--cycle;
\gpcolor{color=gp lt color border}
\draw[gp path] (6.390,1.687)--(6.390,4.457)--(7.096,4.457)--(7.096,1.687)--cycle;
\gpfill{rgb color={0.333,0.333,0.333}} (8.084,1.687)--(8.790,1.687)--(8.790,4.663)--(8.084,4.663)--cycle;
\draw[gp path] (8.084,1.687)--(8.084,4.662)--(8.789,4.662)--(8.789,1.687)--cycle;
\gpfill{rgb color={0.333,0.333,0.333}} (9.777,1.687)--(10.484,1.687)--(10.484,1.735)--(9.777,1.735)--cycle;
\draw[gp path] (9.777,1.687)--(9.777,1.734)--(10.483,1.734)--(10.483,1.687)--cycle;
\gpfill{rgb color={0.333,0.333,0.333}} (11.471,1.687)--(12.178,1.687)--(12.178,2.708)--(11.471,2.708)--cycle;
\draw[gp path] (11.471,1.687)--(11.471,2.707)--(12.177,2.707)--(12.177,1.687)--cycle;
\gpfill{rgb color={0.333,0.333,0.333}} (21.632,1.687)--(22.339,1.687)--(22.339,2.230)--(21.632,2.230)--cycle;
\draw[gp path] (21.632,1.687)--(21.632,2.229)--(22.338,2.229)--(22.338,1.687)--cycle;
\gpfill{rgb color={0.333,0.333,0.333}} (23.326,1.687)--(24.033,1.687)--(24.033,5.513)--(23.326,5.513)--cycle;
\draw[gp path] (23.326,1.687)--(23.326,5.512)--(24.032,5.512)--(24.032,1.687)--cycle;
\gpfill{rgb color={0.333,0.333,0.333}} (26.713,1.687)--(27.420,1.687)--(27.420,4.086)--(26.713,4.086)--cycle;
\draw[gp path] (26.713,1.687)--(26.713,4.085)--(27.419,4.085)--(27.419,1.687)--cycle;
\gpfill{rgb color={0.333,0.333,0.333}} (31.794,1.687)--(32.500,1.687)--(32.500,4.289)--(31.794,4.289)--cycle;
\draw[gp path] (31.794,1.687)--(31.794,4.288)--(32.499,4.288)--(32.499,1.687)--cycle;
\gpfill{rgb color={0.333,0.333,0.333}} (33.487,1.687)--(34.194,1.687)--(34.194,7.867)--(33.487,7.867)--cycle;
\draw[gp path] (33.487,1.687)--(33.487,7.866)--(34.193,7.866)--(34.193,1.687)--cycle;
\gpfill{rgb color={0.333,0.333,0.333}} (35.181,1.687)--(35.888,1.687)--(35.888,3.748)--(35.181,3.748)--cycle;
\draw[gp path] (35.181,1.687)--(35.181,3.747)--(35.887,3.747)--(35.887,1.687)--cycle;
\node[gp node center] at (3.350,1.871) {\plotBarValue{-0.000000}};
\node[gp node center] at (5.043,1.871) {\plotBarValue{-0.000000}};
\node[gp node center] at (6.737,4.641) {\plotBarValue{0.103448}};
\node[gp node center] at (8.430,4.846) {\plotBarValue{0.111111}};
\node[gp node center] at (10.124,1.918) {\plotBarValue{0.001747}};
\node[gp node center] at (11.817,2.891) {\plotBarValue{0.038095}};
\node[gp node center] at (13.511,1.871) {\plotBarValue{-0.000000}};
\node[gp node center] at (15.205,1.871) {\plotBarValue{-0.000000}};
\node[gp node center] at (16.898,1.871) {\plotBarValue{-0.000000}};
\node[gp node center] at (18.592,1.871) {\plotBarValue{-0.000000}};
\node[gp node center] at (20.285,1.871) {\plotBarValue{-0.000000}};
\node[gp node center] at (21.979,2.413) {\plotBarValue{0.020236}};
\node[gp node center] at (23.672,5.696) {\plotBarValue{0.142857}};
\node[gp node center] at (25.366,1.871) {\plotBarValue{-0.000000}};
\node[gp node center] at (27.060,4.269) {\plotBarValue{0.089552}};
\node[gp node center] at (28.753,1.871) {\plotBarValue{-0.000000}};
\node[gp node center] at (30.447,1.871) {\plotBarValue{-0.000000}};
\node[gp node center] at (32.140,4.472) {\plotBarValue{0.097147}};
\node[gp node center] at (33.834,8.050) {\plotBarValue{0.230769}};
\node[gp node center] at (35.527,3.931) {\plotBarValue{0.076923}};
\draw[gp path] (1.380,8.381)--(1.380,1.687)--(36.945,1.687)--(36.945,8.381)--cycle;
%% coordinates of the plot area
\gpdefrectangularnode{gp plot 1}{\pgfpoint{1.380cm}{1.687cm}}{\pgfpoint{36.945cm}{8.381cm}}
\end{tikzpicture}
%% gnuplot variables
%
    }%
  }

  \caption[Plot for evaluating value reuse's impact on code quality]%
          {%
            Normalized optimal solution costs for two constraint models: one
            without value reuse support (baseline), and one with such support
            (subject).
            %
            GMI:~\printGMI{%
              \AltValuesVsWithoutCyclesSpeedupCyclesRegularSpeedupGmean%
            },
            CI:~\printGMICI{%
              \AltValuesVsWithoutCyclesSpeedupCyclesRegularSpeedupCiMin%
            }{%
              \AltValuesVsWithoutCyclesSpeedupCyclesRegularSpeedupCiMax%
            }%
          }
  \labelFigure{alt-values-vs-without-cycles-plot}
\end{figure}

\RefFigure{alt-values-vs-without-cycles-plot} shows the normalized
\gls{solution} costs for the two \glspl{constraint model} describe above, with
\glsshort{constraint model}~\refModel{wo-value-reuse} as \gls{baseline} and
\glsshort{constraint model}~\refModel{w-value-reuse} as \gls{subject}.
%
All \glspl{function} are solved to optimality.
%
The costs range from
\printMinCycles{%
  \AltValuesVsWithoutCyclesSpeedupCyclesAvgMin,
  \AltValuesVsWithoutCyclesSpeedupBaselineCyclesAvgMin
} to
\printMaxCycles{%
  \AltValuesVsWithoutCyclesSpeedupCyclesAvgMax,
  \AltValuesVsWithoutCyclesSpeedupBaselineCyclesAvgMax
}.
%
The \gls{GMI} is \printGMI{%
  \AltValuesVsWithoutCyclesSpeedupCyclesRegularSpeedupGmean%
} with \gls{CI}~\printGMICI{%
  \AltValuesVsWithoutCyclesSpeedupCyclesRegularSpeedupCiMin%
}{%
  \AltValuesVsWithoutCyclesSpeedupCyclesRegularSpeedupCiMax%
}.

We see clearly that \glsshort{constraint model}~\refModel{w-value-reuse}
produces \glspl{solution} with significantly lesser cost than those produced by
\glsshort{constraint model}~\refModel{wo-value-reuse}.
%
In one case (\cCode*{putseqdispext}), for example, where two constants are
frequently used as arguments to \gls{function} call \glspl{instruction} and thus
cannot be loaded as immediates, the cycle count is reduced from \num{256}~cycles
to \num{208}~cycles.

Hence we conclude that \gls{value reuse} is essential for reducing the number of
copy \glspl{instruction} and, subsequently, lowering cost.


\paragraph{Conclusions}

Based on the results from this experiment, we conclude:
%
\begin{enumerate*}[label=(\roman*), itemjoin={;\ }, itemjoin*={; and\ }]
  \item that \glspl{alternative value} are superior to \gls{match duplication}
  \item that \gls{value reuse} significantly improves code quality
\end{enumerate*}.


\section{Modeling Block Ordering}
\labelSection{modeling-block-ordering}

Ordering the \glspl{block} in a \gls{function} entails finding a sequence~$s$
such that each \gls{block} appears exactly once in~$s$\hspace{-.8pt}.
%
Depending on the control-flow \glspl{instruction} selected, some \glspl{block}
may need to be adjacent.
%
For example, assume a \gls{block}~$b$ that branches to either of two
\glspl{block}~$c$ and~$d$ depending on whether a condition holds.
%
Assume also that the conditional branching in $b$ is implemented using an
\gls{instruction} that branches to $c$ if the condition holds, otherwise it
continues the execution with the next \gls{instruction} in the \gls{assembly
  code}.
%
This notion of execution is called \gls!{fall-through}, and due to this, $d$
must be placed immediately after $b$ in the sequence.

For some combinations of \glspl{function} and \glspl{target machine} with
\gls{fall-through} \glspl{instruction}, there exist no valid \gls{block}
sequence without inserting one or more additional jump \glspl{instruction}.
%
\begin{figure}
  \setlength{\opNodeDist}{20pt}%
  \tikzset{
    block node/.append style={
      minimum width=2cm,
      minimum height=.9\opNodeSize,
    },
  }%

  \mbox{}%
  \hfill%
  \subcaptionbox{Control-flow graph\labelFigure{jump-insert-example-cfg}}%
                [34mm]%
                {%
                  % Copyright (c) 2017-2018, Gabriel Hjort Blindell <ghb@kth.se>
%
% This work is licensed under a Creative Commons Attribution-NoDerivatives 4.0
% International License (see LICENSE file or visit
% <http://creativecommons.org/licenses/by-nc-nd/4.0/> for details).
%
\begingroup%
\pgfdeclarelayer{background}%
\pgfsetlayers{background,main}%
\begin{tikzpicture}
  \node [block node] (A)
        {%
          \begin{tabular}{l}
            \instrFont br x, B
          \end{tabular}%
        };
  \node [block node, below=of A] (B)
        {%
          \begin{tabular}{l}
            \instrFont br y, B
          \end{tabular}%
        };
  \node [block node, below=of B] (C) {};

  \begin{pgfonlayer}{background}
    \begin{scope}[control flow]
      \draw (A)
            -- node [control-flow label, swap] {T}
            (B);
      \draw (B)
            -- node [control-flow label, swap] {F}
            (C);
      \draw [rounded corners=3pt]
            (A)
            -| node [control-flow label, pos=.25,
                     inner sep=.5\controlFlowLabelXSep] {F}
            ([xshift=\nodeDist] B.east)
            |-
            (C);
      \draw [rounded corners=3pt]
            ($(B.south) !.5! (B.south east)$)
            --
            +(-90:.33\nodeDist)
            -| node [control-flow label, pos=.25, swap,
                     inner sep=.5\controlFlowLabelXSep] {T}
            ([xshift=.5\nodeDist] B.east)
            |-
            ([yshift=.33\nodeDist] B.north east)
            -|
            ($(B.north) !.5! (B.north east)$);
    \end{scope}
  \end{pgfonlayer}

  \foreach \b in {A, B, C} {
    \node [block label, above right=0 and 0 of \b.north west] {\b};
  }
\end{tikzpicture}%
\endgroup%
%
                }%
  \hfill%
  \subcaptionbox{%
                  Valid block sequences, after jump insertion%
                  \labelFigure{jump-insert-example-solutions}%
                }%
                [65mm]%
                {%
                  % Copyright (c) 2018, Gabriel Hjort Blindell <ghb@kth.se>
%
% This work is licensed under a Creative Commons 4.0 International License (see
% LICENSE file or visit <http://creativecommons.org/licenses/by/4.0/> for a copy
% of the license).
%
\begingroup%
\pgfdeclarelayer{background}%
\pgfsetlayers{background,main}%
\begin{tikzpicture}[
    block node/.append style={
      node distance=0,
    },
  ]

  \node [block node] (A)
        {%
          \begin{tabular}{l}
            \instrFont br x, B\\
            \instrFont br C\\
          \end{tabular}%
        };
  \node [block node, below=of A] (B)
        {%
          \begin{tabular}{l}
            \instrFont br y, B
          \end{tabular}%
        };
  \node [block node, below=of B] (C) {};

  \node [block label, below left=0 and 2pt of A.north west] {A};
  \foreach \b in {B, C} {
    \node [block label, left=2pt of \b.west] {\b};
  }
\end{tikzpicture}%
\endgroup%
%
                  \hspace{6mm}%
                  % Copyright (c) 2017, Gabriel Hjort Blindell <ghb@kth.se>
%
% This work is licensed under a Creative Commons 4.0 International License (see
% LICENSE file or visit <http://creativecommons.org/licenses/by/4.0/> for a copy
% of the license).
%
\begingroup%
\pgfdeclarelayer{background}%
\pgfsetlayers{background,main}%
\begin{tikzpicture}[
    block node/.append style={
      node distance=0,
    },
  ]

  \node [block node] (B)
        {%
          \begin{tabular}{l}
            \instrFont br y, B\\
            \instrFont br C\\
          \end{tabular}%
        };
  \node [block node, below=of B] (A)
        {%
          \begin{tabular}{l}
            \instrFont br x, B
          \end{tabular}%
        };
  \node [block node, below=of A] (C) {};

  \node [block label, below left=0 and 2pt of B.north west] {B};
  \foreach \b in {A, C} {
    \node [block label, left=2pt of \b.west] {\b};
  }
\end{tikzpicture}%
\endgroup%
%
                }%
  \hfill%
  \mbox{}

  \caption[Example that requires additional jump instructions]%
          {%
            Example that requires additional jump instructions.
            %
            It is assumed that the conditional {\instrFont br}~instruction
            falls through to the next instruction if the condition is false%
          }
  \labelFigure{jump-insert-example}
\end{figure}
%
See for example \refFigure{jump-insert-example}.
%
\Glspl{block}~\irBlock*{A} and~\irBlock*{B} both contain control-flow
\glspl{instruction} that branches to the beginning of \irBlock*{B} if the
condition holds, otherwise they should branches to block~\irBlock*{C}
(\refFigure{jump-insert-example-cfg}).
%
Because of the fall-through \gls{constraint}, \irBlock*{A} and \irBlock*{B}
cannot both have \irBlock*{C} as its successor \gls{block}.
%
Hence an additional jump \gls{instruction} that directly branches to
\irBlock*{C} must be inserted after the control-flow \gls{instruction} in either
\irBlock*{A} or~\irBlock*{B} (\refFigure{jump-insert-example-solutions}).

In this dissertation, we discuss two methods for inserting jump
\glspl{instruction} when required: \gls{branch extension} and \glspllong{DTB
  pattern}.
%
We first introduce the \glspl{variable} and \glspl{constraint} for modeling
\gls{block ordering} before introducing each method in turn, and then present
experiments showing that one is superior to the other.


\paragraph{Variables}

The set of \glspl{variable} \mbox{$\mVar{succ}[b] \in \mBlockSet$} models the
successor of block~$b$\hspace{-.5pt}.
%
For example, if \mbox{$\mVar{succ}[b] = b'$}, then \gls{block}~$b'$ will appear
immediately after \gls{block}~$b$ in the \gls{block ordering} sequence.


\paragraph{Constraints}

A \gls{solution} to the \gls{block ordering} problem is a sequence of
\gls{block} successors such that they form a \gls{cycle}.
%
Using the \gls{circuit constraint} defined in
\refChapter{constraint-programming} on \refPageOfDefinition{circuit}, this
\gls{constraint} is modeled as
%
\begin{equation}
  \mCircuit(\mVar{succ}[b_1], \ldots, \mVar{succ}[b_n]),
  \labelEquation{block-order}
\end{equation}
%
where \mbox{$b_1, \ldots, b_n = \mBlockSet$}.

If a \gls{match}~$m$ with an \gls{entry block} is derived from an
\gls{instruction} that performs a \gls{fall-through} to \gls{block}~$b$, then
the \gls{constraint} can naively be modeled as \mbox{$\mVar{sel}[m] \mImp
  \mVar{succ}[\mEntry(m)] = b$}.
%
However, this \gls{constraint} is too limiting as it disallows empty
\glspl{block} to be placed between $\mEntry(m)$ and~$b$\hspace{-.5pt}, thus
forcing redundant jump \glspl{instruction} to be emitted.
%
A \gls{block}~$b$ is considered empty if either no \glspl{match} are placed in
$b$ or every \gls{match} in $b$ is a \gls!{null match}, which is a \gls{match}
that emits nothing if selected.
%
As empty \glspl{block} are not uncommon to appear in the \gls{function} under
compilation -- especially when having performed \gls{global code motion} -- this
has a significant impact on code quality.
%
Hence we extend the naive implementation into a disjunction, where the second
clause captures \glspl{fall-through} via single empty \glspl{block}.
%
Let $\mFallThroughSet$ denote a set of pairs~$\mPair{m}{b}$, where $m$ is a
\gls{match} and $b$ is a \gls{block} through which $m$ will fall if selected,
and let $\mNullMatchSet$ denote the set of \glspl{null match}.
%
With these definitions, the \gls{fall-through} \gls{constraint} is modeled as
%
\begin{equation}
  \begin{array}{c}
    \forall \mPair{m}{b} \in \mFallThroughSet :
    \mVar{sel}[m]
    \mImp
    \mVar{succ}[\mEntry(m)] = b \mOr \mbox{} \\
    \big(
      \mVar{succ}[\mVar{succ}[\mEntry(m)]] = b
      \mAnd
      \mEmptyBlock(\mVar{succ}[\mEntry(m)])
    \big),
  \end{array}
  \labelEquation{fall-through}
\end{equation}
%
where
%
\begin{equation}
  \mEmptyBlock(b)
  \equiv
  \mBigAnd_{\mathclap{o \,\in\, \mOpSet}}
  (
  \mVar{oplace}[o] \neq b
  \mOr
  \mVar{omatch}[o] \in \mNullMatchSet
  ).
\end{equation}

If a \gls{block}~$b$ unconditionally branches to another \gls{block}~$b'$ and
$b'$ appears immediately after $b$ in the \gls{block} sequence, then a jump
\gls{instruction} is redundant.
%
To prevent emission of such jump \glspl{instruction}, we extend the \gls{pattern
  set} with a special \gls!{null-jump pattern}, with \gls{graph} structure
\mbox{$\mEdge{b}{\mEdge{c}{b'}}$}\hspace{-2pt}, that covers a \gls{control
  node}~$c$ at zero cost provided that \mbox{$\mVar{succ}[b] =
  b'$}\hspace{-2pt}.

\Glspl{fall-through} to or via the \gls{function}'s \gls{entry block} is never
allowed since that \gls{block} must always be placed first in the \gls{block}
sequence.
%
If $\mFunctionEntryBlock$ denotes the \gls{function}'s \gls{entry block}, then
this \gls{constraint} is modeled as
%
\begin{equation}
  \forall \mPair{m}{\cdot} \in \mFallThroughSet :
  \mVar{sel}[m]
  \mImp
  \mVar{succ}[\mEntry(m)] \neq \mFunctionEntryBlock.
  \labelEquation{no-fall-through-to-fun-entry}
\end{equation}


\subsection{Branch Extension}

One method of inserting jump \glspl{instruction} is to extend the \gls{UF graph}
with additional \glsshort{block node} and \glspl{control node}.
%
The idea is as follows.
%
For each \gls{control-flow edge}~$\mPair{c}{b}$, where $c$ is a \gls{control
  node} and $b$ is a \gls{block node}, we remove this \gls{edge} and insert a
new \gls{block node}~$b'$ and \gls{control node}~$c'$, and \glspl{control-flow
  edge} such that \mbox{$\mEdge{c}{\mEdge{b'}{\mEdge{c'}{b}}}$}.
%
If the new \glspl{control node} are indeed redundant, then each such \gls{node}
can be covered by a \gls{match} derived from the \gls{null-jump pattern},
causing the new \glspl{block} to become empty and appear immediately before the
target \gls{block}.
%
Like with \gls{copy extension}, to retain matching \gls{branch extension} must
also be performed on each \gls{UF graph} in the \gls{pattern set}.
%
\begin{figure}
  \mbox{}%
  \hfill%
  \subcaptionbox{%
                  Original UF subgraph%
                  \labelFigure{branch-extension-example-original}%
                }%
                [35mm]%
                {%
                  \input{%
                    figures/constraint-model/branch-extension-example-before%
                  }%
                }%
  \hfill%
  \subcaptionbox{%
                  After branch extension%
                  \labelFigure{branch-extension-example-after}%
                }%
                [40mm]%
                {%
                  \input{%
                    figures/constraint-model/branch-extension-example-after%
                  }%
                }%
  \hfill%
  \mbox{}

  \caption{Example of branch extension}%
  \labelFigure{branch-extension-example}
\end{figure}
%
An example is shown in \refFigure{branch-extension-example}.

The disadvantage of \gls{branch extension} is that it inflates the \gls{search
  space}.
%
The number of \glsshort{block node} and \glspl{control node} both increase by
$\mBigO(nk)$, where $n$ is the number of \glspl{block} before \gls{branch
  extension} and $k$ is the highest number of \gls{outbound.e}
\glspl{control-flow edge} from a \gls{control node} in the \gls{UF graph}.
%
This leads to more \glspl{operation} to be covered and more \glspl{block}
wherein an \gls{operation} may be placed.
%
In addition, as the majority of the new \glspl{block} will be empty, situations
often arise where a control-flow \gls{instruction} could successfully fall
through more than one \gls{block}.
%
Because of \refEquation{fall-through}, however, it can only fall through at most
one empty \gls{block}, causing emission of redundant jump \glspl{instruction}
that would not have been emitted had \gls{branch extension} not been performed.


\subsection{Dual-target Branch Patterns}

\def\jmpPattern{p_{\textsc{jmp}}}%

\glsreset{DTB pattern}

Another method is to extend the \gls{pattern set} with so-called \glspl!{DTB
  pattern}.
%
Given a \gls{pattern set}~$S$, first find the \gls{pattern}~\mbox{$\jmpPattern
  \in S$} which corresponds to a unconditional jump \gls{instruction} that
directly branches to a given label (it is reasonable to assume such a
\gls{pattern} always exist for any given \gls{target machine}).
%
Let \mbox{$\mCost(p)$} and \mbox{$\mEmits(p)$} denote the cost of pattern~$p$
respectively the sequence of \glspl{instruction} emitted by $p$ if selected.
%
Then, for each \gls{pattern}~\mbox{$p \in S$} which corresponds to a conditional
jump \gls{instruction} that falls through to a given block~$b$, add to $S$ a new
\gls{pattern}~$p'$ that is a copy of $p$ but has no \gls{fall-through}
\gls{constraint}, emits $\mEmits(p)$ followed by $\mEmits(\jmpPattern)$, and has
cost \mbox{$\mCost(p) + \mCost(\jmpPattern)$}.
%
\begin{figure}
  \subcaptionbox{%
                  A pattern that falls through to the \irBlock*{false} block%
                  \labelFigure{dual-target-branch-pattern-example-original}%
                }%
                [68mm]%
                {%
                  \begin{tabular}{c}
                    \input{%
                      figures/constraint-model/%
                      dual-target-branch-pattern-example%
                    }\\[.5\betweensubfigures]
                    \figureFont\figureFontSize%
                    \begin{tabular}{lc}
                      \toprule
                        \multicolumn{1}{c}{\tabhead emit} & \tabhead cost\\
                      \midrule
                        {\instrFont br b, true} & $2$\\
                      \bottomrule
                    \end{tabular}%
                  \end{tabular}%
                }%
  \hfill%
  \subcaptionbox{%
                  New pattern, without fall-through%
                  \labelFigure{dual-target-branch-pattern-example-copy}%
                }%
                [52mm]%
                {%
                  \begin{tabular}{c}
                    \newcommand{\fallthruString}{}%
                    \input{%
                      figures/constraint-model/%
                      dual-target-branch-pattern-example%
                    }\\[.5\betweensubfigures]
                    \figureFont\figureFontSize%
                    \begin{tabular}{lc}
                      \toprule
                        \multicolumn{1}{c}{\tabhead emit} & \tabhead cost\\
                      \midrule
                        {\instrFont br b, true}
                      & $2 + \mCost\hspace{.4pt}(\jmpPattern)$\\
                        $\mEmits(\jmpPattern)$
                      & \\
                      \bottomrule
                    \end{tabular}%
                  \end{tabular}%
                }

  \caption{Example of creating a DTB pattern}%
  \labelFigure{dual-target-branch-pattern-example}
\end{figure}
%
An example is shown in \refFigure{dual-target-branch-pattern-example}.
%
Because a \gls{DTB pattern} has no \gls{fall-through} \gls{constraint}, it
essentially models a conditional jump \gls{instruction} capable of directly
branching to two \glspl{block} (hence the name).

Consequently, if a \gls{pattern set} contains $k$~\glspl{pattern} with
\gls{fall-through} \glspl{constraint} and a \gls{UF graph} contains
$n$~\glspl{control node} representing conditional jumps, then using \glspl{DTB
  pattern} will enlarge the \gls{match set} by $\mBigO(nk)$~\glspl{match}.
%
Unlike \gls{branch extension}, however, the \gls{UF graph} does not need to be
extended with additional \glspl{block} wherein \glspl{operation} may be placed,
which results in a significantly smaller \gls{search space}.


\subsection{Experimental Evaluation}

We evaluate the different methods for inserting jump \glspl{instruction} by
comparing the solving times exhibited and the cost of the optimal
\glspl{solution} produced by two versions of the \gls{constraint model}:
%
\begin{modelList}
  \item \labelModel{branch-ext}
    one based on \gls{branch extension}
  \item \labelModel{dual-target}
    one based on \glspl{DTB pattern}
\end{modelList}.
%
Since \gls{branch extension} yields a larger number of \glspl{block} compared to
\glspl{DTB pattern}, many of which will be empty and thereby causing emission of
redundant jump \glspl{instruction}, we expect \glsshort{constraint
  model}~\refModel{dual-target} perform better, both in terms of solving time
and cost, than \glsshort{constraint model}~\refModel{branch-ext}.

When filtering, we remove all \glspl{function} that have less than less than
\num{20}~\gls{LLVM} \gls{IR} \glspl{instruction} -- anything smaller will most
likely not require any additional jump \glspl{instruction} -- and greater than
\num{100}~\glspl{instruction} -- anything larger will lead to needlessly long
experiment runtimes.
%
This leaves a pool of \num{413}~\glspl{function}, on which we then perform
sampling.

When clustering, we replace the number of memory \glspl{instruction} as feature
with the number of \glspl{block}.
%
This is to evaluate how the methods behave as the number of \glspl{block} grow
larger.


\paragraph{Impact on Solving Time}

\input{%
  \expDir/dual-target-patterns-vs-branch-ext-pre+solving-time-speedup.stats%
}

\begin{figure}
  \centering%
  \maxsizebox{\textwidth}{!}{%
    \trimBarchartPlot{%
      \input{%
        \expDir/%
        dual-target-patterns-vs-branch-ext-pre+solving-time-speedup.plot%
      }%
    }%
  }

  \caption[%
            Plot comparing solving times for two constraint models supporting
            jump instruction insertion%
          ]%
          {%
            Normalized solving times (incl.\ presolving time) for two
            constraint models that supports jump insertion: one based on branch
            extension (baseline), and one based on DTB patterns (subject).
            %
            GMI:~\printGMI{%
              \DualTargetPatternsVsBranchExtPrePlusSolvingTimeSpeedupPrePlusSolvingTimeRegularSpeedupGmean%
            },
            CI~\printGMICI{%
              \DualTargetPatternsVsBranchExtPrePlusSolvingTimeSpeedupPrePlusSolvingTimeRegularSpeedupCiMin%
            }{%
              \DualTargetPatternsVsBranchExtPrePlusSolvingTimeSpeedupPrePlusSolvingTimeRegularSpeedupCiMax%
            }%
          }
  \labelFigure{%
    dual-target-branch-patterns-vs-branch-extension-solving-time-plot%
  }
\end{figure}

\RefFigure{dual-target-branch-patterns-vs-branch-extension-solving-time-plot}
shows the normalized solving times (including \gls{presolving} time) for the two
\glspl{constraint model} described above, with \glsshort{constraint
  model}~\refModel{branch-ext} as \gls{baseline} and \glsshort{constraint
  model}~\refModel{dual-target} as \gls{subject}.
%
All \glspl{function} are solved to optimality and arranged in increasing order
of number of conditional branch \glspl{instruction}.
%
The solving times range from
\printMinSolvingTime{%
  \DualTargetPatternsVsBranchExtPrePlusSolvingTimeSpeedupSolvingTimeAvgMin,
  \DualTargetPatternsVsBranchExtPrePlusSolvingTimeSpeedupBaselineSolvingTimeAvgMin
}
to
\printMaxSolvingTime{%
  \DualTargetPatternsVsBranchExtPrePlusSolvingTimeSpeedupSolvingTimeAvgMax,
  \DualTargetPatternsVsBranchExtPrePlusSolvingTimeSpeedupBaselineSolvingTimeAvgMax
}
with a \gls{CV} of
\numMaxOf{%
  \DualTargetPatternsVsBranchExtPrePlusSolvingTimeSpeedupSolvingTimeCvMax,
  \DualTargetPatternsVsBranchExtPrePlusSolvingTimeSpeedupBaselineSolvingTimeCvMax
}.
The \gls{GMI} is \printGMI{%
  \DualTargetPatternsVsBranchExtPrePlusSolvingTimeSpeedupPrePlusSolvingTimeRegularSpeedupGmean%
} with \gls{CI}~\printGMICI{%
  \DualTargetPatternsVsBranchExtPrePlusSolvingTimeSpeedupPrePlusSolvingTimeRegularSpeedupCiMin%
}{%
  \DualTargetPatternsVsBranchExtPrePlusSolvingTimeSpeedupPrePlusSolvingTimeRegularSpeedupCiMax%
}.

We see clearly that \glsshort{constraint model}~\refModel{dual-target} results
in significantly shorter solving times than \glsshort{constraint
  model}~\refModel{branch-ext}.
%
We also observe that, as expected, when the number of conditional branch
\glspl{instruction} increases -- up to \num{13}, in the case of
\cCode*{nextkeypacket} -- the \gls{search space} for \glsshort{constraint
  model}~\refModel{branch-ext} grows faster than for \glsshort{constraint
  model}~\refModel{dual-target}.
%
Hence we conclude that, in terms of solving time,
\glspl{DTB pattern} is a better design choice over \gls{branch extension} when
implementing insertion of jump \glspl{instruction}.


\paragraph{Impact on Code Quality}

\newcommand{\DualTargetPatternsVsBranchExtCyclesSpeedupSimpleNameAmean}{}
\newcommand{\DualTargetPatternsVsBranchExtCyclesSpeedupSimpleNameGmean}{}
\newcommand{\DualTargetPatternsVsBranchExtCyclesSpeedupSimpleNameMedian}{}
\newcommand{\DualTargetPatternsVsBranchExtCyclesSpeedupSimpleNameMin}{}
\newcommand{\DualTargetPatternsVsBranchExtCyclesSpeedupSimpleNameMax}{}
\newcommand{\DualTargetPatternsVsBranchExtCyclesSpeedupSolutionFoundAvgAmean}{1.0}
\newcommand{\DualTargetPatternsVsBranchExtCyclesSpeedupSolutionFoundAvgGmean}{}
\newcommand{\DualTargetPatternsVsBranchExtCyclesSpeedupSolutionFoundAvgMedian}{1.0}
\newcommand{\DualTargetPatternsVsBranchExtCyclesSpeedupSolutionFoundAvgMin}{1.0}
\newcommand{\DualTargetPatternsVsBranchExtCyclesSpeedupSolutionFoundAvgMax}{1.0}
\newcommand{\DualTargetPatternsVsBranchExtCyclesSpeedupCyclesAvgAmean}{54219.349999999999}
\newcommand{\DualTargetPatternsVsBranchExtCyclesSpeedupCyclesAvgGmean}{}
\newcommand{\DualTargetPatternsVsBranchExtCyclesSpeedupCyclesAvgMedian}{8921.5}
\newcommand{\DualTargetPatternsVsBranchExtCyclesSpeedupCyclesAvgMin}{1088.0}
\newcommand{\DualTargetPatternsVsBranchExtCyclesSpeedupCyclesAvgMax}{266571.0}
\newcommand{\DualTargetPatternsVsBranchExtCyclesSpeedupOptimalAvgAmean}{0.90000000000000002}
\newcommand{\DualTargetPatternsVsBranchExtCyclesSpeedupOptimalAvgGmean}{}
\newcommand{\DualTargetPatternsVsBranchExtCyclesSpeedupOptimalAvgMedian}{1.0}
\newcommand{\DualTargetPatternsVsBranchExtCyclesSpeedupOptimalAvgMin}{0.0}
\newcommand{\DualTargetPatternsVsBranchExtCyclesSpeedupOptimalAvgMax}{1.0}
\newcommand{\DualTargetPatternsVsBranchExtCyclesSpeedupMatchingTimeAvgAmean}{4.4343151205}
\newcommand{\DualTargetPatternsVsBranchExtCyclesSpeedupMatchingTimeAvgGmean}{}
\newcommand{\DualTargetPatternsVsBranchExtCyclesSpeedupMatchingTimeAvgMedian}{2.3663127214999999}
\newcommand{\DualTargetPatternsVsBranchExtCyclesSpeedupMatchingTimeAvgMin}{0.63699954800000003}
\newcommand{\DualTargetPatternsVsBranchExtCyclesSpeedupMatchingTimeAvgMax}{29.715492472000001}
\newcommand{\DualTargetPatternsVsBranchExtCyclesSpeedupLbCompTimeAvgAmean}{0.0}
\newcommand{\DualTargetPatternsVsBranchExtCyclesSpeedupLbCompTimeAvgGmean}{}
\newcommand{\DualTargetPatternsVsBranchExtCyclesSpeedupLbCompTimeAvgMedian}{0.0}
\newcommand{\DualTargetPatternsVsBranchExtCyclesSpeedupLbCompTimeAvgMin}{0.0}
\newcommand{\DualTargetPatternsVsBranchExtCyclesSpeedupLbCompTimeAvgMax}{0.0}
\newcommand{\DualTargetPatternsVsBranchExtCyclesSpeedupDomProcTimeAvgAmean}{0.70055143833160405}
\newcommand{\DualTargetPatternsVsBranchExtCyclesSpeedupDomProcTimeAvgGmean}{}
\newcommand{\DualTargetPatternsVsBranchExtCyclesSpeedupDomProcTimeAvgMedian}{0.18057847023010254}
\newcommand{\DualTargetPatternsVsBranchExtCyclesSpeedupDomProcTimeAvgMin}{0.041315078735351562}
\newcommand{\DualTargetPatternsVsBranchExtCyclesSpeedupDomProcTimeAvgMax}{9.1429851055145264}
\newcommand{\DualTargetPatternsVsBranchExtCyclesSpeedupIllProcTimeAvgAmean}{11.966792607307434}
\newcommand{\DualTargetPatternsVsBranchExtCyclesSpeedupIllProcTimeAvgGmean}{}
\newcommand{\DualTargetPatternsVsBranchExtCyclesSpeedupIllProcTimeAvgMedian}{1.4708490371704102}
\newcommand{\DualTargetPatternsVsBranchExtCyclesSpeedupIllProcTimeAvgMin}{0.25621414184570312}
\newcommand{\DualTargetPatternsVsBranchExtCyclesSpeedupIllProcTimeAvgMax}{194.81219601631165}
\newcommand{\DualTargetPatternsVsBranchExtCyclesSpeedupRedunProcTimeAvgAmean}{0.42463645935058592}
\newcommand{\DualTargetPatternsVsBranchExtCyclesSpeedupRedunProcTimeAvgGmean}{}
\newcommand{\DualTargetPatternsVsBranchExtCyclesSpeedupRedunProcTimeAvgMedian}{0.13723254203796387}
\newcommand{\DualTargetPatternsVsBranchExtCyclesSpeedupRedunProcTimeAvgMin}{0.037169933319091797}
\newcommand{\DualTargetPatternsVsBranchExtCyclesSpeedupRedunProcTimeAvgMax}{4.9268779754638672}
\newcommand{\DualTargetPatternsVsBranchExtCyclesSpeedupModelPrepTimeAvgAmean}{70.984499999999997}
\newcommand{\DualTargetPatternsVsBranchExtCyclesSpeedupModelPrepTimeAvgGmean}{}
\newcommand{\DualTargetPatternsVsBranchExtCyclesSpeedupModelPrepTimeAvgMedian}{26.774999999999999}
\newcommand{\DualTargetPatternsVsBranchExtCyclesSpeedupModelPrepTimeAvgMin}{7.4000000000000004}
\newcommand{\DualTargetPatternsVsBranchExtCyclesSpeedupModelPrepTimeAvgMax}{640.09000000000003}
\newcommand{\DualTargetPatternsVsBranchExtCyclesSpeedupSolvingTimeAvgAmean}{105.29749999999999}
\newcommand{\DualTargetPatternsVsBranchExtCyclesSpeedupSolvingTimeAvgGmean}{}
\newcommand{\DualTargetPatternsVsBranchExtCyclesSpeedupSolvingTimeAvgMedian}{13.345000000000001}
\newcommand{\DualTargetPatternsVsBranchExtCyclesSpeedupSolvingTimeAvgMin}{0.34999999999999998}
\newcommand{\DualTargetPatternsVsBranchExtCyclesSpeedupSolvingTimeAvgMax}{621.26999999999998}
\newcommand{\DualTargetPatternsVsBranchExtCyclesSpeedupPrePlusSolvingTimeAvgAmean}{118.38948050498962}
\newcommand{\DualTargetPatternsVsBranchExtCyclesSpeedupPrePlusSolvingTimeAvgGmean}{}
\newcommand{\DualTargetPatternsVsBranchExtCyclesSpeedupPrePlusSolvingTimeAvgMedian}{17.132939009666444}
\newcommand{\DualTargetPatternsVsBranchExtCyclesSpeedupPrePlusSolvingTimeAvgMin}{0.77924690246582029}
\newcommand{\DualTargetPatternsVsBranchExtCyclesSpeedupPrePlusSolvingTimeAvgMax}{817.30205909729}
\newcommand{\DualTargetPatternsVsBranchExtCyclesSpeedupTotalTimeAvgAmean}{122.82379562548962}
\newcommand{\DualTargetPatternsVsBranchExtCyclesSpeedupTotalTimeAvgGmean}{}
\newcommand{\DualTargetPatternsVsBranchExtCyclesSpeedupTotalTimeAvgMedian}{21.735861468659671}
\newcommand{\DualTargetPatternsVsBranchExtCyclesSpeedupTotalTimeAvgMin}{1.4162464504658203}
\newcommand{\DualTargetPatternsVsBranchExtCyclesSpeedupTotalTimeAvgMax}{847.01755156928994}
\newcommand{\DualTargetPatternsVsBranchExtCyclesSpeedupLbCompTimeCvAmean}{0.0}
\newcommand{\DualTargetPatternsVsBranchExtCyclesSpeedupLbCompTimeCvGmean}{}
\newcommand{\DualTargetPatternsVsBranchExtCyclesSpeedupLbCompTimeCvMedian}{0.0}
\newcommand{\DualTargetPatternsVsBranchExtCyclesSpeedupLbCompTimeCvMin}{0.0}
\newcommand{\DualTargetPatternsVsBranchExtCyclesSpeedupLbCompTimeCvMax}{0.0}
\newcommand{\DualTargetPatternsVsBranchExtCyclesSpeedupDomProcTimeCvAmean}{0.0}
\newcommand{\DualTargetPatternsVsBranchExtCyclesSpeedupDomProcTimeCvGmean}{}
\newcommand{\DualTargetPatternsVsBranchExtCyclesSpeedupDomProcTimeCvMedian}{0.0}
\newcommand{\DualTargetPatternsVsBranchExtCyclesSpeedupDomProcTimeCvMin}{0.0}
\newcommand{\DualTargetPatternsVsBranchExtCyclesSpeedupDomProcTimeCvMax}{0.0}
\newcommand{\DualTargetPatternsVsBranchExtCyclesSpeedupIllProcTimeCvAmean}{0.0}
\newcommand{\DualTargetPatternsVsBranchExtCyclesSpeedupIllProcTimeCvGmean}{}
\newcommand{\DualTargetPatternsVsBranchExtCyclesSpeedupIllProcTimeCvMedian}{0.0}
\newcommand{\DualTargetPatternsVsBranchExtCyclesSpeedupIllProcTimeCvMin}{0.0}
\newcommand{\DualTargetPatternsVsBranchExtCyclesSpeedupIllProcTimeCvMax}{0.0}
\newcommand{\DualTargetPatternsVsBranchExtCyclesSpeedupRedunProcTimeCvAmean}{0.0}
\newcommand{\DualTargetPatternsVsBranchExtCyclesSpeedupRedunProcTimeCvGmean}{}
\newcommand{\DualTargetPatternsVsBranchExtCyclesSpeedupRedunProcTimeCvMedian}{0.0}
\newcommand{\DualTargetPatternsVsBranchExtCyclesSpeedupRedunProcTimeCvMin}{0.0}
\newcommand{\DualTargetPatternsVsBranchExtCyclesSpeedupRedunProcTimeCvMax}{0.0}
\newcommand{\DualTargetPatternsVsBranchExtCyclesSpeedupModelPrepTimeCvAmean}{0.0}
\newcommand{\DualTargetPatternsVsBranchExtCyclesSpeedupModelPrepTimeCvGmean}{}
\newcommand{\DualTargetPatternsVsBranchExtCyclesSpeedupModelPrepTimeCvMedian}{0.0}
\newcommand{\DualTargetPatternsVsBranchExtCyclesSpeedupModelPrepTimeCvMin}{0.0}
\newcommand{\DualTargetPatternsVsBranchExtCyclesSpeedupModelPrepTimeCvMax}{0.0}
\newcommand{\DualTargetPatternsVsBranchExtCyclesSpeedupSolvingTimeCvAmean}{0.0}
\newcommand{\DualTargetPatternsVsBranchExtCyclesSpeedupSolvingTimeCvGmean}{}
\newcommand{\DualTargetPatternsVsBranchExtCyclesSpeedupSolvingTimeCvMedian}{0.0}
\newcommand{\DualTargetPatternsVsBranchExtCyclesSpeedupSolvingTimeCvMin}{0.0}
\newcommand{\DualTargetPatternsVsBranchExtCyclesSpeedupSolvingTimeCvMax}{0.0}
\newcommand{\DualTargetPatternsVsBranchExtCyclesSpeedupPrePlusSolvingTimeCvAmean}{0.0}
\newcommand{\DualTargetPatternsVsBranchExtCyclesSpeedupPrePlusSolvingTimeCvGmean}{}
\newcommand{\DualTargetPatternsVsBranchExtCyclesSpeedupPrePlusSolvingTimeCvMedian}{0.0}
\newcommand{\DualTargetPatternsVsBranchExtCyclesSpeedupPrePlusSolvingTimeCvMin}{0.0}
\newcommand{\DualTargetPatternsVsBranchExtCyclesSpeedupPrePlusSolvingTimeCvMax}{0.0}
\newcommand{\DualTargetPatternsVsBranchExtCyclesSpeedupTotalTimeCvAmean}{0.0}
\newcommand{\DualTargetPatternsVsBranchExtCyclesSpeedupTotalTimeCvGmean}{}
\newcommand{\DualTargetPatternsVsBranchExtCyclesSpeedupTotalTimeCvMedian}{0.0}
\newcommand{\DualTargetPatternsVsBranchExtCyclesSpeedupTotalTimeCvMin}{0.0}
\newcommand{\DualTargetPatternsVsBranchExtCyclesSpeedupTotalTimeCvMax}{0.0}
\newcommand{\DualTargetPatternsVsBranchExtCyclesSpeedupBaselineSimpleNameAmean}{}
\newcommand{\DualTargetPatternsVsBranchExtCyclesSpeedupBaselineSimpleNameGmean}{}
\newcommand{\DualTargetPatternsVsBranchExtCyclesSpeedupBaselineSimpleNameMedian}{}
\newcommand{\DualTargetPatternsVsBranchExtCyclesSpeedupBaselineSimpleNameMin}{}
\newcommand{\DualTargetPatternsVsBranchExtCyclesSpeedupBaselineSimpleNameMax}{}
\newcommand{\DualTargetPatternsVsBranchExtCyclesSpeedupBaselineSolutionFoundAvgAmean}{1.0}
\newcommand{\DualTargetPatternsVsBranchExtCyclesSpeedupBaselineSolutionFoundAvgGmean}{}
\newcommand{\DualTargetPatternsVsBranchExtCyclesSpeedupBaselineSolutionFoundAvgMedian}{1.0}
\newcommand{\DualTargetPatternsVsBranchExtCyclesSpeedupBaselineSolutionFoundAvgMin}{1.0}
\newcommand{\DualTargetPatternsVsBranchExtCyclesSpeedupBaselineSolutionFoundAvgMax}{1.0}
\newcommand{\DualTargetPatternsVsBranchExtCyclesSpeedupBaselineCyclesAvgAmean}{54277.199999999997}
\newcommand{\DualTargetPatternsVsBranchExtCyclesSpeedupBaselineCyclesAvgGmean}{}
\newcommand{\DualTargetPatternsVsBranchExtCyclesSpeedupBaselineCyclesAvgMedian}{8940.0}
\newcommand{\DualTargetPatternsVsBranchExtCyclesSpeedupBaselineCyclesAvgMin}{1088.0}
\newcommand{\DualTargetPatternsVsBranchExtCyclesSpeedupBaselineCyclesAvgMax}{266571.0}
\newcommand{\DualTargetPatternsVsBranchExtCyclesSpeedupBaselineOptimalAvgAmean}{1.0}
\newcommand{\DualTargetPatternsVsBranchExtCyclesSpeedupBaselineOptimalAvgGmean}{}
\newcommand{\DualTargetPatternsVsBranchExtCyclesSpeedupBaselineOptimalAvgMedian}{1.0}
\newcommand{\DualTargetPatternsVsBranchExtCyclesSpeedupBaselineOptimalAvgMin}{1.0}
\newcommand{\DualTargetPatternsVsBranchExtCyclesSpeedupBaselineOptimalAvgMax}{1.0}
\newcommand{\DualTargetPatternsVsBranchExtCyclesSpeedupBaselineMatchingTimeAvgAmean}{4.4795592708999994}
\newcommand{\DualTargetPatternsVsBranchExtCyclesSpeedupBaselineMatchingTimeAvgGmean}{}
\newcommand{\DualTargetPatternsVsBranchExtCyclesSpeedupBaselineMatchingTimeAvgMedian}{2.4114913920000003}
\newcommand{\DualTargetPatternsVsBranchExtCyclesSpeedupBaselineMatchingTimeAvgMin}{0.63854441699999998}
\newcommand{\DualTargetPatternsVsBranchExtCyclesSpeedupBaselineMatchingTimeAvgMax}{29.780336937999998}
\newcommand{\DualTargetPatternsVsBranchExtCyclesSpeedupBaselineLbCompTimeAvgAmean}{0.0}
\newcommand{\DualTargetPatternsVsBranchExtCyclesSpeedupBaselineLbCompTimeAvgGmean}{}
\newcommand{\DualTargetPatternsVsBranchExtCyclesSpeedupBaselineLbCompTimeAvgMedian}{0.0}
\newcommand{\DualTargetPatternsVsBranchExtCyclesSpeedupBaselineLbCompTimeAvgMin}{0.0}
\newcommand{\DualTargetPatternsVsBranchExtCyclesSpeedupBaselineLbCompTimeAvgMax}{0.0}
\newcommand{\DualTargetPatternsVsBranchExtCyclesSpeedupBaselineDomProcTimeAvgAmean}{0.71725193262100218}
\newcommand{\DualTargetPatternsVsBranchExtCyclesSpeedupBaselineDomProcTimeAvgGmean}{}
\newcommand{\DualTargetPatternsVsBranchExtCyclesSpeedupBaselineDomProcTimeAvgMedian}{0.18106794357299805}
\newcommand{\DualTargetPatternsVsBranchExtCyclesSpeedupBaselineDomProcTimeAvgMin}{0.042232990264892578}
\newcommand{\DualTargetPatternsVsBranchExtCyclesSpeedupBaselineDomProcTimeAvgMax}{9.4443011283874512}
\newcommand{\DualTargetPatternsVsBranchExtCyclesSpeedupBaselineIllProcTimeAvgAmean}{12.077459657192231}
\newcommand{\DualTargetPatternsVsBranchExtCyclesSpeedupBaselineIllProcTimeAvgGmean}{}
\newcommand{\DualTargetPatternsVsBranchExtCyclesSpeedupBaselineIllProcTimeAvgMedian}{1.4817935228347778}
\newcommand{\DualTargetPatternsVsBranchExtCyclesSpeedupBaselineIllProcTimeAvgMin}{0.23539400100708008}
\newcommand{\DualTargetPatternsVsBranchExtCyclesSpeedupBaselineIllProcTimeAvgMax}{198.16651105880737}
\newcommand{\DualTargetPatternsVsBranchExtCyclesSpeedupBaselineRedunProcTimeAvgAmean}{0.43540301322937014}
\newcommand{\DualTargetPatternsVsBranchExtCyclesSpeedupBaselineRedunProcTimeAvgGmean}{}
\newcommand{\DualTargetPatternsVsBranchExtCyclesSpeedupBaselineRedunProcTimeAvgMedian}{0.14108788967132568}
\newcommand{\DualTargetPatternsVsBranchExtCyclesSpeedupBaselineRedunProcTimeAvgMin}{0.036217927932739258}
\newcommand{\DualTargetPatternsVsBranchExtCyclesSpeedupBaselineRedunProcTimeAvgMax}{5.1054849624633789}
\newcommand{\DualTargetPatternsVsBranchExtCyclesSpeedupBaselineModelPrepTimeAvgAmean}{96.920000000000002}
\newcommand{\DualTargetPatternsVsBranchExtCyclesSpeedupBaselineModelPrepTimeAvgGmean}{}
\newcommand{\DualTargetPatternsVsBranchExtCyclesSpeedupBaselineModelPrepTimeAvgMedian}{42.055000000000007}
\newcommand{\DualTargetPatternsVsBranchExtCyclesSpeedupBaselineModelPrepTimeAvgMin}{11.529999999999999}
\newcommand{\DualTargetPatternsVsBranchExtCyclesSpeedupBaselineModelPrepTimeAvgMax}{684.80000000000007}
\newcommand{\DualTargetPatternsVsBranchExtCyclesSpeedupBaselineSolvingTimeAvgAmean}{335.33600000000001}
\newcommand{\DualTargetPatternsVsBranchExtCyclesSpeedupBaselineSolvingTimeAvgGmean}{}
\newcommand{\DualTargetPatternsVsBranchExtCyclesSpeedupBaselineSolvingTimeAvgMedian}{22.435000000000002}
\newcommand{\DualTargetPatternsVsBranchExtCyclesSpeedupBaselineSolvingTimeAvgMin}{0.89000000000000001}
\newcommand{\DualTargetPatternsVsBranchExtCyclesSpeedupBaselineSolvingTimeAvgMax}{1595.1800000000001}
\newcommand{\DualTargetPatternsVsBranchExtCyclesSpeedupBaselinePrePlusSolvingTimeAvgAmean}{348.56611460304259}
\newcommand{\DualTargetPatternsVsBranchExtCyclesSpeedupBaselinePrePlusSolvingTimeAvgGmean}{}
\newcommand{\DualTargetPatternsVsBranchExtCyclesSpeedupBaselinePrePlusSolvingTimeAvgMedian}{25.12498140335083}
\newcommand{\DualTargetPatternsVsBranchExtCyclesSpeedupBaselinePrePlusSolvingTimeAvgMin}{1.3022128486633302}
\newcommand{\DualTargetPatternsVsBranchExtCyclesSpeedupBaselinePrePlusSolvingTimeAvgMax}{1600.7503523159028}
\newcommand{\DualTargetPatternsVsBranchExtCyclesSpeedupBaselineTotalTimeAvgAmean}{353.04567387394263}
\newcommand{\DualTargetPatternsVsBranchExtCyclesSpeedupBaselineTotalTimeAvgGmean}{}
\newcommand{\DualTargetPatternsVsBranchExtCyclesSpeedupBaselineTotalTimeAvgMedian}{31.622812543876158}
\newcommand{\DualTargetPatternsVsBranchExtCyclesSpeedupBaselineTotalTimeAvgMin}{1.9407572656633301}
\newcommand{\DualTargetPatternsVsBranchExtCyclesSpeedupBaselineTotalTimeAvgMax}{1617.7166340876583}
\newcommand{\DualTargetPatternsVsBranchExtCyclesSpeedupBaselineLbCompTimeCvAmean}{0.0}
\newcommand{\DualTargetPatternsVsBranchExtCyclesSpeedupBaselineLbCompTimeCvGmean}{}
\newcommand{\DualTargetPatternsVsBranchExtCyclesSpeedupBaselineLbCompTimeCvMedian}{0.0}
\newcommand{\DualTargetPatternsVsBranchExtCyclesSpeedupBaselineLbCompTimeCvMin}{0.0}
\newcommand{\DualTargetPatternsVsBranchExtCyclesSpeedupBaselineLbCompTimeCvMax}{0.0}
\newcommand{\DualTargetPatternsVsBranchExtCyclesSpeedupBaselineDomProcTimeCvAmean}{0.0}
\newcommand{\DualTargetPatternsVsBranchExtCyclesSpeedupBaselineDomProcTimeCvGmean}{}
\newcommand{\DualTargetPatternsVsBranchExtCyclesSpeedupBaselineDomProcTimeCvMedian}{0.0}
\newcommand{\DualTargetPatternsVsBranchExtCyclesSpeedupBaselineDomProcTimeCvMin}{0.0}
\newcommand{\DualTargetPatternsVsBranchExtCyclesSpeedupBaselineDomProcTimeCvMax}{0.0}
\newcommand{\DualTargetPatternsVsBranchExtCyclesSpeedupBaselineIllProcTimeCvAmean}{0.0}
\newcommand{\DualTargetPatternsVsBranchExtCyclesSpeedupBaselineIllProcTimeCvGmean}{}
\newcommand{\DualTargetPatternsVsBranchExtCyclesSpeedupBaselineIllProcTimeCvMedian}{0.0}
\newcommand{\DualTargetPatternsVsBranchExtCyclesSpeedupBaselineIllProcTimeCvMin}{0.0}
\newcommand{\DualTargetPatternsVsBranchExtCyclesSpeedupBaselineIllProcTimeCvMax}{0.0}
\newcommand{\DualTargetPatternsVsBranchExtCyclesSpeedupBaselineRedunProcTimeCvAmean}{0.0}
\newcommand{\DualTargetPatternsVsBranchExtCyclesSpeedupBaselineRedunProcTimeCvGmean}{}
\newcommand{\DualTargetPatternsVsBranchExtCyclesSpeedupBaselineRedunProcTimeCvMedian}{0.0}
\newcommand{\DualTargetPatternsVsBranchExtCyclesSpeedupBaselineRedunProcTimeCvMin}{0.0}
\newcommand{\DualTargetPatternsVsBranchExtCyclesSpeedupBaselineRedunProcTimeCvMax}{0.0}
\newcommand{\DualTargetPatternsVsBranchExtCyclesSpeedupBaselineModelPrepTimeCvAmean}{0.0}
\newcommand{\DualTargetPatternsVsBranchExtCyclesSpeedupBaselineModelPrepTimeCvGmean}{}
\newcommand{\DualTargetPatternsVsBranchExtCyclesSpeedupBaselineModelPrepTimeCvMedian}{0.0}
\newcommand{\DualTargetPatternsVsBranchExtCyclesSpeedupBaselineModelPrepTimeCvMin}{0.0}
\newcommand{\DualTargetPatternsVsBranchExtCyclesSpeedupBaselineModelPrepTimeCvMax}{0.0}
\newcommand{\DualTargetPatternsVsBranchExtCyclesSpeedupBaselineSolvingTimeCvAmean}{0.0}
\newcommand{\DualTargetPatternsVsBranchExtCyclesSpeedupBaselineSolvingTimeCvGmean}{}
\newcommand{\DualTargetPatternsVsBranchExtCyclesSpeedupBaselineSolvingTimeCvMedian}{0.0}
\newcommand{\DualTargetPatternsVsBranchExtCyclesSpeedupBaselineSolvingTimeCvMin}{0.0}
\newcommand{\DualTargetPatternsVsBranchExtCyclesSpeedupBaselineSolvingTimeCvMax}{0.0}
\newcommand{\DualTargetPatternsVsBranchExtCyclesSpeedupBaselinePrePlusSolvingTimeCvAmean}{0.0}
\newcommand{\DualTargetPatternsVsBranchExtCyclesSpeedupBaselinePrePlusSolvingTimeCvGmean}{}
\newcommand{\DualTargetPatternsVsBranchExtCyclesSpeedupBaselinePrePlusSolvingTimeCvMedian}{0.0}
\newcommand{\DualTargetPatternsVsBranchExtCyclesSpeedupBaselinePrePlusSolvingTimeCvMin}{0.0}
\newcommand{\DualTargetPatternsVsBranchExtCyclesSpeedupBaselinePrePlusSolvingTimeCvMax}{0.0}
\newcommand{\DualTargetPatternsVsBranchExtCyclesSpeedupBaselineTotalTimeCvAmean}{0.0}
\newcommand{\DualTargetPatternsVsBranchExtCyclesSpeedupBaselineTotalTimeCvGmean}{}
\newcommand{\DualTargetPatternsVsBranchExtCyclesSpeedupBaselineTotalTimeCvMedian}{0.0}
\newcommand{\DualTargetPatternsVsBranchExtCyclesSpeedupBaselineTotalTimeCvMin}{0.0}
\newcommand{\DualTargetPatternsVsBranchExtCyclesSpeedupBaselineTotalTimeCvMax}{0.0}
\newcommand{\DualTargetPatternsVsBranchExtCyclesSpeedupCyclesSpeedupAmean}{N/A}
\newcommand{\DualTargetPatternsVsBranchExtCyclesSpeedupCyclesSpeedupGmean}{0.0027568784658444923}
\newcommand{\DualTargetPatternsVsBranchExtCyclesSpeedupCyclesSpeedupMedian}{0.0}
\newcommand{\DualTargetPatternsVsBranchExtCyclesSpeedupCyclesSpeedupMin}{-0.0}
\newcommand{\DualTargetPatternsVsBranchExtCyclesSpeedupCyclesSpeedupMax}{0.023462783171521034}


\begin{figure}
  \centering%
  \maxsizebox{\textwidth}{!}{%
    \trimBarchartPlot{%
      \begin{tikzpicture}[gnuplot]
%% generated with GNUPLOT 5.0p4 (Lua 5.2; terminal rev. 99, script rev. 100)
%% fre  5 jan 2018 08:29:07
\path (0.000,0.000) rectangle (12.500,8.750);
\gpcolor{rgb color={0.753,0.753,0.753}}
\gpsetlinetype{gp lt axes}
\gpsetdashtype{gp dt axes}
\gpsetlinewidth{0.50}
\draw[gp path] (1.196,1.779)--(36.945,1.779);
\gpcolor{color=gp lt color border}
\node[gp node right] at (1.196,1.779) {\plotSpeedupTics{0}};
\gpcolor{rgb color={0.753,0.753,0.753}}
\draw[gp path] (1.196,3.099)--(36.945,3.099);
\gpcolor{color=gp lt color border}
\node[gp node right] at (1.196,3.099) {\plotSpeedupTics{0.1}};
\gpcolor{rgb color={0.753,0.753,0.753}}
\draw[gp path] (1.196,4.420)--(36.945,4.420);
\gpcolor{color=gp lt color border}
\node[gp node right] at (1.196,4.420) {\plotSpeedupTics{0.2}};
\gpcolor{rgb color={0.753,0.753,0.753}}
\draw[gp path] (1.196,5.740)--(36.945,5.740);
\gpcolor{color=gp lt color border}
\node[gp node right] at (1.196,5.740) {\plotSpeedupTics{0.3}};
\gpcolor{rgb color={0.753,0.753,0.753}}
\draw[gp path] (1.196,7.061)--(36.945,7.061);
\gpcolor{color=gp lt color border}
\node[gp node right] at (1.196,7.061) {\plotSpeedupTics{0.4}};
\gpcolor{rgb color={0.753,0.753,0.753}}
\draw[gp path] (1.196,8.381)--(36.945,8.381);
\gpcolor{color=gp lt color border}
\node[gp node right] at (1.196,8.381) {\plotSpeedupTics{0.5}};
\node[gp node left,rotate=-30] at (3.114,1.534) {\functionName{g72x_init_state}};
\node[gp node left,rotate=-30] at (4.995,1.534) {\functionName{gl_Normal3fv}};
\node[gp node left,rotate=-30] at (6.877,1.534) {\functionName{jinit_huff_enco.}};
\node[gp node left,rotate=-30] at (8.758,1.534) {\functionName{predictor_pole}};
\node[gp node left,rotate=-30] at (10.640,1.534) {\functionName{gs_reversepath}};
\node[gp node left,rotate=-30] at (12.521,1.534) {\functionName{jpeg_has_multip.}};
\node[gp node left,rotate=-30] at (14.403,1.534) {\functionName{rsaref2reg}};
\node[gp node left,rotate=-30] at (16.284,1.534) {\functionName{gluNextContour}};
\node[gp node left,rotate=-30] at (18.166,1.534) {\functionName{inflateEnd}};
\node[gp node left,rotate=-30] at (20.047,1.534) {\functionName{internal_transp.}};
\node[gp node left,rotate=-30] at (21.929,1.534) {\functionName{is_tempfile}};
\node[gp node left,rotate=-30] at (23.810,1.534) {\functionName{mp_dmul}};
\node[gp node left,rotate=-30] at (25.692,1.534) {\functionName{gl_swap2}};
\node[gp node left,rotate=-30] at (27.573,1.534) {\functionName{trueRandEvent}};
\node[gp node left,rotate=-30] at (29.455,1.534) {\functionName{finish_pass_mas.}};
\node[gp node left,rotate=-30] at (31.336,1.534) {\functionName{test_nurbs_curv.}};
\node[gp node left,rotate=-30] at (33.218,1.534) {\functionName{gluEndPolygon}};
\node[gp node left,rotate=-30] at (35.099,1.534) {\functionName{nextkeypacket}};
\gpsetlinetype{gp lt border}
\gpsetdashtype{gp dt solid}
\gpsetlinewidth{1.00}
\draw[gp path] (1.196,8.381)--(1.196,1.779)--(36.945,1.779)--(36.945,8.381)--cycle;
\gpcolor{rgb color={0.000,0.000,0.000}}
\draw[gp path] (1.196,1.779)--(1.557,1.779)--(1.918,1.779)--(2.279,1.779)--(2.640,1.779)%
  --(3.002,1.779)--(3.363,1.779)--(3.724,1.779)--(4.085,1.779)--(4.446,1.779)--(4.807,1.779)%
  --(5.168,1.779)--(5.529,1.779)--(5.890,1.779)--(6.251,1.779)--(6.613,1.779)--(6.974,1.779)%
  --(7.335,1.779)--(7.696,1.779)--(8.057,1.779)--(8.418,1.779)--(8.779,1.779)--(9.140,1.779)%
  --(9.501,1.779)--(9.862,1.779)--(10.224,1.779)--(10.585,1.779)--(10.946,1.779)--(11.307,1.779)%
  --(11.668,1.779)--(12.029,1.779)--(12.390,1.779)--(12.751,1.779)--(13.112,1.779)--(13.473,1.779)%
  --(13.835,1.779)--(14.196,1.779)--(14.557,1.779)--(14.918,1.779)--(15.279,1.779)--(15.640,1.779)%
  --(16.001,1.779)--(16.362,1.779)--(16.723,1.779)--(17.084,1.779)--(17.446,1.779)--(17.807,1.779)%
  --(18.168,1.779)--(18.529,1.779)--(18.890,1.779)--(19.251,1.779)--(19.612,1.779)--(19.973,1.779)%
  --(20.334,1.779)--(20.695,1.779)--(21.057,1.779)--(21.418,1.779)--(21.779,1.779)--(22.140,1.779)%
  --(22.501,1.779)--(22.862,1.779)--(23.223,1.779)--(23.584,1.779)--(23.945,1.779)--(24.306,1.779)%
  --(24.668,1.779)--(25.029,1.779)--(25.390,1.779)--(25.751,1.779)--(26.112,1.779)--(26.473,1.779)%
  --(26.834,1.779)--(27.195,1.779)--(27.556,1.779)--(27.917,1.779)--(28.279,1.779)--(28.640,1.779)%
  --(29.001,1.779)--(29.362,1.779)--(29.723,1.779)--(30.084,1.779)--(30.445,1.779)--(30.806,1.779)%
  --(31.167,1.779)--(31.528,1.779)--(31.890,1.779)--(32.251,1.779)--(32.612,1.779)--(32.973,1.779)%
  --(33.334,1.779)--(33.695,1.779)--(34.056,1.779)--(34.417,1.779)--(34.778,1.779)--(35.139,1.779)%
  --(35.501,1.779)--(35.862,1.779)--(36.223,1.779)--(36.584,1.779)--(36.945,1.779);
\gpfill{rgb color={0.333,0.333,0.333}} (16.170,1.779)--(16.955,1.779)--(16.955,1.981)--(16.170,1.981)--cycle;
\gpcolor{color=gp lt color border}
\draw[gp path] (16.170,1.779)--(16.170,1.980)--(16.954,1.980)--(16.954,1.779)--cycle;
\gpfill{rgb color={0.333,0.333,0.333}} (18.051,1.779)--(18.836,1.779)--(18.836,2.380)--(18.051,2.380)--cycle;
\draw[gp path] (18.051,1.779)--(18.051,2.379)--(18.835,2.379)--(18.835,1.779)--cycle;
\gpfill{rgb color={0.333,0.333,0.333}} (21.814,1.779)--(22.599,1.779)--(22.599,2.407)--(21.814,2.407)--cycle;
\draw[gp path] (21.814,1.779)--(21.814,2.406)--(22.598,2.406)--(22.598,1.779)--cycle;
\gpfill{rgb color={0.333,0.333,0.333}} (25.577,1.779)--(26.362,1.779)--(26.362,1.799)--(25.577,1.799)--cycle;
\draw[gp path] (25.577,1.779)--(25.577,1.798)--(26.361,1.798)--(26.361,1.779)--cycle;
\gpfill{rgb color={0.333,0.333,0.333}} (29.340,1.779)--(30.125,1.779)--(30.125,2.054)--(29.340,2.054)--cycle;
\draw[gp path] (29.340,1.779)--(29.340,2.053)--(30.124,2.053)--(30.124,1.779)--cycle;
\gpfill{rgb color={0.333,0.333,0.333}} (33.104,1.779)--(33.889,1.779)--(33.889,1.914)--(33.104,1.914)--cycle;
\draw[gp path] (33.104,1.779)--(33.104,1.913)--(33.888,1.913)--(33.888,1.779)--cycle;
\gpfill{rgb color={0.333,0.333,0.333}} (34.985,1.779)--(35.770,1.779)--(35.770,2.197)--(34.985,2.197)--cycle;
\draw[gp path] (34.985,1.779)--(34.985,2.196)--(35.769,2.196)--(35.769,1.779)--cycle;
\node[gp node center] at (3.354,1.963) {\plotBarValue{-0.000000}};
\node[gp node center] at (5.235,1.963) {\plotBarValue{-0.000000}};
\node[gp node center] at (7.117,1.963) {\plotBarValue{-0.000000}};
\node[gp node center] at (8.998,1.963) {\plotBarValue{-0.000000}};
\node[gp node center] at (10.880,1.963) {\plotBarValue{-0.000000}};
\node[gp node center] at (12.761,1.963) {\plotBarValue{-0.000000}};
\node[gp node center] at (14.643,1.963) {\plotBarValue{-0.000000}};
\node[gp node center] at (16.524,2.164) {\plotBarValue{0.015248}};
\node[gp node center] at (18.406,2.563) {\plotBarValue{0.045455}};
\node[gp node center] at (20.287,1.963) {\plotBarValue{-0.000000}};
\node[gp node center] at (22.169,2.590) {\plotBarValue{0.047458}};
\node[gp node center] at (24.050,1.963) {\plotBarValue{-0.000000}};
\node[gp node center] at (25.932,1.982) {\plotBarValue{0.001469}};
\node[gp node center] at (27.813,1.963) {\plotBarValue{-0.000000}};
\node[gp node center] at (29.695,2.237) {\plotBarValue{0.020778}};
\node[gp node center] at (31.576,1.963) {\plotBarValue{-0.000000}};
\node[gp node center] at (33.458,2.097) {\plotBarValue{0.010163}};
\node[gp node center] at (35.339,2.380) {\plotBarValue{0.031551}};
\draw[gp path] (1.196,8.381)--(1.196,1.779)--(36.945,1.779)--(36.945,8.381)--cycle;
%% coordinates of the plot area
\gpdefrectangularnode{gp plot 1}{\pgfpoint{1.196cm}{1.779cm}}{\pgfpoint{36.945cm}{8.381cm}}
\end{tikzpicture}
%% gnuplot variables
%
    }%
  }

  \caption[%
            Plot comparing optimal solution costs for two constraint models
            supporing jump instruction insertion%
          ]%
          {%
            Normalized optimal solution costs for two constraint models that
            supports jump insertion: one based on branch extension (baseline),
            and one based on DTB patterns (subject).
            %
            GMI:~\printGMI{%
              \DualTargetPatternsVsBranchExtCyclesSpeedupCyclesRegularSpeedupGmean%
            },
            CI~\printGMICI{%
              \DualTargetPatternsVsBranchExtCyclesSpeedupCyclesRegularSpeedupCiMin%
            }{%
              \DualTargetPatternsVsBranchExtCyclesSpeedupCyclesRegularSpeedupCiMax%
            }%
          }
  \labelFigure{dual-target-branch-patterns-vs-branch-extension-cycles-plot}
\end{figure}

\RefFigure{dual-target-branch-patterns-vs-branch-extension-cycles-plot} shows
the normalized \gls{solution} costs for the two \glspl{constraint model}
described above, with \glsshort{constraint model}~\refModel{branch-ext} as
\gls{baseline} and \glsshort{constraint model}~\refModel{dual-target} as
\gls{subject}.
%
All \glspl{function} are solved to optimality and arranged in increasing order
of number of conditional branch \glspl{instruction}.
%
The costs range from
\printMinCycles{
  \DualTargetPatternsVsBranchExtCyclesSpeedupCyclesAvgMin,
  \DualTargetPatternsVsBranchExtCyclesSpeedupBaselineCyclesAvgMin
} to
\printMaxCycles{
  \DualTargetPatternsVsBranchExtCyclesSpeedupCyclesAvgMax,
  \DualTargetPatternsVsBranchExtCyclesSpeedupBaselineCyclesAvgMax
}.
%
The \gls{GMI} is \printGMI{%
  \DualTargetPatternsVsBranchExtCyclesSpeedupCyclesRegularSpeedupGmean%
} with \gls{CI}~\printGMICI{%
  \DualTargetPatternsVsBranchExtCyclesSpeedupCyclesRegularSpeedupCiMin%
}{%
    \DualTargetPatternsVsBranchExtCyclesSpeedupCyclesRegularSpeedupCiMax%
}.

We see clearly that \glsshort{constraint model}~\refModel{dual-target} produces
optimal \glspl{solution} with slightly lower cost compared those produced by
\glsshort{constraint model}~\refModel{branch-ext}.
%
Hence we conclude that, in terms of code quality,
\glspl{DTB pattern} is a better design choice over \gls{branch extension} when
implementing insertion of jump \glspl{instruction}.


\paragraph{Conclusions}

Based on the results from these experiments, we conclude that \glspl{DTB
  pattern} are superior, both in terms of solving time and code quality, to
\gls{branch extension}.


\section{Objective Function}
\labelSection{cm-objective-function}

In \gls{instruction selection} most \glspl{compiler} optimize for performance,
meaning they attempt to minimize the total cost -- this is typically the
\gls{instruction} latency -- of the selected \glspl{match} per \gls{block}~$b$
while factoring in the execution frequency of~$b$.
%
The intuition is that the \glspl{instruction} selected for ``hot'' \glspl{block}
-- that is, those which are part of a loop with many iterations -- will have
greater impact than those selected for scarcely executed \glspl{block}.
%
In this chapter we introduce a straightforward but naive implementation of the
\gls{objective function}, which will be refined in the next chapter.


\paragraph{Variables}

The \gls{cost variable} \mbox{$\mVar{cost} \in \mathbb{N}$} models the total
cost of the selected \glspl{match}.
%
It is assumed the total cost can never be negative.


\paragraph{Constraints}

A straightforward implementation of the \gls{objective function} described above
can be modeled as
%
\begin{equation}
  \mVar{cost} =
  \sum_{\mathclap{m \in \mMatchSet}}
  \mVar{sel}[m] \times \mCost(m) \times \mFreq(\mBlockOf(m)),
  \labelEquation{naive-objective-function}
\end{equation}
%
where \mbox{$\mCost(m) \in \mNatNumSet$} denotes the cost of \gls{match}~$m$,
\mbox{$\mFreq(b) \in \mNatNumSet$} denotes the execution frequency of
block~$b$\!,\hspace{-1pt}%
%
\footnote{%
  In order to contain the size of the \gls{domain} of the \gls{cost variable},
  the execution frequencies must often be scaled down.
  %
  In our experiments, limiting the execution frequencies to \num{1000} proved
  sufficient.
}
%
and
%
\begin{equation}
  \mBlockOf(m)
  \equiv
  \left\{
  \begin{array}{ll}
      \mVar{oplace}[\mMin(\mCovers(m))]
    & \text{if } \mCovers(m) \neq \mEmptySet, \\
      \mVar{dplace}[\mVar{alt}[\mMin(\mDefines(m))]]
    & \text{otherwise}.
  \end{array}
  \right.
  \labelEquation{block-of-function}
\end{equation}
%
However, as previously stated \refEquation{naive-objective-function} is a naive
implementation and we therefore use a refined version, introduced in
\refChapter{solving-techniques}, which offers much stronger \gls{propagation}.


\section{Limitations}
\labelSection{cm-limitations}

The \gls{constraint model} described in this chapter has several limitations
that affect code quality.
%
Consequently, a \gls{solution} that is considered optimal with respect to this
\glsshort{constraint model} may still be inferior to the code produced by an
\gls{instruction selector} without these limitations.
%
The first limitation concerns \gls{recomputation} of values, the second concerns
elimination of common subexpressions, and the third concerns implicit sign and
zero extensions and truncations.


\paragraph{Recomputation}

For some combinations of \glspl{function} and \glspl{target machine}, it may be
beneficial to \glsshort!{recomputation} values appearing in common
subexpressions instead of reusing it.
%
\begin{filecontents*}{recomputation-example-ir.c}
$\irAssign{a}{x + y}$
store a, $\ldots$
$\ldots$
store a, $\ldots$
\end{filecontents*}
%
\begin{figure}
  \centering%
  \mbox{}%
  \hfill%
  \subcaptionbox{Code snippet\labelFigure{recomputation-example-ir}}%
                [25mm]%
                {%
                  \lstinputlisting[language=c,mathescape]%
                                  {recomputation-example-ir.c}%
                }%
  \hfill%
  \subcaptionbox{%
                  UF subgraph, covered by two matches derived from a
                  store instruction with base-plus-index addressing mode.
                  %
                  For brevity, the state nodes are not included%
                  \labelFigure{recomputation-example-graph}%
                }%
                [74mm]%
                {%
                  % Copyright (c) 2018, Gabriel Hjort Blindell <ghb@kth.se>
%
% This work is licensed under a Creative Commons 4.0 International License (see
% LICENSE file or visit <http://creativecommons.org/licenses/by/4.0/> for a copy
% of the license).
%
\begingroup%
\setlength{\opNodeDist}{12pt}%
\figureFont\figureFontSize%
\pgfdeclarelayer{background}%
\pgfsetlayers{background,main}%
\begin{tikzpicture}[
    outer match node/.style={
      match node,
      draw=none,
      inner sep=3pt,
    },
  ]

  \node [computation node] (add) {\opAdd};
  \node [value node, position=135 degrees from add] (x) {\opVar{x}};
  \node [value node, position= 45 degrees from add] (y) {\opVar{y}};
  \node [value node, below=of add] (a) {\opVar{a}};
  \node [computation node, position=-135 degrees from a] (st1) {\opStore};
  \node [computation node, position=- 45 degrees from a] (st2) {\opStore};
  \node [value node, position=135 degrees from st1] (v1) {};
  \node [value node, position= 45 degrees from st2] (v2) {};

  \begin{scope}[data flow]
    \draw (x) -- (add);
    \draw (y) -- (add);
    \draw (add) -- (a);
    \draw (a) -- (st1);
    \draw (a) -- (st2);
    \draw (v1) -- (st1);
    \draw (v2) -- (st2);
  \end{scope}

  \begin{pgfonlayer}{background}
    % m1
    \node [outer match node, fit=(v1)]               (m1a) {};
    \node [outer match node, fit=(x)]                (m1b) {};
    \node [outer match node, fit=(y)]                (m1c) {};
    \node [outer match node, fit=(add), inner sep=0] (m1d) {};
    \node [outer match node, fit=(a)]                (m1e) {};
    \node [outer match node, fit=(st1), inner sep=0] (m1f) {};

    \def\pathMI{
      [rounded corners=4pt]
      (m1a.north west)
      --
      (m1b.north west)
      --
      (m1c.north east)
      --
      (m1c.south east)
      [rounded corners=8pt]
      --
      (m1d.east)
      [rounded corners=4pt]
      --
      (m1e.south east)
      --
      (m1f.south -| m1f.south east)
      --
      (m1f.south -| m1f.south west)
      --
      (m1a.south west)
      --
      cycle
    }
    \path [fill=shade1] \pathMI;

    % m2
    \node [outer match node, fit=(v2)]               (m2a) {};
    \node [outer match node, fit=(y)]                (m2b) {};
    \node [outer match node, fit=(x)]                (m2c) {};
    \node [outer match node, fit=(add), inner sep=0] (m2d) {};
    \node [outer match node, fit=(a)]                (m2e) {};
    \node [outer match node, fit=(st2), inner sep=0] (m2f) {};

    \def\pathMII{
      [rounded corners=4pt]
      (m2a.north east)
      --
      (m2b.north east)
      --
      (m2c.north west)
      --
      (m2c.south west)
      [rounded corners=8pt]
      --
      (m2d.west)
      [rounded corners=4pt]
      --
      (m2e.south west)
      --
      (m2f.south -| m2f.south west)
      --
      (m2f.south -| m2f.south east)
      --
      (m2a.south east)
      --
      cycle
    }
    \path [fill=shade1] \pathMII;

    \begin{scope}
      \path [clip] \pathMI;
      \path [fill=shade2] \pathMII;
    \end{scope}

    \draw [match line] \pathMI;
    \draw [match line] \pathMII;
  \end{pgfonlayer}
\end{tikzpicture}%
\endgroup%
%
                }%
  \hfill%
  \mbox{}

  \caption{%
    Example illustrating when recomputation is preferred over value reuse%
  }%
  \labelFigure{recomputation-example}
\end{figure}
%
See for example \refFigure{recomputation-example}.
%
The \gls{function} performs two memory stores using the same address
value~\irVar*{a} (\refFigure{recomputation-example-ir}).
%
If the \gls{target machine} provides a memory \gls{instruction} with
base-plus-index addressing mode (that is, the address to be used is the sum of
the values appearing in a base and an index \gls{register}), then no
add~\gls{instruction} is needed for computing~\irVar*{a}.
%
In the context of \gls{instruction selection}, this means letting the
\gls{operation} representing the addition to be covered by more than one
\gls{match} (\refFigure{recomputation-example-graph}).
%
However, \refEquationList{operation-coverage, data-definitions} require that
every \gls{operation} and \gls{datum} must be covered respectively
\gls{define.d}[d] by exactly one selected \gls{match}, thus forbidding such
\glspl{solution}.
%
Although these \glspl{constraint} can be relaxed to allow \glspl{operation} and
\glspl{datum} to be covered respectively \gls{define.d}[d] by at least one
selected \gls{match}, many of the solving techniques introduced in this
dissertation which are needed for scalability and robustness rely on exact
\gls{cover}[age].


\paragraph{If-Conversions}

In most cases, performing a branch incurs a performance penalty.
%
Some architectures therefore allow the \glspl{instruction} to be predicated with
a boolean flag for optional execution, which allows \glspl{function} with
if-then-else structures to be transformed into linear code.
%
This process is called \gls!{if-conversion}.

Although the \gls{universal representation} enables predicated versions of the
\glspl{instruction} to be captured as \glspl{pattern}, selection of such
\glspl{pattern} is typically prevented by the \gls{constraint model}.
%
\begin{filecontents*}{if-conversions-example-ir.c}
entry:
  $\ldots$
  if p goto body
  else goto end;
body:
  $\irAssign{a}{x + y}$
  $\irAssign{b}{v + w}$
end:
  $\ldots$
\end{filecontents*}
%
\begin{filecontents*}{if-conversions-example-instrs.c}
entry:
  $\ldots$
  if (p) add a, x, y
  if (p) add b, v, w
  $\ldots$
\end{filecontents*}
%
\begin{figure}
  \centering%
  \mbox{}%
  \hfill%
  \begin{minipage}{36mm}%
    \centering%
    \subcaptionbox{Code snippet\labelFigure{if-conversions-example-ir}}%
                  [28mm]
                  {%
                    \lstinputlisting[language=c,mathescape]%
                                    {if-conversions-example-ir.c}%
                  }

    \vspace{\betweensubfigures}

    \subcaptionbox{%
                    Assembly code with predicated instructions%
                    \labelFigure{if-conversions-example-instrs}%
                  }{%
                    \lstinputlisting[mathescape]%
                                    {if-conversions-example-instrs.c}%
                  }
  \end{minipage}%
  \hfill\hfill\hfill%
  \adjustbox{valign=M}{%
    \subcaptionbox{%
                    UF subgraph, covered by two matches derived predicated
                    add instructions%
                    \labelFigure{if-conversions-example-graph}%
                  }{%
                    \input{%
                      figures/constraint-model/if-conversions-example-graph%
                    }%
                  }%
  }%
  \hfill%
  \mbox{}

  \caption{%
    Example illustrating if-conversions%
  }%
  \labelFigure{if-conversions-example}
\end{figure}
%
See for example \refFigure{if-conversions-example}.
%
Assume a \gls{function} containing two sums, \irVar*{a} and~\irVar*{b}, which
are conditionally computed given a certain predicate~\irVar*{p}
(\refFigure{if-conversions-example-ir}).
%
Because this constitutes an if-then-else structure, this code snippet is eligble
for \gls{if-conversion} (\refFigure{if-conversions-example-instrs}).
%
Representing the predicated versions of add~\glspl{instruction} as
\glspl{pattern} gives rise to two \glspl{match}, $m_1$ and~$m_2$, which can
collectively cover the \glsshort{computation node} and \glspl{control node} in
the corresponding \gls{UF graph} (\refFigure{if-conversions-example-graph}).
%
However, since $m_1$ and $m_2$ both cover the same \glspl{control node}, only
one of the \glspl{match} can be selected (due to
\refEquationList{operation-coverage, data-definitions}).
%
In addition, because both \glspl{match} \gls{consume.b} the \irBlock*{body}
\gls{block}, no other \glspl{operation} may be placed in \irBlock*{body} if
either is selected (due to \refEquation{consumption}).
%
This means either both or none of the \glspl{match} must be selected.
%
Together with the \gls{constraint} of exact \gls{cover}[age], in all
\glspl{solution} neither of $m_1$ or $m_2$ is selected.
%
This problem can be fixed by relaxing the \gls{constraint} of exact
\gls{cover}[age], but as in the case of \gls{recomputation} this inhibits many
of the necessary solving techniques.


\paragraph{Implicit Sign or Zero Extensions}

In cases where the \gls{function} contains sign or zero extensions, depending on
the hardware the \gls{constraint model} may produce code with redundant
\glspl{instruction}.
%
\begin{figure}
  \centering%
  % Copyright (c) 2017-2018, Gabriel Hjort Blindell <ghb@kth.se>
%
% This work is licensed under a Creative Commons Attribution-NoDerivatives 4.0
% International License (see LICENSE file or visit
% <http://creativecommons.org/licenses/by-nc-nd/4.0/> for details).
%
\begingroup%
\setlength{\nodeDist}{12pt}%
\figureFont\figureFontSize%
\pgfdeclarelayer{background}%
\pgfsetlayers{background,main}%
\begin{tikzpicture}[
    outer match node/.style={
      match node,
      draw=none,
      inner sep=3pt,
    },
  ]

  \node [computation node] (eq) {\nEQ};
  \node [value node, position=135 degrees from eq] (a) {\nVar{a}};
  \node [value node, position= 45 degrees from eq] (b) {\nVar{b}};
  \node [value node, below=of eq] (c) {\nVar{c}};
  \node [computation node, above=of a] (call) {\nCall{foo}};
  \node [computation node, above=of b] (load) {\nLoad};

  \begin{scope}[data flow]
    \draw (call) -- (a);
    \draw (load) -- (b);
    \draw (a) -- (eq);
    \draw (b) -- (eq);
    \draw (eq) -- (c);
  \end{scope}

  \begin{pgfonlayer}{background}
    % m1
    \node [outer match node, fit=(call), inner sep=1pt] (m1a) {};
    \node [outer match node, fit=(a)]                   (m1b) {};

    \def\pathMI{
      [bend left=45]
      (m1a.west)
      to
      (m1a.north)
      to
      (m1a.east)
      --
      (m1a.east |- m1b.east)
      to
      (m1b.south)
      to
      (m1a.west |- m1b.west)
      -- coordinate (m1)
      cycle
    }
    \path [fill=shade1] \pathMI;

    % m2
    \node [outer match node, fit=(load), inner sep=1pt] (m2a) {};
    \node [outer match node, fit=(b)]                   (m2b) {};

    \def\pathMII{
      [bend left=45]
      (m2a.west)
      to
      (m2a.north)
      to
      (m2a.east)
      -- coordinate (m2)
      (m2a.east |- m2b.east)
      to
      (m2b.south)
      to
      (m2a.west |- m2b.west)
      --
      cycle
    }
    \path [fill=shade1] \pathMII;

    % m3
    \begin{scope}[outer match node/.append style={inner sep=5pt}]
      \node [outer match node, fit=(a)] (m3a) {};
      \node [outer match node, fit=(b)] (m3b) {};
      \node [outer match node, fit=(c)] (m3c) {};
    \end{scope}

    \def\pathMIII{
      [rounded corners=8pt]
      (m3a.north west)
      --
      (m3b.north east)
      --
      (m3b.south east)
      --
      (m3c.south east)
      --
      (m3c.south west)
      -- coordinate (m3)
      (m3a.south west)
      --
      cycle
    }
    \path [fill=shade1] \pathMIII;

    \begin{scope}
      \path [clip] \pathMI;
      \path [fill=shade2] \pathMIII;
    \end{scope}
    \begin{scope}
      \path [clip] \pathMII;
      \path [fill=shade2] \pathMIII;
    \end{scope}

    \begin{scope}[match line]
      \draw \pathMI;
      \draw \pathMII;
      \draw \pathMIII;
    \end{scope}
  \end{pgfonlayer}

  % Match labels
  \begin{scope}[overlay]
    \node [match label, left=of m1, outer sep=1.5pt] (m1l) {$\strut m_1$};
    \node [match label, right=of m2, outer sep=1.5pt] (m2l) {$\strut m_2$};
    \node [match label, left=of m3] (m3l) {$\strut m_3$};

    \foreach \i in {1, 2, 3} {
      \draw [match attachment line] (m\i) -- (m\i l);
    }
  \end{scope}
\end{tikzpicture}%
\endgroup%


  \caption{Example illustrating implicit sign or zero extensions}%
  \labelFigure{implicit-extensions-example}
\end{figure}
%
See for example \refFigure{implicit-extensions-example}, which depicts a
\glsshort{UF graph}~\gls{subgraph} coverable by \glspl{match}~$m_1$, $m_2$,
and~$m_3$.
%
Assume that \irCode*{foo} represents a \gls{function} call and that \irVar*{a},
\irVar*{b}, and \irVar*{c} represent 8-bit values stored in 32-bit
\glspl{register}.
%
As is common, $m_3$ will be derived from an \gls{instruction} that checks
whether the full contents of two \glspl{register} are equal.
%
Consequently, as a precaution the upper bytes of the \glspl{register} need to be
zero-extended (that is, those bits are all set to~0) before doing the
comparison.
%
While this is certainly necessary for the \gls{register} of \irVar*{a} as
nothing can be assumed about the value returned by \irCode*{foo}, it may be
redundant for the register of \irVar*{b}.
%
For example, $m_2$ may be derived from a single-byte load \gls{instruction} that
clears the entire \gls{register} before loading the value.
%
But since this information is lost in the \gls{constraint model}, $m_3$ must
assume both \glspl{register} need to be zero-extended.

One solution to this problem is to extend the \gls{pattern set} with additional
\glspl{pattern} that capture these situations.
%
For example, merging the \glspl{pattern} of $m_2$ and $m_3$ results in a
\gls{match} which, if selected, emits the \gls{instruction} of $m_2$ followed by
the \glspl{instruction} of $m_3$ without the redundant clearing of \irVar*{b}.
%
Depending on the \gls{instruction set}, however, this may result in an
exponential number of \glspl{pattern}.
%
If the \gls{instruction set} contains $n$~\glspl{instruction} with implicit
extensions and $m$~\gls{instruction} each taking $k$~values that must first be
sign- or zero-extended, then this will result in $\mBigO(n^k m)$ additional
\glspl{pattern}.

Another solution is to apply the same mechanism used in \gls{copy extension}.
%
In the same manner as with \glspl{copy node}, first the \gls{UF graph} is
extended with \glspl!{extension node}.
%
Hence, for each \gls{data-flow edge}~$\mEdge{v}{o}$, where $v$ is a \gls{value
  node} and $o$ is an \gls{operation}, we remove this \gls{edge} and insert a
new \gls{extension node}~$e$, \gls{value node}~$v'$, and \glspl{data-flow edge}
such that \mbox{$\mEdge{v}{\mEdge{e}{\mEdge{v'}{o}}}$}.
%
Then the \gls{constraint model} is extended with two sets of \glspl{variable},
\mbox{$\mVar{sext}[d] \in \mSet{0, 1}$} and \mbox{$\mVar{zext}[d] \in \mSet{0,
    1}$}, denoting whether a value has been sign- respectively zero-extended.
%
We also extend the \gls{pattern set} with a special \gls!{null-extend pattern},
with \gls{graph} structure \mbox{$\mEdge{v}{\mEdge{e}{v'}}$}, that covers $e$ at
zero cost provided that \mbox{$(\mVar{sext}[v] \mOr \neg\mVar{sext}[v']) \mAnd
  (\mVar{zext}[v] \mOr \neg\mVar{zext}[v'])$} holds.
%
Obviously, a \gls{match} derived from the \gls{null-extend pattern} emits
nothing if selected.
%
If the \gls{null-extend pattern} cannot be selected for covering a particular
\gls{extension node}, then this means an appropriate extension \gls{instruction}
must be emitted.


\section{Summary}
\labelSection{model-summary}

In this chapter, we have introduced a \gls{constraint model} that integrates the
problems of \gls{global.is} \gls{instruction selection}, \gls{global code
  motion}, \gls{data copying}, \gls{value reuse}, and \gls{block ordering}.
%
A complete implementation of this \glsshort{constraint model}, written in
\gls{MiniZinc} (a high-level \gls{constraint} modeling language), is available
in \refAppendix{minizinc-implementation}.
%
When multiple design choice exist for a given task, we have performed a thorough
evaluation to decide which design choice is better.
%
We have also discussed the limitations of this \glsshort{constraint model} and
how these affect the assembly code that can be produced.
