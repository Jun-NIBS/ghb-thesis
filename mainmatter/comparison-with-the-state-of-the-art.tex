% Copyright (c) 2017, Gabriel Hjort Blindell <ghb@kth.se>
%
% This work is licensed under a Creative Commons 4.0 International License (see
% LICENSE file or visit <http://creativecommons.org/licenses/by/4.0/> for a copy
% of the license).

\chapter{Comparison with the State of the Art}
\labelChapter{comparison-with-the-state-of-the-art}

\todo{write outline}


\section{Unison \versus LLVM}
\labelSection{comparison-unison-vs-llvm}

\todo{write}


\section{Selection of SIMD instructions}
\labelSection{comparison-with-or-without-simds}

\def\patternSetA{\textsc{i}}
\def\patternSetB{\textsc{ii}}

We evaluate the impact of being able to select \gls{SIMD.instr}
\glspl{instruction} by comparing the cost (that is, the total number of cycles,
as described in \refSection{cm-objective-function} on
\refPageOfSection{cm-objective-function}) of \glspl{solution} produced from two
\glspl{pattern set}: one derived from \gls{Hexagon} without \gls{SIMD.instr}
\glspl{instruction}, and another with such \glspl{instruction}.
%
We refer to these as \glspl{pattern set}~\patternSetA{} and~\patternSetB,
respectively.

The experimental setup is the same as in
\refSection{cm-alt-values-experimental-evaluation} on
\refPageOfSection{cm-alt-values-experimental-evaluation} with two exceptions.
%
First, when filtering we remove all \glspl{function} that have less than less
than \num{50}~\gls{LLVM} \gls{IR} \glspl{instruction} -- anything smaller will
most likely not have enough data parallelism for selection of \gls{SIMD.instr}
\glspl{instruction} -- and greater than \num{150}~\glspl{instruction} --
anything larger will lead to unreasonably long experiment runtimes.
%
To increase the amount of data parallelism, we also remove all \glspl{function}
not containing at least two addition, subtraction, logical and, or logical or
\glspl{instruction}.
%
This leaves a pool of \num{221}~\glspl{function}, from which
\num{20}~\glspl{function} are sampled using the same method as before.
%
Second, when solving we apply time limit of \SI{600}{\s} to the \gls{constraint
  solver}.
%
For any given \gls{function}, the last \gls{solution} found is considered
optimal if and only if the \glsshort{constraint solver} has finished its
execution within the time limit.

\newcommand{\SimdVsWithoutCyclesSpeedupSimpleNameAmean}{}
\newcommand{\SimdVsWithoutCyclesSpeedupSimpleNameGmean}{}
\newcommand{\SimdVsWithoutCyclesSpeedupSimpleNameMedian}{}
\newcommand{\SimdVsWithoutCyclesSpeedupSimpleNameMin}{}
\newcommand{\SimdVsWithoutCyclesSpeedupSimpleNameMax}{}
\newcommand{\SimdVsWithoutCyclesSpeedupSolutionFoundAvgAmean}{1.0}
\newcommand{\SimdVsWithoutCyclesSpeedupSolutionFoundAvgGmean}{}
\newcommand{\SimdVsWithoutCyclesSpeedupSolutionFoundAvgMedian}{1.0}
\newcommand{\SimdVsWithoutCyclesSpeedupSolutionFoundAvgMin}{1.0}
\newcommand{\SimdVsWithoutCyclesSpeedupSolutionFoundAvgMax}{1.0}
\newcommand{\SimdVsWithoutCyclesSpeedupCyclesAvgAmean}{6160.5500000000002}
\newcommand{\SimdVsWithoutCyclesSpeedupCyclesAvgGmean}{}
\newcommand{\SimdVsWithoutCyclesSpeedupCyclesAvgMedian}{5909.0}
\newcommand{\SimdVsWithoutCyclesSpeedupCyclesAvgMin}{390.0}
\newcommand{\SimdVsWithoutCyclesSpeedupCyclesAvgMax}{16963.0}
\newcommand{\SimdVsWithoutCyclesSpeedupOptimalAvgAmean}{0.94999999999999996}
\newcommand{\SimdVsWithoutCyclesSpeedupOptimalAvgGmean}{}
\newcommand{\SimdVsWithoutCyclesSpeedupOptimalAvgMedian}{1.0}
\newcommand{\SimdVsWithoutCyclesSpeedupOptimalAvgMin}{0.0}
\newcommand{\SimdVsWithoutCyclesSpeedupOptimalAvgMax}{1.0}
\newcommand{\SimdVsWithoutCyclesSpeedupMatchingTimeAvgAmean}{1.9552516058875}
\newcommand{\SimdVsWithoutCyclesSpeedupMatchingTimeAvgGmean}{}
\newcommand{\SimdVsWithoutCyclesSpeedupMatchingTimeAvgMedian}{1.4186189429999998}
\newcommand{\SimdVsWithoutCyclesSpeedupMatchingTimeAvgMin}{0.67505047250000005}
\newcommand{\SimdVsWithoutCyclesSpeedupMatchingTimeAvgMax}{5.4705011900000002}
\newcommand{\SimdVsWithoutCyclesSpeedupLbCompTimeAvgAmean}{0.0}
\newcommand{\SimdVsWithoutCyclesSpeedupLbCompTimeAvgGmean}{}
\newcommand{\SimdVsWithoutCyclesSpeedupLbCompTimeAvgMedian}{0.0}
\newcommand{\SimdVsWithoutCyclesSpeedupLbCompTimeAvgMin}{0.0}
\newcommand{\SimdVsWithoutCyclesSpeedupLbCompTimeAvgMax}{0.0}
\newcommand{\SimdVsWithoutCyclesSpeedupDomProcTimeAvgAmean}{0.28086245656013487}
\newcommand{\SimdVsWithoutCyclesSpeedupDomProcTimeAvgGmean}{}
\newcommand{\SimdVsWithoutCyclesSpeedupDomProcTimeAvgMedian}{0.22015199065208435}
\newcommand{\SimdVsWithoutCyclesSpeedupDomProcTimeAvgMin}{0.075460672378540039}
\newcommand{\SimdVsWithoutCyclesSpeedupDomProcTimeAvgMax}{0.80706125497817993}
\newcommand{\SimdVsWithoutCyclesSpeedupIllProcTimeAvgAmean}{0.47936028838157652}
\newcommand{\SimdVsWithoutCyclesSpeedupIllProcTimeAvgGmean}{}
\newcommand{\SimdVsWithoutCyclesSpeedupIllProcTimeAvgMedian}{0.40341660380363464}
\newcommand{\SimdVsWithoutCyclesSpeedupIllProcTimeAvgMin}{0.13042497634887695}
\newcommand{\SimdVsWithoutCyclesSpeedupIllProcTimeAvgMax}{1.2770726680755615}
\newcommand{\SimdVsWithoutCyclesSpeedupRedunProcTimeAvgAmean}{0.16675203740596772}
\newcommand{\SimdVsWithoutCyclesSpeedupRedunProcTimeAvgGmean}{}
\newcommand{\SimdVsWithoutCyclesSpeedupRedunProcTimeAvgMedian}{0.13693201541900635}
\newcommand{\SimdVsWithoutCyclesSpeedupRedunProcTimeAvgMin}{0.052652955055236816}
\newcommand{\SimdVsWithoutCyclesSpeedupRedunProcTimeAvgMax}{0.41885948181152344}
\newcommand{\SimdVsWithoutCyclesSpeedupModelPrepTimeAvgAmean}{41.449500000000008}
\newcommand{\SimdVsWithoutCyclesSpeedupModelPrepTimeAvgGmean}{}
\newcommand{\SimdVsWithoutCyclesSpeedupModelPrepTimeAvgMedian}{25.766249999999999}
\newcommand{\SimdVsWithoutCyclesSpeedupModelPrepTimeAvgMin}{5.7950000000000008}
\newcommand{\SimdVsWithoutCyclesSpeedupModelPrepTimeAvgMax}{232.42249999999999}
\newcommand{\SimdVsWithoutCyclesSpeedupSolvingTimeAvgAmean}{76.091125000000005}
\newcommand{\SimdVsWithoutCyclesSpeedupSolvingTimeAvgGmean}{}
\newcommand{\SimdVsWithoutCyclesSpeedupSolvingTimeAvgMedian}{2.9262499999999996}
\newcommand{\SimdVsWithoutCyclesSpeedupSolvingTimeAvgMin}{0.30249999999999999}
\newcommand{\SimdVsWithoutCyclesSpeedupSolvingTimeAvgMax}{606.04500000000007}
\newcommand{\SimdVsWithoutCyclesSpeedupPrePlusSolvingTimeAvgAmean}{77.01809978234769}
\newcommand{\SimdVsWithoutCyclesSpeedupPrePlusSolvingTimeAvgGmean}{}
\newcommand{\SimdVsWithoutCyclesSpeedupPrePlusSolvingTimeAvgMedian}{3.7349955034255977}
\newcommand{\SimdVsWithoutCyclesSpeedupPrePlusSolvingTimeAvgMin}{0.64042061328887945}
\newcommand{\SimdVsWithoutCyclesSpeedupPrePlusSolvingTimeAvgMax}{608.54799340486534}
\newcommand{\SimdVsWithoutCyclesSpeedupTotalTimeAvgAmean}{78.973351388235159}
\newcommand{\SimdVsWithoutCyclesSpeedupTotalTimeAvgGmean}{}
\newcommand{\SimdVsWithoutCyclesSpeedupTotalTimeAvgMedian}{5.4774381373005987}
\newcommand{\SimdVsWithoutCyclesSpeedupTotalTimeAvgMin}{1.5165221129020079}
\newcommand{\SimdVsWithoutCyclesSpeedupTotalTimeAvgMax}{610.67039389661522}
\newcommand{\SimdVsWithoutCyclesSpeedupCyclesCvAmean}{0.0}
\newcommand{\SimdVsWithoutCyclesSpeedupCyclesCvGmean}{}
\newcommand{\SimdVsWithoutCyclesSpeedupCyclesCvMedian}{0.0}
\newcommand{\SimdVsWithoutCyclesSpeedupCyclesCvMin}{0.0}
\newcommand{\SimdVsWithoutCyclesSpeedupCyclesCvMax}{0.0}
\newcommand{\SimdVsWithoutCyclesSpeedupLbCompTimeCvAmean}{0.0}
\newcommand{\SimdVsWithoutCyclesSpeedupLbCompTimeCvGmean}{}
\newcommand{\SimdVsWithoutCyclesSpeedupLbCompTimeCvMedian}{0.0}
\newcommand{\SimdVsWithoutCyclesSpeedupLbCompTimeCvMin}{0.0}
\newcommand{\SimdVsWithoutCyclesSpeedupLbCompTimeCvMax}{0.0}
\newcommand{\SimdVsWithoutCyclesSpeedupDomProcTimeCvAmean}{0.0083464455721563909}
\newcommand{\SimdVsWithoutCyclesSpeedupDomProcTimeCvGmean}{}
\newcommand{\SimdVsWithoutCyclesSpeedupDomProcTimeCvMedian}{0.0071121885858765489}
\newcommand{\SimdVsWithoutCyclesSpeedupDomProcTimeCvMin}{0.0020826932453696253}
\newcommand{\SimdVsWithoutCyclesSpeedupDomProcTimeCvMax}{0.031283546799839447}
\newcommand{\SimdVsWithoutCyclesSpeedupIllProcTimeCvAmean}{0.0059961179144342458}
\newcommand{\SimdVsWithoutCyclesSpeedupIllProcTimeCvGmean}{}
\newcommand{\SimdVsWithoutCyclesSpeedupIllProcTimeCvMedian}{0.0061818400806308415}
\newcommand{\SimdVsWithoutCyclesSpeedupIllProcTimeCvMin}{0.0014826898821171585}
\newcommand{\SimdVsWithoutCyclesSpeedupIllProcTimeCvMax}{0.010794371458082232}
\newcommand{\SimdVsWithoutCyclesSpeedupRedunProcTimeCvAmean}{0.010369764425853042}
\newcommand{\SimdVsWithoutCyclesSpeedupRedunProcTimeCvGmean}{}
\newcommand{\SimdVsWithoutCyclesSpeedupRedunProcTimeCvMedian}{0.0079718182478213696}
\newcommand{\SimdVsWithoutCyclesSpeedupRedunProcTimeCvMin}{0.0015989651795336284}
\newcommand{\SimdVsWithoutCyclesSpeedupRedunProcTimeCvMax}{0.032752048991081507}
\newcommand{\SimdVsWithoutCyclesSpeedupModelPrepTimeCvAmean}{0.0068948852476148209}
\newcommand{\SimdVsWithoutCyclesSpeedupModelPrepTimeCvGmean}{}
\newcommand{\SimdVsWithoutCyclesSpeedupModelPrepTimeCvMedian}{0.0064673413097429393}
\newcommand{\SimdVsWithoutCyclesSpeedupModelPrepTimeCvMin}{0.0011657927102274889}
\newcommand{\SimdVsWithoutCyclesSpeedupModelPrepTimeCvMax}{0.016616826251698073}
\newcommand{\SimdVsWithoutCyclesSpeedupSolvingTimeCvAmean}{0.0070348937146989601}
\newcommand{\SimdVsWithoutCyclesSpeedupSolvingTimeCvGmean}{}
\newcommand{\SimdVsWithoutCyclesSpeedupSolvingTimeCvMedian}{0.0069988455567890698}
\newcommand{\SimdVsWithoutCyclesSpeedupSolvingTimeCvMin}{0.0}
\newcommand{\SimdVsWithoutCyclesSpeedupSolvingTimeCvMax}{0.021259416970306505}
\newcommand{\SimdVsWithoutCyclesSpeedupPrePlusSolvingTimeCvAmean}{0.0052359487203888265}
\newcommand{\SimdVsWithoutCyclesSpeedupPrePlusSolvingTimeCvGmean}{}
\newcommand{\SimdVsWithoutCyclesSpeedupPrePlusSolvingTimeCvMedian}{0.0057968254431859205}
\newcommand{\SimdVsWithoutCyclesSpeedupPrePlusSolvingTimeCvMin}{7.9087231231518549e-05}
\newcommand{\SimdVsWithoutCyclesSpeedupPrePlusSolvingTimeCvMax}{0.01528343156014148}
\newcommand{\SimdVsWithoutCyclesSpeedupTotalTimeCvAmean}{0.0037465087738599834}
\newcommand{\SimdVsWithoutCyclesSpeedupTotalTimeCvGmean}{}
\newcommand{\SimdVsWithoutCyclesSpeedupTotalTimeCvMedian}{0.0028638587138181807}
\newcommand{\SimdVsWithoutCyclesSpeedupTotalTimeCvMin}{7.7563860806206781e-05}
\newcommand{\SimdVsWithoutCyclesSpeedupTotalTimeCvMax}{0.010220724953850189}
\newcommand{\SimdVsWithoutCyclesSpeedupBaselineSimpleNameAmean}{}
\newcommand{\SimdVsWithoutCyclesSpeedupBaselineSimpleNameGmean}{}
\newcommand{\SimdVsWithoutCyclesSpeedupBaselineSimpleNameMedian}{}
\newcommand{\SimdVsWithoutCyclesSpeedupBaselineSimpleNameMin}{}
\newcommand{\SimdVsWithoutCyclesSpeedupBaselineSimpleNameMax}{}
\newcommand{\SimdVsWithoutCyclesSpeedupBaselineSolutionFoundAvgAmean}{1.0}
\newcommand{\SimdVsWithoutCyclesSpeedupBaselineSolutionFoundAvgGmean}{}
\newcommand{\SimdVsWithoutCyclesSpeedupBaselineSolutionFoundAvgMedian}{1.0}
\newcommand{\SimdVsWithoutCyclesSpeedupBaselineSolutionFoundAvgMin}{1.0}
\newcommand{\SimdVsWithoutCyclesSpeedupBaselineSolutionFoundAvgMax}{1.0}
\newcommand{\SimdVsWithoutCyclesSpeedupBaselineCyclesAvgAmean}{6362.1000000000004}
\newcommand{\SimdVsWithoutCyclesSpeedupBaselineCyclesAvgGmean}{}
\newcommand{\SimdVsWithoutCyclesSpeedupBaselineCyclesAvgMedian}{5909.0}
\newcommand{\SimdVsWithoutCyclesSpeedupBaselineCyclesAvgMin}{390.0}
\newcommand{\SimdVsWithoutCyclesSpeedupBaselineCyclesAvgMax}{18963.0}
\newcommand{\SimdVsWithoutCyclesSpeedupBaselineOptimalAvgAmean}{0.94999999999999996}
\newcommand{\SimdVsWithoutCyclesSpeedupBaselineOptimalAvgGmean}{}
\newcommand{\SimdVsWithoutCyclesSpeedupBaselineOptimalAvgMedian}{1.0}
\newcommand{\SimdVsWithoutCyclesSpeedupBaselineOptimalAvgMin}{0.0}
\newcommand{\SimdVsWithoutCyclesSpeedupBaselineOptimalAvgMax}{1.0}
\newcommand{\SimdVsWithoutCyclesSpeedupBaselineMatchingTimeAvgAmean}{1.750885684025}
\newcommand{\SimdVsWithoutCyclesSpeedupBaselineMatchingTimeAvgGmean}{}
\newcommand{\SimdVsWithoutCyclesSpeedupBaselineMatchingTimeAvgMedian}{1.374084919625}
\newcommand{\SimdVsWithoutCyclesSpeedupBaselineMatchingTimeAvgMin}{0.65881363425000006}
\newcommand{\SimdVsWithoutCyclesSpeedupBaselineMatchingTimeAvgMax}{4.1732398062500007}
\newcommand{\SimdVsWithoutCyclesSpeedupBaselineLbCompTimeAvgAmean}{0.0}
\newcommand{\SimdVsWithoutCyclesSpeedupBaselineLbCompTimeAvgGmean}{}
\newcommand{\SimdVsWithoutCyclesSpeedupBaselineLbCompTimeAvgMedian}{0.0}
\newcommand{\SimdVsWithoutCyclesSpeedupBaselineLbCompTimeAvgMin}{0.0}
\newcommand{\SimdVsWithoutCyclesSpeedupBaselineLbCompTimeAvgMax}{0.0}
\newcommand{\SimdVsWithoutCyclesSpeedupBaselineDomProcTimeAvgAmean}{0.27449409663677216}
\newcommand{\SimdVsWithoutCyclesSpeedupBaselineDomProcTimeAvgGmean}{}
\newcommand{\SimdVsWithoutCyclesSpeedupBaselineDomProcTimeAvgMedian}{0.21194049715995789}
\newcommand{\SimdVsWithoutCyclesSpeedupBaselineDomProcTimeAvgMin}{0.075448513031005859}
\newcommand{\SimdVsWithoutCyclesSpeedupBaselineDomProcTimeAvgMax}{0.79878449440002441}
\newcommand{\SimdVsWithoutCyclesSpeedupBaselineIllProcTimeAvgAmean}{0.46707802116870878}
\newcommand{\SimdVsWithoutCyclesSpeedupBaselineIllProcTimeAvgGmean}{}
\newcommand{\SimdVsWithoutCyclesSpeedupBaselineIllProcTimeAvgMedian}{0.38655206561088562}
\newcommand{\SimdVsWithoutCyclesSpeedupBaselineIllProcTimeAvgMin}{0.13164490461349487}
\newcommand{\SimdVsWithoutCyclesSpeedupBaselineIllProcTimeAvgMax}{1.2904265522956848}
\newcommand{\SimdVsWithoutCyclesSpeedupBaselineRedunProcTimeAvgAmean}{0.16693266332149506}
\newcommand{\SimdVsWithoutCyclesSpeedupBaselineRedunProcTimeAvgGmean}{}
\newcommand{\SimdVsWithoutCyclesSpeedupBaselineRedunProcTimeAvgMedian}{0.13730928301811218}
\newcommand{\SimdVsWithoutCyclesSpeedupBaselineRedunProcTimeAvgMin}{0.052985489368438721}
\newcommand{\SimdVsWithoutCyclesSpeedupBaselineRedunProcTimeAvgMax}{0.40079480409622192}
\newcommand{\SimdVsWithoutCyclesSpeedupBaselineModelPrepTimeAvgAmean}{40.65325}
\newcommand{\SimdVsWithoutCyclesSpeedupBaselineModelPrepTimeAvgGmean}{}
\newcommand{\SimdVsWithoutCyclesSpeedupBaselineModelPrepTimeAvgMedian}{25.8825}
\newcommand{\SimdVsWithoutCyclesSpeedupBaselineModelPrepTimeAvgMin}{5.8175000000000008}
\newcommand{\SimdVsWithoutCyclesSpeedupBaselineModelPrepTimeAvgMax}{230.31}
\newcommand{\SimdVsWithoutCyclesSpeedupBaselineSolvingTimeAvgAmean}{62.795249999999996}
\newcommand{\SimdVsWithoutCyclesSpeedupBaselineSolvingTimeAvgGmean}{}
\newcommand{\SimdVsWithoutCyclesSpeedupBaselineSolvingTimeAvgMedian}{2.8587499999999997}
\newcommand{\SimdVsWithoutCyclesSpeedupBaselineSolvingTimeAvgMin}{0.3075}
\newcommand{\SimdVsWithoutCyclesSpeedupBaselineSolvingTimeAvgMax}{605.94000000000005}
\newcommand{\SimdVsWithoutCyclesSpeedupBaselinePrePlusSolvingTimeAvgAmean}{63.703754781126975}
\newcommand{\SimdVsWithoutCyclesSpeedupBaselinePrePlusSolvingTimeAvgGmean}{}
\newcommand{\SimdVsWithoutCyclesSpeedupBaselinePrePlusSolvingTimeAvgMedian}{3.6330001497268674}
\newcommand{\SimdVsWithoutCyclesSpeedupBaselinePrePlusSolvingTimeAvgMin}{0.64147972822189325}
\newcommand{\SimdVsWithoutCyclesSpeedupBaselinePrePlusSolvingTimeAvgMax}{608.43000585079199}
\newcommand{\SimdVsWithoutCyclesSpeedupBaselineTotalTimeAvgAmean}{65.454640465151982}
\newcommand{\SimdVsWithoutCyclesSpeedupBaselineTotalTimeAvgGmean}{}
\newcommand{\SimdVsWithoutCyclesSpeedupBaselineTotalTimeAvgMedian}{5.2232746501018674}
\newcommand{\SimdVsWithoutCyclesSpeedupBaselineTotalTimeAvgMin}{1.4929413288099671}
\newcommand{\SimdVsWithoutCyclesSpeedupBaselineTotalTimeAvgMax}{610.51807884129198}
\newcommand{\SimdVsWithoutCyclesSpeedupBaselineCyclesCvAmean}{0.0}
\newcommand{\SimdVsWithoutCyclesSpeedupBaselineCyclesCvGmean}{}
\newcommand{\SimdVsWithoutCyclesSpeedupBaselineCyclesCvMedian}{0.0}
\newcommand{\SimdVsWithoutCyclesSpeedupBaselineCyclesCvMin}{0.0}
\newcommand{\SimdVsWithoutCyclesSpeedupBaselineCyclesCvMax}{0.0}
\newcommand{\SimdVsWithoutCyclesSpeedupBaselineLbCompTimeCvAmean}{0.0}
\newcommand{\SimdVsWithoutCyclesSpeedupBaselineLbCompTimeCvGmean}{}
\newcommand{\SimdVsWithoutCyclesSpeedupBaselineLbCompTimeCvMedian}{0.0}
\newcommand{\SimdVsWithoutCyclesSpeedupBaselineLbCompTimeCvMin}{0.0}
\newcommand{\SimdVsWithoutCyclesSpeedupBaselineLbCompTimeCvMax}{0.0}
\newcommand{\SimdVsWithoutCyclesSpeedupBaselineDomProcTimeCvAmean}{0.0088390070129996477}
\newcommand{\SimdVsWithoutCyclesSpeedupBaselineDomProcTimeCvGmean}{}
\newcommand{\SimdVsWithoutCyclesSpeedupBaselineDomProcTimeCvMedian}{0.0084463856025207978}
\newcommand{\SimdVsWithoutCyclesSpeedupBaselineDomProcTimeCvMin}{0.0012551704229638578}
\newcommand{\SimdVsWithoutCyclesSpeedupBaselineDomProcTimeCvMax}{0.031534778053369197}
\newcommand{\SimdVsWithoutCyclesSpeedupBaselineIllProcTimeCvAmean}{0.0079966488555641511}
\newcommand{\SimdVsWithoutCyclesSpeedupBaselineIllProcTimeCvGmean}{}
\newcommand{\SimdVsWithoutCyclesSpeedupBaselineIllProcTimeCvMedian}{0.008377670752122153}
\newcommand{\SimdVsWithoutCyclesSpeedupBaselineIllProcTimeCvMin}{0.0016235484859931711}
\newcommand{\SimdVsWithoutCyclesSpeedupBaselineIllProcTimeCvMax}{0.013384541759695268}
\newcommand{\SimdVsWithoutCyclesSpeedupBaselineRedunProcTimeCvAmean}{0.015191169082603992}
\newcommand{\SimdVsWithoutCyclesSpeedupBaselineRedunProcTimeCvGmean}{}
\newcommand{\SimdVsWithoutCyclesSpeedupBaselineRedunProcTimeCvMedian}{0.010763069431720864}
\newcommand{\SimdVsWithoutCyclesSpeedupBaselineRedunProcTimeCvMin}{0.0035199124645044948}
\newcommand{\SimdVsWithoutCyclesSpeedupBaselineRedunProcTimeCvMax}{0.038568872430675256}
\newcommand{\SimdVsWithoutCyclesSpeedupBaselineModelPrepTimeCvAmean}{0.0070918084845909262}
\newcommand{\SimdVsWithoutCyclesSpeedupBaselineModelPrepTimeCvGmean}{}
\newcommand{\SimdVsWithoutCyclesSpeedupBaselineModelPrepTimeCvMedian}{0.0070151664830669005}
\newcommand{\SimdVsWithoutCyclesSpeedupBaselineModelPrepTimeCvMin}{0.0014630140841956642}
\newcommand{\SimdVsWithoutCyclesSpeedupBaselineModelPrepTimeCvMax}{0.016242087500600036}
\newcommand{\SimdVsWithoutCyclesSpeedupBaselineSolvingTimeCvAmean}{0.010073600618552229}
\newcommand{\SimdVsWithoutCyclesSpeedupBaselineSolvingTimeCvGmean}{}
\newcommand{\SimdVsWithoutCyclesSpeedupBaselineSolvingTimeCvMedian}{0.0096473120510125188}
\newcommand{\SimdVsWithoutCyclesSpeedupBaselineSolvingTimeCvMin}{0.00012994698937204513}
\newcommand{\SimdVsWithoutCyclesSpeedupBaselineSolvingTimeCvMax}{0.035001599823423823}
\newcommand{\SimdVsWithoutCyclesSpeedupBaselinePrePlusSolvingTimeCvAmean}{0.0078539713893773823}
\newcommand{\SimdVsWithoutCyclesSpeedupBaselinePrePlusSolvingTimeCvGmean}{}
\newcommand{\SimdVsWithoutCyclesSpeedupBaselinePrePlusSolvingTimeCvMedian}{0.0074779345561491002}
\newcommand{\SimdVsWithoutCyclesSpeedupBaselinePrePlusSolvingTimeCvMin}{0.00010275572347979443}
\newcommand{\SimdVsWithoutCyclesSpeedupBaselinePrePlusSolvingTimeCvMax}{0.025293951796409291}
\newcommand{\SimdVsWithoutCyclesSpeedupBaselineTotalTimeCvAmean}{0.0052431350333524722}
\newcommand{\SimdVsWithoutCyclesSpeedupBaselineTotalTimeCvGmean}{}
\newcommand{\SimdVsWithoutCyclesSpeedupBaselineTotalTimeCvMedian}{0.0039749226545396834}
\newcommand{\SimdVsWithoutCyclesSpeedupBaselineTotalTimeCvMin}{0.00010312119011845932}
\newcommand{\SimdVsWithoutCyclesSpeedupBaselineTotalTimeCvMax}{0.016181753471245022}
\newcommand{\SimdVsWithoutCyclesSpeedupCyclesZeroCenteredSpeedupAmean}{n/a}
\newcommand{\SimdVsWithoutCyclesSpeedupCyclesZeroCenteredSpeedupGmean}{n/a}
\newcommand{\SimdVsWithoutCyclesSpeedupCyclesZeroCenteredSpeedupMedian}{0.0}
\newcommand{\SimdVsWithoutCyclesSpeedupCyclesZeroCenteredSpeedupMin}{-0.0}
\newcommand{\SimdVsWithoutCyclesSpeedupCyclesZeroCenteredSpeedupMax}{0.11790367269940459}
\newcommand{\SimdVsWithoutCyclesSpeedupCyclesRegularSpeedupAmean}{n/a}
\newcommand{\SimdVsWithoutCyclesSpeedupCyclesRegularSpeedupGmean}{1.0161049791727153}
\newcommand{\SimdVsWithoutCyclesSpeedupCyclesRegularSpeedupMedian}{1.0}
\newcommand{\SimdVsWithoutCyclesSpeedupCyclesRegularSpeedupMin}{1.0}
\newcommand{\SimdVsWithoutCyclesSpeedupCyclesRegularSpeedupMax}{1.1179036726994045}
\newcommand{\SimdVsWithoutCyclesSpeedupCyclesRegularSpeedupCiAmean}{n/a}
\newcommand{\SimdVsWithoutCyclesSpeedupCyclesRegularSpeedupCiGmean}{n/a}
\newcommand{\SimdVsWithoutCyclesSpeedupCyclesRegularSpeedupCiMedian}{n/a}
\newcommand{\SimdVsWithoutCyclesSpeedupCyclesRegularSpeedupCiMin}{1.0034477975382945}
\newcommand{\SimdVsWithoutCyclesSpeedupCyclesRegularSpeedupCiMax}{1.0316245985704637}


\begin{figure}
  \centering%
  \maxsizebox{\textwidth}{!}{%
    \trimBarchartPlot{%
      \begin{tikzpicture}[gnuplot]
%% generated with GNUPLOT 5.0p4 (Lua 5.2; terminal rev. 99, script rev. 100)
%% lör  3 feb 2018 16:43:20
\path (0.000,0.000) rectangle (12.500,8.750);
\gpcolor{rgb color={0.753,0.753,0.753}}
\gpsetlinetype{gp lt axes}
\gpsetdashtype{gp dt axes}
\gpsetlinewidth{0.50}
\draw[gp path] (1.380,1.779)--(36.945,1.779);
\gpcolor{color=gp lt color border}
\node[gp node right] at (1.380,1.779) {\plotZCNormTics{0}};
\gpcolor{rgb color={0.753,0.753,0.753}}
\draw[gp path] (1.380,2.879)--(36.945,2.879);
\gpcolor{color=gp lt color border}
\node[gp node right] at (1.380,2.879) {\plotZCNormTics{0.02}};
\gpcolor{rgb color={0.753,0.753,0.753}}
\draw[gp path] (1.380,3.980)--(36.945,3.980);
\gpcolor{color=gp lt color border}
\node[gp node right] at (1.380,3.980) {\plotZCNormTics{0.04}};
\gpcolor{rgb color={0.753,0.753,0.753}}
\draw[gp path] (1.380,5.080)--(36.945,5.080);
\gpcolor{color=gp lt color border}
\node[gp node right] at (1.380,5.080) {\plotZCNormTics{0.06}};
\gpcolor{rgb color={0.753,0.753,0.753}}
\draw[gp path] (1.380,6.180)--(36.945,6.180);
\gpcolor{color=gp lt color border}
\node[gp node right] at (1.380,6.180) {\plotZCNormTics{0.08}};
\gpcolor{rgb color={0.753,0.753,0.753}}
\draw[gp path] (1.380,7.281)--(36.945,7.281);
\gpcolor{color=gp lt color border}
\node[gp node right] at (1.380,7.281) {\plotZCNormTics{0.1}};
\gpcolor{rgb color={0.753,0.753,0.753}}
\draw[gp path] (1.380,8.381)--(36.945,8.381);
\gpcolor{color=gp lt color border}
\node[gp node right] at (1.380,8.381) {\plotZCNormTics{0.12}};
\node[gp node left,rotate=-30] at (3.110,1.534) {\functionName{color_cmyk_to_r.}};
\node[gp node left,rotate=-30] at (4.803,1.534) {\functionName{debug_print_str.}};
\node[gp node left,rotate=-30] at (6.497,1.534) {\functionName{delete_contours}};
\node[gp node left,rotate=-30] at (8.190,1.534) {\functionName{fill_input_buff.}};
\node[gp node left,rotate=-30] at (9.884,1.534) {\functionName{free_new_ctrl}};
\node[gp node left,rotate=-30] at (11.577,1.534) {\functionName{gl_read_alpha_s.}};
\node[gp node left,rotate=-30] at (13.271,1.534) {\functionName{gluBeginPolygon}};
\node[gp node left,rotate=-30] at (14.965,1.534) {\functionName{gx_curve_cursor.}};
\node[gp node left,rotate=-30] at (16.658,1.534) {\functionName{inflate_block}};
\node[gp node left,rotate=-30] at (18.352,1.534) {\functionName{is_compromised}};
\node[gp node left,rotate=-30] at (20.045,1.534) {\functionName{jinit_inverse_d.}};
\node[gp node left,rotate=-30] at (21.739,1.534) {\functionName{jpeg_finish_out.}};
\node[gp node left,rotate=-30] at (23.432,1.534) {\functionName{jpeg_stdio_src}};
\node[gp node left,rotate=-30] at (25.126,1.534) {\functionName{motion_vector}};
\node[gp node left,rotate=-30] at (26.820,1.534) {\functionName{mp_dmul}};
\node[gp node left,rotate=-30] at (28.513,1.534) {\functionName{mp_shortmod}};
\node[gp node left,rotate=-30] at (30.207,1.534) {\functionName{pack_tree_iter}};
\node[gp node left,rotate=-30] at (31.900,1.534) {\functionName{pbm_getint}};
\node[gp node left,rotate=-30] at (33.594,1.534) {\functionName{post_process_pr.}};
\node[gp node left,rotate=-30] at (35.287,1.534) {\functionName{zero}};
\gpsetlinetype{gp lt border}
\gpsetdashtype{gp dt solid}
\gpsetlinewidth{1.00}
\draw[gp path] (1.380,8.381)--(1.380,1.779)--(36.945,1.779)--(36.945,8.381)--cycle;
\gpcolor{rgb color={0.000,0.000,0.000}}
\draw[gp path] (1.380,1.779)--(1.739,1.779)--(2.098,1.779)--(2.458,1.779)--(2.817,1.779)%
  --(3.176,1.779)--(3.535,1.779)--(3.895,1.779)--(4.254,1.779)--(4.613,1.779)--(4.972,1.779)%
  --(5.332,1.779)--(5.691,1.779)--(6.050,1.779)--(6.409,1.779)--(6.769,1.779)--(7.128,1.779)%
  --(7.487,1.779)--(7.846,1.779)--(8.206,1.779)--(8.565,1.779)--(8.924,1.779)--(9.283,1.779)%
  --(9.643,1.779)--(10.002,1.779)--(10.361,1.779)--(10.720,1.779)--(11.080,1.779)--(11.439,1.779)%
  --(11.798,1.779)--(12.157,1.779)--(12.517,1.779)--(12.876,1.779)--(13.235,1.779)--(13.594,1.779)%
  --(13.953,1.779)--(14.313,1.779)--(14.672,1.779)--(15.031,1.779)--(15.390,1.779)--(15.750,1.779)%
  --(16.109,1.779)--(16.468,1.779)--(16.827,1.779)--(17.187,1.779)--(17.546,1.779)--(17.905,1.779)%
  --(18.264,1.779)--(18.624,1.779)--(18.983,1.779)--(19.342,1.779)--(19.701,1.779)--(20.061,1.779)%
  --(20.420,1.779)--(20.779,1.779)--(21.138,1.779)--(21.498,1.779)--(21.857,1.779)--(22.216,1.779)%
  --(22.575,1.779)--(22.935,1.779)--(23.294,1.779)--(23.653,1.779)--(24.012,1.779)--(24.372,1.779)%
  --(24.731,1.779)--(25.090,1.779)--(25.449,1.779)--(25.808,1.779)--(26.168,1.779)--(26.527,1.779)%
  --(26.886,1.779)--(27.245,1.779)--(27.605,1.779)--(27.964,1.779)--(28.323,1.779)--(28.682,1.779)%
  --(29.042,1.779)--(29.401,1.779)--(29.760,1.779)--(30.119,1.779)--(30.479,1.779)--(30.838,1.779)%
  --(31.197,1.779)--(31.556,1.779)--(31.916,1.779)--(32.275,1.779)--(32.634,1.779)--(32.993,1.779)%
  --(33.353,1.779)--(33.712,1.779)--(34.071,1.779)--(34.430,1.779)--(34.790,1.779)--(35.149,1.779)%
  --(35.508,1.779)--(35.867,1.779)--(36.227,1.779)--(36.586,1.779)--(36.945,1.779);
\gpfill{rgb color={0.333,0.333,0.333}} (4.697,1.779)--(5.403,1.779)--(5.403,3.233)--(4.697,3.233)--cycle;
\gpcolor{color=gp lt color border}
\draw[gp path] (4.697,1.779)--(4.697,3.232)--(5.402,3.232)--(5.402,1.779)--cycle;
\gpfill{rgb color={0.333,0.333,0.333}} (11.471,1.779)--(12.178,1.779)--(12.178,3.707)--(11.471,3.707)--cycle;
\draw[gp path] (11.471,1.779)--(11.471,3.706)--(12.177,3.706)--(12.177,1.779)--cycle;
\gpfill{rgb color={0.333,0.333,0.333}} (14.858,1.779)--(15.565,1.779)--(15.565,4.409)--(14.858,4.409)--cycle;
\draw[gp path] (14.858,1.779)--(14.858,4.408)--(15.564,4.408)--(15.564,1.779)--cycle;
\gpfill{rgb color={0.333,0.333,0.333}} (26.713,1.779)--(27.420,1.779)--(27.420,8.267)--(26.713,8.267)--cycle;
\draw[gp path] (26.713,1.779)--(26.713,8.266)--(27.419,8.266)--(27.419,1.779)--cycle;
\gpfill{rgb color={0.333,0.333,0.333}} (35.181,1.779)--(35.888,1.779)--(35.888,7.310)--(35.181,7.310)--cycle;
\draw[gp path] (35.181,1.779)--(35.181,7.309)--(35.887,7.309)--(35.887,1.779)--cycle;
\node[gp node center] at (3.350,1.963) {\plotBarNormValue{-0.000000}};
\node[gp node center] at (5.043,3.416) {\plotBarNormValue{0.026414}};
\node[gp node center] at (6.737,1.963) {\plotBarNormValue{-0.000000}};
\node[gp node center] at (8.430,1.963) {\plotBarNormValue{-0.000000}};
\node[gp node center] at (10.124,1.963) {\plotBarNormValue{-0.000000}};
\node[gp node center] at (11.817,3.890) {\plotBarNormValue{0.035018}};
\node[gp node center] at (13.511,1.963) {\plotBarNormValue{-0.000000}};
\node[gp node center] at (15.205,4.592) {\plotBarNormValue{0.047793}};
\node[gp node center] at (16.898,1.963) {\plotBarNormValue{-0.000000}};
\node[gp node center] at (18.592,1.963) {\plotBarNormValue{-0.000000}};
\node[gp node center] at (20.285,1.963) {\plotBarNormValue{-0.000000}};
\node[gp node center] at (21.979,1.963) {\plotBarNormValue{-0.000000}};
\node[gp node center] at (23.672,1.963) {\plotBarNormValue{-0.000000}};
\node[gp node center] at (25.366,1.963) {\plotBarNormValue{-0.000000}};
\node[gp node center] at (27.060,8.450) {\plotBarNormValue{0.117904}};
\node[gp node center] at (28.753,1.963) {\plotBarNormValue{-0.000000}};
\node[gp node center] at (30.447,1.963) {\plotBarNormValue{-0.000000}};
\node[gp node center] at (32.140,1.963) {\plotBarNormValue{-0.000000}};
\node[gp node center] at (33.834,1.963) {\plotBarNormValue{-0.000000}};
\node[gp node center] at (35.527,7.493) {\plotBarNormValue{0.100507}};
\node[gp node center] at (18.592,1.994) {\plotSubOptSymbol};
\draw[gp path] (1.380,8.381)--(1.380,1.779)--(36.945,1.779)--(36.945,8.381)--cycle;
%% coordinates of the plot area
\gpdefrectangularnode{gp plot 1}{\pgfpoint{1.380cm}{1.779cm}}{\pgfpoint{36.945cm}{8.381cm}}
\end{tikzpicture}
%% gnuplot variables
%
    }%
  }

  \caption[Plot for evaluating the impact of SIMD instructions on code quality]%
          {%
            Normalized solution costs for two pattern sets: one without SIMD
            instructions (baseline), and another with such instruction
            (subject).
            %
            GMI:~\printGMI{%
              \SimdVsWithoutCyclesSpeedupCyclesRegularSpeedupGmean%
            },
            CI:~\printGMICI{%
              \SimdVsWithoutCyclesSpeedupCyclesRegularSpeedupCiMin%
            }{%
              \SimdVsWithoutCyclesSpeedupCyclesRegularSpeedupCiMax%
            }.
            %
            \Glspl{function} whose bars are marked with two dots are those
            for which the \gls{subject} fails to find the optimal solution%
          }
  \labelFigure{simd-vs-without-cycles-plot}
\end{figure}

\RefFigure{simd-vs-without-cycles-plot} shows the normalized \gls{solution}
costs for the two \glspl{pattern set} describe above, with \gls{pattern
  set}~\patternSetA{} as \gls{baseline} and \gls{pattern set}~\patternSetB{} as
\gls{subject}.
%
The costs range from
\printMinCycles{%
  \SimdVsWithoutCyclesSpeedupCyclesAvgMin,
  \SimdVsWithoutCyclesSpeedupBaselineCyclesAvgMin
} to
\printMaxCycles{%
  \SimdVsWithoutCyclesSpeedupCyclesAvgMax,
  \SimdVsWithoutCyclesSpeedupBaselineCyclesAvgMax
}.
%
As the \gls{GMI} is \printGMI{%
  \SimdVsWithoutCyclesSpeedupCyclesRegularSpeedupGmean%
} with \gls{CI}~\printGMICI{%
  \SimdVsWithoutCyclesSpeedupCyclesRegularSpeedupCiMin%
}{%
  \SimdVsWithoutCyclesSpeedupCyclesRegularSpeedupCiMax%
}, we see that the \gls{pattern set}~\patternSetB{} yields \glspl{solution} with
significantly lesser cost than those yielded by \glsshort{pattern
  set}~\patternSetA.
%
The five cases with lesser cost ({\codeFont debug\_print\_str}, {\codeFont
  gl\_read\_alpha}, {\codeFont gx\_curve\_cursor}, {\codeFont mp\_dmul}, and
{\codeFont zero}), our approach is able to combine pairs of additions or
subtractions into \num{2}-way \gls{SIMD.instr} \glspl{instruction}.
%
One one of these cases ({\codeFont gl\_read\_alpha}) the additions originally
reside in different \glspl{block}, but due to \gls{global code motion} our
approach is able to move the operations to the same block in order to implement
these using a single \gls{instruction}.
%
Hence we conclude that there is sufficient data parallelism to be exploited
through selection of \gls{SIMD.instr} \glspl{instruction}, resulting in
significantly better code quality, and that this exploitation is benefitted from
\gls{global code motion}.


\section{Summary}
\labelSection{comparison-summary}

\todo{write}

