% Copyright (c) 2017, Gabriel Hjort Blindell <ghb@kth.se>
%
% This work is licensed under a Creative Commons 4.0 International License (see
% LICENSE file or visit <http://creativecommons.org/licenses/by/4.0/> for a copy
% of the license).

%======
% TEXT
%======

% Builds \labelX, \refX, \RefX, \refXRange, \RefXRange, \refPageOfX, and
% \RefPageOfX commands:
%    #1: The identifier (X) in singular
%    #2: Plural identifier term
%    #3: Short term to appear in reference (without ending dot)
\makeatletter
\newcounter{refListLength}
\NewDocumentCommand{\buildLabelRefCommands}{mom}{%
  \expandafter\newcommand\csname label#1\endcsname[1]{\label{#1:##1}}%
  \expandafter\NewDocumentCommand\csname @ref#1\endcsname{smm}{%
    \IfBooleanF{##1}{##3}%
    \ref{#1:##2}%
  }%
  \expandafter\NewDocumentCommand\csname ref#1\endcsname{sm}{%
    \def\refcmd{\csname @ref#1\endcsname}%
    \def\prefix{#3.\thinspace}%
    \IfBooleanTF{##1}{\refcmd*{##2}{\prefix}}%
                     {\refcmd{##2}{\prefix}}%
  }%
  \expandafter\NewDocumentCommand\csname Ref#1\endcsname{sm}{%
    \def\refcmd{\csname @ref#1\endcsname}%
    \def\prefix{#1~}%
    \IfBooleanTF{##1}{\refcmd*{##2}{\prefix}}%
                     {\refcmd{##2}{\prefix}}%
  }%
  \expandafter\NewDocumentCommand\csname @ref#1List\endcsname{smm}{%
    \setcounter{refListLength}{0}%
    \foreach \r in {##2} {\stepcounter{refListLength}}%
    \edef\numRefs{\arabic{refListLength}}%
    \IfBooleanF{##1}{##3}%
    \foreach \r [count=\ri] in {##2} {%
      \ifnum \ri=\numRefs
        \ifnum \numRefs>1
          and~%
        \fi%
      \fi%
      \ref{#1:\r}%
      \ifnum \ri<\numRefs
        \ifnum \numRefs>2
          ,%
        \fi
        \ %
      \fi%
    }%
  }%
  \expandafter\NewDocumentCommand\csname ref#1List\endcsname{sm}{%
    \def\refcmd{\csname @ref#1List\endcsname}%
    \def\prefix{#3s.\thinspace}%
    \IfBooleanTF{##1}{\refcmd*{##2}{\prefix}}%
                     {\refcmd{##2}{\prefix}}%
  }%
  \expandafter\NewDocumentCommand\csname Ref#1List\endcsname{sm}{%
    \def\refcmd{\csname @ref#1List\endcsname}%
    \def\prefix{\IfValueTF{#2}{#2}{#1s}~}%
    \IfBooleanTF{##1}{\refcmd*{##2}{\prefix}}%
                     {\refcmd{##2}{\prefix}}%
  }%
  \expandafter\NewDocumentCommand\csname @ref#1Range\endcsname{smmm}{%
    \IfBooleanF{##1}{##4}%
    \mbox{\ref{#1:##2}--\ref{#1:##3}}%
  }%
  \expandafter\NewDocumentCommand\csname ref#1Range\endcsname{smm}{%
    \def\refcmd{\csname @ref#1Range\endcsname}%
    \def\prefix{#3.\thinspace}%
    \IfBooleanTF{##1}{\refcmd*{##2}{##3}{\prefix}}%
                     {\refcmd{##2}{##3}{\prefix}}%
  }%
  \expandafter\NewDocumentCommand\csname Ref#1Range\endcsname{smm}{%
    \def\refcmd{\csname @ref#1Range\endcsname}%
    \def\prefix{\IfValueTF{#2}{#2}{#1s}~}%
    \IfBooleanTF{##1}{\refcmd*{##2}{##3}{\prefix}}%
                     {\refcmd{##2}{##3}{\prefix}}%
  }%
  \expandafter\NewDocumentCommand\csname @refPageOf#1\endcsname{smm}{%
    \IfBooleanF{##1}{##3}%
    \pageref{#1:##2}%
  }%
  \expandafter\NewDocumentCommand\csname refPageOf#1\endcsname{sm}{%
    \def\refcmd{\csname @refPageOf#1\endcsname}%
    \def\prefix{p.\thinspace}%
    \IfBooleanTF{##1}{\refcmd*{##2}{\prefix}}%
                     {\refcmd{##2}{\prefix}}%
  }%
  \expandafter\NewDocumentCommand\csname RefPageOf#1\endcsname{sm}{%
    \def\refcmd{\csname @refPageOf#1\endcsname}%
    \def\prefix{Page~}%
    \IfBooleanTF{##1}{\refcmd*{##2}{\prefix}}%
                     {\refcmd{##2}{\prefix}}%
  }%
  \expandafter\NewDocumentCommand\csname @refPagesOf#1List\endcsname{smm}{%
    \setcounter{refListLength}{0}%
    \foreach \r in {##2} {\stepcounter{refListLength}}%
    \edef\numRefs{\arabic{refListLength}}%
    \IfBooleanF{##1}{##3}%
    \foreach \r [count=\ri] in {##2} {%
      \ifnum \ri=\numRefs
        \ifnum \numRefs>1
          and~%
        \fi%
      \fi%
      \pageref{#1:\r}%
      \ifnum \ri<\numRefs
        \ifnum \numRefs>2
          ,%
        \fi
        \ %
      \fi%
    }%
  }%
  \expandafter\NewDocumentCommand\csname refPagesOf#1List\endcsname{sm}{%
    \def\refcmd{\csname @refPagesOf#1List\endcsname}%
    \def\prefix{pp.\thinspace}%
    \IfBooleanTF{##1}{\refcmd*{##2}{\prefix}}%
                     {\refcmd{##2}{\prefix}}%
  }%
  \expandafter\NewDocumentCommand\csname RefPagesOf#1List\endcsname{sm}{%
    \def\refcmd{\csname @refPagesOf#1List\endcsname}%
    \def\prefix{Pages~}%
    \IfBooleanTF{##1}{\refcmd*{##2}{\prefix}}%
                     {\refcmd{##2}{\prefix}}%
  }%
  \expandafter\NewDocumentCommand\csname nameref#1\endcsname{sm}{%
    \def\refcmd{\csname ref#1\endcsname}%
    \IfBooleanF{##1}{\refcmd*{##2}.\thinspace}%
    \nameref{#1:##2}%
  }%
}
\makeatother

% Build labeling and reference commands
\buildLabelRefCommands{Algorithm}{Alg}
\buildLabelRefCommands{Appendix}[Appendices]{Ap}
\buildLabelRefCommands{Chapter}{Chap}
\buildLabelRefCommands{Definition}{Def}
\buildLabelRefCommands{Equation}{Eq}
\buildLabelRefCommands{Figure}{Fig}
\buildLabelRefCommands{Section}{Sect}
\buildLabelRefCommands{Table}{Tab}

% Macro for labeling and references pages
\newcommand{\labelPage}[1]{\label{page:#1}}
\newcommand{\refPage}[1]{p.\thinspace\pageref{page:#1}}

% Other text commands
\NewDocumentCommand{\etal}{s}{%
  et al%
  \IfBooleanF{#1}{.\ }%
}
\NewDocumentCommand{\etc}{s}{%
  etc%
  \IfBooleanF{#1}{.\ }%
}
\NewDocumentCommand{\st}{s}{%
  s.t%
  \IfBooleanF{#1}{.\ }%
}
\NewDocumentCommand{\versus}{s}{%
  vs%
  \IfBooleanF{#1}{.\ }%
}
\NewDocumentCommand{\wrt}{s}{%
  w.r.t%
  \IfBooleanF{#1}{.\ }%
}
\newcommand{\todo}[1]{%
  \textcolor{black!25!red}{\textsc{todo}:~#1}%
}
\newcommand{\toolFont}[1]{\textsc{#1}}

% Enable bold version of the monotype font
% See http://www.macfreek.nl/memory/LaTeX_Bold_Typewriter_Font
\DeclareFontShape{OT1}{cmtt}{bx}{n}%
  {<5><6><7><8><9><10><10.95><12><14.4><17.28><20.74><24.88>cmttb10}{}

% For highlighting text
% See https://tex.stackexchange.com/a/74469/2634
\newcommand{\hlDiffColor}{black!25}
\newcommand{\hlStrut}{%
  \vrule width 0pt height .9\ht\strutbox depth .9\dp\strutbox\relax%
}
\NewDocumentCommand{\hlDiff}{mo}{%
  \IfValueT{#2}{\hspace{#2}}%
  \begingroup%
  \setlength{\fboxsep}{0pt}%
  \ifmmode%
    \colorbox{\hlDiffColor}{\hlStrut$#1$\/}%
  \else%
    \colorbox{\hlDiffColor}{\hlStrut#1\/}%
  \fi%
  \endgroup%
  \IfValueT{#2}{\hspace{#2}}%
}

\NewDocumentEnvironment{statement}{}{%
  \begin{list}{}{}
    \item\itshape%
}{%
  \end{list}%
}


%=======
% LISTS
%=======

\setlist{noitemsep}

\setlist[itemize]{%
  label=\raisebox{1pt}{\rule{4pt}{4pt}},
}

\newlist{inlinelist}{enumerate*}{1}
\setlist[inlinelist]{%
  label={},
  itemjoin={},
}

\newlist{contributions}{enumerate}{2}
\setlist[contributions,1]{%
  label=C\arabic*,
  labelsep=8pt,
}
\setlist[contributions,2]{%
  label=\alph*.,
  ref=\alph*,
  topsep=0pt,
}

\NewDocumentCommand{\labelContribution}{m}{%
  \label{cont:#1}%
}
\NewDocumentCommand{\refContribution}{m}{%
  \ref{cont:#1}%
}

\newlist{publications}{enumerate}{1}
\setlist[publications]{%
  label=P\arabic*,
  labelsep=8pt,
  itemsep=.5\baselineskip,
}

\NewDocumentCommand{\labelPublication}{m}{%
  \label{pub:#1}%
}
\NewDocumentCommand{\refPublication}{m}{%
  \ref{pub:#1}%
}

\newlist{requirements}{enumerate}{1}
\setlist[requirements]{%
  label=R\arabic*,
  labelsep=8pt,
}

\NewDocumentCommand{\labelRequirement}{m}{%
  \label{req:#1}%
}
\NewDocumentCommand{\refRequirement}{m}{%
  \ref{req:#1}%
}

\newlist{modelList}{enumerate*}{1}
\setlist[modelList]{%
  label={(\textsc{\roman*})},
  ref=\textsc{\roman*},
  itemjoin={;\ },
  itemjoin*={; and\ },
}

\NewDocumentCommand{\labelModel}{m}{%
  \label{model:#1}%
}
\NewDocumentCommand{\refModel}{m}{%
  \ref{model:#1}%
}

\newlist{patternList}{enumerate*}{1}
\setlist[patternList]{%
  label={(\textsc{\roman*})},
  ref=\textsc{\roman*},
  itemjoin={;\ },
  itemjoin*={; and\ },
}

\NewDocumentCommand{\labelPattern}{m}{%
  \label{pattern:#1}%
}
\NewDocumentCommand{\refPattern}{m}{%
  \ref{pattern:#1}%
}

\setlist[description]{%
  font=\sffamily,
}


%==============
% PUBLICATIONS
%==============

\NewDocumentEnvironment{authorsContribution}{}{%

  \vspace{.5\baselineskip}
  {\sffamily\bfseries Contribution}\quad%
}{%
}


%=========
% FIGURES
%=========

\newlength{\betweensubfigures}
\setlength{\betweensubfigures}{\baselineskip}

% Declares a paragraph where a figure will be inlined with the text. Commands
% for the figure itself must be as they would appear within a 'figure'
% environment (however, do not explicitly declare such an environment as it will
% taken care of by the macro.
%
% Attributes for controlling the position of the figure are 'l' (always on the
% left), 'r' (always on the right), and 'p' (right if the page is odd and left
% if the page is even).
%
% It might happen that the paragraph is split over two pages. In such cases it
% is necessary to manually break the paragraph into two but allowing the last
% line of the first paragraph to fill the entire width of the page. This is done
% by appending
%
%      {\parfillskip=0pt\relax\par}
%
% to the first paragraph.
%
% Arguments:
%    #1: Width of figure.
%    #2: Positioning attribute: l, r, or p. Optional (default is p).
%    #3: Number-of-lines offset from the top of the paragraph. Optional (default
%        is 0).
\newbool{isInsideInParFigure}
\newtoggle{positionLeft}
\newlength{\leftParOffset}
\newlength{\rightParOffset}
\newsavebox{\inParFigureBox}
\newlength{\inParFigureHeight}
\newcounter{totalArgs}
\NewDocumentEnvironment{inParFigure}{mO{p}O{0}}{%
  \inParFigureX{#1}{#2}{#3}%
}{%
  \endinParFigureX%
}
\NewEnviron{inParFigureX}[3]{%
  % Check position attribute
  \ifthenelse{\equal{#2}{l}}{%
    \global\toggletrue{positionLeft}%
  }{}%
  \ifthenelse{\equal{#2}{r}}{%
    \global\togglefalse{positionLeft}%
  }{}%
  \ifthenelse{\equal{#2}{p}}{%
    \checkoddpage
    \ifthenelse{\boolean{oddpage}}{%
      \global\togglefalse{positionLeft}
    }{%
      \global\toggletrue{positionLeft}
    }%
  }{}%
  \ifthenelse{\equal{#2}{l} \OR \equal{#2}{r} \OR \equal{#2}{p}}{}{%
    \PackageError{}{Position argument must be either 'l', 'r', or 'p'!}%
  }%

  % Make minipage to contain figure
  \savebox{\inParFigureBox}{%
    \begin{minipage}{#1}%
      \booltrue{isInsideInParFigure}%
      \iftoggle{positionLeft}{}{%
        \mbox{}\hfill%
      }%
      \parbox{\widthof{\BODY}}{\BODY}%
      \boolfalse{isInsideInParFigure}%
    \end{minipage}%
  }%
  \setlength{\inParFigureHeight}{\ht\inParFigureBox+\dp\inParFigureBox}%

  % Put figure as a zero-width entity with depth such that it will be placed at
  % desired position within paragraph
  \setlength{\parindent}{0pt}%
  \iftoggle{positionLeft}{}{%
    \hfill%
  }%
  \adjustbox{%
    raise={-\height+\inParFigureHeightTweak\baselineskip-#3\baselineskip}%
          {0pt}%
          {0pt}%
  }{%
    \usebox{\inParFigureBox}%
  }%
  \vspace{-\baselineskip}% Undo line skip
  \par%

  % Calculate height of minipage in number of text lines
  \pgfmathtruncatemacro{\numLinesPGF}{ceil(\inParFigureHeight/\baselineskip)}%
  \global\edef\numLines{\numLinesPGF}%

  % Make paragraph shape to allow figure to be inlined
  \gdef\makeParShape{%
    \iftoggle{positionLeft}{%
      \setlength{\leftParOffset}{#1+\inParFigureTextPadding}%
      \setlength{\rightParOffset}{\textwidth-\leftParOffset}%
    }{%
      \setlength{\leftParOffset}{0pt}%
      \setlength{\rightParOffset}{\textwidth-#1-\inParFigureTextPadding}%
    }%
    \setcounter{totalArgs}{#3+\numLines+1}%
    \parshape = \value{totalArgs}%
      \MyRepeat{#3}{0pt \textwidth}%
      \MyRepeat{\numLines}{\leftParOffset \rightParOffset}%
      0pt \textwidth%
  }%
}
\AfterEndEnvironment{inParFigure}{%
  \makeParShape%
}
\NewDocumentCommand{\inParFigureTextPadding}{}{1.5em}
\NewDocumentCommand{\inParFigureHeightTweak}{}{0.55}

% Command for repeating something n times.
\makeatletter
\newcommand{\MyRepeat}[1]{%
  \expandafter\@MyRepeat\expandafter{\the\numexpr #1\relax}%
}
\def\@MyRepeat#1{%
  \ifnum#1>0
    \expandafter\@@MyRepeat\expandafter{%
      \the\numexpr #1-1\expandafter\relax\expandafter%
    }%
  \else%
    \expandafter\@gobble%
  \fi%
}
\def\@@MyRepeat#1#2{%
  \@MyRepeat{#1}{#2}#2%
}
\makeatother


%========
% TABLES
%========

\newcommand{\tabhead}{\sffamily\bfseries}
\newcommand{\mathtabhead}[1]{\bm{#1}}


%======
% MATH
%======

% Define 'definition' environment
\newtheoremstyle{def}%
                {\topsep}%
                {\topsep}%
                {}%
                {}%
                {\sffamily\relsize{-0.8}\bfseries}%
                {}%
                {1em}%
                {%
                  \thmname{#1}%
                  \thmnumber{ #2}%
                  \thmnote{ -- %
                    \setboolean{consFontForceTextRM}{false}%
                    #3%
                    \setboolean{consFontForceTextRM}{true}%
                  }%
                }
\theoremstyle{def}
\newtheorem{definition}{Definition}[chapter]

% Reduce vertical spacing of displaymath inside a 'definition' environment
\AtBeginEnvironment{definition}{%
  \setlength{\abovedisplayskip}{5pt plus 2pt minus 2pt}%
  \setlength{\belowdisplayskip}{5pt plus 2pt minus 2pt}%
}

% General commands
\newcommand{\mathsc}[1]{\textrm{\textsc{#1}}}
\newcommand{\overbar}[1]{
  \mkern 1.5mu\overline{\mkern-1.5mu#1\mkern-1.5mu}\mkern 1.5mu
}
\NewDocumentCommand{\mPowerset}{m}{
  2^{#1}
}
\newcommand{\transp}{\mathsf{T}}
\newcommand{\mNatNumSet}{\mathbb{N}}
\newcommand{\mPhi}{\varphi}
\newcommand{\mSigma}{\Sigma}
\NewDocumentCommand{\mSequence}{sm}{
  \IfBooleanT{#1}{\left}\langle
    #2
  \IfBooleanT{#1}{\right}\rangle
}
\NewDocumentCommand{\mSet}{sm}{
  \IfBooleanT{#1}{\left}\{
    #2
  \IfBooleanT{#1}{\right}\}
}
\def\mSetSep{\mid}%
\NewDocumentCommand{\mSetBuilder}{smm}{
  \IfBooleanTF{#1}{
    \left\{
    \begin{array}{@{}c|c@{}}
      #2 & #3
    \end{array}
    \right\}
  }{
    \mSet{#2 \mSetSep #3}
  }
}
\NewDocumentCommand{\mCard}{sm}{
  \IfBooleanT{#1}{\left}|
    #2
  \IfBooleanT{#1}{\right}|
}
\newcommand{\mEmptySet}{\varnothing}
\newcommand{\mFunFont}[1]{\mathit{#1}}
\NewDocumentCommand{\mEdge}{mm}{
  #1 \rightarrow #2
}
\NewDocumentCommand{\mPair}{smm}{
  \IfBooleanT{#1}{\left}(
    #2, #3
  \IfBooleanT{#1}{\right})
}
\NewDocumentCommand{\mTuple}{sm}{
  \IfBooleanT{#1}{\left}\langle
    #2
  \IfBooleanT{#1}{\right}\rangle
}
\newcommand{\mAnd}{\wedge}
\newcommand{\mBigAnd}{\bigwedge}
\newcommand{\mOr}{\vee}
\newcommand{\mImp}{\Rightarrow}
\newcommand{\mEq}{\Leftrightarrow}
\newcommand{\mNot}{\neg}
\newcommand{\mFunDecl}[3]{
  #1 : #2 \rightarrow #3
}
\newcommand{\mMatrix}[1]{\mathit{#1}}
\newcommand{\mVector}[1]{\vec{#1}}
\newcommand{\mBigO}{O}
\DeclareMathOperator{\mInEdgeNr}{\mFunFont{in}}
\DeclareMathOperator{\mOutEdgeNr}{\mFunFont{out}}
\DeclareMathOperator{\mMax}{\mFunFont{max}}
\DeclareMathOperator{\mMin}{\mFunFont{min}}
\DeclareMathOperator{\mPred}{\mFunFont{pred}}
\DeclareMathOperator{\mRank}{\mFunFont{rank}}
\DeclareMathOperator{\mSucc}{\mFunFont{succ}}
\DeclareMathOperator{\mSource}{\mFunFont{source}}
\DeclareMathOperator{\mTarget}{\mFunFont{target}}
\DeclareMathOperator{\mType}{\mFunFont{type}}

\newcommand{\mNodeMatchRel}{\backsimeq}
\newcommand{\mVFTwoAttrCmp}{\mNodeMatchRel}

\newcommand{\mAnchor}{\bullet}
\newlength{\boxSize}
\setlength{\boxSize}{1.08ex}
\newlength{\boxLineWidth}
\setlength{\boxLineWidth}{.13ex}
\newlength{\hexagonEdgeLength}
\pgfmathsetlengthmacro{\hexagonEdgeLength}{\boxSize/(2*sin(60))}
\newcommand{\mBox}{%
  \tikz%
  \draw [line width=\boxLineWidth]%
        (0,0) -- (0, \boxSize) -- (\boxSize, \boxSize) -- (\boxSize, 0) --
        cycle;%
  \mkern 1mu%
}
\newcommand{\mStop}{%
  \tikz%
  \draw [line width=\boxLineWidth]%
        (0,0) --
        ++(120:\hexagonEdgeLength) --
        ++(60:\hexagonEdgeLength) --
        ++(0:\hexagonEdgeLength) --
        ++(-60:\hexagonEdgeLength) --
        ++(-120:\hexagonEdgeLength) --
        cycle;
}
% Constraint model-related commands
\renewcommand{\emptyset}{\varnothing}
\newcommand{\mStronger}{\leq}
\newboolean{consFontForceTextRM}
\setboolean{consFontForceTextRM}{true}
\NewDocumentCommand{\mConstraintFont}{sm}{%
  \IfBooleanTF{#1}{%
    \StrLeft{#2}{1}[\tempa]%
    \StrGobbleLeft{#2}{1}[\tempb]%
    \ifthenelse{\boolean{consFontForceTextRM}}{%
      \mathsc{\MakeUppercase{\tempa}\MakeLowercase{\tempb}}%
    }{%
      \textsc{\MakeUppercase{\tempa}\MakeLowercase{\tempb}}%
    }%
  }{%
    \ifthenelse{\boolean{consFontForceTextRM}}{%
      \mathsc{\MakeLowercase{#2}}%
    }{%
      \textsc{\MakeLowercase{#2}}%
    }%
  }%
}
\newcommand{\mBrPattern}{g_{\mathrm{br}}}
\newcommand{\mConst}{\rule{3pt}{3pt}}
\newcommand{\mCopy}{\circ}
\newcommand{\mKill}{\times}
\newcommand{\mNull}{\bot}
\newcommand{\mNullCopy}{\rlap{$\mCopy$} \mkern-.125mu \mNull}
\newcommand{\mState}{\square}
\newcommand{\mUFGraph}{G}
\newcommand{\mPatternSet}{S}
\newcommand{\mExtPatternSet}{\mPatternSet_\mathrm{ext}}
\newcommand{\mArgSet}{A}
\NewDocumentCommand{\mOpSet}{o}{
  \IfValueTF{#1}{O_{#1}}{O}
}
\NewDocumentCommand{\mOpCompSet}{m}{
  \mOpSet_{\overbar{#1}}
}
\NewDocumentCommand{\mOperandSet}{o}{
  \IfValueTF{#1}{P_{#1}}{P}
}
\NewDocumentCommand{\mOperandCompSet}{m}{
  \mOperandSet_{\overbar{#1}}
}
\newcommand{\mPhiOperandSet}{\mOperandSet[\mPhi]}
\newcommand{\mPhiOperandCompSet}{\mOperandCompSet{\mPhi}}
\newcommand{\mForbiddenCombSet}{F}
\newcommand{\mCostMatrix}{\mMatrix{C}}
\newcommand{\mDomMatrix}{\mMatrix{R}}
\NewDocumentCommand{\mDataSet}{o}{
  \IfValueTF{#1}{D_{#1}}{D}
}
\NewDocumentCommand{\mDataCompSet}{m}{
  \mDataSet_{\overbar{#1}}
}
\newcommand{\mConstDataSet}{\mDataSet[\mConst]}
\newcommand{\mConstDataCompSet}{\mDataCompSet{\mConst}}
\newcommand{\mStateDataSet}{\mDataSet[\mState]}
\newcommand{\mStateDataCompSet}{\mDataCompSet{\mState}}
\newcommand{\mBlockSet}{B}
\newcommand{\mFunctionDefEdgeSet}{E}
\newcommand{\mFunctionEntryBlock}{b_{\mathsc{f}}}
\newcommand{\mMatchDefEdgeSet}{E_{M}}
\NewDocumentCommand{\mMatchSet}{o}{
  \IfValueTF{#1}{M_{#1}}{M}
}
\NewDocumentCommand{\mMatchCompSet}{m}{
  \mMatchSet_{\overbar{#1}}
}
\newcommand{\mCopyMatchSet}{\mMatchSet[\mCopy]}
\newcommand{\mCopyMatchCompSet}{\mMatchCompSet{\mCopy}}
\newcommand{\mKillMatchSet}{\mMatchSet[\mKill]}
\newcommand{\mKillMatchCompSet}{\mMatchCompSet{\mKill}}
\newcommand{\mNullMatchSet}{\mMatchSet[\!\mNull]}
\newcommand{\mNullCopyMatchSet}{\mMatchSet[\mNullCopy]}
\newcommand{\mPhiMatchSet}{\mMatchSet[\mPhi]}
\newcommand{\mPhiMatchCompSet}{\mMatchCompSet{\mPhi}}
\newcommand{\mLocationSet}{L}
\newcommand{\mFallThroughSet}{J}
\NewDocumentCommand{\mInterchDataSet}{o}{
  \IfValueTF{#1}{I_{#1}}{I}
}
\newcommand{\mIntLocation}{l_{\mathsc{int}}}
\newcommand{\mKilledLocation}{l_{\mathsc{killed}}}
\NewDocumentCommand{\mVar}{mo}{
  \mathbf{#1}{\IfValueTF{#2}{[#2]}{}}
}
\NewDocumentCommand{\mBoolRepVar}{m}{%
  \def\vLine{\vrule width 0.4pt height 1.6ex depth 2pt}
  [ \mathllap{\vLine\hspace*{.85pt}} #1 \mathrlap{\hspace*{.85pt}\vLine} ]
}
\newcommand{\mResourceSet}{R}
\newcommand{\mHeuristicCost}{C_{\mathsc{heur}}}
\newcommand{\mRelaxedCost}{C_{\mathsc{rlx}}}
\DeclareMathOperator{\mArgLoc}{\mFunFont{argLoc}}
\DeclareMathOperator{\mBranches}{\mFunFont{branches}}
\DeclareMathOperator{\mBlockOf}{\mFunFont{blockOf}}
\DeclareMathOperator{\mCapOf}{\mFunFont{cap}}
\DeclareMathOperator{\mCapUseLocsOf}{\mFunFont{capUseLocsOf}}
\DeclareMathOperator{\mCapDefLocsOf}{\mFunFont{capDefLocsOf}}
\DeclareMathOperator{\mCupUseLocsOf}{\mFunFont{cupUseLocsOf}}
\DeclareMathOperator{\mCupDefLocsOf}{\mFunFont{cupDefLocsOf}}
\DeclareMathOperator{\mChild}{\mFunFont{child}}
\DeclareMathOperator{\mConsumes}{\mFunFont{consumes}}
\DeclareMathOperator{\mCost}{\mFunFont{cost}}
\DeclareMathOperator{\mCovers}{\mFunFont{covers}}
\DeclareMathOperator{\mDataOf}{\mFunFont{dataOf}}
\DeclareMathOperator{\mDefines}{\mFunFont{defines}}
\DeclareMathOperator{\mDisableIncompUsers}{\mFunFont{disableUses}}
\DeclareMathOperator{\mDom}{\mFunFont{dom}}
\DeclareMathOperator{\mDomain}{\mFunFont{D}}
\DeclareMathOperator{\mEmits}{\mFunFont{emits}}
\DeclareMathOperator{\mEmptyBlock}{\mFunFont{isEmpty}}
\DeclareMathOperator{\mEntry}{\mFunFont{entry}}
\DeclareMathOperator{\mExtValues}{\mFunFont{extValues}}
\DeclareMathOperator{\mIsExt}{\mFunFont{isExt}}
\DeclareMathOperator{\mFreq}{\mFunFont{freq}}
\DeclareMathOperator{\mLat}{\mFunFont{lat}}
\DeclareMathOperator{\mMatched}{\mFunFont{matched}}
\DeclareMathOperator{\mMTrees}{\mFunFont{mtrees}}
\DeclareMathOperator{\mNumChildren}{\mFunFont{numch}}
\DeclareMathOperator{\mOpCost}{\mFunFont{cost}}
\DeclareMathOperator{\mResourceUse}{\mFunFont{req}}
\DeclareMathOperator{\mSpans}{\mFunFont{spans}}
\DeclareMathOperator{\mStates}{\mFunFont{states}}
\DeclareMathOperator{\mStores}{\mFunFont{stores}}
\DeclareMathOperator{\mUses}{\mFunFont{uses}}
\DeclareMathOperator{\mUsersOf}{\mFunFont{usersOf}}
\DeclareMathOperator{\mWeight}{\mFunFont{weight}}
\DeclareMathOperator{\mWidthOf}{\mFunFont{widthOf}}

\NewDocumentCommand{\mkGlobalConCommands}{om}{%
  \IfValueTF{#1}{%
    \expandafter\DeclareMathOperator\csname m#1\endcsname{\mConstraintFont{#2}}
    \expandafter\DeclareMathOperator\csname M#1\endcsname{\mConstraintFont*{#2}}
  }{%
    \expandafter\DeclareMathOperator\csname m#2\endcsname{\mConstraintFont{#2}}
    \expandafter\DeclareMathOperator\csname M#2\endcsname{\mConstraintFont*{#2}}
  }
}

\mkGlobalConCommands{AllDifferent}
\mkGlobalConCommands{Circuit}
\mkGlobalConCommands{Cumulative}
\mkGlobalConCommands{GCC}
\mkGlobalConCommands{Increasing}
\mkGlobalConCommands{NoOverlap}
\mkGlobalConCommands{Table}
\mkGlobalConCommands[ValuePrec]{ValuePre}
\mkGlobalConCommands[ValuePrecChain]{VPC}

% blkarray-related commands
\NewDocumentEnvironment{adjblockarray}{mm}{%
  \begin{lrbox}{\adjblockarraybox}%
    $\begin{blockarray}{#1}
}{%
    \end{blockarray}$%
  \end{lrbox}%
  \raisebox{-#2}[\dimexpr\height-#2][\dimexpr\depth-#2]{%
    \usebox{\adjblockarraybox}%
  }%
}
\newsavebox{\adjblockarraybox}

% Typesets a tree that appears in the sigma set.
\NewDocumentCommand{\mSigmaTree}{mm}{%
  #1(#2)%
}

\newcommand{\mSubsumes}{\geq}
\newcommand{\mStrictSubsumes}{>}
\newcommand{\mImmSubsumes}{>_i}
\newcommand{\mMatchSetBase}[1]{#1_0}
\NewDocumentCommand{\mReduceCost}{mm}{
  \mFunFont{reducecost}(
    #1 \xrightarrow{\raisebox{-2pt}[-6pt]{\scriptsize$*$}} #2
  )
}


%============
% REFERENCES
%============

\addbibresource{references.bib}

% Command alias.
\NewDocumentCommand{\printreferences}{}{%
  \printbibliography%
}

% Bibliography options:
%    - Write first and middle names as initials in the bibliography
%    - Sort references by label (when citing)
%    - Set 2 names as max before invoking "et al." when citing
%    - Print all names in bibliography
\ExecuteBibliographyOptions{%
  giveninits=true,
  sortcites=true,
  maxcitenames=2,
  maxbibnames=100,
  hyperref=true,
  urldate=iso8601,
}

% Include prefixes in citations.
% https://tex.stackexchange.com/a/23397/2634
\ExecuteBibliographyOptions{%
  useprefix=false,
}
\makeatletter
\AtBeginDocument{%
  \toggletrue{blx@useprefix}% Use prefixes in running text
  \renewcommand*{\mkbibnameprefix}[1]{\MakeCapital{#1}}}% Capitalize prefixes in
                                                        % running text
\AtBeginBibliography{%
  \renewcommand*{\mkbibnameprefix}[1]{#1}}% Uncapitalize prefixes in
                                          % bibliography
\makeatother

% Customize appearance of chapter heading
\newcommand{\refname}{References}
\defbibheading{bibliography}[\refname]{%
  \chapter*{#1}%
  \addcontentsline{toc}{chapter}{\refname}%
  \markboth{#1}{#1}%
}

% Set URLs in smaller font in bibliography
\DeclareFieldFormat{url}{URL:~\relsize{-.5}\url{#1}}

% Set URL field in normal font
\DeclareFieldFormat{url}{\textsc{url}:~#1}

% Use "Doctoral thesis" instead of "PhD thesis"
\DefineBibliographyStrings{english}{%
  phdthesis={doctoral thesis},
  urlseen={accessed},
}

% Prevent pagebreaks within entries
% http://tex.stackexchange.com/a/43275/2634
\patchcmd{\bibsetup}{\interlinepenalty=5000}{\interlinepenalty=10000}{}{}

% Modify \fullcite to include all authors.
\let\oldfullcite\fullcite
\renewcommand{\fullcite}[1]{%
  \AtNextCite{\AtEachCitekey{\defcounter{maxnames}{100}}}%
  \oldfullcite{#1}%
}


%============
% GLOSSARIES
%============

% Add "glossary" to table of content
\glstoctrue

% Disable hyperlinks from terms to index
\glsdisablehyper

% Add new field to glossary entries
\glsaddkey{indextext}%
          {}%
          {\glsentryindextext}%
          {\Glsentryindextext}%
          {\glsindextext}%
          {\Glsindextext}%
          {\GLSindextext}

% Customize index style
\newglossarystyle{myindex}{%
  \setglossarystyle{mcolindex}
  \setlength{\columnsep}{8mm}
  \renewcommand*{\glstreenamefmt}[1]{##1}
  \renewcommand*{\glossentry}[2]{%
     \item\glsentryitem{##1}%
       \glstreenamefmt{%
         \glstarget{##1}{%
           \glsletentryfield{\tmpa}{##1}{indextext}%
           \expandafter\ifblank\expandafter{\tmpa}{%
             \glsentryfirst{##1}%
           }{%
             \glsentryindextext{##1}%
           }%
         }%
       }%
       \space\hfill\space##2%
  }
  \renewcommand{\subglossentry}[3]{%
    \ifcase##1\relax
      % level 0
      \item
    \or
      % level 1
      \subitem
      \glssubentryitem{##2}%
    \else
      % all other levels
      \subsubitem
    \fi
    \glstreenamefmt{\glstarget{##2}{\glossentryname{##2}}}%
    \space\hfill\space##3%
  }%
}

\renewcommand{\printindex}[1][]{%
  \setglossarystyle{myindex}
  \printglossary[title={Index}, #1]
}


% Commands for introducing terms and acronyms
\NewDocumentCommand{\newTerm}{omO{}}{%
  \IfValueTF{#1}{%
    \newglossaryentry{#1}{name={#2}, description={#2}, #3}%
  }{%
    \newglossaryentry{#2}{name={#2}, description={#2}, #3}%
  }%
}
\NewDocumentCommand{\newAcronym}{ommO{}}{%
  \IfValueTF{#1}{%
    \newTerm[#1]{#2}[description={#3}, indextext={#2}, first={#3 (#2)}, #4]%
  }{%
    \newTerm{#2}[description={#3}, indextext={#2}, first={#3 (#2)}, #4]%
  }%
}
\NewDocumentCommand{\newToolTerm}{omO{}}{%
  \IfValueTF{#1}{%
    \newTerm[#1]{\toolFont{#2}}[sort={#2}, #3]%
  }{%
    \newTerm[#2]{\toolFont{#2}}[sort={#2}, #3]%
  }%
}

% Commands for referring to a short version of a term or acronym
\let\glsshort\glsuseri
\let\Glsshort\Glsuseri
\let\glsplshort\glsuserii
\let\Glsplshort\Glsuserii
\let\glshyphened\glsuseriii
\let\Glshyphened\Glsuseriii
\let\glslong\glsuseriv
\let\Glslong\Glsuseriv
\let\glspllong\glsuserv
\let\Glspllong\Glsuserv

% Adds a modifier to all \gls-like commands for emphasizing the term or acronym
\newcommand{\glsEmph}[1]{\emph{#1}}
\GlsXtrSetAltModifier{!}{format=hyperit}
\makeatletter
\renewcommand*{\glslinkpostsetkeys}{%
  \ifdefstring\@glsnumberformat{hyperit}%
                               {\let\glstextformat\glsEmph}%
                               {\let\glstextformat\@firstofone}%
}
\makeatother

% Force 'see also' to appear on a new line in the index
\renewcommand\glsseeformat[3][\seename]{%
  \\*\raggedleft\emph{#1} \glsseelist{#2}%
}

% Rename certain items
\renewcommand{\seename}{see also}


%===================
% TABLE OF CONTENTS
%===================

% Change ToC title
\renewcommand{\contentsname}{Table of Contents}

% Increase width for chapter numbers
\addtolength{\cftchapternumwidth}{4pt}
\addtolength{\cftsectionindent}{4pt}
\addtolength{\cftsectionnumwidth}{3pt}

% Right-align chapter numbers
\renewcommand{\cftchapterpresnum}{\hfill}
\renewcommand{\cftchapteraftersnum}{\hspace*{8pt}}

% Reduce space between leading dots
\renewcommand{\cftsectiondotsep}{4}

% Set titles in ragged right mode
\setrmarg{2.55em plus 1fil}

% Force same indentation for entries in List of Algorithms as other List of Fig.
% https://tex.stackexchange.com/a/381303/2634
\makeatletter
\renewcommand*\l@algocf{\l@figure}
\makeatother


%======================================
% TITLE, CHAPTER, AND SECTION HEADINGS
%======================================

\makeatletter
\apptocmd{\@titleFont}{\sffamily}{}{}
\apptocmd{\@frontMatterChapFont}{\sffamily}{}{}
\makeatother

\maxsecnumdepth{subsubsection}

\setlength{\midchapskip}{50pt}
\renewcommand*{\chapterheadstart}{%
  % Remove all space above chapter heading
}
\renewcommand*{\chapnamefont}{%
  \sffamily\large\scshape%
}
\renewcommand*{\chapnumfont}{%
  \normalfont\fontsize{80}{64}\selectfont%
}
\renewcommand*{\printchaptername}{%
  % Do not print 'Chapter' or 'Appendix'
}
\makeatletter
\renewcommand*{\printchapternum}{%
  \flushright%
  \begin{tabular}{@{}c@{}}
    \chapnamefont\MakeLowercase{\@chapapp}\\[.5ex]
    \chapnumfont\thechapter%
  \end{tabular}%
}
\renewcommand*{\printchaptertitle}[1]{%
  \flushleft\chaptitlefont#1%
}
\renewcommand*{\chaptitlefont}{%
  \sffamily\Huge\bfseries%
}
\setsecheadstyle{\sffamily\Large\bfseries}
\setsubsecheadstyle{\sffamily\large\bfseries}
\setsubsubsecheadstyle{\sffamily\bfseries}
\setparaheadstyle{\sffamily\bfseries}

% Customize typesetting of page headers and footers
\newcommand{\hfMarkText}[1]{%
  \footnotesize\textosf{\textsc{\MakeLowercase{#1}}}%
}
\nouppercaseheads
\makeevenhead{headings}%
             {\hfMarkText{\thepage}}%
             {}%
             {\hfMarkText{\leftmark}}
\makeoddhead{headings}%
            {\hfMarkText{\rightmark}}%
            {}%
            {\hfMarkText{\thepage}}
\makeoddfoot{plain}%
            {}%
            {\hfMarkText{\thepage}}%
            {}
\makeatletter
\renewcommand{\@hfMarkText}[1]{\hfMarkText{#1}}
\makeatother

% Remove chapter section numbers from header marks
\addtopsmarks{headings}{}{
  \createmark{chapter}{left}{shownumber}{}{ \ }
}
\addtopsmarks{headings}{}{
  \createmark{section}{right}{shownumber}{}{ \ }
}

% Activate changes
\pagestyle{headings}


%==================
% FOR WRITING CODE
%==================

\NewDocumentCommand{\codeFont}{s}{%
  \ttfamily%
  \IfBooleanT{#1}{\relsize{-.5}}%
}
\NewDocumentCommand{\cCode}{sm}{%
  \mbox{%
    \IfBooleanTF{#1}{\codeFont*}{\codeFont}%
    #2%
  }%
}
\NewDocumentCommand{\cVar}{smo}{%
  \IfBooleanTF{#1}{%
    \cCode*{#2\IfValueTF{#3}{$_{\text{#3}}$}{}}%
  }{%
    \cCode{#2\IfValueTF{#3}{$_{\text{#3}}$}{}}%
  }%
}
\let\irFont\codeFont
\NewDocumentCommand{\irCode}{sm}{%
  \mbox{%
    \IfBooleanTF{#1}{\irFont*}{\irFont}%
    #2%
  }%
}
\NewDocumentCommand{\irVar}{smo}{%
  \IfBooleanTF{#1}{%
    \irCode*{#2\IfValueTF{#3}{$_{\text{#3}}$}{}}%
  }{%
    \irCode{#2\IfValueTF{#3}{$_{\text{#3}}$}{}}%
  }%
}
\NewDocumentCommand{\irTemp}{sm}{%
  \IfBooleanTF{#1}{%
    \irCode*{t$_{\text{#2}}$}%
  }{%
    \irCode{t$_{\text{#2}}$}%
  }%
}
\NewDocumentCommand{\irAssign}{smm}{%
  \mbox{#2 $\leftarrow$ #3}%
}
\newcommand{\irAddText}{$+$}
\NewDocumentCommand{\irAdd}{smm}{%
  \mbox{#2 \irAddText{} #3}%
}
\NewDocumentCommand{\irBlock}{sm}{%
  \IfBooleanTF{#1}{\irCode*{#2}}{\irCode{#2}}%
}
\newcommand{\irBrText}{br}
\NewDocumentCommand{\irBr}{ms}{%
  \mbox{%
    \IfBooleanTF{#1}{%
      \irCode*{\bfseries\irBrText}%
    }{%
      \irCode{\bfseries\irBrText}%
    }
    \IfBooleanTF{#1}{\irBlock*{#2}}{\irBlock{#2}}%
  }%
}
\newcommand{\irCallText}{call}
\NewDocumentCommand{\irCall}{sm}{%
  \mbox{%
    \IfBooleanTF{#1}{%
      \irCode*{\bfseries\irCallText}%
    }{%
      \irCode{\bfseries\irCallText}%
    }
    \IfBooleanTF{#1}{\irCode*{#2}}{\irCode{#2}}%
  }%
}
\newcommand{\irCondBrText}{c.br}
\NewDocumentCommand{\irCondBr}{smmm}{%
  \mbox{%
    \IfBooleanTF{#1}{%
      \irCode*{\bfseries\irCondBrText}%
    }{%
      \irCode{\bfseries\irCondBrText}%
    }
    \IfBooleanTF{#1}{%
      \irCode*{#2, }\irBlock*{#3}\irCode*{, }\irBlock*{#4}%
    }{%
      \irCode{#2, }\irBlock{#3}\irCode{, }\irBlock{#4}%
    }%
  }%
}
\newcommand{\irCpText}{cp}
\newcommand{\irEQText}{$=$}
\NewDocumentCommand{\irEQ}{smm}{\mbox{#2 \irEQText{} #3}}
\newcommand{\irNEText}{$\neq$}
\NewDocumentCommand{\irNE}{smm}{\mbox{#2 \irNEText{} #3}}
\newcommand{\irGEText}{$\geq$}
\NewDocumentCommand{\irGE}{smm}{\mbox{#2 \irGEText{} #3}}
\newcommand{\irGTText}{$>$}
\NewDocumentCommand{\irGT}{smm}{\mbox{#2 \irGTText{} #3}}
\newcommand{\irLEText}{$\leq$}
\NewDocumentCommand{\irLE}{smm}{\mbox{#2 \irLEText{} #3}}
\newcommand{\irLTText}{$<$}
\NewDocumentCommand{\irLT}{smm}{\mbox{#2 \irLTText{} #3}}
\newcommand{\irLSHText}{$\ll$}
\NewDocumentCommand{\irLSH}{smm}{\mbox{#2 \irLSHText{} #3}}
\newcommand{\irMulText}{$\times$}
\NewDocumentCommand{\irMul}{smm}{\mbox{#2 \irMulText{} #3}}
\newcommand{\irLoadText}{load}
\NewDocumentCommand{\irLoad}{sm}{%
  \IfBooleanTF{#1}{%
    \irCode*{\bfseries\irLoadText}%
  }{%
    \irCode{\bfseries\irLoadText}%
  }
  #2%
}
\newcommand{\irPhiText}{$\mPhi$}
\NewDocumentCommand{\irPhi}{sm}{\mbox{\irPhiText(#2)}}
\newcommand{\irRetText}{ret}
\NewDocumentCommand{\irRet}{sm}{%
  \IfBooleanTF{#1}{%
    \irCode*{\bfseries\irRetText}%
  }{%
    \irCode{\bfseries\irRetText}%
  }
  #2%
}
\newcommand{\irStoreText}{store}
\NewDocumentCommand{\irStore}{smm}{%
  \IfBooleanTF{#1}{%
    \irCode*{\bfseries\irStoreText}%
  }{%
    \irCode{\bfseries\irStoreText}%
  }
  #2, #3%
}
\newcommand{\irSubText}{$-$}
\NewDocumentCommand{\irSub}{smm}{\mbox{#2 \irSubText{} #3}}
\newcommand{\irZextText}{zxt}
\NewDocumentCommand{\irZext}{sm}{%
  \IfBooleanTF{#1}{%
    \irCode*{\bfseries\irZextText}%
  }{%
    \irCode{\bfseries\irZextText}%
  }
  #2%
}

% Commands to be used inside TikZ nodes
\NewDocumentCommand{\noWidth}{mO{0pt}}{%
  \raisebox{#2}[\height][0pt]{#1}%
}
\newcommand{\nAdd}{\irCode{\noWidth{\irAddText}}}
\newcommand{\nBlock}[1]{\irCode{#1}}
\newcommand{\nBr}{\irCode{\irBrText}}
\newcommand{\nCall}[1]{\irCode{\noWidth{#1}}}
\NewDocumentCommand{\nCondBr}{s}{%
  \IfBooleanTF{#1}{%
    \irCode{\irCondBrText}%
  }{%
    \irCode{c{\kern-1pt}.{\kern-1.2pt}b{\kern-.3pt}r}%
  }%
}
\NewDocumentCommand{\nRet}{s}{%
  \IfBooleanTF{#1}{%
    \irCode{\irRetText}%
  }{%
    \irCode{r{\kern-.3pt}e{\kern-.6pt}t}%
  }%
}
\newcommand{\nCopy}{\irCode{\noWidth{\irCpText}}}
\newcommand{\nEQ}{\irCode{\noWidth{\irEQText}}}
\newcommand{\nGT}{\irCode{\noWidth{\irGTText}}}
\newcommand{\nLT}{\irCode{\noWidth{\irLTText}}}
\newcommand{\nGE}{\irCode{\noWidth{\irGEText}}}
\newcommand{\nLE}{\irCode{\noWidth{\irLEText}}}
\newcommand{\nLSH}{\irCode{\noWidth{\irLSHText}}}
\newcommand{\nLoad}{\irCode{\noWidth{ld}}}
\newcommand{\nMul}{\irCode{\noWidth{\irMulText}}}
\newcommand{\nPhi}{\irCode{\noWidth{\irPhiText}[.5pt]}}
\NewDocumentCommand{\nStore}{s}{%
  \IfBooleanTF{#1}{%
    \irCode{st}%
  }{%
    \irCode{s{\kern-.2pt}t}%
  }%
}
\newcommand{\nSub}{\irCode{\noWidth{\irSubText}[-1pt]}}
\newcommand{\nZext}{\irCode{\noWidth{\irZextText}}}
\newcommand{\nTemp}[1]{%
  \irTemp{#1}\hspace*{-1pt}%
}
\NewDocumentCommand{\nVar}{mo}{%
  \irVar{#1\IfValueTF{#2}{$_{\text{#2}}$}{}}%
  \IfValueTF{#2}{\hspace*{-1pt}}{}%
}

% Commands used for assembly instructions
\let\instrFont\irFont
\let\instrCode\irCode
\let\instrBlock\irBlock
\let\instrEQ\irEQ
\let\instrNE\irNE
\let\instrGE\irGE
\let\instrLE\irLE
\let\instrTemp\irTemp
\let\instrVar\irVar

% Commands for grammars
\newcommand{\mNTFont}[1]{%
  % This command must not be declared using \NewDocumentCommand because it
  % will appear inside a \bm{...}, which cannot handle \NewDocumentCommand-
  % declared macros.
  \mathit{#1}%
}
\NewDocumentCommand{\mNT}{mO{}}{%
  \mNTFont{#1}_{#2}%
}
\NewDocumentCommand{\mAttr}{m}{%
  \mNTFont{#1}%
}
\NewDocumentCommand{\mPredicate}{m}{%
  \mNTFont{#1}%
}
\NewDocumentCommand{\mAction}{m}{%
  \mNTFont{#1}%
}


%========================
% FOR WRITING ALGORITHMS
%========================

% Set height of rules
\setlength{\algoheightrule}{\heavyrulewidth}%

% Commands used for algorithms
\newcommand{\Or}{\textbf{or}\ }
\newcommand{\Assign}{$\leftarrow$\ }

\DontPrintSemicolon
\SetNlSty{tiny}{}{}
\SetAlFnt{\figureFont\figureFontSize}
\SetAlCapSkip{\abovecaptionskip}
\SetAlCapSty{}

% Customize algorithm typesetting
\newcommand{\algStyle}[1]{\textsf{#1}}
\newcommand{\kwStyle}[1]{\algStyle{\textbf{#1}}}
\newcommand{\commentStyle}[1]{\emph{#1}}
\SetKwSty{kwStyle}
\SetArgSty{algStyle}
\SetFuncArgSty{algStyle}
\patchcmd{\SetProgSty}{ArgSty}{ProgSty}{}{}% There is a bug in \SetProgSty
\SetProgSty{algStyle}
\SetCommentSty{commentStyle}
\SetKwComment{cmt}{\righttriangle\:}{}
\SetKw{Return}{return}
\SetNlSty{figureFont}{}{}

\SetKw{And}{and}
\SetKw{Not}{not}
\SetKw{Or}{or}

% Command for removing algorithm line number for a single line
\let\oldnl\nl
\newcommand{\nonl}{\renewcommand{\nl}{\let\nl\oldnl}}

% Hook for checking whether the 'vlined' option is used
\newbool{VlinedOptionUsed}
\boolfalse{VlinedOptionUsed}
\apptocmd{\SetAlgoVlined}{\booltrue{VlinedOptionUsed}}{}{}

% Command for declaring a function in an algorithm
\newbool{InsideDeclFunction}
\boolfalse{InsideDeclFunction}
\makeatletter
\newcommand{\DeclFunction}[3]{%
  \ifbool{InsideDeclFunction}{}{\nonl}%
  \KwSty{function} #1\,(#2):%
  \begingroup%
    \booltrue{InsideDeclFunction}%
    \algocf@block{#3}{}{}
  \endgroup%
  \ifbool{VlinedOptionUsed}{}{%
    \ifbool{InsideDeclFunction}{}{\nonl}%
    \KwSty{end}\;%
  }%
}
\makeatother

% Command for calling a function in an algorithm
\newcommand{\Call}[2]{%
  #1\,(#2)%
}

% Redefined \listofalgorithms for consistent appearance
\makeListOfCommand{\listofalgorithms}{\listalgorithmcfname}{loa}{chapter}


%==========
% LISTINGS
%==========

\lstset{%
  basicstyle=\codeFont\footnotesize,
  frame=lines,
  framerule=\heavyrulewidth,
  framexleftmargin=.5em,
  framexrightmargin=.5em,
  aboveskip=0pt,
  belowskip=0pt,
}

\lstdefinelanguage{minizinc}{%
  morekeywords={
    array,
    array1d,
    array2d,
    bool2int,
    card,
    circuit,
    constraint,
    domain,
    diff,
    div,
    endif,
    else,
    exists,
    false,
    fix,
    forall,
    if,
    in,
    include,
    increasing,
    index\_set\_1of2,
    int,
    int\_search,
    intersect,
    let,
    max,
    min,
    minimize,
    mod,
    not,
    of,
    output,
    seq\_search,
    set,
    show,
    solve,
    subset,
    sum,
    table,
    test,
    then,
    true,
    union,
    value\_precede\_chain,
    var,
    where,
  },
  sensitive=true,
  morecomment=[l]{\%},
  morestring=[b]",
  numbers=left,
  numberstyle=\tiny\ttfamily,
}

% Creates a minipage of certain width where the frame from the listings will
% not overflow.
\newenvironment{lstpage}[1]{%
  \begin{minipage}{#1}%
    \centering%
    \begin{minipage}{#1-1em}%
}{%
    \end{minipage}%
  \end{minipage}%
}


%=========
% CAPTION
%=========

% Add period at end of every caption and subcaption
\captionsetup{textformat=period}

% algorithm2e does not obey the caption settings above and requires a patch
\makeatletter
\patchcmd{\algocf@captiontext}{\endgraf}{\unskip.\endgraf}{}{}
\makeatother

% Fix captions of algorithms so that they use the entire text width
\makeatletter
\pretocmd{\algocf@makecaption}{%
  \addtolength{\hsize}{1.5\algomargin}%
  \setlength{\algomargin}{0pt}%
}{}{}
\makeatother


%==================
% EXPERIMENT PLOTS
%==================

\newcommand{\expDir}{experiments/20180111-113629}

\newcommand{\functionNameFont}{\LARGE}
\newcommand{\functionName}[1]{{\functionNameFont\path{#1}}}
\newcommand{\plotPercentageFont}{\huge}
\newcommand{\plotPercentage}[1]{{\plotPercentageFont$#1$\%}}
\newcommand{\speedupValue}[1]{%
  \pgfmathparse{and(-1 < #1, #1 < 1) ? 1 : 0}%
  \ifnum \pgfmathresult=1
    % Decimal rounding
    \pgfmathprintnumber[fixed, precision=3]{#1}%
  \else
    % Significant figure rounding
    \pgfmathprintnumberto[fixed, precision=3, verbatim]{#1}{\tempa}%
    \num[round-mode=figures, round-precision=3]{\tempa}%
  \fi%
}
\newcommand{\plotSpeedupTicsFont}{\huge}
\newcommand{\plotSpeedupTics}[1]{%
  {\plotSpeedupTicsFont$\speedupValue{#1}\times$}%
}
\newcommand{\plotSecondsFont}{\Large}
\newcommand{\plotSeconds}[1]{{\plotSecondsFont$#1$\thinspace{}s}}
\newcommand{\barValueFont}{}
\newcommand{\barSymbolFont}{\LARGE}
\newcommand{\plotBarValue}[1]{%
  {\barValueFont\speedupValue{#1}}%
}
\newcommand{\barValueNotOptimal}{$*$}
\newcommand{\plotBarValueNotOptimal}{%
  {\barSymbolFont\barValueNotOptimal}%
}
\newcommand{\barValueNoBaselineSolution}{$**$}
\newcommand{\plotBarValueNoBaselineSolution}{%
  {\barSymbolFont\barValueNoBaselineSolution}%
}
\newcommand{\barValueNoSolution}{$***$}
\newcommand{\plotBarValueNoSolution}{%
  {\barSymbolFont\barValueNoSolution}%
}
\newcommand{\plotSubOptSymbol}{%
  \Huge\bfseries%
  \hspace*{4.6pt}%
  \raisebox{9pt}{$\colon$}%
}

\newcommand{\trimBarchartPlot}[1]{\trimbox{6pt 12pt 8pt 7pt}{#1}}
\newcommand{\trimLinechartPlot}[1]{\trimbox{7pt 9pt 9pt 7pt}{#1}}


%=========
% SIUNITX
%=========

\sisetup{
  round-mode = places,
  round-precision = 2,
}


%======
% MISC
%======

\usepgflibrary{fpu}

\NewDocumentCommand{\minOf}{m}{%
  \pgfkeys{/pgf/fpu}% Enable large values
  \pgfkeys{/pgf/fpu/output format=fixed}%
  \pgfmathparse{min(#1)}%
  \pgfkeys{/pgf/fpu=false}% Restore
}

\NewDocumentCommand{\maxOf}{m}{%
  \pgfkeys{/pgf/fpu}% Enable large values
  \pgfkeys{/pgf/fpu/output format=fixed}%
  \pgfmathparse{max(#1)}%
  \pgfkeys{/pgf/fpu=false}% Restore
}

\NewDocumentCommand{\numMinOf}{O{}m}{%
  \minOf{#2}%
  \num[#1]{\pgfmathresult}%
}

\NewDocumentCommand{\numMaxOf}{O{}m}{%
  \maxOf{#2}%
  \num[#1]{\pgfmathresult}%
}

\NewDocumentCommand{\SIMinOf}{O{}mm}{%
  \minOf{#2}%
  \SI[#1]{\pgfmathresult}{#3}%
}

\NewDocumentCommand{\SIMaxOf}{O{}mm}{%
  \maxOf{#2}%
  \SI[#1]{\pgfmathresult}{#3}%
}

\NewDocumentCommand{\printSolvingTime}{O{}m}{%
  \SI[round-mode=figures, round-precision=3, #1]{#2}{\s}%
}

\NewDocumentCommand{\printMinSolvingTime}{O{}m}{%
  \minOf{#2}%
  \printSolvingTime[#1]{\pgfmathresult}%
}

\NewDocumentCommand{\printMaxSolvingTime}{O{}m}{%
  \maxOf{#2}%
  \printSolvingTime[#1]{\pgfmathresult}%
}

\NewDocumentCommand{\printCycles}{O{}m}{%
  \num[round-precision=0, #1]{#2}~cycles%
}

\NewDocumentCommand{\printMinCycles}{O{}m}{%
  \minOf{#2}%
  \printCycles[#1]{\pgfmathresult}%
}

\NewDocumentCommand{\printMaxCycles}{O{}m}{%
  \maxOf{#2}%
  \printCycles[#1]{\pgfmathresult}%
}

\NewDocumentCommand{\printGMI}{O{}m}{%
  \num[round-mode=figures, round-precision=3, #1]{#2}%
}

\NewDocumentCommand{\printGMICI}{O{}mO{}m}{%
  \mbox{$[\printGMI[#1]{#2}, \printGMI[#3]{#4}]$}%
}

\NewDocumentCommand{\printSpeedup}{O{}m}{%
  \num[round-mode=figures, round-precision=3, #1]{#2}$\times$%
}

\NewDocumentCommand{\printMaxSpeedup}{O{}m}{%
  \maxOf{#2}%
  \printSpeedup[#1]{\pgfmathresult}%
}

\NewDocumentCommand{\printMinSpeedup}{O{}m}{%
  \minOf{#2}%
  \printSpeedup[#1]{\pgfmathresult}%
}

\newcommand{\supportNo}{$\cdot$}
\newcommand{\supportYes}{%
  \begin{tikzpicture}
    \draw [line width=1.5pt] (0,0) -- ++(-45:4pt) -- ++(45:7.5pt);
  \end{tikzpicture}%
}
\newcommand{\righttriangle}{%
  \begin{tikzpicture}
    \draw [line width=.65pt] (0,0) -- ++(-90:4pt) -- ++(30:4pt) -- cycle;
  \end{tikzpicture}%
}

% Length to be used for storing temporary values
\newlength{\tmpLength}
