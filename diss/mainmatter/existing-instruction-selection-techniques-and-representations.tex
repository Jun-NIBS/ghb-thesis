% Copyright (c) 2017, Gabriel Hjort Blindell <ghb@kth.se>
%
% This work is licensed under a Creative Commons 4.0 International License (see
% LICENSE file or visit <http://creativecommons.org/licenses/by/4.0/> for a copy
% of the license).

\chapter{Existing Instruction Selection Techniques and Representations}
\labelChapter{existing-isel-techniques-and-reps}

\todo{write chapter overview}



\section{TODO}

\Gls{instruction selection} can be reduced into two subproblems:%
%
\begin{enumerate}
  \item Finding all instances of instructions that can implement one or more
    \glspl{operation} in the \gls{program}.
  \item Selecting a subset of these instances such that all \glspl{operation}
    are implemented.
\end{enumerate}
%
Without loss of generality, we assume the input to the \gls{instruction
  selector} to consist of a single \gls{function}, which in turn consists of
many \glspl{basic block} (henceforth referred to as simply \glspl!{block}).

Most \glspl{compiler} solve the two problems above using graph-based methods.
%
First, the \gls{IR} code is transformed into a \gls!{data-flow graph}, where
nodes represent \glspl{operation} in the function and edges represent data
dependencies between the \glspl{operation}.
%
\Glspl{data-flow graph} limited to \glspl{block} are called \glspl!{block
  graph}, and \glspl{data-flow graph} capturing data flow of whole functions are
called \glspl!{function graph}.
%
Corresponding \glspl{data-flow graph}, called \glspl!{pattern graph} (or simply
\glspl!{pattern}), are also built to represent the \gls{instruction} provided by
the \gls{target machine}.
%
The set of \glspl{pattern graph} for a particular \gls{target machine}
constitute a \gls!{pattern set}.

\begin{filecontents*}{p-match-sel-example.c}
x = A[i + 1];
\end{filecontents*}

\begin{figure}
  \centering%
  \subcaptionbox{C code\labelFigure{p-match-sel-example-c}}%
                {\lstinputlisting[language=c]{p-match-sel-example.c}}%
  \hspace{5mm}%
  \subcaptionbox{%
                  Instructions. The $*s$ notion means ``get value at address $s$
                  in memory''%
                  \labelFigure{p-match-sel-example-instrs}%
                }%
                [50mm]%
                {%
                  \small
                  \begin{tabular}{%
                                   >{\instrFont}r@{\hspace{4pt}}%
                                   >{$}l<{$}@{ $\leftarrow$ }%
                                   >{$}l<{$}%
                                 }
                    add  & r & s + t\\
                    mul  & r & s \times t\\
                    mad  & r & s \times t + u\\
                    load & r & *s\\
                    load & r & *(s \times t + u)
                  \end{tabular}%
                }%
  \hspace{5mm}%
  \subcaptionbox{%
                   Corresponding \gls{block graph} and \glspl{match}%
                   \labelFigure{p-match-sel-example-graph}%
                }{%
                  \small%
                  % Copyright (c) 2017, Gabriel Hjort Blindell <ghb@kth.se>
%
% This work is licensed under a Creative Commons 4.0 International License (see
% LICENSE file or visit <http://creativecommons.org/licenses/by/4.0/> for a copy
% of the license).
%
\begingroup%
\figureFont%
\pgfdeclarelayer{background1}%
\pgfdeclarelayer{background2}%
\pgfdeclarelayer{background3}%
\pgfdeclarelayer{background4}%
\pgfsetlayers{background4,background3,background2,background1,main}%
\begin{tikzpicture}[%
    data node/.style={%
      op node,
    },
    match node/.style={%
      match line,
      draw,
      circle,
      inner sep=.5pt,
      outer sep=0,
    },
    outer match node/.style={%
      match node,
      draw=none,
      inner sep=0,
    },
    match label/.style={%
      minimum size=0,
      inner sep=0,
      outer sep=1pt,
      node distance=10pt,
      font=\footnotesize,
    },
  ]

  % Graph
  \node [op node] (add1) {\opAdd};
  \node [data node, position=135 degrees from add1] (i) {\opVar{i}};
  \node [data node, position= 45 degrees from add1] (c1) {\opVar{1}};
  \node [op node, position=-45 degrees from add1] (mul) {\opMul};
  \node [data node, position= 45 degrees from mul] (c4) {\opVar{4}};
  \node [op node, position=-45 degrees from mul] (add2) {\opAdd};
  \node [data node, position= 45 degrees from add2] (A) {\opVar{A}};
  \node [op node, below=of add2] (load) {\opLoad};
  \begin{scope}[data flow]
    \draw (i) -- (add1);
    \draw (c1) -- (add1);
    \draw (add1) -- (mul);
    \draw (c4) -- (mul);
    \draw (mul) -- (add2);
    \draw (A) -- (add2);
    \draw (add2) -- (load);
  \end{scope}

  \begin{pgfonlayer}{background1}
    % Matches
    \node [match node, fill=shade2, fit=(add1)] (m1) {};
    \node [match node, fill=shade2, fit=(add2)] (m2) {};
    \node [match node, fill=shade2, fit=(mul)] (m3) {};
    % m4
    \begin{pgfonlayer}{background2}
      \node [outer match node, inner sep=-1pt, fit=(m3)] (m4a) {};
      \node [outer match node, inner sep=-1pt, fit=(m2)] (m4b) {};
      \draw [match line, bend left=45, fill=shade1]
            (m4a.south west)
            to
            (m4a.north west)
            to
            (m4a.north east)
            --
            (m4b.north east)
            to
            (m4b.south east)
            to
            (m4b.south west)
            -- coordinate (m4)
            cycle;
    \end{pgfonlayer}
    \node [match node, fill=shade1, fit=(load)] (m5) {};
    % m6
    \begin{pgfonlayer}{background4}
      \node [outer match node, inner sep=1pt, fit=(m3)] (m6a) {};
      \node [outer match node, inner sep=1pt, fit=(m2)] (m6b) {};
      \node [outer match node, inner sep=1pt, fit=(m5)] (m6c) {};
    \end{pgfonlayer}
    % The named paths must be declared outside of pgfonlayer, or else they will
    % disappear when the environment ends
    \path [name path=m6a-to-m6b-north]
          (m6a.north east)
          --
          (m6b.north east)
          --
          +(-45:5mm);
    \path [name path=m6a-to-m6b-south]
          (m6a.south west)
          --
          (m6b.south west)
          --
          +(-45:5mm);
    \path [name path=m6c-to-m6b-east]
          (m6c.east)
          --
          (m6c.east |- m6b.north);
    \path [name path=m6c-to-m6b-west]
          (m6c.west)
          --
          (m6c.west |- m6b.west);
    \begin{pgfonlayer}{background4}
      \draw [match line, bend left=45, fill=shade3,
             name intersections={of=m6a-to-m6b-north and m6c-to-m6b-east,
                                 by=x-east},
             name intersections={of=m6a-to-m6b-south and m6c-to-m6b-west,
                                 by=x-west},
             rounded corners=4pt,
            ]
            (m6a.south west)
            to
            (m6a.north west)
            to
            (m6a.north east)
            --
            (x-east)
            --
            (m6c.east)
            to
            (m6c.south)
            to
            (m6c.west)
            -- coordinate [pos=.25] (m6)
            (x-west)
            --
            cycle;
    \end{pgfonlayer}

    % Fill overdrawn edges of m6
    \draw [match line, bend left=45, fill=none,
           name intersections={of=m6a-to-m6b-north and m6c-to-m6b-east,
                               by=x-east},
           name intersections={of=m6a-to-m6b-south and m6c-to-m6b-west,
                               by=x-west},
           rounded corners=4pt,
          ]
          (m6a.south west)
          to
          (m6a.north west)
          to
          (m6a.north east)
          --
          (x-east)
          --
          (m6c.east)
          to
          (m6c.south)
          to
          (m6c.west)
          --
          (x-west)
          --
          cycle;
  \end{pgfonlayer}

  % Match labels
  \node [match label, left=of m1] (m1l) {$\strut m_1$};
  \node [match label, right=of m2] (m2l) {$\strut m_2$};
  \node [match label, left=of m3] (m3l) {$\strut m_3$};
  \node [match label, position=-135 degrees from m4] (m4l) {$m_4$};
  \node [match label, right=of m5] (m5l) {$\strut m_5$};
  \node [match label, left=of m6] (m6l) {$\strut m_6$};
  \foreach \i in {1, ..., 6} {
    \draw [thick] (m\i) -- (m\i l);
  }
\end{tikzpicture}%
\endgroup%
%
                }

  \caption[An example of the pattern matching and selection problem]%
          {%
            An example demonstrating the pattern matching and selection
            problem for a program that loads a value from integer array
            \irVar{A} at offset \irVar{i} $+$ \irVar{1} (it is assumed that
            \irVar{A} is stored in memory and that an integer is 4~bytes).
            Valid covers are \mbox{$\mSet{m_1, m_2, m_3, m_5}$},
            \mbox{$\mSet{m_1, m_4, m_5}$}, and \mbox{$\mSet{m_1, m_6}$}%
          }
  \labelFigure{p-match-sel-example}
\end{figure}

The first subproblem can be reduced to finding all instances where a
\gls{pattern} from the \gls{pattern set} is subgraph isomorphic to~$G$, where
$G$ denotes either a \gls{block graph} or a \gls{function graph}.
%
Each such instance is called a \gls!{match}, and the set of all \glspl{match}
constitute a \gls!{match set}, which is denoted by~$M$.
%
Hence we refer to this subproblem as the \gls!{pattern matching}[ problem].
%
\Gls{pattern matching} can be done in linear time if both $G$ and all
\glspl{pattern} are tree-shaped, otherwise it is an NP-complete
problem~\cite{GareyJohnson:1979,HoffmannODonnell:1982}.
%
Having found $M$, the second subproblem -- which we refer to as the
\gls!{pattern selection}[ problem] -- can be reduced to selecting a set of
\glspl{match} that \gls{cover}[s]~$G$.
%
We say that a subset~\mbox{$S \subseteq M$}, where $M$ is the \gls{match set},
\gls!{cover}[s] $G$ if every \gls{operation} in $G$ appears in exactly one match
from~$S$.
%
An example is shown in \refFigure{p-match-sel-example}.

For a given \gls{program} and \gls{target machine}, there often exists many
valid combinations of \glspl{instruction} -- in terms of $G$ and $M$, this means
there exist many \glspl{cover} of~$G$ -- which may result in code where quality
differs significantly.
%
For example, depending on the hardware the efficiency of two sets of selected
instructions may differ by as much as two orders of
magnitude~\cite{ZivojnovicEtAl:1994}.
%
Consequently, we augment the \gls{pattern selection} problem -- originally
defined to accept any \gls{cover} -- into an optimization problem, where only
\glspl{cover} with least cost are accepted.
