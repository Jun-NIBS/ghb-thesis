% Copyright (c) 2017, Gabriel Hjort Blindell <ghb@kth.se>
%
% This work is licensed under a Creative Commons 4.0 International License (see
% LICENSE file or visit <http://creativecommons.org/licenses/by/4.0/> for a copy
% of the license).

\chapter{Existing Instruction Selection Techniques and Representations}
\labelChapter{existing-isel-techniques-and-reps}

\todo{write chapter overview}



\section{TODO}

\Gls{instruction selection} can be reduced into two subproblems:%
%
\begin{enumerate}
  \item Finding all instances of instructions that can implement one or more
    \glspl{operation} in the \gls{program}.
  \item Selecting a subset of these instances such that all \glspl{operation}
    are implemented.
\end{enumerate}
%
Without loss of generality, assume that the input to the \gls{instruction
  selector} to consist of a single \gls{function}, which in turn consists of
many \glspl{basic block} (henceforth referred to as simply \glspl!{block}).

Most \glspl{compiler} solve the two problems above using graph-based methods.
%
First, the \gls{IR} code is transformed into a \gls!{data-flow graph}, where
nodes represent \glspl{operation} in the function and edges represent data
dependencies between the \glspl{operation}.
%
\Glspl{data-flow graph} are called \glspl!{expression tree} if they are limited
to single, tree-shaped expressions, \glspl!{block graph} if they capture many
expressions in a \gls{block} as a \gls{DAG}, and \glspl{data-flow graph} if they
capture the data flow of entire functions.
%
Corresponding \glspl{data-flow graph}, called \glspl!{pattern graph} (or simply
\glspl!{pattern}), are also built to represent the \gls{instruction} provided by
the \gls{target machine}.
%
The set of \glspl{pattern graph} for a particular \gls{target machine}
constitute a \gls!{pattern set}.

\begin{filecontents*}{p-match-sel-example.c}
x = A[i + 1];
\end{filecontents*}

\begin{figure}
  \centering%
  \subcaptionbox{C code\labelFigure{p-match-sel-example-c}}%
                {\lstinputlisting[language=c]{p-match-sel-example.c}}%
  \hspace{5mm}%
  \subcaptionbox{%
                  Instructions. The $*s$ notion means ``get value at address $s$
                  in memory''%
                  \labelFigure{p-match-sel-example-instrs}%
                }%
                [50mm]%
                {%
                  \small
                  \begin{tabular}{%
                                   >{\instrFont}r@{\hspace{4pt}}%
                                   >{$}l<{$}@{ $\leftarrow$ }%
                                   >{$}l<{$}%
                                 }
                    add  & r & s + t\\
                    mul  & r & s \times t\\
                    mad  & r & s \times t + u\\
                    load & r & *s\\
                    load & r & *(s \times t + u)
                  \end{tabular}%
                }%
  \hspace{5mm}%
  \subcaptionbox{%
                   Corresponding \gls{block graph} and \glspl{match}%
                   \labelFigure{p-match-sel-example-graph}%
                }{%
                  \small%
                  % Copyright (c) 2017, Gabriel Hjort Blindell <ghb@kth.se>
%
% This work is licensed under a Creative Commons 4.0 International License (see
% LICENSE file or visit <http://creativecommons.org/licenses/by/4.0/> for a copy
% of the license).
%
\begingroup%
\figureFont%
\pgfdeclarelayer{background1}%
\pgfdeclarelayer{background2}%
\pgfdeclarelayer{background3}%
\pgfdeclarelayer{background4}%
\pgfsetlayers{background4,background3,background2,background1,main}%
\begin{tikzpicture}[%
    data node/.style={%
      op node,
    },
    match node/.style={%
      match line,
      draw,
      circle,
      inner sep=.5pt,
      outer sep=0,
    },
    outer match node/.style={%
      match node,
      draw=none,
      inner sep=0,
    },
    match label/.style={%
      minimum size=0,
      inner sep=0,
      outer sep=1pt,
      node distance=10pt,
      font=\footnotesize,
    },
  ]

  % Graph
  \node [op node] (add1) {\opAdd};
  \node [data node, position=135 degrees from add1] (i) {\opVar{i}};
  \node [data node, position= 45 degrees from add1] (c1) {\opVar{1}};
  \node [op node, position=-45 degrees from add1] (mul) {\opMul};
  \node [data node, position= 45 degrees from mul] (c4) {\opVar{4}};
  \node [op node, position=-45 degrees from mul] (add2) {\opAdd};
  \node [data node, position= 45 degrees from add2] (A) {\opVar{A}};
  \node [op node, below=of add2] (load) {\opLoad};
  \begin{scope}[data flow]
    \draw (i) -- (add1);
    \draw (c1) -- (add1);
    \draw (add1) -- (mul);
    \draw (c4) -- (mul);
    \draw (mul) -- (add2);
    \draw (A) -- (add2);
    \draw (add2) -- (load);
  \end{scope}

  \begin{pgfonlayer}{background1}
    % Matches
    \node [match node, fill=shade2, fit=(add1)] (m1) {};
    \node [match node, fill=shade2, fit=(add2)] (m2) {};
    \node [match node, fill=shade2, fit=(mul)] (m3) {};
    % m4
    \begin{pgfonlayer}{background2}
      \node [outer match node, inner sep=-1pt, fit=(m3)] (m4a) {};
      \node [outer match node, inner sep=-1pt, fit=(m2)] (m4b) {};
      \draw [match line, bend left=45, fill=shade1]
            (m4a.south west)
            to
            (m4a.north west)
            to
            (m4a.north east)
            --
            (m4b.north east)
            to
            (m4b.south east)
            to
            (m4b.south west)
            -- coordinate (m4)
            cycle;
    \end{pgfonlayer}
    \node [match node, fill=shade1, fit=(load)] (m5) {};
    % m6
    \begin{pgfonlayer}{background4}
      \node [outer match node, inner sep=1pt, fit=(m3)] (m6a) {};
      \node [outer match node, inner sep=1pt, fit=(m2)] (m6b) {};
      \node [outer match node, inner sep=1pt, fit=(m5)] (m6c) {};
    \end{pgfonlayer}
    % The named paths must be declared outside of pgfonlayer, or else they will
    % disappear when the environment ends
    \path [name path=m6a-to-m6b-north]
          (m6a.north east)
          --
          (m6b.north east)
          --
          +(-45:5mm);
    \path [name path=m6a-to-m6b-south]
          (m6a.south west)
          --
          (m6b.south west)
          --
          +(-45:5mm);
    \path [name path=m6c-to-m6b-east]
          (m6c.east)
          --
          (m6c.east |- m6b.north);
    \path [name path=m6c-to-m6b-west]
          (m6c.west)
          --
          (m6c.west |- m6b.west);
    \begin{pgfonlayer}{background4}
      \draw [match line, bend left=45, fill=shade3,
             name intersections={of=m6a-to-m6b-north and m6c-to-m6b-east,
                                 by=x-east},
             name intersections={of=m6a-to-m6b-south and m6c-to-m6b-west,
                                 by=x-west},
             rounded corners=4pt,
            ]
            (m6a.south west)
            to
            (m6a.north west)
            to
            (m6a.north east)
            --
            (x-east)
            --
            (m6c.east)
            to
            (m6c.south)
            to
            (m6c.west)
            -- coordinate [pos=.25] (m6)
            (x-west)
            --
            cycle;
    \end{pgfonlayer}

    % Fill overdrawn edges of m6
    \draw [match line, bend left=45, fill=none,
           name intersections={of=m6a-to-m6b-north and m6c-to-m6b-east,
                               by=x-east},
           name intersections={of=m6a-to-m6b-south and m6c-to-m6b-west,
                               by=x-west},
           rounded corners=4pt,
          ]
          (m6a.south west)
          to
          (m6a.north west)
          to
          (m6a.north east)
          --
          (x-east)
          --
          (m6c.east)
          to
          (m6c.south)
          to
          (m6c.west)
          --
          (x-west)
          --
          cycle;
  \end{pgfonlayer}

  % Match labels
  \node [match label, left=of m1] (m1l) {$\strut m_1$};
  \node [match label, right=of m2] (m2l) {$\strut m_2$};
  \node [match label, left=of m3] (m3l) {$\strut m_3$};
  \node [match label, position=-135 degrees from m4] (m4l) {$m_4$};
  \node [match label, right=of m5] (m5l) {$\strut m_5$};
  \node [match label, left=of m6] (m6l) {$\strut m_6$};
  \foreach \i in {1, ..., 6} {
    \draw [thick] (m\i) -- (m\i l);
  }
\end{tikzpicture}%
\endgroup%
%
                }

  \caption[An example of the pattern matching and selection problem]%
          {%
            An example demonstrating the pattern matching and selection
            problem for a program that loads a value from integer array
            \irVar{A} at offset \irVar{i} $+$ \irVar{1} (it is assumed that
            \irVar{A} is stored in memory and that an integer is 4~bytes).
            Valid covers are \mbox{$\mSet{m_1, m_2, m_3, m_5}$},
            \mbox{$\mSet{m_1, m_4, m_5}$}, and \mbox{$\mSet{m_1, m_6}$}%
          }
  \labelFigure{p-match-sel-example}
\end{figure}

The first subproblem can be reduced to finding all instances where a
\gls{pattern} from the \gls{pattern set} is subgraph isomorphic to~$G$, where
$G$ denotes either a \gls{block graph} or a \gls{function graph}.
%
Each such instance is called a \gls!{match}, and the set of all \glspl{match}
constitute a \gls!{match set}, which is denoted by~$M$.
%
Hence this subproblem is referred to as the \gls!{pattern matching}[ problem].
%
\Gls{pattern matching} can be done in linear time if both $G$ and all
\glspl{pattern} are tree-shaped, otherwise it is an NP-complete
problem~\cite{GareyJohnson:1979,HoffmannODonnell:1982}.
%
Having found $M$, the second subproblem -- which is referred to as the
\gls!{pattern selection}[ problem] -- can be reduced to selecting a set of
\glspl{match} that \gls{cover}[s]~$G$.
%
A subset~\mbox{$C \subseteq M$}, where $M$ is a \gls{match set},
\gls!{cover}[s] $G$ if every \gls{operation} in $G$ appears in exactly one match
from~$C$.
%
Such a subset is called a \gls!{cover}.
%
See \refFigure{p-match-sel-example} for an example.

For a given \gls{program} and \gls{target machine}, there often exists many
valid combinations of \glspl{instruction} -- in terms of $G$ and $M$, this means
there exist many \glspl{cover} of~$G$ -- which may result in code where quality
differs significantly.
%
In certain cases, the performance of two sets of selected instructions may
differ by as much as two orders of magnitude~\cite{ZivojnovicEtAl:1994}.
%
Consequently, the \gls{pattern selection} problem -- originally defined to
accept any valid \gls{cover} -- is augmented into an optimization problem called
\gls!{optimal.ps} \gls{pattern selection}, where only \glspl{cover} with least
cost are accepted.
%
The cost of a \gls{cover}~$C$ is the sum of the costs for the \glspl{match}
appearing in~$C$, where the cost of a \gls{match} has been assigned to reflect a
desired characteristic in the produced code.
%
For example, assume that the \instrCode{add}, \instrCode{mul}, and
\instrCode{mad} \glspl{instruction} in \refFigure{p-match-sel-example} take one
cycle to execute whereas the \instrCode{load} \glspl{instruction} take five
cycles to execute.
%
Assume further that the \gls{compiler} should maximize performance.
%
The corresponding matches \mbox{$m_1, m_2, m_3, m_4, m_5, m_6$} are therefore
assigned costs \mbox{$1, 1, 1, 1, 5, 5$}, respectively.
%
Then, of the valid \glspl{cover} \mbox{$\mSet{m_1, m_2, m_3, m_5}$},
\mbox{$\mSet{m_1, m_4, m_5}$}, and \mbox{$\mSet{m_1, m_6}$}, only the last
\gls{cover} is considered \gls{optimal.ps} as it has a total cost of~6 whereas
the other two \glspl{cover} have costs~8 and~7, respectively.
%
There is typically a strong correlation between the size of a \gls{cover} and
its cost -- smaller \glspl{cover} lead to less cost, and ultimately better code
-- but this depends heavily on the properties of the \gls{target machine}.



\section{Traditional Approaches}

In the interest of producing decent code while maintaining short compilation
times, most modern \glspl{compiler} either limit themselves to \glspl{block
  graph} or apply greedy, linear-time heuristics.
%
For \glspl{block graph} one can find \gls{least-cost.c} \glspl{cover} in linear
time, but at the price of precluding optimization decisions that only become
possible when operating on \glspl{function graph}.



\subsection{Operating on Block Graphs}

The most common and well known technique that operates on \glspl{block graph}
was introduced by \textcite{AhoEtAl:1989}.
%
The approach assumes the \glspl{instruction} to be expressed in a particular
form, which will therefore be discussed first.



\subsubsection{Machine Grammars}

The \gls{instruction set} of a \gls{target machine} is typically described in a
machine-readable \gls{machine description}, which can take many forms.
%
A common form is to describe the \glspl{instruction} as a \gls!{machine grammar}
(or simply called \gls!{grammar}).
%
\Glspl{machine grammar} are based on \glspl{context-free
  grammar}~\cite{AhoEtAl:2006}, which are used for describing language syntax.
%
A \gls{grammar} consists of \glspl!{terminal}, \glspl!{nonterminal}, and
\glspl!{rule}.
%
In this context, a \gls{terminal} is a symbol representing an \gls{operation}
(e.g.\ \opAdd, \opLT, \opLoad), and a \gls{nonterminal} is a symbol representing
an abstract result (e.g.\ $\mNT{Reg}$) produced by the \gls{instruction}.
%
To distinguish between the two, \glspl{terminal} are written entirely in lower
case whereas \glspl{nonterminal} start with a capital letter and are set in
italics.
%
A \gls{rule} describes the behavior of an \gls{instruction} and consists of a
\gls!{production}, a cost, and an action.
%
\Glspl{production} describe how derive \glspl{nonterminal}, and are written as
%
\begin{displaymath}
  A \rightarrow B C \ldots
\end{displaymath}
%
where the left-hand side is consists a \gls{nonterminal} and the right-hand side
consists of a series of \glspl{terminal} and \glspl{nonterminal}.
%
Each \gls{instruction} therefore gives rise to one or more \glspl{production},
where the right-hand side of a \gls{production} captures a \gls{pattern} of the
\gls{instruction} and the left-hand side denotes the result produced by the
\gls{instruction}.
%
The \glspl{production} are typically written in Polish notation to avoid the
need for parentheses (for example, \mbox{$1 + 2$} is written as \mbox{$+ \; 1 \;
  2$}).
%
An example is shown in \refTable{grammar-rules-example}.

\begin{table}[t]
  \centering%
  \begin{tabular}{cr@{ $\rightarrow$ }lcl}
    \toprule
    \# & \multicolumn{2}{c}{Production} & Cost & \multicolumn{1}{c}{Action}\\
    \midrule
    1 & $\mNT{Reg}[1]$ & $\opLoad + \mNT{Reg}[2] \; \irCode{imm}$
      & 1
      & emit: {\instrFont load \$$\mNT{Reg}[1]$, imm(\$$\mNT{Reg}[2]$)}\\
    2 & $\mNT{Reg}[1]$ & $\opLoad + \irCode{imm} \; \mNT{Reg}[2]$
      & 1
      & emit: {\instrFont load \$$\mNT{Reg}[1]$, imm(\$$\mNT{Reg}[2]$)}\\
    3 & $\mNT{Reg}[1]$ & $\opLoad \; \mNT{Reg}[2]$
      & 1
      & emit: {\instrFont load \$$\mNT{Reg}[1]$, 0(\$$\mNT{Reg}[2]$)}\\
    \bottomrule
  \end{tabular}%

  \caption[An example of machine grammar rules]%
          {%
            An example of machine grammar rules corresponding to a
            \mbox{\instrFont load \$t, o(\$s)} \gls{instruction} that loads a
            value from memory at the address given in register~\instrCode{s},
            offset by an immediate value~\instrCode{o}, and stores the loaded
            value in register~\instrCode{t}, in one cycle.
            %
            The subscripts are only needed for reference%
          }
  \labelTable{grammar-rules-example}
\end{table}

\todo{explain rule reduction}



\paragraph{Linear-form Grammars}

To simplify \gls{pattern matching} and \gls{pattern selection}, a \gls{machine
  grammar} can be rewritten into \gls!{linear form.g}. A \gls{grammar} is in
\gls{linear form.g} if every \gls{rule} in the \gls{grammar} has a
\gls{production} in one of the following forms:
%
\begin{enumerate}
  \item \mbox{$\mNT{N} \rightarrow \irCode{op} \; \mNT{A}[1] \; \mNT{A}[2]
    \ldots \mNT{A}[n]$}, where \irCode{op} is a \gls{terminal}, representing an
    \gls{operation} that takes $n$ arguments, and all $\mNT{A}[i]$ are
    \glspl{nonterminal}.
    %
    Such rules are called \gls!{base.r}[ \glspl{rule}].
  \item \mbox{$\mNT{N} \rightarrow \irCode{t}$}, where \irCode{t} is a
    \gls{terminal}.
    %
    Such rules are also called \gls{base.r} \glspl{rule}.
  \item \mbox{$\mNT{N} \rightarrow \mNT{A}$}, where $\mNT{A}$ is a
    \gls{nonterminal}.
    %
    Such rules are called \gls!{chain.r}[ \glspl{rule}].
\end{enumerate}

\begin{table}[t]
  \centering%
  \begin{tabular}{cr@{ $\rightarrow$ }lcl}
    \toprule
    \# & \multicolumn{2}{c}{Production} & Cost & \multicolumn{1}{c}{Action}\\
    \midrule
    1 & $\mNT{Reg}$ & $\opLoad \; \mNT{A}$
      & 1
      & emit: {\instrFont load \$$\mNT{Reg}$, $A.\text{imm}$(\$$A.\mNT{Reg}$)}\\
    4 & $\mNT{A}$ & $+ \; \mNT{Reg} \; \irCode{imm}$
      & 0
      & \\
    2 & $\mNT{Reg}$ & $\opLoad \; \mNT{B}$
      & 1
      & emit: {\instrFont load \$$\mNT{Reg}$, $B.\text{imm}$(\$$B.\mNT{Reg}$)}\\
    5 & $\mNT{B}$ & $+ \; \irCode{imm} \; \mNT{Reg}$
      & 0
      & \\
    3 & $\mNT{Reg}[1]$ & $\opLoad \; \mNT{Reg}[2]$
      & 1
      & emit: {\instrFont load \$$\mNT{Reg}[1]$, 0(\$$\mNT{Reg}[2]$)}\\
    \bottomrule
  \end{tabular}%

  \caption[The grammar from \refTable{grammar-rules-example} in linear form]%
          {%
            The grammar from \refTable{grammar-rules-example} in linear form.
            %
            The nonterminals~$\mNT{A}$ and~$\mNT{B}$ are introduced in order to
            connect rules~1 with~4 and~2 with~5, respectively%
          }
  \labelTable{linear-form-grammar-example}
\end{table}

A \gls{machine grammar} can be mechanically rewritten into \gls{linear form.g}
by introducing new \glspl{nonterminal} and breaking down illegal \glspl{rule}
into multiple, smaller \glspl{rule} until the \gls{grammar} is in \gls{linear
  form.g}.
%
For example, rewriting the \gls{grammar} shown in
\refTable{grammar-rules-example} into \gls{linear form.g} results in the
\gls{grammar} shown in \refTable{linear-form-grammar-example}.
%
Note that the new \glspl{rule} have zero cost and no action as these are only
intermediary steps towards enabling reduction of the original \gls{rule}.

Since all \glspl{production} in a \glsshort{linear form.g} \gls{grammar} have at
most one \gls{terminal}, the \gls{pattern matching} problem becomes trivial.

\todo{why is this better}



\subsubsection{TODO}

The technique by \citeauthor{AhoEtAl:1989} is centered around the following
assumption: the cost of reducing node~$n$ in an \gls{expression tree} to
\gls{nonterminal}~$s$ using \gls{rule}~$r$ is the cost of $r$ plus the costs of
reducing all children of $n$ to the appropriate \glspl{nonterminal} appearing on
the right-hand side of~$r$.
%
If $r$ is a \gls{chain.r} \gls{rule} then the cost is computed as the cost of
$r$ plus the cost of reducing $n$ to the right-hand-side \gls{nonterminal}
of~$r$.
%
The recursive nature of these costs can be exploited using dynamic programming,
resulting in the algorithm shown in \refAlgorithm{aho-etal-cost-algorithm} which
computes the least cost of reducing a given \gls{expression tree} to a
particular \gls{nonterminal}.

\begin{algorithm}[t]
  \DeclFunction{ComputeCosts (expression tree $T$, grammar $G$)}{
    $S$ \Assign $\mSetBuilder{s}%
                             {\text{$s$ is a nonterminal in $G$}}$\;
    $C$ \Assign matrix of size $|T| \times |S|$, initialized with $\infty$
                cost\;
    ComputeCostsRec (root node of $T$)\;
    \Return{$C$}\;
    \BlankLine
    \DeclFunction{ComputeCostsRec (node $n$)}{
      \ForEach{child $m$ of $n$}{
        ComputeCostsRec ($m$)\;
      }
      $R$ \Assign $\mSetBuilder{r}{\text{%
                                          $r \in G$,
                                          $r$ is a base rule that matches at $n$
                                            \Or $r$ is a chain rule%
                                        }}$\;
      \ForEach{rule $r \in R$, in transitive reduction order}{%
        $c$ \Assign ComputeRuleCost ($n$, $r$)\;
        $l$ \Assign left-hand-side nonterminal of $r$\;
        \If{$c < c_{n,l}$.cost}{%
          $c_{n,l}$.cost \Assign $c$\;
          $c_{n,l}$.rule \Assign $r$\;
        }
      }
    }

    \BlankLine
    \DeclFunction{ComputeRuleCost (node $n$, rule $r$)}{%
      $c$ \Assign cost of $r$\;
      \eIf{$r$ is a chain rule}{%
        $s$ \Assign right-hand-side nonterminal of $r$\;
        $c$ \Assign $c$ $+$ $c_{n,s}$.cost
        \cmt*{here cost of node itself is taken instead of its children}
      }{%
        \For{$i \leftarrow 1$ \KwTo $|n|$}{%
          $m$ \Assign $i$th child of $n$\;
          $s$ \Assign $i$th right-hand-side nonterminal of $r$\;
          $c$ \Assign $c$ $+$ $c_{m,s}$.cost\;
        }
      }
      \Return{$c$}\;
    }
  }

  \caption[%
            Algorithm for computing the least cost of reducing a given
            expression tree to a particular nonterminal%
          ]{%
            Algorithm for computing the least cost of reducing a given
            expression tree to a particular nonterminal. It runs in linear time
            and assumed the given grammar to be in linear form%
          }
  \labelAlgorithm{aho-etal-cost-algorithm}
\end{algorithm}

The algorithm works as follows.
%
It first constructs a cost matrix~$C$, where each row represents a node in the
\gls{expression tree} and each column represents a \gls{nonterminal} in the
\gls{grammar}, which is assumed to be in \gls{linear form.g}.
%
The cost in each element~\mbox{$c_{i, j}$} is initialized to infinity,
indicating that there exists no chain of \glspl{rule reduction} which reduces
node~$i$ to \gls{nonterminal}~$j$.
%
It then computes the costs by traversing the \gls{expression tree} bottom up.
%
At each node~$n$ it constructs a \gls{rule} set~$R$, containing all matching
\gls{base.r} \glspl{rule} -- that is, all \glspl{rule} with a \gls{terminal} on
the right-hand side that matches the type of~$n$ (this can be trivially found in
linear time as the grammar is in \gls{linear form.g}) -- as well as all
\gls{chain.r} \glspl{rule}.
%
For each \gls{rule}~$r$ in~$R$, it computes the cost of applying $r$ at~$n$
according to the scheme stated above.
%
If done carelessly, the \glspl{rule} need to be processed until the costs for
the current node reaches a fixpoint.
%
But if iterated in so-called \gls!{transitive reduction order} -- that is, all
\gls{base.r} \glspl{rule} appear before any \gls{chain.r} \gls{rule}, and if two
\gls{chain.r} \glspl{rule}~$i$ and~$j$ are of the form \mbox{$a \rightarrow b$}
and \mbox{$b \rightarrow c$}, then $j$ appears before $i$ -- then each
\gls{rule} need to be processed only once per node.
%
Since every node is also only processed once, the algorithm runs in linear time
with respect to the size of the \gls{expression tree}.

\begin{algorithm}[t]
  \DeclFunction{Select (expression tree $T$,
                        goal nonterminal $g$,
                        cost matrix $C$)}{%
    $n$ \Assign root node of $T$\;
    $r$ \Assign $c_{n,g}$.rule\;
    \eIf{$r$ is a chain rule}{%
      $s$ \Assign right-hand-side nonterminal of $r$\;
      Select ($T$, $s$)\;
    }{%
      \For{$i \leftarrow 1$ \KwTo $|n|$}{%
        $m$ \Assign $i$th child of $n$\;
        $s$ \Assign $i$th right-hand-side nonterminal of $r$\;
        Select (expression tree rooted at $m$, $s$)\;
      }
    }
    execute actions associated with $r$\;
  }

  \caption{%
            Algorithm for selecting rules based on costs computed by
            \refAlgorithm{aho-etal-cost-algorithm}%
          }
  \labelAlgorithm{aho-etal-select-algorithm}
\end{algorithm}

Having computed the costs, the optimal order of \glspl{rule reduction} -- which
is equivalent to the \gls{least-cost.c} \gls{cover} -- can be found using the
algorithm shown in \refAlgorithm{aho-etal-select-algorithm}.

\todo{explain benefits?}
\todo{explain drawbacks}



\subsection{Operating on Function Graphs}

\todo{write}



\section{Combinatorial Approaches}

\todo{write}
