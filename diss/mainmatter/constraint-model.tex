% Copyright (c) 2017, Gabriel Hjort Blindell <ghb@kth.se>
%
% This work is licensed under a Creative Commons 4.0 International License (see
% LICENSE file or visit <http://creativecommons.org/licenses/by/4.0/> for a copy
% of the license).

\chapter{Constraint Model}
\labelChapter{constraint-model}

This chapter introduces the \gls{constraint model} for \gls{universal
  instruction selection}.
%
\RefSectionRange{modeling-global-instruction-selection}{modeling-block-ordering}
introduce the \glspl{variable} and \gls{constraint} for modeling \gls{global.is}
\gls{instruction selection}, \gls{global code motion}, \gls{data copying},
\gls{value reuse}, and \gls{block ordering}, respectively.
%
The \glsshort{constraint model} limitations are discussed in
\refSection{cm-limitations}.



\section{Modeling Global Instruction Selection}
\labelSection{modeling-global-instruction-selection}

Modeling \gls{global.is} \gls{instruction selection} entails that all
\glspl{operation} must be \gls{cover}[ed] and all \glspl{datum} must be
\gls{define.d}[d].
%
This could, however, lead to situations resulting in cyclic data dependencies,
which must be forbidden.


\subsection{Covering Operations and Defining Data}

In \gls{global.is}[ \gls{instruction selection}], a set of \glspl{match} must be
selected such that every \gls{operation} in a given \gls{UF graph} is covered.
%
There are two variants of this problem:
%
\begin{enumerate*}[label=(\arabic*)]
  \item each \gls{operation} must appear in \emph{exactly} one selected
    \gls{match}; and
%
  \item each \gls{operation} must appear in \emph{at least} one selected
    \gls{match}, hence allowing matches to \gls{overlap}.
\end{enumerate*}
%
The former problem is more common as it is stricter, resulting in simpler models
with smaller \glspl{solution space}.
%
It also allows use of \glspl{constraint} that enable strong \gls{propagation},
which is essential for curbing solving time and increasing scalability.

Depending on the \gls{instruction set}, the latter problem permits
\glspl{solution} with potentially higher code quality.
%
For example, assume a \gls{UF graph} where a sum is used as address in two
memory operations and a \gls{target machine} where the address can be computed
as part of the memory instructions.
%
A \gls{solution} to the latter problem would therefore only need two
instructions, whereas the former problem would require three instructions -- one
to compute the sum and two to perform the memory operations -- since the
addition is not allowed to be covered by both memory instructions.
%
In certain conditions, however, an add instruction may still be required.
%
For example, assume a \gls{UF graph} where the sum is also used in a
subtraction.
%
For this \gls{UF graph}, unless the \gls{target machine} has an instruction that
performs both an addition and a subtraction, a \gls{solution} to either problem
requires an add instruction to compute the sum.
%
Due to the increased complexity of the relaxed version of the problem, we model
exact coverage in this dissertation.

Similarly, every value and state must be produced by exactly one selected match.
%
If a \gls!{datum}~$d$ denotes either a \glsshort{state node} or \gls{value node}
in the \gls{UF graph}, then we say that a \gls{match}~$m$ \gls!{define.d}[s] $d$
if there exists an inbound \glsshort{state-flow edge} or \gls{data-flow edge} to
$d$ in the \gls{UP graph} from which $m$ was derived.
%
Similarly, $m$ \gls!{use.d}[s] $d$ if there exists an outbound
\glsshort{state-flow edge} or \gls{data-flow edge} to $d$ in the \gls{UP graph}
of $m$.


\subsubsection{Variables}

Given a \gls{UF graph}~$\mUFGraph$ and a set~$\mMatchSet$ of \glspl{match} found
for $\mUFGraph$, the set of \glspl{variable} \mbox{$\mVar{sel}[m] \in \mSet{0,
    1}$} models whether \gls{match}~\mbox{$m \in \mMatchSet$} is selected.
%
Hence $m$ is selected if \mbox{$\mVar{sel}[m] = 1$}, abbreviated
$\mVar{sel}[m]$, and not selected if \mbox{$\mVar{sel}[m] = 0$}, abbreviated
\mbox{$\neg\mVar{sel}[m]$}.

The set of \glspl{variable} \mbox{$\mVar{omatch}[o] \in \mMatchSet[o]$} models
which selected \gls{match} covers \gls{operation}~\mbox{$o \in
  \mOpSet$\hspace{-1pt}}, where $\mOpSet$ denotes the set of \glspl{operation}
in $\mUFGraph$, and \mbox{$\mMatchSet[o] \subseteq \mMatchSet$} denotes the set
of \glspl{match} that can cover~$o$.
%
Similarly, the set of \glspl{variable} \mbox{$\mVar{dmatch}[d] \in
  \mMatchSet[o]$} models which selected \gls{match} \gls{define.d}[s]
\gls{datum}~\mbox{$d \in \mDataSet$\hspace{-1.5pt}}, where $\mDataSet$ denotes
the set of \glspl{datum} in $\mUFGraph$, and \mbox{$\mMatchSet[d] \subseteq
  \mMatchSet$} denotes the set of \glspl{match} that can \gls{define.d}~$d$.


\subsubsection{Constraints}

The \gls{constraint} that every \gls{operation} must be covered is modeled as
%
\begin{equation}
  \mVar{omatch}[o] = m \mEq \mVar{sel}[m],
  \forall o \in \mOpSet \hspace{-1pt},
  \forall m \in \mMatchSet[o] \hspace{-1pt}.
  \labelEquation{operation-coverage}
\end{equation}
%
In other words, an \gls{operation}~$o$ is covered by a match~$m$ if and only if
$m$ is selected.

Likewise, the \gls{constraint} that every \gls{datum} must be \gls{define.d}[d]
is modeled as
%
\begin{equation}
  \mVar{dmatch}[d] = m \mEq \mVar{sel}[m],
  \forall d \in \mDataSet \hspace{-1.5pt},
  \forall m \in \mMatchSet[d] \hspace{-1pt}.
  \labelEquation{data-definitions}
\end{equation}
%
In other words, a \gls{datum}~$d$ is \gls{define.d}[d] by a match~$m$ if and
only if $m$ is selected.


\subsection{Forbidding Cyclic Data Dependencies}
\labelSection{forbidding-cyclic-data-dependencies}

In certain cases, selecting \glspl{match} of \glspl{instruction} producing
multiple results -- for example, many modern processors provide memory
\glspl{instruction} that automatically increment or decrement the address value
-- could lead to cyclic data dependencies~\cite{EbnerEtAl:2008}.
%
\begin{filecontents*}{cyclic-data-deps-example-ir.c}
$\ldots$
$\irAssign{\irVar{p}[2]}{\irAdd{\irVar{p}[1]}{\irVar{4}}}$
$\irStore{\irVar{q}[1]}{\irVar{p}[2]}$
$\irAssign{\irVar{q}[2]}{\irAdd{\irVar{q}[1]}{\irVar{4}}}$
$\irStore{\irVar{p}[1]}{\irVar{q}[2]}$
\end{filecontents*}%
%
\begin{figure}
  \subcaptionbox{IR\labelFigure{cyclic-data-deps-example-ir}}%
                {%
                  \lstinputlisting[language=c,mathescape]%
                                  {cyclic-data-deps-example-ir.c}%
                }%
  \hfill%
  \subcaptionbox{%
                  UF graph, covered by two matches derived from an
                  auto-increment store instruction.
                  %
                  For brevity, the state node are not included%
                  \labelFigure{cyclic-data-deps-example-uf-graph}%
                }%
                [62mm]%
                {%
                  \input{%
                    figures/modeling-global-instruction-selection/%
                    cyclic-data-deps-example-uf-graph%
                  }%
                }%
  \hfill%
  \subcaptionbox{%
                  Dependency graph%
                  \labelFigure{cyclic-data-deps-example-dep-graph}%
                }%
                [32mm]%
                {%
                  \input{%
                    figures/modeling-global-instruction-selection/%
                    cyclic-data-deps-example-dep-graph%
                  }%
                }

  \caption{Example illustrating cyclic data dependencies}%
  \labelFigure{cyclic-data-deps-example}%
\end{figure}
%
An example of such a situation is given in \refFigure{cyclic-data-deps-example}.
%
If both \glspl{match} are selected, then either value~\irVar{p}[2] or
value~\irVar{q}[2] will be \gls{use.d}[d] before it is available (depending on
the instruction order), thus resulting in incorrect code.
%
Consequently, such combinations must be identified and forbidden.

We detect such combinations -- which could involve more than two \glspl{match}
-- by first constructing a \gls!{dependency graph}, where each \gls{node}
represents a \gls{match} and each directed \gls{edge}~$\mPair{n}{m}$ indicates
that \gls{match}~$m$ \gls{use.d}[s] \glspl{datum} produced by \gls{match}~$n$.
%
A \gls{cycle} in this \gls{graph} corresponds a combination of \glspl{match}
which will lead to a cyclic data dependency if all \glspl{match} are selected.
%
Hence, we find all \glspl{cycle} in the \gls{dependency graph} -- we applied
\citeauthor{Johnson:1975}'s algorithm for this task~\cite{Johnson:1975} -- and
add \glspl{constraint} forbidding selection of all \glspl{match} appearing in a
\gls{cycle}.


\subsubsection{Constraints}

Given a set~\mbox{$\mForbiddenCombSet \subseteq \mPowerset{\mMatchSet}$} of
\glspl{cycle} found for the \gls{dependency graph} built from a \gls{UF graph}
and \gls{match set}, the \gls{constraint} forbidding cyclic data dependencies is
modeled as
%
\begin{equation}
  \sum_{m \in f} \mVar{sel}[m] < \mCard{f},
  \forall f \in \mForbiddenCombSet.
  \labelEquation{cyclic-data-deps}
\end{equation}
%
In other words, the number of selected \glspl{match} must be strictly less than
the number of \gls{match} appearing in the \gls{cycle}.


\section{Modeling Global Code Motion}
\labelSection{modeling-global-code-motion}

The \gls{global code motion} problem entails that \glspl{datum} must be placed
in \glspl{block} such that each definition of a \gls{datum}~$d$ precedes all
\gls{use.d}[s] of~$d$.
%
This condition can be expressed in terms of \gls{block} dominance.
%
Given a \gls{function}~$f$, a \gls{block}~$b$ in $f$ \gls!{dominate.b}[s]
another \gls{block}~$c$ in $f$ if $b$ appears on every control-flow path from
$f$'s \gls{entry block} to $c$ (see \refFigure{block-dominance-example} for an
example).
%
\begin{figure}
  \mbox{}%
  \hfill%
  \subcaptionbox{Control-flow graph\labelFigure{block-dominance-example-cfg}}%
                [34mm]%
                {%
                  \input{%
                    figures/modeling-global-code-motion/block-dominance-example%
                  }%
                }%
  \hfill%
  \subcaptionbox{Dominance\labelFigure{block-dominance-example-doms}}%
                {%
                  \figureFontSize%
                  \begin{tabular}{cc}
                    \toprule
                      \tabhead Block
                    & \tabhead dominates\\
                    \midrule
                      \irBlock{entry}
                    & $\mSet{\irBlock{entry}, \irBlock{A}, \irBlock{B},
                        \irBlock{C}, \irBlock{D}, \irBlock{E}}$\\
                      \irBlock{A}
                    & $\mSet{\irBlock{A}, \irBlock{B}, \irBlock{C},
                        \irBlock{D}}$\\
                      \irBlock{B}
                    & $\mSet{\irBlock{B}}$\\
                      \irBlock{C}
                    & $\mSet{\irBlock{C}}$\\
                      \irBlock{D}
                    & $\mSet{\irBlock{D}}$\\
                      \irBlock{E}
                    & $\mSet{\irBlock{E}}$\\
                    \bottomrule
                  \end{tabular}%
                }%
  \hfill%
  \mbox{}

  \caption{Example of block dominance}
  \labelFigure{block-dominance-example}
\end{figure}%
%
By definition, a \gls{block} always \gls{dominate.b}[s] itself.
%
Hence a placement of \glspl{match} into \glspl{block} is a \gls{solution} to the
\gls{global code motion} problem if each \gls{datum}~$d$ is \gls{define.d}[d] by
some selected \gls{match} placed in a \gls{block}~$b$, and every
non-\gls{phi-match} \glsshort{use.d}[ing] $d$ is placed in a \gls{block}
\gls{dominate.b}[d] by~$b$.
%
The \glspl{phi-match} must be excluded since, due to the \glspl{definition
  edge}, at least one \gls{datum} used by such \glspl{match} must be
\gls{define.d}[d] in a \gls{block} that does not \gls{dominate.b} the
\gls{block} wherein the \gls{phi-match} must be placed.


\subsubsection{Variables}

The set of \glspl{variable} \mbox{$\mVar{oplace}[o] \in \mBlockSet$} models in
which \gls{block} \gls{operation}~$o$ is placed, where $\mBlockSet$ denotes the
set of \glspl{block} in $\mUFGraph$.
%
Likewise, the set of \glspl{variable} \mbox{$\mVar{dplace}[d] \in \mBlockSet$}
models in which \gls{block} the definition of \gls{datum}~$d$ is placed.


\subsubsection{Constraints}

Intuitively, all \glspl{operation} covered by a \gls{match}~$m$ must be placed
in the same \gls{block} wherein $m$ itself is placed.
%
Hence, if \mbox{$\mCovers(m) \subseteq \mOpSet$} denotes the set of
\glspl{operation} covered by \gls{match}~$m$, then this \gls{constraint} is
modeled as
%
\begin{equation}
  \begin{array}{c}
    \mVar{sel}[m] \mImp \mVar{oplace}[o_1] = \mVar{oplace}[o_2], \\
    \forall m \in \mMatchSet,
    \forall o_1\hspace{-1pt}, o_2 \in \mCovers(m).
  \end{array}
  \labelEquation{operation-placement}
\end{equation}
%
This also enables the placement of $m$ to be deduced from any of the
corresponding $\mVar{oplace}$~\glspl{variable}.

We prevent control-flow \glspl{operation} from being moved to another
\gls{block}, which in all likelihood would break \gls{program} semantics, by
forcing selected \glspl{match} with control flow to be placed in the \gls{block}
matched by the \gls{UP graph}'s \gls{entry block}.
%
Hence, if \mbox{$\mEntry(m) \subseteq \mBlockSet$} returns either the empty set
or a set containing only the \gls{entry block} of match~$m$ (when the \gls{UP
  graph} of $m$ has such a node), then this \gls{constraint} is modeled as
%
\begin{equation}
  \begin{array}{c}
    \mVar{sel}[m] \mImp \mVar{oplace}[o] = b\hspace{-1.2pt}, \\
    \forall m \in \mMatchSet,
    \forall o \in \mCovers(m),
    \forall b \in \mEntry(m).
  \end{array}
  \labelEquation{preventing-control-flow-op-moves}
\end{equation}

As stated previously, each \gls{datum}~$d$ must be \gls{define.d}[d] in some
\gls{block}~\mbox{$b \in \mBlockSet$} such that $b$ \gls{dominate.b}[s] every
\gls{block} wherein $d$ is \gls{use.d}[d], excluding \gls{use.d}[s] made by the
\glspl{phi-match}.
%
To this end, let \mbox{$\mDefines(m) \subseteq \mOperandSet$} and
\mbox{$\mUses(m) \subseteq \mOperandSet$} denote the set of \glspl{datum}
\gls{define.d}[d] respectively \gls{use.d}[d] by \gls{match}~$m$.
%
Let also \mbox{$\mDom(b) \subseteq \mBlockSet$} denote the set of \glspl{block}
\gls{dominate.b}[d] by \gls{block}~$b$.
%
With these definitions, the \gls{constraint} is modeled as
%
\begin{equation}
  \begin{array}{c}
    \mVar{dplace}[d] \in \mDom(\mVar{oplace}[o]), \\
    \forall m \in \mMatchCompSet{\mPhi},
    \forall d \in \mUses(m),
    \forall o \in \mCovers(m),
  \end{array}
  \labelEquation{dom}
\end{equation}
%
where $\mMatchCompSet{\mPhi}$ denotes the set~$\mMatchSet$ without the
\glspl{phi-match}.

The restrictions imposed by the \glspl{definition edge} are modeled as
%
\begin{equation}
  \mVar{dplace}[d] = b,
  \forall \mPair{d}{b} \in \mDefEdgeSet,
  \labelEquation{def-edges}
\end{equation}
%
where $\mDefEdgeSet$ denotes the set of \glspl{definition edge} in $\mUFGraph$.
%
It is assumed that the \glspl{edge} in $\mDefEdgeSet$ have been reoriented such
that all \glspl{source} are either \glsshort{state node} or \glspl{value node}
and all \glspl{target} are \glspl{block node}.

So far, there is no connection between the $\mVar{oplace}$ and $\mVar{dplace}$
\glspl{variable}.
%
Intuitively, every \gls{datum} \gls{define.d}[d] by a \gls{match}~$m$ should be
placed in the same \gls{block} as $m$ together with all \glspl{operation}
covered by~$m$.
%
This alone, however, could result in an over-constrained \glsshort{constraint
  model} that prevents selection of certain \glspl{match}.
%
For example, assume a \gls{match}~$m$ of the \gls{UP graph} shown in
\refFigure{up-graph-examples-add-graph} on
\refPageOfFigure{up-graph-examples-add-graph}, where the \glspl{block node}
\irBlock{entry}, \irBlock{clamp}, and \irBlock{end} are matched to \glspl{block}
in $\mUFGraph$ labeled \irBlock{A}, \irBlock{B}, and \irBlock{C}, respectively.
%
Because of \refEquation{preventing-control-flow-op-moves}, $m$ must be placed in
the \irBlock{A}~\gls{block}.
%
But because of \refEquation{def-edges}, one of its \glspl{value node} must be
placed in the \irBlock{B}~\gls{block}.
%
Consequently, for such conditions the \gls{constraint} is relaxed as follows.
%
We say that a \gls{match} \gls!{span.b}[s] the \glspl{block} matched by the
\gls{UP graph}'s \glspl{block node} (hence $m$ \gls{span.b}[s] \glspl{block}
\irBlock{A}, \irBlock{B}, and~\irBlock{C}).
%
We also say that a \gls{match} \gls!{consume.b}[s] any matched \glspl{block}
where the corresponding \gls{block node} has both \gls{inbound.e} and
\gls{outbound.e} control-flow \glspl{edge} in the \gls{UP graph} (hence $m$
\gls{consume.b}[s] \gls{block}~\irBlock{B}).
%
Consequently, every \gls{datum}~$d$ \gls{define.d}[d] by a \gls{match}~$m$ must
be placed in the same \gls{block} as $m$ only if $m$ \gls{span.b}[s] no
\glspl{block}, otherwise $d$ may be \gls{define.d}[d] in any of the
\glspl{block} \gls{span.b}[ned] by~$m$.
%
If \mbox{$\mSpans(m) \subseteq B$} denotes the set of \glspl{block}
\gls{span.b}[ned] by \gls{match}~$m$, then this constraint is modeled as
%
\begin{equation}
  \begin{array}{c}
    \mVar{sel}[m]
    \mImp
    \mVar{dplace}[d] \in \mSet{\mVar{oplace}[o]} \cup \mSpans(m), \\
    \forall m \in \mMatchSet,
    \forall d \in \mDefines(m),
    \forall o \in \mCovers(m).
  \end{array}
  \labelEquation{spanning}
\end{equation}

\Glsshort{consume.b}[ing] a \gls{block} entails that the corresponding
\gls{instruction} assumes full control of the control flow to and from that
\gls{block}, which in turn means no \glspl{operation} covered by other
\glspl{match} may be placed in that \gls{block}.
%
Hence, if \mbox{$\mConsumes(m) \subseteq B$} denotes the set of \glspl{block}
\gls{consume.b}[d] by \gls{match}~$m$, then this constraint is modeled as
%
\begin{equation}
  \begin{array}{c}
    \mVar{sel}[m]
    \mImp
    \mVar{oplace}[o] \neq b\hspace{-1.2pt}, \\
    \forall m \in \mMatchSet,
    \forall o \in \mOpSet \setminus \mCovers(m),
    \forall b \in \mConsumes(m).
  \end{array}
  \labelEquation{consumption}
\end{equation}


\section{Modeling Data Copying}
\labelSection{modeling-data-copying}

\todo{write introduction}


\subsubsection{Variables}

The set of \glspl{variable} \mbox{$\mVar{loc}[d] \subseteq \mLocationSet \cup
  \mSet{\mNullLocation}$} models in which location \gls{datum}~$d$ is available,
where $\mLocationSet$ denotes the set of locations provided by the \gls{target
  machine} and $\mNullLocation$ denotes a special location not available for any
\glsshort{input datum} or \glspl{output datum}.
%
The special location is used for intermediate values produced internally by an
\gls{instruction} which can only be accessed by this very \gls{instruction}.
%
For example, the address computed by a memory load \gls{instruction} with a
sophisticated addressing mode is produced within the pipeline and thus cannot be
reused by other \glspl{instruction}.


\subsubsection{Constraints}

Every \gls{datum} must be made available in a location that is compatible for
all selected \glspl{match}.
%
To this end, let \mbox{$\mStores(m, d) \subseteq \mLocationSet$} denote the set
of compatible locations for a \gls{datum}~$d$ which is either an \glsshort{input
  datum} or \gls{output datum} to a \gls{match}~$m$.
%
Let also \mbox{$\mExtValues(m) \subseteq \mDataSet$} denote the set of values
which are either \glsshort{input datum} or \glspl{output datum} to~$m$ (note
that this does not include any \glspl{state node} \gls{define.d}[d] or
\gls{use.d}[d] by $m$).
%
Likewise, let \mbox{$\mIntValues(m) \subseteq \mDataSet$} denote the set of
values which are intermediate values produced by~$m$.
%
With these definitions, this \gls{constraint} is modeled as
%
\begin{equation}
  \begin{array}{c}
    \mVar{sel}[m]
    \mImp
    \mVar{loc}[d] \in \mStores(m, p), \\
    \forall m \in \mMatchSet,
    \forall d \in \mExtValues(m).
  \end{array}
  \labelEquation{valid-locations}
\end{equation}

The \gls{constraint} that intermediate values must be placed in the special
location is modeled as
%
\begin{equation}
  \begin{array}{c}
    \mVar{sel}[m]
    \mImp
    \mVar{loc}[d] = \mNullLocation, \\
    \forall m \in \mMatchSet,
    \forall d \in \mIntValues(m).
  \end{array}
  \labelEquation{locs-of-intermediate-values}
\end{equation}


\subsection{Copy Extension}

\todo{describe method}


\subsection{Handling Calling Conventions}

\paragraph{Constraints}


\section{Modeling Value Reuse}
\labelSection{modeling-value-reuse}

\subsection{Match Duplication}

\paragraph{Variables}
\paragraph{Constraints}

\subsection{Alternative Values}

\paragraph{Variables}

\paragraph{Constraints}

\begin{equation}
  \begin{array}{c}
    \mVar{dplace}[\hlDiff{\mVar{alt}[p]}[1pt]] \in \mDom(\mVar{oplace}[o]), \\
    \forall m \in \mMatchCompSet{\mPhi},
    \forall \hlDiff{p} \in \mUses(m),
    \forall o \in \mCovers(m)
  \end{array}
  \labelEquation{dom-alt}
\end{equation}

\begin{equation}
  \begin{array}{c}
    \mVar{sel}[m]
    \mImp
    \mVar{dplace}[\hlDiff{\mVar{alt}[p]}[1pt]] \in
      \mSet{\mVar{oplace}[o]} \cup \mSpans(m), \\
    \forall m \in \mMatchSet,
    \forall \hlDiff{p} \in \mDefines(m),
    \forall o \in \mCovers(m)
  \end{array}
  \labelEquation{spanning-alt}
\end{equation}

\begin{equation}
  \begin{array}{c}
    \mVar{sel}[m]
    \mImp
    \mVar{loc}[\hlDiff{\mVar{alt}[p]}[1pt]] \in \mStores(m, p), \\
    \forall m \in \mMatchSet,
    \forall \hlDiff{p} \in
      \hlDiff{\mOperandSet} \;\text{s.t.}
      \mStores(m, \hlDiff{p}) \neq \mEmptySet
  \end{array}
  \labelEquation{valid-locations-alt}
\end{equation}

\begin{equation}
  \begin{array}{c}
    \mVar{sel}[m]
    \mImp
    \mVar{loc}[\hlDiff{\mVar{alt[p]}}[1pt]] = \mNullLocation, \\
    \forall m \in \mMatchSet,
    \forall \hlDiff{p} \in \mIntValues(m)
  \end{array}
  \labelEquation{locs-of-intermediate-values-alt}
\end{equation}

\begin{equation}
  \begin{array}{c}
    \mVar{sel}[m]
    \mEq
    \mVar{inactive}[\mVar{alt}[p]], \\
    \forall m \in \mMatchSet[\mKill],
    \forall p \in \mDefines(m)
  \end{array}
  \labelEquation{inactivity-when-killed}
\end{equation}

\begin{equation}
  \begin{array}{c}
    \mVar{sel}[m]
    \mImp
    \neg \mVar{inactive}[\mVar{alt}[p]], \\
    \forall m \in \mMatchCompSet{\mKill},
    \forall p \in \mUses(m)
  \end{array}
  \labelEquation{inactivity-when-used}
\end{equation}

\subsection{Experimental Evaluation}

\subsection{Discussion}


\section{Modeling Block Ordering}
\labelSection{modeling-block-ordering}

\paragraph{Variables}

\paragraph{Constraints}

\begin{equation}
  \mCircuit\left(
    \cup_{b \in \mBlockSet} \{\mVar{succ}[b]\}
  \right)
  \labelEquation{block-order}
\end{equation}



\subsection{Handling Branch Fallthroughs}

\subsection{Branch Extension}

\paragraph{Variables}

\paragraph{Constraints}

\subsection{Dual-target Branch Patterns}

\paragraph{Variables}

\paragraph{Constraints}

\begin{equation}
  \begin{array}{c}
    \mVar{succ}[\mEntry(m)] = b \mOr \mbox{} \\
    \big(
      \mVar{succ}[\mVar{succ}[\mEntry(m)]] = b
      \mAnd
      \mEmptyBlock(\mVar{succ}[\mEntry(m)])
    \big) \\
    \forall \mPair{m}{b} \in \mFallThroughSet,
  \end{array}
  \labelEquation{fall-through}
\end{equation}
%
where
%
\begin{equation*}
  \mEmptyBlock(b) =
  \mVar{oplace}[o] \neq b
  \mOr
  \mVar{omatch}[o] \in \mMatchSet[\mNull],
  \forall o \in \mOpSet
\end{equation*}


\subsection{Experimental Evaluation}
\subsubsection{With or Without Block Ordering}
\subsubsection{Branch Extension Vs. Dual-target Branch Patterns}

\subsection{Discussion}


\section{Limitations}
\labelSection{cm-limitations}

\subsection{Recomputation}
\subsection{Common Subexpression Elimination}
\subsection{Implicit Sign/Zero Extensions and Truncations}
