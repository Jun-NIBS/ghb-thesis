% Copyright (c) 2017, Gabriel Hjort Blindell <ghb@kth.se>
%
% This work is licensed under a Creative Commons 4.0 International License (see
% LICENSE file or visit <http://creativecommons.org/licenses/by/4.0/> for a copy
% of the license).

\chapter{Constraint Model}
\labelChapter{constraint-model}

This chapter introduces the \gls{constraint model} for \gls{universal
  instruction selection}.
%
\RefSectionRange{modeling-global-instruction-selection}{modeling-block-ordering}
introduce the \glspl{variable} and \gls{constraint} for modeling \gls{global.is}
\gls{instruction selection}, \gls{global code motion}, \gls{data copying},
\gls{value reuse}, and \gls{block ordering}, respectively.
%
The \glsshort{constraint model} limitations are discussed in
\refSection{cm-limitations}.



\section{Modeling Global Instruction Selection}
\labelSection{modeling-global-instruction-selection}

Modeling \gls{global.is} \gls{instruction selection} entails that all
\glspl{operation} must be \gls{cover}[ed] and all \glspl{datum} must be
\gls{define.d}[d].
%
This could, however, lead to situations resulting in cyclic data dependencies,
which must be forbidden.


\subsection{Covering Operations and Defining Data}

In \gls{global.is}[ \gls{instruction selection}], a set of \glspl{match} must be
selected such that every \gls{operation} in a given \gls{UF graph} is covered.
%
There are two variants of this problem:
%
\begin{enumerate*}[label=(\arabic*)]
  \item each \gls{operation} must appear in \emph{exactly} one selected
    \gls{match}; and
%
  \item each \gls{operation} must appear in \emph{at least} one selected
    \gls{match}, hence allowing matches to \gls{overlap}.
\end{enumerate*}
%
The former problem is more common as it is stricter, resulting in simpler models
with smaller \glspl{solution space}.
%
It also allows use of \glspl{constraint} that enable strong \gls{propagation},
which is essential for curbing solving time and increasing scalability.

Depending on the \gls{instruction set}, the latter problem permits
\glspl{solution} with potentially higher code quality.
%
For example, assume a \gls{UF graph} where a sum is used as address in two
memory operations and a \gls{target machine} where the address can be computed
as part of the memory instructions.
%
A \gls{solution} to the latter problem would therefore only need two
instructions, whereas the former problem would require three instructions -- one
to compute the sum and two to perform the memory operations -- since the
addition is not allowed to be covered by both memory instructions.
%
In certain conditions, however, an add instruction may still be required.
%
For example, assume a \gls{UF graph} where the sum is also used in a
subtraction.
%
For this \gls{UF graph}, unless the \gls{target machine} has an instruction that
performs both an addition and a subtraction, a \gls{solution} to either problem
requires an add instruction to compute the sum.
%
Due to the increased complexity of the relaxed version of the problem, we model
exact coverage in this dissertation.

Similarly, every value and state must be produced by exactly one selected match.
%
If a \gls!{datum}~$d$ denotes either a \glsshort{state node} or \gls{value node}
in the \gls{UF graph}, then we say that a \gls{match}~$m$ \gls!{define.d}[s] $d$
if there exists an inbound \glsshort{state-flow edge} or \gls{data-flow edge} to
$d$ in the \gls{UP graph} from which $m$ was derived.
%
Similarly, $m$ \gls!{use.d}[s] $d$ if there exists an outbound
\glsshort{state-flow edge} or \gls{data-flow edge} to $d$ in the \gls{UP graph}
of $m$.


\subsubsection{Variables}

Given a \gls{UF graph}~$\mUFGraph$ and a set~$\mMatchSet$ of \glspl{match} found
for $\mUFGraph$, the set of \glspl{variable} \mbox{$\mVar{sel}[m] \in \mSet{0,
    1}$} models whether \gls{match}~\mbox{$m \in \mMatchSet$} is selected.
%
Hence $m$ is selected if \mbox{$\mVar{sel}[m] = 1$}, abbreviated
$\mVar{sel}[m]$, and not selected if \mbox{$\mVar{sel}[m] = 0$}, abbreviated
\mbox{$\neg\mVar{sel}[m]$}.

The set of \glspl{variable} \mbox{$\mVar{omatch}[o] \in \mMatchSet[o]$} models
which selected \gls{match} covers \gls{operation}~\mbox{$o \in
  \mOpSet$\hspace{-1pt}}, where $\mOpSet$ denotes the set of \glspl{operation}
in $\mUFGraph$, and \mbox{$\mMatchSet[o] \subseteq \mMatchSet$} denotes the set
of \glspl{match} that can cover~$o$.
%
Similarly, the set of \glspl{variable} \mbox{$\mVar{dmatch}[d] \in
  \mMatchSet[o]$} models which selected \gls{match} \gls{define.d}[s]
\gls{datum}~\mbox{$d \in \mDataSet$\hspace{-1.5pt}}, where $\mDataSet$ denotes
the set of \glspl{datum} in $\mUFGraph$, and \mbox{$\mMatchSet[d] \subseteq
  \mMatchSet$} denotes the set of \glspl{match} that can \gls{define.d}~$d$.


\subsubsection{Constraints}

The \gls{constraint} that every \gls{operation} must be covered is modeled as
%
\begin{equation}
  \mVar{omatch}[o] = m \mEq \mVar{sel}[m],
  \forall o \in \mOpSet \hspace{-1pt},
  \forall m \in \mMatchSet[o] \hspace{-1pt}.
  \labelEquation{operation-coverage}
\end{equation}
%
Hence an \gls{operation}~$o$ is covered by a match~$m$ if and only if $m$ is
selected.

Likewise, the \gls{constraint} that every \gls{datum} must be \gls{define.d}[d]
is modeled as
%
\begin{equation}
  \mVar{dmatch}[d] = m \mEq \mVar{sel}[m],
  \forall d \in \mDataSet \hspace{-1.5pt},
  \forall m \in \mMatchSet[d] \hspace{-1pt}.
  \labelEquation{data-definitions}
\end{equation}
%
Hence a \gls{datum}~$d$ is \gls{define.d}[d] by a match~$m$ if and only if $m$
is selected.


\subsection{Forbidding Cyclic Data Dependencies}
\labelSection{forbidding-cyclic-data-dependencies}

In certain cases, selecting \glspl{match} of \glspl{instruction} producing
multiple results -- for example, many modern processors provide memory
\glspl{instruction} that automatically increment or decrement the address value
-- could lead to cyclic data dependencies~\cite{EbnerEtAl:2008}.
%
\begin{filecontents*}{cyclic-data-deps-example-ir.c}
$\ldots$
$\irAssign{\irVar{p}[2]}{\irAdd{\irVar{p}[1]}{\irVar{4}}}$
$\irStore{\irVar{q}[1]}{\irVar{p}[2]}$
$\irAssign{\irVar{q}[2]}{\irAdd{\irVar{q}[1]}{\irVar{4}}}$
$\irStore{\irVar{p}[1]}{\irVar{q}[2]}$
\end{filecontents*}%
%
\begin{figure}
  \subcaptionbox{IR\labelFigure{cyclic-data-deps-example-ir}}%
                {%
                  \lstinputlisting[language=c,mathescape]%
                                  {cyclic-data-deps-example-ir.c}%
                }%
  \hfill%
  \subcaptionbox{%
                  UF graph, covered by two matches derived from an
                  auto-increment store instruction.
                  %
                  For brevity, the state node are not included%
                  \labelFigure{cyclic-data-deps-example-uf-graph}%
                }%
                [62mm]%
                {%
                  \input{%
                    figures/constraint-model/cyclic-data-deps-example-uf-graph%
                  }%
                }%
  \hfill%
  \subcaptionbox{%
                  Dependency graph%
                  \labelFigure{cyclic-data-deps-example-dep-graph}%
                }%
                [32mm]%
                {%
                  \input{%
                    figures/constraint-model/cyclic-data-deps-example-dep-graph%
                  }%
                }

  \caption{Example illustrating cyclic data dependencies}%
  \labelFigure{cyclic-data-deps-example}%
\end{figure}
%
An example of such a situation is given in \refFigure{cyclic-data-deps-example}.
%
If both \glspl{match} are selected, then either value~\irVar{p}[2] or
value~\irVar{q}[2] will be \gls{use.d}[d] before it is available (depending on
the instruction order), thus resulting in incorrect code.
%
Consequently, such combinations must be identified and forbidden.

We detect such combinations -- which could involve more than two \glspl{match}
-- by first constructing a \gls!{dependency graph}, where each \gls{node}
represents a \gls{match} and each \gls{edge}~$\mEdge{n}{m}$ indicates
that \gls{match}~$m$ \gls{use.d}[s] \glspl{datum} produced by \gls{match}~$n$.
%
A \gls{cycle} in this \gls{graph} corresponds a combination of \glspl{match}
which will lead to a cyclic data dependency if all \glspl{match} are selected.
%
Hence, we find all \glspl{cycle} in the \gls{dependency graph} -- we applied
\citeauthor{Johnson:1975}'s algorithm for this task~\cite{Johnson:1975} -- and
add \glspl{constraint} forbidding selection of all \glspl{match} appearing in a
\gls{cycle}.


\subsubsection{Constraints}

Given a set~\mbox{$\mForbiddenCombSet \subseteq \mPowerset{\mMatchSet}$} of
\glspl{cycle} found for the \gls{dependency graph} built from a \gls{UF graph}
and \gls{match set}, the \gls{constraint} forbidding cyclic data dependencies is
modeled as
%
\begin{equation}
  \sum_{m \in f} \mVar{sel}[m] < \mCard{f},
  \forall f \in \mForbiddenCombSet.
  \labelEquation{cyclic-data-deps}
\end{equation}
%
In other words, the number of selected \glspl{match} must be strictly less than
the number of \gls{match} appearing in the \gls{cycle}.


\section{Modeling Global Code Motion}
\labelSection{modeling-global-code-motion}

The \gls{global code motion} problem entails that \glspl{datum} must be placed
in \glspl{block} such that each definition of a \gls{datum}~$d$ precedes all
\gls{use.d}[s] of~$d$.
%
This condition can be expressed in terms of \gls{block} dominance.
%
Given a \gls{function}~$f$, a \gls{block}~$b$ in $f$ \gls!{dominate.b}[s]
another \gls{block}~$c$ in $f$ if $b$ appears on every control-flow path from
$f$'s \gls{entry block} to $c$ (see \refFigure{block-dominance-example} for an
example).
%
\begin{figure}
  \mbox{}%
  \hfill%
  \subcaptionbox{Control-flow graph\labelFigure{block-dominance-example-cfg}}%
                [34mm]%
                {%
                  \input{%
                    figures/constraint-model/block-dominance-example%
                  }%
                }%
  \hfill%
  \subcaptionbox{Dominance\labelFigure{block-dominance-example-doms}}%
                {%
                  \figureFontSize%
                  \begin{tabular}{cc}
                    \toprule
                      \tabhead Block
                    & \tabhead dominates\\
                    \midrule
                      \irBlock{entry}
                    & $\mSet{\irBlock{entry}, \irBlock{A}, \irBlock{B},
                        \irBlock{C}, \irBlock{D}, \irBlock{E}}$\\
                      \irBlock{A}
                    & $\mSet{\irBlock{A}, \irBlock{B}, \irBlock{C},
                        \irBlock{D}}$\\
                      \irBlock{B}
                    & $\mSet{\irBlock{B}}$\\
                      \irBlock{C}
                    & $\mSet{\irBlock{C}}$\\
                      \irBlock{D}
                    & $\mSet{\irBlock{D}}$\\
                      \irBlock{E}
                    & $\mSet{\irBlock{E}}$\\
                    \bottomrule
                  \end{tabular}%
                }%
  \hfill%
  \mbox{}

  \caption{Example of block dominance}
  \labelFigure{block-dominance-example}
\end{figure}%
%
By definition, a \gls{block} always \gls{dominate.b}[s] itself.
%
Hence a placement of \glspl{match} into \glspl{block} is a \gls{solution} to the
\gls{global code motion} problem if each \gls{datum}~$d$ is \gls{define.d}[d] by
some selected \gls{match} placed in a \gls{block}~$b$, and every
non-\gls{phi-match} \glsshort{use.d}[ing] $d$ is placed in a \gls{block}
\gls{dominate.b}[d] by~$b$.
%
The \glspl{phi-match} must be excluded since, due to the \glspl{definition
  edge}, at least one \gls{datum} used by such \glspl{match} must be
\gls{define.d}[d] in a \gls{block} that does not \gls{dominate.b} the
\gls{block} wherein the \gls{phi-match} must be placed.


\subsubsection{Variables}

The set of \glspl{variable} \mbox{$\mVar{oplace}[o] \in \mBlockSet$} models in
which \gls{block} \gls{operation}~$o$ is placed, where $\mBlockSet$ denotes the
set of \glspl{block} in $\mUFGraph$.
%
Likewise, the set of \glspl{variable} \mbox{$\mVar{dplace}[d] \in \mBlockSet$}
models in which \gls{block} the definition of \gls{datum}~$d$ is placed.


\subsubsection{Constraints}

Intuitively, all \glspl{operation} covered by a \gls{match}~$m$ must be placed
in the same \gls{block} wherein $m$ itself is placed.
%
Hence, if \mbox{$\mCovers(m) \subseteq \mOpSet$} denotes the set of
\glspl{operation} covered by \gls{match}~$m$, then this \gls{constraint} is
modeled as
%
\begin{equation}
  \begin{array}{c}
    \mVar{sel}[m] \mImp \mVar{oplace}[o_1] = \mVar{oplace}[o_2], \\
    \forall m \in \mMatchSet,
    \forall o_1\hspace{-1pt}, o_2 \in \mCovers(m).
  \end{array}
  \labelEquation{operation-placement}
\end{equation}
%
This also enables the placement of $m$ to be deduced from any of the
corresponding $\mVar{oplace}$~\glspl{variable}.

We prevent control-flow \glspl{operation} from being moved to another
\gls{block}, which in all likelihood would break \gls{program} semantics, by
forcing selected \glspl{match} with control flow to be placed in the \gls{block}
matched by the \gls{UP graph}'s \gls{entry block}.
%
Hence, if \mbox{$\mEntry(m) \subseteq \mBlockSet$} returns either the empty set
or a set containing only the \gls{entry block} of match~$m$ (when the \gls{UP
  graph} of $m$ has such a node), then this \gls{constraint} is modeled as
%
\begin{equation}
  \begin{array}{c}
    \mVar{sel}[m] \mImp \mVar{oplace}[o] = b\hspace{-1.2pt}, \\
    \forall m \in \mMatchSet,
    \forall o \in \mCovers(m),
    \forall b \in \mEntry(m).
  \end{array}
  \labelEquation{preventing-control-flow-op-moves}
\end{equation}

As stated previously, each \gls{datum}~$d$ must be \gls{define.d}[d] in some
\gls{block}~\mbox{$b \in \mBlockSet$} such that $b$ \gls{dominate.b}[s] every
\gls{block} wherein $d$ is \gls{use.d}[d], excluding \gls{use.d}[s] made by the
\glspl{phi-match}.
%
To this end, let \mbox{$\mDefines(m) \subseteq \mOperandSet$} and
\mbox{$\mUses(m) \subseteq \mOperandSet$} denote the set of \glspl{datum}
\gls{define.d}[d] respectively \gls{use.d}[d] by \gls{match}~$m$.
%
Let also \mbox{$\mDom(b) \subseteq \mBlockSet$} denote the set of \glspl{block}
\gls{dominate.b}[d] by \gls{block}~$b$.
%
With these definitions, the \gls{constraint} is modeled as
%
\begin{equation}
  \begin{array}{c}
    \mVar{dplace}[d] \in \mDom(\mVar{oplace}[o]), \\
    \forall m \in \mMatchCompSet{\mPhi},
    \forall d \in \mUses(m),
    \forall o \in \mCovers(m),
  \end{array}
  \labelEquation{dom}
\end{equation}
%
where $\mMatchCompSet{\mPhi}$ denotes the set~$\mMatchSet$ without the
\glspl{phi-match}.

The restrictions imposed by the \glspl{definition edge} are modeled as
%
\begin{equation}
  \mVar{dplace}[d] = b,
  \forall \mEdge{d}{b} \in \mDefEdgeSet,
  \labelEquation{def-edges}
\end{equation}
%
where $\mDefEdgeSet$ denotes the set of \glspl{definition edge} in $\mUFGraph$.
%
It is assumed that the \glspl{edge} in $\mDefEdgeSet$ have been reoriented such
that all \glspl{source} are either \glsshort{state node} or \glspl{value node}
and all \glspl{target} are \glspl{block node}.

So far, there is no connection between the $\mVar{oplace}$ and $\mVar{dplace}$
\glspl{variable}.
%
Intuitively, every \gls{datum} \gls{define.d}[d] by a \gls{match}~$m$ should be
placed in the same \gls{block} as $m$ together with all \glspl{operation}
covered by~$m$.
%
This alone, however, could result in an over-constrained \glsshort{constraint
  model} that prevents selection of certain \glspl{match}.
%
For example, assume a \gls{match}~$m$ of the \gls{UP graph} shown in
\refFigure{up-graph-examples-add-graph} on
\refPageOfFigure{up-graph-examples-add-graph}, where the \glspl{block node}
\irBlock{entry}, \irBlock{clamp}, and \irBlock{end} are matched to \glspl{block}
in $\mUFGraph$ labeled \irBlock{A}, \irBlock{B}, and \irBlock{C}, respectively.
%
Because of \refEquation{preventing-control-flow-op-moves}, $m$ must be placed in
the \irBlock{A}~\gls{block}.
%
But because of \refEquation{def-edges}, one of its \glspl{value node} must be
placed in the \irBlock{B}~\gls{block}.
%
Consequently, for such conditions the \gls{constraint} is relaxed as follows.
%
We say that a \gls{match} \gls!{span.b}[s] the \glspl{block} matched by the
\gls{UP graph}'s \glspl{block node} (hence $m$ \gls{span.b}[s] \glspl{block}
\irBlock{A}, \irBlock{B}, and~\irBlock{C}).
%
We also say that a \gls{match} \gls!{consume.b}[s] any matched \glspl{block}
where the corresponding \gls{block node} has both \gls{inbound.e} and
\gls{outbound.e} control-flow \glspl{edge} in the \gls{UP graph} (hence $m$
\gls{consume.b}[s] \gls{block}~\irBlock{B}).
%
Consequently, every \gls{datum}~$d$ \gls{define.d}[d] by a \gls{match}~$m$ must
be placed in the same \gls{block} as $m$ only if $m$ \gls{span.b}[s] no
\glspl{block}, otherwise $d$ may be \gls{define.d}[d] in any of the
\glspl{block} \gls{span.b}[ned] by~$m$.
%
If \mbox{$\mSpans(m) \subseteq B$} denotes the set of \glspl{block}
\gls{span.b}[ned] by \gls{match}~$m$, then this constraint is modeled as
%
\begin{equation}
  \begin{array}{c}
    \mVar{sel}[m]
    \mImp
    \mVar{dplace}[d] \in \mSet{\mVar{oplace}[o]} \cup \mSpans(m), \\
    \forall m \in \mMatchSet,
    \forall d \in \mDefines(m),
    \forall o \in \mCovers(m).
  \end{array}
  \labelEquation{spanning}
\end{equation}

\Glsshort{consume.b}[ing] a \gls{block} entails that the corresponding
\gls{instruction} assumes full control of the control flow to and from that
\gls{block}, which in turn means no \glspl{operation} covered by other
\glspl{match} may be placed in that \gls{block}.
%
Hence, if \mbox{$\mConsumes(m) \subseteq B$} denotes the set of \glspl{block}
\gls{consume.b}[d] by \gls{match}~$m$, then this constraint is modeled as
%
\begin{equation}
  \begin{array}{c}
    \mVar{sel}[m]
    \mImp
    \mVar{oplace}[o] \neq b\hspace{-1.2pt}, \\
    \forall m \in \mMatchSet,
    \forall o \in \mOpSet \setminus \mCovers(m),
    \forall b \in \mConsumes(m).
  \end{array}
  \labelEquation{consumption}
\end{equation}


\section{Modeling Data Copying}
\labelSection{modeling-data-copying}

The cost of \gls{data copying} is taken into account by keeping track of the
location requirements for the data used and produced by the selected
\glspl{match}.
%
The idea is as follows.
%
For each value~$v$ in the \gls{UF graph}, let a \gls{variable}~$\mVar{x}$ decide
in which location provided by the \gls{target machine} $v$ is stored.
%
A \gls{match}~$m$ that either \gls{use.d}[s] or \gls{define.d}[s] $v$ and
requires $v$ to be in one of a set~$L$ of locations can then enforce, if
selected, that \mbox{$\mVar{x} \in L$}.


\subsubsection{Variables}

The set of \glspl{variable} \mbox{$\mVar{loc}[d] \subseteq \mLocationSet \cup
  \mSet{\mNullLocation}$} models in which location \gls{datum}~$d$ is available,
where $\mLocationSet$ denotes the set of locations provided by the \gls{target
  machine} and $\mNullLocation$ denotes a special location for values that
cannot be reused across \glspl{instruction}.
%
In this context a location is an abstract representation, typically representing
a \gls{register}, but it could also indicate that the value is for example
stored in memory.
%
The special location is used for intermediate values produced internally by an
\gls{instruction} which can only be accessed by this very \gls{instruction}.
%
For example, the address computed by a memory load \gls{instruction} with a
sophisticated addressing mode is produced within the pipeline and thus cannot be
reused by other \glspl{instruction}.


\subsubsection{Constraints}

Every \gls{datum} must be made available in a location that is compatible for
all selected \glspl{match}.
%
If \mbox{$\mStores(m, d) \subseteq \mLocationSet \cup \mSet{\mNullLocation}$}
denotes the set of compatible locations (including the special location for
intermediate values) for a \gls{datum}~$d$ which is either \gls{define.d}[d] or
\gls{use.d}[d] by a \gls{match}~$m$, then this \gls{constraint} is modeled as
%
\begin{equation}
  \begin{array}{c}
    \mVar{sel}[m]
    \mImp
    \mVar{loc}[d] \in \mStores(m, d), \\
    \forall m \in \mMatchSet,
    \forall d \in \mDefines(m) \cup \mUses(m).
  \end{array}
  \labelEquation{compatible-locations}
\end{equation}


\subsection{Copy Extension}

Depending on the \gls{target machine}, \refEquation{compatible-locations} can
result in an over-constrained \glsshort{constraint model}.
%
For example, in many \glspl{target machine} the \gls{SIMD.instr}
\glspl{instruction} use a different set of \glspl{register} than the other,
general \glspl{instruction}.
%
In such situations, the \glspl{match} derived from the \gls{SIMD.instr}
\glspl{instruction} and the \glspl{match} derived from the general
\glspl{instruction} will have non-overlapping locations on the same
\glspl{datum} (that is, \mbox{$\mStores(m_1, d) \cap \mStores(m_2, d) =
  \mEmptySet$}), preventing selection of such \glspl{match}.

Since non-overlapping locations entails the need for copy \glspl{instruction},
we extend the \gls{UF graph} with \glspl!{copy node} through a process called
\gls!{copy extension}.
%
For each \gls{data-flow edge}~$\mEdge{v}{o}$, where $v$ is a \gls{value node}
and $o$ is an \gls{operation}, we remove this \gls{edge} and insert a new
\gls{copy node}~$c$, \gls{value node}~$v'$, and \glspl{data-flow edge} such that
\mbox{$\mEdge{v}{\mEdge{c}{\mEdge{v'}{o}}}$}.
%
If $o$ is a \gls{phi-node} then the corresponding \gls{definition edge}
connected to $v$ -- this is the \gls{edge} with the same \gls{outbound.en}
\gls{edge number} as the \gls{data-flow edge} -- is also moved to $v'$.
%
We also extend the \gls{pattern set} with a special \gls!{null-copy pattern},
with the \gls{graph} structure \mbox{$\mEdge{v}{\mEdge{c}{v'}}$}, that covers
$c$ at zero cost provided that \mbox{$\mVar{loc}[v] = \mVar{loc}[v']$}.
%
If the \gls{null-copy pattern} cannot be selected for covering a particular
\gls{copy node}, then this means an appropriate copy \gls{instruction} must be
emitted whose cost will be accounted for.


\subsection{Handling Calling Conventions}

The method for handling \gls{data copying} can also be used for handling calling
conventions of the specific \gls{target machine}.
%
\Glspl{constraint} that callee arguments (that is, arguments to the
\gls{function} under compilation) must reside in a specific location are modeled
as
%
\begin{equation}
  \mVar{loc}[d] \in \mArgLoc(d),
  \forall d \in \mArgSet,
  \labelEquation{function-args}
\end{equation}
%
where \mbox{$\mArgSet \subseteq \mDataSet$} denotes the set of \gls{function}
arguments in $\mUFGraph$ and \mbox{$\mArgLoc(d) \subseteq \mLocationSet$}
denotes the set of locations in which argument~$d$ resides.
%
Arguments residing on the stack can be signified using a special $\mMemLocation$
location, thus requiring a memory load \gls{instruction} in order to be used by
other \glspl{instruction}.

Locations for caller arguments (that is, arguments to \glspl{function} calls
made from within the \gls{function} under compilation) can be enforced either
through \refEquation{function-args} or through
\refEquation{compatible-locations}.
%
If the calling convention depends on the \gls{instruction} selected then the
former is needed, otherwise the latter is more suitable as the restrictions can
be enforced before a \gls{match} is selected.\!%
%
\footnote{%
  If exactly one \gls{match}~$m$ can cover a given \gls{function} call
  \gls{node}, then both \glspl{constraint} provide the same amount of
  propagation as \mbox{$\mVar{sel}[m] = 1$} will always hold for the implication
  in \refEquation{compatible-locations}.
}


\section{Modeling Value Reuse}
\labelSection{modeling-value-reuse}

Code quality can be increased if \glspl{instruction} are allowed to reuse copies
of values, which is a crucial feature to be expected in the code emitted by any
modern \gls{instruction selector}.
%
\begin{figure}
  \setlength{\opNodeDist}{12pt}%

  \mbox{}%
  \hfill%
  \subcaptionbox{%
                  A UF graph with two matches%
                  \labelFigure{value-reuse-example-uf-graph}%
                }%
                [50mm]%
                {% Copyright (c) 2017-2018, Gabriel Hjort Blindell <ghb@kth.se>
%
% This work is licensed under a Creative Commons Attribution-NoDerivatives 4.0
% International License (see LICENSE file or visit
% <http://creativecommons.org/licenses/by-nc-nd/4.0/> for details).
%
\begingroup%
\figureFont\figureFontSize%
\pgfdeclarelayer{background}%
\pgfsetlayers{background,main}%
\begin{tikzpicture}[
    outer match node/.style={%
      match node,
      draw=none,
      inner sep=0,
    },
  ]

  \node [value node] (v) {\strut\nVar{v}};
  \node [computation node, position=-135 degrees from v] (op1) {};
  \node [computation node, position=- 45 degrees from v] (op2) {};

  \begin{scope}[data flow]
    \draw (v) -- (op1);
    \draw (v) -- (op2);
  \end{scope}

  \begin{pgfonlayer}{background}
    % m1
    \node [outer match node, inner sep=3pt, fit=(v)] (m1a) {};
    \node [outer match node, fit=(op1)] (m1b) {};
    \def\pathMI{
      [bend left=45]
      (m1a.north west)
      to
      (m1a.north east)
      to
      (m1a.south east)
      --
      (m1b.south east)
      to
      (m1b.south west)
      to
      (m1b.north west)
      -- coordinate (m1)
      cycle
    }
    \path [fill=shade1]
          \pathMI;

    % m2
    \node [outer match node, inner sep=1.5pt, fit=(v)] (m2a) {};
    \node [outer match node, fit=(op2)] (m2b) {};
    \def\pathMII{
      [bend left=45]
      (m2a.south west)
      to
      (m2a.north west)
      to
      (m2a.north east)
      -- coordinate (m2)
      (m2b.north east)
      to
      (m2b.south east)
      to
      (m2b.south west)
      --
      cycle
    }
    \path [fill=shade1]
          \pathMII;

    \begin{scope}
      \path [clip] \pathMI;
      \path [fill=shade2]
            \pathMII;
    \end{scope}

    \draw [match line]
          \pathMI;
    \draw [match line]
          \pathMII;
  \end{pgfonlayer}

  % Match labels
  \begin{scope}[overlay]
    \node [match label, position=135 degrees from m1] (m1l) {$\strut m_1$};
    \node [match label, position= 45 degrees from m2] (m2l) {$\strut m_2$};
    \foreach \i in {1, 2} {
      \draw [match attachment line] (m\i) -- (m\i l);
    }
  \end{scope}
\end{tikzpicture}%
\endgroup%
}%
  \hfill%
  \subcaptionbox{%
                  UF graph after copy extension%
                  \labelFigure{value-reuse-example-after-ce}%
                }%
                [50mm]%
                {% Copyright (c) 2018, Gabriel Hjort Blindell <ghb@kth.se>
%
% This work is licensed under a Creative Commons 4.0 International License (see
% LICENSE file or visit <http://creativecommons.org/licenses/by/4.0/> for a copy
% of the license).
%
\begingroup%
\figureFont\figureFontSize%
\pgfdeclarelayer{background}%
\pgfsetlayers{background,main}%
\begin{tikzpicture}[
    outer match node/.style={%
      match node,
      draw=none,
      inner sep=0,
    },
  ]

  \node [value node] (v) {\strut\nVar{v}};
  \node [computation node, position=-135 degrees from v] (cp1) {\nCopy};
  \node [computation node, position=- 45 degrees from v] (cp2) {\nCopy};
  \node [value node, below=of cp1] (v1) {\strut\nVar{v}[\hspace{-1pt}1]};
  \node [value node, below=of cp2] (v2) {\strut\nVar{v}[\hspace{-1pt}2]};
  \node [computation node, below=of v1] (op1) {};
  \node [computation node, below=of v2] (op2) {};

  \begin{scope}[data flow]
    \draw (v) -- (cp1);
    \draw (v) -- (cp2);
    \draw (cp1) -- (v1);
    \draw (cp2) -- (v2);
    \draw (v1) -- (op1);
    \draw (v2) -- (op2);
  \end{scope}

  \begin{pgfonlayer}{background}
    % m1
    \node [outer match node, inner sep=2pt, fit=(v1)] (m1a) {};
    \node [outer match node, fit=(op1)] (m1b) {};
    \def\pathMI{
      [bend left=45]
      (m1a.west)
      to
      (m1a.north)
      to
      (m1a.east)
      --
      (m1b.east)
      to
      (m1b.south)
      to
      (m1b.west)
      -- coordinate (m1)
      cycle
    }
    \draw [match line, fill=shade1]
          \pathMI;

    % m2
    \node [outer match node, inner sep=2pt, fit=(v2)] (m2a) {};
    \node [outer match node, fit=(op2)] (m2b) {};
    \def\pathMII{
      [bend left=45]
      (m2a.west)
      to
      (m2a.north)
      to
      (m2a.east)
      -- coordinate (m2)
      (m2b.east)
      to
      (m2b.south)
      to
      (m2b.west)
      --
      cycle
    }
    \draw [match line, fill=shade1]
          \pathMII;
  \end{pgfonlayer}

  % Match labels
  \begin{scope}[overlay]
    \node [match label, left=of m1] (m1l) {$\strut m_1$};
    \node [match label, right=of m2] (m2l) {$\strut m_2$};
    \foreach \i in {1, 2} {
      \draw [match attachment line] (m\i) -- (m\i l);
    }
  \end{scope}
\end{tikzpicture}%
\endgroup%
}%
  \hfill%
  \mbox{}

  \vspace{\betweensubfigures}

  \mbox{}%
  \hfill%
  \subcaptionbox{%
                  Redundant copying of values%
                  \labelFigure{value-reuse-example-redundant-copies}%
                }%
                [50mm]%
                {%
                  \input{%
                    figures/constraint-model/%
                    value-reuse-example-redundant-copies%
                  }%
                }%
  \hfill%
  \subcaptionbox{%
                  Reuse of copied value%
                  \labelFigure{value-reuse-example-one-copy}%
                }%
                [50mm]%
                {%
                  \input{%
                    figures/constraint-model/value-reuse-example-one-copy%
                  }%
                }%
  \hfill%
  \mbox{}

  \caption{Example of value reuse}
  \labelFigure{value-reuse-example}
\end{figure}
%
See for example \refFigure{value-reuse-example-uf-graph}, where the
value~\irVar{v} is used by two \glspl{operation} which are coverable by
\glspl{match}~$m_1$ and~$m_2$.
%
After \gls{copy extension}, $m_1$ and $m_2$ use their own private copy --
\irVar{v}[1] and \irVar{v}[2], respectively -- of \irVar{v} (see
\refFigure{value-reuse-example-after-ce}).
%
Consequently, if both $m_1$ and $m_2$ require \irVar{v} to be copied -- assume,
for example, that \irVar{v} resides on the stack -- then two copy
\glspl{instruction} will be emitted (see
\refFigure{value-reuse-example-redundant-copies}).
%
However, if \irVar{v}[1] and \irVar{v}[2] could reside in the same location then
either of the values could be reused by either \gls{match}, thus necessitating
only one copy \gls{instruction} (see \refFigure{value-reuse-example-one-copy}).
%
We call this notion \gls!{value reuse}.

In this dissertation, we discuss two approaches for reusing values: \gls{match
  duplication} and \glspl{alternative value}.
%
We first introduce each in turn and then present experiments showing that one is
superior to the other.


\subsection{Match Duplication}

\todo{describe idea}


\subsection{Alternative Values}

\todo{describe idea}

\subsubsection{Variables}

\todo{introduce variables}

\subsubsection{Constraints}

\todo{explain constraints}

\begin{equation}
  \begin{array}{c}
    \mVar{dplace}[\hlDiff{\mVar{alt}[p]}[1pt]] \in \mDom(\mVar{oplace}[o]), \\
    \forall m \in \mMatchCompSet{\mPhi},
    \forall \hlDiff{p} \in \mUses(m),
    \forall o \in \mCovers(m)
  \end{array}
  \labelEquation{dom-alt}
\end{equation}

\begin{equation}
  \begin{array}{c}
    \mVar{sel}[m]
    \mImp
    \mVar{dplace}[\hlDiff{\mVar{alt}[p]}[1pt]] \in
      \mSet{\mVar{oplace}[o]} \cup \mSpans(m), \\
    \forall m \in \mMatchSet,
    \forall \hlDiff{p} \in \mDefines(m),
    \forall o \in \mCovers(m)
  \end{array}
  \labelEquation{spanning-alt}
\end{equation}

\begin{equation}
  \begin{array}{c}
    \mVar{sel}[m]
    \mImp
    \mVar{loc}[\hlDiff{\mVar{alt}[p]}[1pt]] \in \mStores(m, \hlDiff{p}), \\
    \forall m \in \mMatchSet,
    \forall \hlDiff{p} \in \mDefines(m) \cup \mUses(m).
  \end{array}
  \labelEquation{compatible-locations-alt}
\end{equation}

\begin{equation}
  \begin{array}{c}
    \mVar{sel}[m]
    \mEq
    \mVar{inactive}[\mVar{alt}[p]], \\
    \forall m \in \mMatchSet[\mKill],
    \forall p \in \mDefines(m)
  \end{array}
  \labelEquation{inactivity-when-killed}
\end{equation}

\begin{equation}
  \begin{array}{c}
    \mVar{sel}[m]
    \mImp
    \neg \mVar{inactive}[\mVar{alt}[p]], \\
    \forall m \in \mMatchCompSet{\mKill},
    \forall p \in \mUses(m)
  \end{array}
  \labelEquation{inactivity-when-used}
\end{equation}


\subsection{Experimental Evaluation}

\todo{insert data}

\todo{discuss data}


\section{Modeling Block Ordering}
\labelSection{modeling-block-ordering}

\paragraph{Variables}

\paragraph{Constraints}

\begin{equation}
  \mCircuit\left(
    \cup_{b \in \mBlockSet} \{\mVar{succ}[b]\}
  \right)
  \labelEquation{block-order}
\end{equation}



\subsection{Handling Branch Fallthroughs}

\subsection{Branch Extension}

\paragraph{Variables}

\paragraph{Constraints}

\subsection{Dual-target Branch Patterns}

\paragraph{Variables}

\paragraph{Constraints}

\begin{equation}
  \begin{array}{c}
    \mVar{succ}[\mEntry(m)] = b \mOr \mbox{} \\
    \big(
      \mVar{succ}[\mVar{succ}[\mEntry(m)]] = b
      \mAnd
      \mEmptyBlock(\mVar{succ}[\mEntry(m)])
    \big) \\
    \forall \mPair{m}{b} \in \mFallThroughSet,
  \end{array}
  \labelEquation{fall-through}
\end{equation}
%
where
%
\begin{equation*}
  \mEmptyBlock(b) =
  \mVar{oplace}[o] \neq b
  \mOr
  \mVar{omatch}[o] \in \mMatchSet[\mNull],
  \forall o \in \mOpSet
\end{equation*}


\subsection{Experimental Evaluation}
\subsubsection{With or Without Block Ordering}
\subsubsection{Branch Extension Vs. Dual-target Branch Patterns}

\subsection{Discussion}


\section{Limitations}
\labelSection{cm-limitations}

\subsection{Recomputation}
\subsection{Common Subexpression Elimination}
\subsection{Implicit Sign/Zero Extensions and Truncations}
