% Copyright (c) 2017, Gabriel Hjort Blindell <ghb@kth.se>
%
% This work is licensed under a Creative Commons 4.0 International License (see
% LICENSE file or visit <http://creativecommons.org/licenses/by/4.0/> for a copy
% of the license).

\chapter{Future Work}
\labelChapter{future-work}

\todo{add chapter outline}


\section{Integrating Instruction Scheduling}

In this context, the \gls{instruction scheduling} problem can be defined as
follows.
%
Given a \gls{block}~$b$ and a set of selected \glspl{match} placed in
$b$\hspace{-1pt}, assign to each \gls{match}~$m$ an issue cycle $c_m$ such that
\mbox{$c_m \!+ \mLat(m) \leq c_{m'}$}, where $\mLat(m)$ denotes the latency of
$m$\hspace{-1pt}, holds for every \gls{match}~$m'$ that \gls{use.d}[s]
\glspl{datum} \gls{define.d}[d] by $m$\hspace{-1pt}.
%
To model this problem, we require two new \glspl{variable}.


\subsubsection{Variables}

The set of \glspl{variable} \mbox{$\mVar{cycle}[m] \in \mNatNumSet$} models at
which cycle a \gls{match}~$m$ is scheduled, and the set of \glspl{variable}
\mbox{$\mVar{sched}[m, b] \in \mSet{0, 1}$} models whether $m$ is to be
scheduled in \gls{block}~$b$\hspace{-1pt}.


\subsubsection{Constraints}

If a selected \gls{match}~$m_1$ \gls{define.d}[s] a \gls{datum}~$d$ which is
used by another selected \gls{match}~$m_2$ and $m_1$ and $m_2$ are both placed
in the same \gls{block}, then $m_1$ must be scheduled before~$m_2$.
%
This can be modeled as
%
\begin{equation}
  \hspace*{-1em}
  \begin{array}{c}
    \mVar{sel}[m_1] \mAnd \mVar{sel}[m_2]
    \mAnd \mBlockOf(m_1) = \mBlockOf(m_2)
    \mAnd \mVar{alt}[p_1] = \mVar{alt}[p_2] \\
    \mbox{} \mImp
    \mVar{cycle}[m_1] + \mLat(m_1) \leq \mVar{cycle}[m_2] \\
    \forall m_1 \hspace{-1pt}, m_2 \in \mMatchSet,
    \forall p_1 \in \mDefines(m_1),
    \forall p_2 \in \mUses(m_2) \setminus \mDefines(m_2)
    \text{ \st }
    m_1 \neq m_2 \hspace{-.8pt},
  \end{array}
  \labelEquation{precedence}
\end{equation}
%
where \mbox{$\mBlockOf(m) \in \mBlockSet$} denotes the block wherein $m$ is
placed (see \refEquation{block-of-function}).

If the \gls{target machine} is a \gls{VLIW} architecture, then special care must
be taken to ensure that the resource capacities are not exceeded by the
\glspl{instruction}.
%
In this context, a resource capacity could be the number of \glspl{issue slot}
(that is, the number of \glspl{instruction} that can run in parallel) or the
number of functional units available.\!%
%
\footnote{%
  Non-\gls{VLIW} architectures can be modeled using the same approach by setting
  the number of \glspl{issue slot} to 1.%
}
%
To model these restrictions, we will apply the same approach as in
\cite{CastanedaLozanoEtAl:2014:LCTES} and use the \gls!{cumulative constraint},
which constrains the scheduling times for a given set of tasks such that the
capacity of a given resource is not exceeded.
%
\begin{figure}
  \centering%
  % Copyright (c) 2017, Gabriel Hjort Blindell <ghb@kth.se>
%
% This work is licensed under a Creative Commons 4.0 International License (see
% LICENSE file or visit <http://creativecommons.org/licenses/by/4.0/> for a copy
% of the license).
%
\begingroup%
\figureFont\figureFontSize%
\pgfdeclarelayer{foreground}%
\pgfsetlayers{main,foreground}%
\def\gridYSize{6mm}%
\def\gridXSize{3mm}%
\def\axisWidth{1pt}%
\begin{tikzpicture}[%
    axis/.style={
      ->,
      line width=\axisWidth,
    },
    limit/.style={
      -,
      draw=shade3,
      dashed,
      line width=\normalLineWidth,
    },
    task/.style={
      nothing,
      draw,
      fill=shade1,
      line width=\normalLineWidth,
    },
    label/.style={
      nothing,
      node distance=4pt,
    },
  ]

  % Grid
  \foreach \y in {0, ..., 3} {
    \foreach \x in {0, ..., 20} {
      \pgfmathtruncatemacro{\py}{\y - 1};
      \pgfmathtruncatemacro{\px}{\x - 1};
      \ifnum \x>0
        \coordinate (\y-\x) at ([xshift=\gridXSize] \y-\px);
      \else
        \ifnum \y>0
          \coordinate (\y-\x) at ([yshift=\gridYSize] \py-\x);
        \else
          \coordinate (\y-\x);
        \fi
      \fi
    }
  }

  % Axes
  \begin{pgfonlayer}{foreground}
    \begin{scope}[axis]
      \draw ([yshift=-.5*\axisWidth] 0-0)
            -- coordinate (y-axis)
            (3-0);
      \draw ([xshift=-.5*\axisWidth] 0-0)
            -- coordinate (x-axis)
            (0-20);
    \end{scope}
  \end{pgfonlayer}
  \node [label, left=of y-axis] {\rotatebox{90}{capacity}};
  \node [label, below=of x-axis] {time};

  % Limit
  \draw [limit] (2-0)
        -- coordinate [pos=.075] (limit)
        (2-20);
  \node [label, node distance=3pt, above=of limit] {limit};

  % Tasks
  \node [task, fit=(0-0) (1-5)] (t1) {};
  \node [task, fit=(1-3) (2-7)] (t2) {};
  \node [task, fit=(0-7) (2-12)] (t3) {};
  \node [task, fit=(0-12) (1-19)] (t4) {};
  \node [task, fit=(1-12) (2-17)] (t5) {};

  \foreach \i in {1, ..., 5} {
    \node [nothing] at (t\i) {$t_\i$};
  }
\end{tikzpicture}%
\endgroup%


  \caption[Example illustrating the cumulative constraint]%
          {%
            Example of a solution to the cumulative constraint.
            %
            Each box represents a task%
          }
  \labelFigure{cumulative-example}
\end{figure}
%
An example is shown in \refFigure{cumulative-example}.
%
Formally, the \gls{constraint} is defined as follows.
%
\begin{definition}[Cumulative Constraint]
  Let \mbox{$c \in \mNatNumSet$} represent the capacity of a resource to be used
  by $k$ optional tasks.
  %
  For each task~$i$, let \mbox{$\mVar{s}_i \in \mNatNumSet$} be a \gls{variable}
  representing the time at which $i$ is scheduled to start, \mbox{$l_i \in
    \mNatNumSet$} represent its latency, \mbox{$u_i \in \mNatNumSet$} represent
  the amount of resource required by~$i$, and \mbox{$\mVar{b}_i \in \mSet{0,
      1}$} be a \gls{variable} representing whether $i$ is scheduled.
  %
  Then
  %
  \begin{displaymath}
    \begin{array}{c}
      \mCumulative(
        c \hspace{-1pt},
        \mTuple{
          \mVar{s}_1 \hspace{-.8pt},
          l_1 \hspace{-.8pt},
          u_1 \hspace{-.8pt},
          \mVar{b}_1
        } \hspace{-1pt},
        \ldots,
        \mTuple{
          \mVar{s}_k \hspace{-.8pt},
          l_k \hspace{-.8pt},
          u_k \hspace{-.8pt},
          \mVar{b}_k
        }
      )
      \equiv \mbox{} \\
      \mSetBuilder*{%
                     \langle d_{\mVar{s}_1} \hspace{-.8pt}, \ldots, d_{\mVar{s}_k}
                             \hspace{-.8pt},
                             d_{\mVar{b}_1} \hspace{-.8pt}, \ldots, d_{\mVar{b}_k}
                             \hspace{-1pt}
                     \rangle
                   }%
                   {
                     \begin{array}{@{}l@{}}
                       \forall_{\! i} \: d_{\mVar{s}_i} \in \mDomain(\mVar{s}_i),
                       \forall_{\! i} \: d_{\mVar{b}_i} \in \mDomain(\mVar{b}_i)
                       \text{ \st} \\
                       \forall_{\! 1 \,\leq\, t \,<\, \mMax(\cup_i \mDomain(\mVar{s}_i))}
                         \displaystyle
                         \hspace{-1.2em}
                         \sum_{%
                                \substack{
                                  \forall{i} \text{ \st} \\
                                  d_{\mVar{s}_i} \,\leq\, t \,<\,
                                  d_{\mVar{s}_i} \hspace{-.8pt}+\, l_i
                                }
                              }
                         \hspace{-1.2em}
                           d_{\mVar{b}_i} \times u_i
                         \leq c
                     \end{array}
                   }\!.
    \end{array}
  \end{displaymath}
  \labelDefinition{cumulative}
\end{definition}

If \mbox{$\mCapOf(r) \in \mNatNumSet$} denotes the capacity of a
resource~\mbox{$r \in \mResourceSet$}, where $\mResourceSet$ denotes the set of
resources in the \gls{target machine}, \mbox{$\mLat(m) \in \mNatNumSet$} denotes
the latency of a \gls{match}~$m$\hspace{-.8pt}, and \mbox{$\mResourceUse(m, r)
  \in \mNatNumSet$} denotes the amount of $r$ used by $m$\hspace{-.8pt}, then
the \gls{constraint} can be modeled as
%
\begin{equation}
  \begin{array}{c}
    \mCumulative
    \!
    \left(
      \hspace{1pt}
      \mCapOf(r),
      \cup_{m \in \mMatchSet}
      \mTuple{
        \mVar{cycle}[m],
        \mLat(m),
        \mResourceUse(m, r),
        \mVar{sched}[m \hspace{-.8pt}, b]
      }
    \right) \\
    \forall r \in \mResourceSet,
    \forall b \in \mBlockSet \!.
  \end{array}
  \labelEquation{resource-capacities}
\end{equation}

Lastly, a \gls{match}~$m$ must be scheduled in \gls{block}~$b$ if and only if
$m$ is selected and placed in $b$, which can be modeled as
%
\begin{equation}
  \begin{array}{c}
    \mVar{sched}[m, b]
    \mEq
    \mVar{sel}[m] \mAnd \mBlockOf(m) = b \\
    \forall b \in \mBlockSet,
    \forall m \in \mMatchSet.
  \end{array}
  \labelEquation{optional-scheduling}
\end{equation}


\section{Integrating Register Allocation}

In this context, \gls{register allocation} can be described as the problem of
assigning a location to each \gls{datum}~$d$ such that the value in the location
is preserved until the last use of~$d$.
%
Like with \gls{instruction selection}, the problem can be considered at two
different scopes: \gls{local.ra} \gls{register allocation} and \gls{global.ra}
\gls{register allocation}.

\todo{model local register allocation}

\todo{model global register allocation}


\subsubsection{Constraints}

\todo{add constraints for local register allocation}

\todo{add constraints for global register allocation}
