% Copyright (c) 2017, Gabriel Hjort Blindell <ghb@kth.se>
%
% This work is licensed under a Creative Commons 4.0 International License (see
% LICENSE file or visit <http://creativecommons.org/licenses/by/4.0/> for a copy
% of the license).

\chapter{Introduction}

\section{Background and Motivation}

\todo{Explain code generation}
\todo{Explain instruction selection (what, why)}
\todo{Explain global code motion (what, why)}
\todo{Explain block ordering (what, why)}
\todo{Traditional approaches (how)}
\todo{Limitations (why)}
\todo{Combinatorial optimization (what)}
\todo{Combinatorial approaches (how)}
\todo{Limitations (why)}
\todo{Research goal (what)}

\section{Thesis Statement}

\begin{statement}
  Constraint programming is a flexible, practical, and competitive approach to
  instruction selection.
\end{statement}

\todo{Define ``flexible''}
\todo{Define ``practical''}
\todo{Define ``competitive''}

\section{Approach}

\todo{Overview figure}

\section{Contributions}

This dissertation makes six contributions to the areas of code generation and
constraint programming:

\begin{contributions}
\item \labelContribution{survey}
    a comprehensive and structured survey that covers over four decades of
    research in instruction selection;
  \item \labelContribution{representations}
    program and instruction representations that enable
    \begin{contributions}
      \item uniform treatment of data- and control-flow operations, and
      \item global instruction selection and global code motion to be combined
        and modeled as a constraint model;
    \end{contributions}
  \item \labelContribution{transformations}
    transformations on the representations that include
    \begin{contributions}
      \item copy extension (to perform and account for the cost of data
        copying), and
      \item alternative value extension (to enable value reuse);
    \end{contributions}
  \item \labelContribution{constraint-model}
    a constraint model that integrates -- for the first time -- global
    instruction selection together with global code motion, including data
    copying, block ordering, and value reuse;
  \item \labelContribution{solving-techniques}
    solving techniques that enables practical solving of the constraint model;
    and
  \item \labelContribution{experiments}
    thorough experiments demonstrating that the approach scales to medium-sized
    programs and yields better code than traditional approaches.
\end{contributions}

\todo{Where are these discussed further?}

\section{Outline}

\todo{Describe chapters}
\todo{Connect contributions with chapters}

\section{Publications}

This dissertation is based on material presented in the following publications:

\begin{publications}
  \item \labelPublication{survey}
    \fullcite{HjortBlindell:2016:Survey}.
  \item \labelPublication{cp-paper}
    \fullcite{HjortBlindellEtAl:2015:CP}.
  \item \labelPublication{cases-paper}
    \fullcite{HjortBlindellEtAl:2017:CASES}.
\end{publications}

\todo{Describe the contribution of each publication}
\todo{Connect contributions with publications}



\begin{table}
  \centering%
  \begin{tabular}{lcccccccc}
    & \refContribution{survey}
    & \refContribution{representations}
    & \refContribution{transformations}
    & \refContribution{constraint-model}
    & \refContribution{solving-techniques}
    & \refContribution{experiments} \\
    \refPublication{survey}
    & \supportYes
    & \supportNo
    & \supportNo
    & \supportNo
    & \supportNo
    & \supportNo \\
    \refPublication{cp-paper}
    & \supportNo
    & \supportYes
    & \supportYes
    & \supportYes
    & \supportYes
    & \supportYes \\
    \refPublication{cases-paper}
    & \supportNo
    & \supportNo
    & \supportYes
    & \supportYes
    & \supportYes
    & \supportYes
  \end{tabular}
\end{table}

The author also participated in the following publications, which are out of
scope for the dissertation:

\begin{publications}[resume]
  \item \fullcite{HjortBlindell:2013:Survey}.
  \item \fullcite{HjortBlindellEtAl:2014:FDL}.
  \item \fullcite{HjortBlindellEtAl:2016:FDL}.
  \item \fullcite{CastanedaLozanoEtAl:2014:LCTES}.
  \item \fullcite{CastanedaLozanoEtAl:2016:CC}.
\end{publications}

\todo{Describe briefly each publication and why they are excluded}
