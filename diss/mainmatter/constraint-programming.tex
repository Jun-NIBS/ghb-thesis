% Copyright (c) 2017, Gabriel Hjort Blindell <ghb@kth.se>
%
% This work is licensed under a Creative Commons 4.0 International License (see
% LICENSE file or visit <http://creativecommons.org/licenses/by/4.0/> for a copy
% of the license).

\chapter{Constraint Programming}
\labelChapter{constraint-programming}

This chapter describes \glsdesc{CP}, which is the method used in this
dissertation for modeling and solving the problems described in
\refChapter{introduction}.
%
\refSection{cp-overview} gives a brief overview (for a comprehensive overview of
\gls{CP}, see \cite{RossiEtAl:2006}).
%
\refSection{cp-modeling} describes how to describe problems as a \gls{constraint
  model}, and \refSection{cp-solving} describes the techniques applied in
solving these \glsplshort{constraint model}.



\section{Overview}
\labelSection{cp-overview}

As already mentioned, \glsdesc{CP} is a method for solving computationally hard
problems.
%
These problems are typically optimization problems, but the method can also be
applied to solve satisfaction problems.
%
In terms of modeling, \gls{CP} offers a higher level of abstraction than similar
methods such as \gls{IP} and \gls{SAT}~\cite{BiereEtAl:2009}.
%
For example, \gls{CP} provides dedicated constraints for capturing many
recurring problem structures that must be decomposed and reformulated in
\gls{IP} and \gls{SAT}~models.
%
This also makes \gls{CP} particularly suited for solving problems that appear in
\gls{instruction scheduling} and \gls{register allocation}, which is essential
for these tasks to be integrated with \gls{instruction selection}.



\section{Modeling}
\labelSection{cp-modeling}

To solve a problem using \gls{CP}, it must first be formulated as a
\gls{constraint model}~\cite{Smith:2006}.
%
A \gls!{constraint model} consists of two elements:%
%
\begin{inlinelist}[itemjoin={, }, itemjoin*={, and}]
  \item a set of \glspl{variable}
  \item a set of \glspl{constraint}
\end{inlinelist}.
%
\Glspl!{variable} represent problem decisions and take their values from a
finite domain.
%
The domain is typically is a range of integers, but it can also consist of real
numbers and complex structures such as string, sets, and
\glspl{graph}~\cite{Gervet:2006}.
%
\Glspl!{constraint} express relations between \glspl{variable} and forbid
\gls{variable} assignments that are illegal in the problem.
%
A \gls{variable} assignment that fulfills all \glspl{constraint} in the
\gls{constraint model} constitute a \gls!{solution}.
%
An example is shown in \refFigure{cp-example}.

\begin{figure}
  \centering%
  \figureFont\figureFontSize%
  \mbox{}%
  \hfill\hfill%
  \subcaptionbox{Constraint model\labelFigure{cp-example-model}}%
                {%
                  \begin{tabular}{lc}
                    \toprule
                    \multicolumn{1}{c}{\tabhead Variables}
                      & \tabhead Constraints\\
                    \midrule
                      $\mVar{x} \in \mSet{1, 2}$ & $\mVar{x} \neq \mVar{y}$\\
                      $\mVar{y} \in \mSet{1, 2}$ & $\mVar{x} \neq \mVar{z}$\\
                      $\mVar{z} \in \mSet{1, 2, 3, 4}$
                        & $\mVar{y} \neq \mVar{z}$\\
                    \bottomrule
                  \end{tabular}%
                }%
  \hfill%
  \subcaptionbox{Solutions\labelFigure{cp-example-solutions}}%
                {%
                  $\begin{array}{c}
                     \langle \mVar{x} = 1, \mVar{y} = 2, \mVar{z} = 3 \rangle \\
                     \langle \mVar{x} = 2, \mVar{y} = 1, \mVar{z} = 3 \rangle \\
                     \langle \mVar{x} = 1, \mVar{y} = 2, \mVar{z} = 4 \rangle \\
                     \langle \mVar{x} = 2, \mVar{y} = 1, \mVar{z} = 4 \rangle \\
                   \end{array}$%
                }%
  \hfill\hfill%
  \mbox{}

  \caption[Example of a constraint model]%
          {%
            Example of a constraint model, corresponding to a problem where
            three variables must be assigned values which are different from one
            another%
          }
  \labelFigure{cp-example}
\end{figure}



\paragraph{Global Constraints}

If a \gls!{binary.c}[ \gls{constraint}] is a \gls{constraint} involving two
\glspl{variable}, then a \gls!{global.c}[ \gls{constraint}] is a
\gls{constraint} involving three or more
\glspl{variable}~\cite{VanHoeveKatriel:2006}.
%
\Gls{global.c} \glspl{constraint} capture recurring problem structures and
improve solving compared to relations modeled using multiple \gls{binary.c}
\glspl{constraint}.
%
An example of a \gls{global.c} \gls{constraint} is $\mDistinct$ (also known as
$\mAllDiff$), which is defined as follows:
%
\begin{equation}
  \mDistinct(\mVar{x}_1, \ldots, \mVar{x}_k)
  \equiv
  \todo{fix}.
\end{equation}
%
In other words, $\mDistinct$ enforces all \glspl{variable} in a given set to
take distinct values.
%
Hence the \glspl{constraint} in \refFigure{cp-example} can be replaced by
$\mDistinct(\mVar{x}, \mVar{y}, \mVar{z})$.

\todo{discuss circuit}

\todo{discuss table}



\section{Solving}
\labelSection{cp-solving}

\todo{write}



\paragraph{Propagation}

\todo{write}



\paragraph{Search}

\todo{write}



\paragraph{Implied Constraints and Dominance Breaking}

\todo{write}



\paragraph{Presolving}

\todo{write}
